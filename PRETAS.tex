
\textbf{Gabriel Soares de Sousa} nasceu em Portugal, provavelmente na
década de 1540, e chegou ao Brasil, proveniente do Reino e a caminho das
Índias Orientais, em 1569. Não se sabe o que fez este viajante e
explorador desistir de seguir viagem com o restante da tripulação e
desembarcar no litoral da capitania da Bahia. Soares se fixou no
Recôncavo, onde se estabeleceu como senhor de engenho e proprietário
de terras, imóveis na cidade, bois e escravos. Após quinze anos vivendo
na colônia, viajou à Corte espanhola com o intuito de solicitar ao rei
Filipe \textsc{ii} permissão para empreender uma expedição para além do rio São
Francisco, em busca de pedras e metais preciosos. Enquanto esperava, em
Madri, o despacho real, Gabriel Soares colocou no papel as lembranças
dos dezessete anos em que viveu no Brasil, produzindo dois
textos, \textit{Roteiro geral com largas informações de toda a costa do
Brasil} e \textit{Memorial e declaração das grandezas da Bahia de Todos
os Santos}, que enviou ao valido do rei, D.~Cristóvão de Moura, em 1º
de março de 1587. Após seus pedidos serem aceitos pela Coroa e em posse
das concessões reais para realizar sua empreitada colonial, Soares
retornou à Bahia em 1591 para, no mesmo ano, partir para o sertão, onde
veio a falecer logo em seguida.
     
\textbf{Tratado descritivo do Brasil em 1587} reúne dois textos de
Gabriel Soares de Sousa enviados a um influente conselheiro do rei
Filipe \textsc{ii} de Espanha, no intuito de oferecer à Coroa informações acerca
da situação da colônia portuguesa e demonstrar o conhecimento do autor
sobre aquelas terras.  O \textit{Roteiro geral com largas informações
de toda a costa do Brasil} e o \textit{Memorial e declaração das
grandezas da Bahia de Todos os Santos, de sua fertilidade e das
notáveis partes que tem}, ainda que circulassem em cópias manuscritas
pela Europa e fossem citados por religiosos e viajantes, permaneceram
inéditos e anônimos ou apócrifos até o século \textsc{xix}, quando foram
recuperados, reunidos e publicados integralmente, com sua autoria
restituída, pelo historiador brasileiro Francisco Adolfo de Varnhagen.
Desde então, a obra tem despertado grande interesse dos estudiosos do
início da colonização do Brasil e é considerada por muitos o mais
importante texto quinhentista sobre o assunto. É fonte indispensável a
diferentes áreas do conhecimento, como botânica, geografia, história e
antropologia, pois as minuciosas descrições apresentadas por Soares
fornecem preciosas informações a respeito da fauna, flora, acidentes
geográficos, povos nativos e engenhos da costa do Brasil no
século \textsc{xvi}, sobretudo da Bahia de Todos os Santos.
  
\textbf{Fernanda Trindade Luciani} é mestre em História Social pela Universidade de São Paulo (\textsc{usp}) na área de Brasil Colonial. Atualmente, é doutoranda pela Faculdade de Educação da mesma universidade e desenvolve pesquisa na área de História da Educação sobre reformas curriculares no século \textsc{xxi}.

 
