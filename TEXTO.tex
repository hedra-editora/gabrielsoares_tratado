\chapter[Epístola do autor a Dom Cristóvão de Moura]{Epístola do autor a\break Dom
Cristóvão de Moura\footnoteInSection{ Transcrição, a partir do códice do acervo da
Biblioteca Guita e José Mindlin (daqui por diante \textsc{bgjm}), da carta
enviada por Gabriel Soares de Sousa a D.~Cristóvão de Moura, valido do rei
Filipe~\textsc{ii}. Ver nota 5 da Introdução.}}

\begin{linenumbers}

\textsc{Obrigado} de minha curiosidade, fiz por espaço de dezessete anos que residi no
Estado do Brasil muitas lembranças por escrito do que me pareceu digno de notar, as quais
tirei a limpo nesta Corte em este caderno, enquanto a dilação de meus requerimentos me deu
para isso lugar, ao que me dispus entendendo convir ao serviço del"-Rei nosso senhor,
compadecendo"-me da pouca notícia que nestes reinos se tem nas grandezas e estranhezas
desta Província: no que anteparei algumas vezes movido do conhecimento de mim mesmo,
entendendo que as obras que se escrevem não têm mais valor que o da reputação dos autores
delas; mas, como minha intenção não foi escrever história que deleitasse com estilo e boa
linguagem, não espero tirar louvor desta escritura e breve relação, em que se contém o que
pude alcançar da cosmografia e descrição deste Estado, que a Vossa Senhoria ofereço e me
fará mercê aceitá"-la, como está merecendo a vontade com que o faço, passando pelos
desconcertos dela, pois a confiança disso me fez suave o trabalho e tempo que em o
escrever gastei, de cuja substância se podem fazer muitas lembranças a Sua Majestade para
que folgue de as ter deste seu Estado. Para o engrandecer como está merecendo a quem Vossa
Senhoria faça dar a valia que lhe é devida para os moradores dele rogarem a Nosso Senhor
guarde a mui ilustre pessoa de Vossa Senhoria e lhe acrescente a vida por muitos anos. Em
Madri, ao primeiro de março de mil quinhentos e oitenta e sete.

\end{linenumbers}

\chapter[Primeira parte: Roteiro geral]{Primeira parte \subtitulo{Roteiro geral com largas
informações de toda a costa que pertence ao Estado do Brasil e a descrição de muitos
lugares dela, especialmente da Bahia de Todos os Santos}}
\hedramarkboth{Roteiro geral}{}

\section[Declaração do que se contém neste caderno]{Declaração do que se\break contém
neste caderno}

\begin{linenumbers}

Como todas as coisas têm fim, convém que tenham princípio, e como o de minha pretensão é
manifestar a grandeza, fertilidade e outras grandes partes que tem a Bahia de Todos os
Santos e demais Estados do Brasil, do que os reis passados tanto se descuidaram, a el"-rei
nosso senhor convém, e ao bem do seu serviço, que lhe mostre, por estas lembranças, os
grandes merecimentos deste seu Estado, as qualidades e estranhezas dele etc., para que lhe
ponha os olhos e bafeje com o seu poder, o qual se engrandeça e estenda a felicidade, com
que se engrandeceram todos os Estados que reinam debaixo de sua proteção, porque está
muito desamparado depois que el"-rei D.~João~\textsc{iii} passou desta vida para a
eterna,\footnote{ O rei português D. João \textsc{iii} (1502-1557), filho do rei
D. Manuel, governou entre 1521 e 1557.} o qual principiou com tanto zelo, que para o
engrandecer meteu nisso tanto cabedal, como é notório, o qual se vivera mais dez anos
deixara nele edificadas muitas cidades, vilas e fortalezas mui populosas, o que se não
efetuou depois do seu falecimento, antes se arruinaram algumas povoações que em seu tempo
se fizeram, em cujo reparo e acrescentamento estará bem empregado todo o cuidado que Sua
Majestade mandar ter deste novo reino, para se edificar nele um grande império, o
qual com pouca despesa destes reinos se fará tão soberano que seja um dos Estados do mundo
porque terá de costa mais de mil léguas, como se verá por este \textit{Tratado} no tocante
à cosmografia dele, cuja terra é quase toda muito fértil, mui sadia, fresca e lavada de
bons ares e regada de frescas e frias águas. Pela qual costa tem muitos, mui seguros e
grandes portos, para nele entrarem grandes armadas, com muita facilidade, para as quais
tem mais quantidade de madeira que nenhuma parte do mundo, e outros muitos aparelhos para
se poderem fazer.

É esta província mui abastada de mantimentos de muita substância e menos trabalhosos que
os de Espanha. Dão"-se nela muitas carnes, assim naturais dela, como das de Portugal, e
maravilhosos pescados; onde se dão melhores algodões que em outra parte sabida, e muitos
açúcares tão bons como na ilha da Madeira. Tem muito pau de que se fazem as tintas. Em
algumas partes dela se dá trigo, cevada e vinho muito bem, e em todas todos os frutos e
sementes de Espanha, do que haverá muita qualidade, se Sua Majestade mandar prover nisso
com muita instância e no descobrimento dos metais que nesta terra há, porque lhe não falta
ferro, aço, cobre, ouro, esmeralda, cristal e muito salitre; em cuja costa sai do mar
todos os anos muito bom âmbar; e de todas estas e outras podiam vir todos os anos a estes
reinos em tanta abastança, que se escusem os que vêm a eles dos estrangeiros, o que se
pode facilitar sem Sua Majestade meter mais cabedal neste Estado que o rendimento dele nos
primeiros anos; com o que pode mandar fortificar e prover do necessário à sua defensão, o
qual está hoje em tamanho perigo, que se nisso caírem os corsários, com mui pequena armada
se senhorearão desta província, por razão de não estarem as povoações dela fortificadas,
nem terem ordem com que possam resistir a qualquer afronta que se oferecer, do que vivem
os moradores dela tão atemorizados que estão sempre com o fato entrouxado para se
recolherem para o mato, como fazem com a vista de qualquer nau grande, temendo"-se serem
corsários, a esta afronta devia Sua Majestade mandar acudir com muita brevidade, pelo
perigo que há na tardança, o que não convém que haja, porque se os estrangeiros se
apoderarem desta terra custará muito lançá"-los fora dela pelo grande aparelho que têm para
nela se fortificarem, com o que se inquietará toda Espanha e custará a vida de muitos
capitães e soldados, e muitos milhões de ouro em armadas e no aparelho delas, ao que agora
se pode atalhar acudindo"-lhe com a presteza devida. Não se crê que Sua Majestade não tenha
a isto por falta de providência, pois lhe sobeja para as maiores empresas do mundo, mas de
informação do sobredito, que lhe não tem dado quem disso tem obrigação. E como a eu também
tenho de seu leal vassalo, satisfaço da minha parte com o que se contém neste Memorial,
que ordenei pela maneira seguinte.

\paragraph{[1] Em que se declara quem foram os primeiros descobridores da província do
Brasil, e como está arrumada} \quad
A província do Brasil está situada além da linha equinocial da parte do sul, debaixo da
qual começa ela a correr junto do rio que se diz das Amazonas, onde se principia o norte
da linha de demarcação e repartição; e vai correndo esta linha pelo sertão desta província
até 45 graus, pouco mais ou menos.

Esta terra se descobriu aos 24 dias do mês de abril de 1500 anos por Pedro Álvares
Cabral,\footnote{ Nas edições de Gabriel Soares de Sousa de 1851 e 1879, organizadas por
Adolfo de Varnhagen, encontra"-se ``aos 25 dias do mês de abril''. Talvez seja um erro
tipográfico da edição de 1851 não corrigido nas edições posteriores, pois Varnhagen, em
seus comentários a esse primeiro capítulo, assim se refere ao equívoco de Gabriel Soares:
``A costa do Brasil foi avistada por Cabral aos 22 de abril, e não aos
24''.\textit{Tratado Descritivo do Brasil em 1587}, Rio de Janeiro, Typographia Universal
de Laemmert, 1851; e \textit{Tratado Descritivo do Brasil em 1587}, Rio de Janeiro,
Typographia de João Inácio da Silva, 1879.}
que neste tempo ia por capitão"-mor para a Índia por mandado de el"-rei D.~Manuel, em cujo
nome tomou posse desta província, onde agora é a capitania de Porto Seguro, no lugar onde
já esteve a vila de Santa Cruz, que assim se chamou por se aqui arvorar uma muito grande,
por mando de Pedro Álvares Cabral, ao pé da qual mandou dizer, em seu dia, a 3 de maio,
uma solene missa, com muita festa, pelo qual respeito se chama a vila do mesmo nome, e a
província muitos anos foi nomeada por de Santa Cruz e de muitos Nova Lusitânia; e para
solenidade desta posse plantou este capitão no mesmo lugar um padrão com as armas de
Portugal, dos que trazia para o descobrimento da Índia para onde levava sua derrota.\footnote{
Derrota: rota.}

A estas partes foi depois mandado por Sua Alteza Gonçalo Coelho com três caravelas de
armada, para que descobrisse esta costa, com as quais andou por elas muitos meses
buscando"-lhe os portos e rios, em muitos dos quais entrou e assentou marcos dos que para
este descobrimento levava, no que passou grandes trabalhos pela pouca experiência e
informação que se até então tinha de como a costa corria, e do curso dos ventos com que se
navegava. E recolhendo"-se Gonçalo Coelho com perda de dois navios, com as informações que
pode alcançar, as veio dar a el"-rei D. João, o \textsc{iii}, que já neste tempo reinava, o
qual logo ordenou outra armada de caravelas que mandou a estas conquistas, a qual entregou
a Cristóvão Jacques,\footnote{ Cristóvão Jacques era filho bastardo do fidalgo Pêro
Jacques e veio ao Brasil a mando de D. Manuel, estabelecendo uma feitoria na capitania de
Pernambuco. Esteve também no recôncavo da Bahia, na enseada do Iguape e no Paraguaçu, onde
combateu e aprisionou naus francesas.} fidalgo da sua casa que nela foi por capitão"-mor, o
qual foi continuando no descobrimento desta costa e trabalhou um bom pedaço sobre aclarar
a navegação dela, e plantou em muitas partes padrões que para isso levava.

Contestando com a obrigação do seu regimento, e andando correndo a costa, foi dar com a
boca da Bahia, a que pôs o nome de Todos os Santos, pela qual entrou dentro, e andou
especulando por ela todos os seus recôncavos, em um dos quais, a que chamam o rio do
Paraguaçu, achou duas naus francesas que estavam ancoradas resgatando com o gentio, com as
quais se pôs às bombardas, e as meteu no fundo, com o que se satisfez e se recolheu para o
Reino, onde deu suas informações a Sua Alteza, que, com elas, e com as primeiras e outras
que lhe tinha dado Pero Lopes de Sousa,\footnote{ Em Varnhagen (1851 e 1879), ``Pedro
Lopes de Sousa''. Pero Lopes de Sousa e seu irmão Martim Afonso de Sousa partiram em 1530
para o Brasil em missão ordenada pelo rei português D. João \textsc{iii}. Em 1532, Pero
Lopes decidiu retornar ao reino e nessa viagem enfrentou e aprisionou dois navios
franceses ao largo de Pernambuco. Essa aventura lhe rendeu cinquenta léguas de terras no
litoral do Brasil, oferecidas pela Coroa. A ele foram atribuídas as capitanias de
Itamaracá, Santo Amaro e Santana.} que por esta costa também tinha andado com outra
armada, ordenou de fazer povoar essa província e repartir a terra dela em capitanias por
pessoas que se ofereciam a meter nisso todo o cabedal de suas fazendas,\footnote{ Em
Varnhagen (1851 e 1879), ``repartir as terras dela em capitães e pessoas que se
ofereceram''.} do que faremos particular menção em seu lugar.

\paragraph{[2] Em que se declara a repartição que fizeram os reis católicos de Castela com
el"-rei D. João \textsc{ii} de Portugal} \quad
Para se ficar bem entendendo onde demora e se estende o Estado do Brasil, convém que em
suma declaremos como se avieram os reis na repartição de suas conquistas, o que se fez por
esta maneira. Os reis católicos de Castela, D.~Fernando e D.~Isabel, sua mulher,\footnote{
Referência aos dois reis ibéricos, Isabel \textsc{i} de Castela e Fernando \textsc{ii} de
Aragão, cujo casamento resultou na unificação de seus reinos e na formação da atual
Espanha.} tinham começado de entender no descobrimento das Índias Ocidentais e algumas
ilhas, e porque esperavam de ir este descobrimento em tanto crescimento como foi, por
atalharem as diferenças que sobre isso se podiam oferecer, concertaram"-se com el"-rei D.
João, o \textsc{ii} de Portugal,\footnote{ Em Varnhagen (1851 e 1879) e nas edições que se
seguiram, à exceção da organizada por Pirajá da Silva, ``D.~João, o \textsc{iii}''. É
possível afirmar, contudo, que este trecho se refere ao rei português D.~João \textsc{ii},
que governou entre 1481-1495.} se fizesse uma repartição líquida, para cada um mandar
conquistar para sua parte livremente, sem escrúpulo de se prejudicarem. E acordados os
reis desta maneira, deram conta deste concerto ao Papa, que além de o aprovar, o louvou
muito. E como tiveram o consentimento de Sua Santidade, ordenaram a repartição desta
concordância, fazendo baliza na ilha do Cabo Verde, de barlavento mais ocidental, que se
entende a de Santo Antão, e contando dela vinte e um graus e meio equinociais de dezessete
léguas e meia de cada grau, e lançada daqui uma linha meridiana de norte"-sul, que ficassem
as terras e ilhas que estavam por descobrir para a parte do Oriente, da Coroa de
Portugal;\footnote{ O acordo aqui descrito é o conhecido Tratado de Tordesilhas, assim
denominado por ter sido assinado na povoação castelhana de Tordesillas, a 7 de junho de
1494, pelos reis de Portugal e Castela.} linha mental como está declarado, fica o Estado
do Brasil da dita Coroa, qual se começa além da ponta do rio das Amazonas da banda de
este,\footnote{ Em Varnhagen (1851 e 1879), ``banda de oeste''.} pela terra dos charybas, donde se principia o norte desta província, e indo
correndo esta linha pelo sertão dela ao sul parte o Brasil e conquistas dele além da Baía
de São Matias, por quarenta e cinco graus pouco mais ou menos, distantes da linha
equinocial, e altura do pólo antártico, e por esta conta tem de costa mil e cinquenta
léguas, como pelas cartas se pode ver segundo a opinião de Pero Nunes,\footnote{ Em
Varnhagen (1851 e 1879), ``Pedro Nunes''.} que nesta arte atinou melhor que todos os do
seu tempo.

\paragraph{[3] Em que se declara o princípio de onde começa a correr a costa do Estado do
Brasil} \quad
Mostra"-se claramente, segundo o que se contém neste capítulo atrás, que se começa a costa
do Brasil além do rio das Amazonas da banda de oeste pela terra que se diz dos Caribas do
rio de Vicente Pinzon,\footnote{ O então rio de Vicente Yañez Pinzon foi mais tarde
chamado de Wiapoc ou Oyapoc, pelos franceses. A referência aqui é, portanto, ao atual rio
Oiapoque, situado no Amapá.} que mora debaixo da linha. Desse rio de Vicente Pinzon à
ponta do rio das Amazonas, a que chamam o cabo Corso, são quinze léguas, a qual ponta está
debaixo da linha equinocial; dessa ponta do rio à outra ponta da banda de leste são trinta
e seis léguas. E ao mar doze léguas da boca desse rio estão ilhas, as quais demoram em
altura de um terço de grau da banda do sul. Essas ilhas se mostram na carta mais chegadas
à terra, o que é erro manifesto. Nessas ilhas há bons portos para surgirem navios, mas
para bem hão se de buscar de baixa"-mar, nordeste"-sudoeste, porque nesta conjunção se
descobre melhor o canal. A este rio chama o gentio de Mar Doce,\footnote{ Mar Doce era o
próprio rio Amazonas, assim designado pelas populações nativas pela sua extensão e volume
de água.} por ser um dos maiores do mundo, o qual é muito povoado de gentio doméstico e
bem acondicionado, e segundo a informação que se deste rio tem, vem do sertão mais de mil
léguas até o mar; pelo qual há muitas ilhas grandes e pequenas quase todas povoadas de
gentio de diferentes nações e costumes, e muito dele costuma pelejar com flechas ervadas.
Mas toda a gente que por estas ilhas vive, anda despida ao modo do mais gentio do Brasil e
usam dos mesmos mantimentos e muita parte dos seus costumes; e na boca deste rio, e por
ele acima algumas léguas, com parte da costa da banda do leste, é povoado de Tapuyas,
gente branda e mais tratável e doméstica que o mais gentio que há na costa do Brasil, de
cujos costumes diremos adiante em seu lugar.

\paragraph{[4] Em que se dão em suma algumas informações que se têm deste rio das Amazonas} \quad
Como não há coisa que se encubra aos homens que querem cometer grandes empresas, não pôde
estar encoberto este rio do mar Doce ou das Amazonas ao capitão Francisco de Orellana
que,\footnote{ Em Varnhagen (1851), ``Francisco Arelhana''. Conquistador espanhol,
Orellana (c.1490-1550) participou da conquista do Peru junto a Francisco Pizarro.} andando
na conquista do Peru em companhia do governador Francisco Pizarro,\footnote{ Francisco
Pizarro foi o conquistador do Peru que subjugou o Império Inca, conquistando Cuzco em 1533
e fundando a cidade de Lima em 1535.} e indo por seu mandado com certa gente de cavalo
descobrindo a terra, entrou por ela adentro tanto espaço que se achou perto do nascimento
deste rio. E vendo"-o caudaloso, fez junto dele embarcações, segundo o costume daquelas
partes, em as quais se embarcou com a gente que trazia e se veio por este rio abaixo, em o
qual se houveram de perder por levar grande fúria a correnteza, e com muito trabalho
tornou a tomar porto em povoado, na qual jornada teve muitos encontros de guerra com o
gentio e com um grande exército de mulheres que com ele pelejaram com arcos e flechas, de
onde o rio tomou o nome das Amazonas. Livrando"-se este capitão deste perigo e dos mais por
onde passou, veio tanto por este rio abaixo até que chegou ao mar; e dele foi ter a uma
ilha que se chama a Margarita,\footnote{ A Ilha de Margarita, que pertence à Venezuela,
situa"-se no mar do Caribe à nordeste de Caracas.} donde se passou à Espanha. Dando suas
informações ao imperador Carlos \textsc{v}, que está em glória, lhe ordenou uma armada de
quatro naus para cometer esta empresa, em a qual partiu, do porto de São Lucar,\footnote{
Sanlúcar de Barrameda é uma cidade espanhola na Província de Cádiz, Andaluzia, banhada
pelo Oceano Atlântico.} com sua mulher para ir povoar a boca deste rio, e o ir
conquistando por ele acima, o que não houve efeito por na mesma boca deste rio falecer
este capitão de sua doença, de onde sua mulher se tornou com a mesma armada para a
Espanha.

Neste tempo, pouco mais ou menos, andava correndo a costa do Brasil em uma caravela, como
aventureiro, Luís de Melo da Silva,\footnote{ Em Varnhagen (1851 e 1879), ``Luís de Melo''.
Foi o segundo donatário da capitania do Maranhão, concedida pelo rei D. João
\textsc{iii}.} filho do alcaide"-mor de Elvas, o qual, querendo passar a Pernambuco,
desgarrou com o tempo e as águas por esta costa abaixo, e vindo correndo a ribeira, entrou
no rio do Maranhão, e neste das Amazonas, de cuja grandeza se contentou muito; e tomou
língua do gentio de cuja facilidade ficou satisfeito,\footnote{ Em Varnhagen (1851 e
1879), ``de cuja fertilidade ficou satisfeito''.} e muito mais das grandes informações que
na ilha da Margarita lhe deram alguns soldados, que ali achou, que ficaram da companhia do
capitão Francisco de Orellana, os quais facilitaram a Luís de Melo a navegação deste rio,
e que com pouco cabedal e trabalho adquirisse por ele acima muito ouro e prata, do que
movido Luís de Melo, se veio à Espanha, e alcançou licença de el"-rei D.~João \textsc{iii}
de Portugal para armar a sua custa e cometer esta empresa, para o que se fez prestes na
cidade de Lisboa; e partiu do porto dela com três naus e duas caravelas com as quais se
perdeu nos baixos do Maranhão, com a maior parte da gente que levava; e ele com algumas
pessoas escaparam nos batéis e uma caravela em que foi ter às Antilhas. E depois deste
fidalgo ter em Portugal, se passou à Índia, onde acabou valorosos feitos; e vindo"-se para
o Reino muito rico e com tenção de tornar a cometer esta jornada, acabou no caminho em a
nau São Francisco, que desapareceu sem até hoje se saber novas dele.

\paragraph{[5] Que declara a costa da ponta do rio das Amazonas até o do Maranhão} \quad
A ponta do leste do rio das Amazonas está em um grau da banda do sul; desta ponta ao rio
da Lama são trinta e cinco léguas, a qual está em altura de um grau e três quartos; e
ainda que este rio se chame da Lama, podem entrar por ele adentro e estarem muito seguras
de todo o tempo, naus de duzentos tonéis, o qual rio entra pela terra adentro muitas
léguas.

Deste rio à ponta dos baixos são nove léguas, a qual está na mesma altura de um grau e
três quartos. Nesta ponta há abrigada para os barcos da costa poderem ancorar.

Da ponta dos baixos à ponta do rio Maranhão são dez léguas, onde chega a serra Escalvada,
e entre ponta e ponta tem a costa algumas abrigadas, onde podem ancorar navios da costa, a
qual ponta está em dois graus da banda do sul.

Até aqui se corre a costa noroeste"-sueste e toma da quarta de leste"-oeste; e dessa ponta
do rio a outra ponta são dezessete léguas, a qual está em altura de dois graus e três
quartos. Tem este rio do Maranhão na boca entre ponta e ponta delas para dentro uma ilha
que se chama das Vacas, que será de três léguas, onde esteve Ayres da Cunha quando se
perdeu com sua armada nestes baixos; e aqui nesta ilha estiveram também os filhos de João
de Barros\footnote{ Recebendo de D.~João \textsc{iii} duas capitanias em parceria com
Ayres da Cunha e Fernando Álvares de Andrade, João de Barros (c.1496-1570) decidiu
financiar uma expedição para ocupar suas terras na região do atual Maranhão. Contudo,
essas tentativas de colonização foram fracassadas e Barros perdeu grande parte de sua
fortuna. Conhecido também por seus escritos e considerado um dos primeiros historiadores e
gramáticos da língua portuguesa, publicou, entre outras obras, \textit{Gramática da Língua
Portuguesa} (1540) e a primeira parte de \textit{Décadas da Ásia} (1552), uma narrativa
dos feitos lusos nas Índias orientais.} e aí tiveram povoado, quando também se perderam
nos baixos deste rio, onde fizeram pazes com o gentio Tapuya, que tem povoado parte desta
costa, e por este rio acima, onde mandavam resgatar mantimentos e outras coisas para
remédio de sua mantença.

Por este rio entrou um Bastião Marinho, piloto da costa, com um caravelão, e foi por ele
acima algumas vinte léguas, onde achou muitas ilhas cheias de arvoredo e a terra delas
alcantilada com sofrível fundo; e muitos braços em que entram muitos rios que se metem
neste, o qual afirmou ser toda a terra fresca, cheia de arvoredo e povoada de gentio, e as
ilhas também. Neste rio entra o de Pindaré, que vem de muito longe.

Para se entrar neste rio do Maranhão, vindo do mar em fora, há de se chegar bem à terra da
banda de leste por fugir dos baixios e do aparcelado, e quem entrar por entre ela e a ilha
entra seguro.

Quem houver de ir deste rio do Maranhão para o da Lama ou para o das Amazonas, há de se
lançar por fora dos baixios com a sonda na mão, e não vá por menos de doze
braças,\footnote{ Medida de comprimento que equivale, no Brasil e em Portugal, à distância
entre um punho e outro, ou entre a extremidade de uma mão aberta e a outra, em um adulto
com os braços estendidos horizontalmente para os lados, o que acabou por se definir como
equivalente a 2,2 metros.} porque esta costa tem aqui dez léguas ao mar, vaza e enche
nela a maré muito depressa, e em conjunção de Lua tem grandes macaréus;\footnote{ Macaréu
ou pororoca é o fenômeno natural de elevação das águas próximo à foz do rio, provocado
pelo encontro das correntes fluviais com as águas oceânicas.} mas para bem não se há de
cometer o canal de nenhum destes rios senão de baixa"-mar na costa, o que se pode saber
pela Lua, o que convém que seja, pelos grandes perigos que nesta entrada se oferecem,
assim de macaréus, como por espraiar e esparcelar o mar oito e dez léguas da terra, pelo
que é forçado a chegar"-se à terra de baixa"-mar, pois então se descobre o canal mui bem; e
neste rio do Maranhão não podem entrar, por este respeito, navios grandes.

\paragraph{[6] Em que se declara a costa do rio do Maranhão até o Rio Grande} \quad
Atrás fica dito como a ponta de sudeste do rio do Maranhão, que se chama esparcelada, está
em dois graus e três quartos. Desta ponta à baía dos Santos são treze léguas, a qual está
na mesma altura, e esta baía é muito suja e tem alguns ilhéus; mas também entram nela
muitos navios da costa, onde têm surgidouro\footnote{Surgidouro: ancoradouro.} e boa abrigada e maneira para se fazer aguada
nela. Desta baía dos Santos ao rio de João de Lisboa são quatro léguas, o qual está na
mesma altura, onde também entram caravelões, por terem nele grande abrigada. Do rio de
João de Lisboa à baía dos Reis são nove léguas, a qual está em dois graus. Nesta baía
estão algumas ilhas alagadas da maré de águas vivas por entre as quais entram caravelões e
surgem à vontade. Desta baía ao rio do Meio são dezessete léguas, o qual está na mesma
altura de dois graus, onde também entram caravelões. Entre este e a baía dos Reis entra
outro rio que se chama do Parcel,\footnote{ Recife que aflora à água; leito do mar de pouca
profundidade.} onde também os navios da costa têm boa colheita. Deste rio do Meio à baía
do Ano Bom são onze léguas, a qual costa está na mesma altura de dois graus, aonde entram
navios da costa e têm muito boa colheita, a qual baía tem um grande baixo. No meio e
dentro dela se vêm meter no mar o Rio Grande dos Tapuyas, e se navega um grande espaço
pela terra adentro e vem de muito longe; o qual se chama dos Tapuyas por eles virem por
ele abaixo em canoas a mariscar ao mar desta baía, da qual à baía da Coroa são dez léguas;
e está na mesma altura onde entram e surgem caravelões da costa. Da baía da Coroa até o
Rio Grande são três léguas, onde começaremos o capítulo que segue. E corre"-se a costa até
aqui leste"-oeste.

\paragraph{[7] Em que se declara a costa do Rio Grande até a do Jagoarive} \quad
Como fica dito, o Rio Grande está em dois graus da parte do sul, o qual vem de muito longe
e traz muita água, por se meterem nele muitos rios; e, segundo a informação do gentio,
nasce de uma lagoa em que se afirma acharem"-se muitas pérolas. Perdendo"-se, haverá
dezesseis anos, um navio nos baixos do Maranhão, da gente que escapou dele que veio por
terra, afirmou um Nicolau de Rezende, desta companhia, que a terra toda ao longo do mar
até este Rio Grande era escalvada a maior parte dela, e outra cheia de palmares bravos, e
que achara uma lagoa muito grande, que seria de vinte léguas pouco mais ou menos; e que ao
longo dela era a terra fresca e coberta de arvoredo; e que mais adiante achara outra muito
maior a que não vira o fim, mas que a terra que vizinhava com ela era fresca e escalvada,
e que em uma e em outra havia grandes pescarias, de que se aproveitavam os Tapuyas que
viviam por esta costa até este Rio Grande, dos quais disse que recebera com os mais
companheiros bom tratamento. Por este Rio Grande entram navios da costa e têm nele boa
colheita, o qual se navega com barcos algumas léguas. Deste Rio Grande ao dos Negros são
sete léguas, o qual está em altura de dois graus e um quarto; e do rio dos Negros às
Barreiras Vermelhas são seis léguas, que estão na mesma altura; e numa parte e noutra têm
os navios da costa surgidouro e abrigada. Das Barreiras Vermelhas à ponta dos Fumos são
quatro léguas, a qual está em dois graus e um terço. Desta ponta do rio da Cruz são sete
léguas e está em dois graus e meio em que também têm colheita os navios da costa. Afirma o
gentio que nasce este rio de uma lagoa, ou junto dela, onde também se criam pérolas, e
chama"-se este rio da Cruz, porque se metem nele perto do mar dois riachos, em direito um
do outro, com que fica a água em cruz. Deste rio ao do Parcel são oito léguas, o qual está
em dois graus e meio; e faz"-se na boca deste rio uma baía toda esparcelada. Do rio do
Parcel à enseada do Macoripe são onze léguas, e está na mesma altura, a qual enseada é
muito grande e ao longo dela navegam navios da costa; mas dentro, em toda, têm bom
surgidouro e abrigo; e no rio das Ostras, que fica entre esta enseada e a do Parcel o têm
também. Da enseada do Macoripe ao monte de Li são quinze léguas e está em altura de dois
graus e dois terços, onde há porto e abrigada para os navios da costa; e entre este porto
e a enseada de Macoripe têm os mesmos navios surgidouro e abrigada no porto que se diz dos
parcéis. Do monte de Li ao rio Jagoarive são dez léguas, o qual está em dois graus e três
quartos, e junto da barra deste rio se mete outro nele, que se chama o Rio Grande, que é
extremo entre os Tapuyas e os pitiguoares.\footnote{ Em Varnhagen (1851 e 1879),
``pitigoares''.} Neste rio entram navios de honesto porte até onde se corre a costa
leste"-oeste; a terra daqui até o Maranhão é quase toda escalvada; e quem quiser navegar
por ela e entrar em qualquer porto dos nomeados há de entrar neste rio de Jagoarive por
entre os baixos e a terra porque tudo até o Maranhão defronte da costa são baixos, e pode
navegar sempre por entre eles e a terra, por fundo de três braças e duas e meia, achando
tudo limpo, e quanto se chegar mais à terra se achará mais fundo. Nesta boca do Jagoarive
está uma enseada onde navios de todo o porte podem ancorar e estar seguros.

\paragraph{[8] Em que se declara a costa do rio de Jagoarive até o cabo de São Roque} \quad
Do rio Jagoarive de que se trata acima até a baía dos Arrecifes são oito léguas, a qual
demora em altura de três graus. Nesta baía se descobrem de baixa"-mar muitas fontes de água
doce muito boa, onde bebem os peixes"-bois, de que aí há muitos, que se matam arpoando"-os
assim o gentio pitiguoar, que aqui vinha, como os caravelões da costa, que por aqui passam
desgarrados, onde acham bom surgidouro e abrigada.

Desta baía ao rio São Miguel são sete léguas, a qual está em altura de três graus e dois
terços, em a qual os navios da costa surgem por acharem nela boa abrigada. Desta baía ao
Rio Grande são quatro léguas o qual está em altura de quatro graus. Este rio tem duas
pontas saídas para o mar, e entre uma e outra há uma ilhota, que lhe faz duas barras,
pelas quais entram navios da costa. Defronte deste rio se começam os baixos de São Roque,
e deste Rio Grande ao cabo de São Roque são dez léguas, o qual está em altura de quatro
graus e um seismo;\footnote{ Sexto.} entre este cabo e a ponta do Rio Grande se faz de uma
ponta a outra uma grande baía, cuja terra é boa e cheia de mato, em cuja ribeira ao longo
do mar se acha muito sal feito. Defronte desta baía estão os baixos de São Roque, os quais
arrebentam em três ordens, e entra"-se nesta baía por cinco canais que vêm ter ao canal que
está entre um arrecife e outro, pelos quais se acha fundo de duas, três, quatro e cinco
braças, por onde entram os navios da costa à vontade.

\paragraph{[9] Em que se declara a costa do cabo de São Roque até o porto dos Búzios} \quad
Do cabo de São Roque até a ponta de Goaripari são seis léguas, a qual está em quatro graus
e um quarto, onde a costa é limpa e a terra escalvada, de pouco arvoredo e sem gentio. De
Goaripari à enseada da Itapitanga são sete léguas, a qual está em quatro graus e um
quarto; da ponta desta enseada à ponta de Goaripari são tudo arrecifes, e entre eles e a
terra entram naus francesas e surgem nesta enseada à vontade, sobre a qual está um grande
médão de areia;\footnote{ Duna; monte de areia junto ao mar.} a terra por aqui ao longo do mar
está despovoada do gentio por ser estéril e fraca. Da Itapitanga ao rio Pequeno, a que os
índios chamam Baquipe, são oito léguas, a qual está entre cinco graus e um seismo. Neste
rio entram chalupas francesas a resgatar com o gentio e carregar do pau de tinta, as quais
são das naus que se recolhem na enseada de Itapitanga.

Andando os filhos de João de Barros correndo esta costa, depois que se perderam, lhes
mataram neste lugar os pitiguoares com favor dos franceses, induzidos deles muitos homens.
Deste Rio Pequeno ao outro Rio Grande são três léguas, o qual está em altura de cinco
graus e um quarto; neste Rio Grande podem entrar muitos navios de todo o porte, porque tem
a barra funda de dezoito até seis braças, entra"-se nele como pelo arrecife de Pernambuco,
por ser da mesma feição. Tem este rio um baixo à entrada da banda do norte, onde corre
água muito à vazante, e tem dentro algumas ilhas de mangues, pelo qual vão barcos por ele
acima quinze ou vinte léguas e vem de muito longe. Esta terra do rio Grande é muito
sofrível para este rio haver de se povoar, em o qual se metem muitas ribeiras em que se
podem fazer engenhos de açúcar pelo sertão. Neste rio há muito pau de tinta, onde os
franceses o vão carregar muitas vezes.

Do Rio Grande ao porto dos Búzios são dez léguas, e está em altura de cinco graus e dois
terços; entre este porto e o rio estão uns lençóis de areia como os de Tapoam junto da
Baía de Todos os Santos. Neste Rio Grande achou Diogo Paes de Pernambuco, língua\footnote{ 
Intérprete, tradutor.} do gentio, um castelhano entre os pitiguoares, com os beiços
furados como eles, entre os quais andava havia muito tempo, o qual se embarcou em uma nau
para a França porque servia de língua dos franceses entre os gentios nos seus resgates.
Neste porto dos Búzios entram caravelões da costa num riacho que neste lugar se vem meter
no mar.

\paragraph{[10] Em que se declara a terra e costa do porto dos Búzios até a baía da Traição,
e como João de Barros mandou povoar a sua capitania} \quad
Do porto dos Búzios a Itacoatigara são nove léguas, e este rio se chama deste nome por
estar em uma ponta dele uma pedra de feição de pipa como ilha, a que o gentio por este
respeito pôs este nome, que quer dizer ponta da pipa. Mas o próprio nome do rio é Garatuí,
o qual está em altura de seis graus. Entre esta ponta e porto dos Búzios está a enseada de
Tabatingua,\footnote{ Em Varnhagen (1851 e 1879), ``Tabatinga''.} onde também há
surgidouro e abrigada para navios em que detrás da ponta costumavam ancorar naus francesas
e fazer sua carga de pau de tinta. De Itacoatigara ao rio de Goaramatai são duas léguas, o
qual está em seis graus esforçados; de Goaramatai ao rio de Caramative são duas léguas, o
qual está em seis graus e um quarto, e entre um e outro rio está a enseada Aratipicaba,
onde dos arrecifes para dentro entram naus francesas e fazem sua carga.

Deste porto para baixo, pouco mais ou menos, se estende a capitania de João de Barros,
feitor que foi da casa da Índia, a quem el"-rei D. João \textsc{iii} de Portugal fez mercê
de cinquenta léguas de costa partindo com a capitania de Pero Lopes de Sousa, de
Tamaraqua.\footnote{ Em Varnhagen (1851 e 1879), ``Tamaracá''.} Desejoso João de Barros de
se aproveitar desta mercê, fez à sua custa uma armada de navios em que embarcou muitos
moradores com todo o necessário para se poder povoar esta sua capitania e em a qual mandou
dois filhos que partiram com ela do Porto de Lisboa,\footnote{ Em Varnhagen (1851 e 1879),
``mandou dois filhos seus que partiram com ela, e prosseguindo''.} e prosseguindo logo sua
viagem em busca da costa do Brasil foram tomar terra junto do rio do Maranhão, em cujos
baixos se perderam. Deste naufrágio escapou muita gente, com a qual os filhos de João de
Barros se recolheram em uma ilha que está na boca deste rio do Maranhão, onde passaram
muitos trabalhos, por se não poderem comunicar desta ilha com os moradores da capitania de
Pernambuco, e das demais capitanias, os quais depois de gastarem alguns anos, despovoaram
e se vieram para este Reino. Nesta armada, e em outros navios que João de Barros depois
mandou por sua conta em socorro de seus filhos, gastou muita soma de mil cruzados, sem
desta despesa lhe resultar nenhum proveito, como fica dito atrás. Também lhe mataram os
pitiguoares muita gente onde se chama o rio Pequeno.

\paragraph{[11] Em que se declara a costa da baía da Traição até a Paraíba} \quad
Do rio de Camaratibe até a baía da Traição são duas léguas, a qual está em seis graus e um
terço, onde ancoram naus francesas e entram dos arrecifes para dentro. Chama"-se esta baía
pelo gentio potiguoar Acajutibiró, e os portugueses, da Traição, por nela matarem uns
poucos de castelhanos e portugueses que nesta costa se perderam.\footnote{ Em Varnhagen
(1851 e 1879), ``por com ela matarem''.} Nesta baía fazem cada ano os franceses muito pau
de tinta e carregam dele muitas naus. Desta baía da Traição ao rio Magoape são três
léguas, o qual está em seis graus e meio. Do rio de Magoape ao da Parahyba são cinco
léguas, o qual está em seis graus e três quartos.\footnote{ Em \textsc{bgjm}, encontra"-se
``Desta baía da Traição ao rio Magoape são três léguas, o qual está em seis graus e três
quartos''.} A este rio chamam na carta de marear o de São Domingos,\footnote{ O Rio São
Domingos é atualmente o rio Paraíba do Norte, ou simplesmente rio Paraíba, e banha o
Estado de mesmo nome.} onde entram naus de duzentos tonéis, e no rio de Magoape entram
caravelas da costa; mas o rio de São Domingos se navega muito pela terra dentro, de onde
ele vem de bem longe. Tem este rio um ilhéu da boca para dentro que lhe faz duas barras, e
pela que está da banda do norte entram caravelões dos que navegam por entre a terra e os
arrecifes até Tamaraqua, e pela outra banda entram as naus grandes;\footnote{ Em Varnhagen
(1851 e 1879), ``e pela outra barra entram''.} e porque entravam cada ano neste rio naus
francesas a carregar do pau de tinta com que abatiam o que ia para o Reino das mais
capitanias por conta dos portugueses e porque o gentio pitiguoar andava mui alevantado
contra os moradores da capitania de Tamaraqua e Pernambuquo, com o favor dos franceses,
com os quais fizeram nestas capitanias grandes danos, queimando engenhos e outras muitas
fazendas, em que mataram muitos homens brancos e escravos. Assentou Sua Majestade de o
mandar povoar e fortificar para o que mandou a isso Frutuoso Barbosa com muitos moradores,
o que se começou a fazer com mui grande alvoroço dos moradores destas duas capitanias, mas
foi Deus servido que lhe sucedesse mal com lhe matarem os pitiguares, em cuja companhia
andavam muitos franceses, trinta e seis homens e alguns escravos numa cilada, com o qual
sucesso se descontentaram muito os moradores de Pernambuquo; e se desavieram com Frutuoso
Barbosa, de feição que se tornaram para suas casas, e ele ficou impossibilitado para poder
pôr em efeito o que lhe era encomendado, o que se depois efetuou com o favor e ajuda que
para isso deu Diogo Flores de Baldes, general da armada que foi ao estreito de
Magalhães.\footnote{ Diogo Flores de Valdez foi um general espanhol que partiu de Cadiz,
em 1551, com uma armada para defender o Estreito de Magalhães dos corsários franceses.}

\paragraph{[12] Em que se trata de como se tornou a cometer a povoação do rio da Parahiba} \quad
Na Baía de Todos os Santos soube o general Diogo Flores, vindo aí do estreito de
Magalhães, com seis naus que lhe ficaram da armada que levou, como os moradores de
Pernambuquo e Tamaraqua pediam muito afincadamente ao governador Manuel Telles
Barreto,\footnote{ Manuel Telles Barreto foi governador"-geral entre 1582 e 1587. Assumiu
este cargo em substituição a Cosme Rangel, no início do reinado de Filipe \textsc{ii} de
Espanha. Para se prevenir dos ataques inimigos e defender os portos, construiu
fortificações e organizou a defesa.} que era então do Estado do Brasil, que os fosse
socorrer contra o gentio pitiguar que os ia destruindo, com o favor e ajuda dos franceses,
os quais tinham neste rio da Parahiba quatro naus para carregar do pau da tinta;\footnote{
Em Varnhagen (1851 e 1879), ``quatro navios para carregar''.} e, posto este negócio em
conselho, se assentou que o governador, naquela conjunção, não era bem que saísse da
Bahia, pois não havia mais de seis meses que era a ela chegado, onde tinha por prover em
grandes negócios convenientes ao serviço de Deus e de el"-rei e ao bem comum, mas pois
naquele porto estava o general Diogo Flores,\footnote{ Em Varnhagen (1851 e 1879), ``e do
bem comum, mas que, pois naquele porto''.} com aquela armada, e Diogo Vaz da Veiga com
duas naus portuguesas da armada em que do Reino fora o governador, das quais vinha por
capitão para o Reino, que um capitão e outro fossem fazer este socorro, indo por cabeça
principal o capitão Diogo Flores de Baldes, o qual chegou a Pernambuquo com a armada toda
junta, com que veio o ouvidor geral Martim Leitão e o provedor"-mor Martim Carvalho para,
em Pernambuquo, a favorecerem com gente e mantimentos, como o fizeram, a qual gente foi
por terra e o general por mar com esta armada, com a qual ancorou fora da barra, e não
entrou dentro com mais que com a sua fragata e duas naus, uma das de Diogo Vaz da
Veiga,\footnote{ Em Varnhagen (1851 e 1879), ``sua fragata e uma nau das de Diogo Vaz da
Veiga''.} de que era capitão Pero Correa de Lacerda,\footnote{ Em Varnhagen (1851 e 1879),
``Pedro Corrêa de Lacerda''.} em a qual o mesmo Diogo Vaz ia, e com todos os batéis das
outras naus. Em os franceses vendo esta armada puseram fogo às suas naus e lançaram"-se com
o gentio, com o qual fizeram mostras de quererem impedir a desembarcação, o que lhes não
serviu de nada, que o general desembarcou a pé enxuto, sem lho poderem impedir, e chegou a
gente de Pernambuquo e Tamaraqua por terra com muitos escravos e todos juntos ordenaram um
forte de terra e faxina onde se recolheram, no qual Diogo Flores deixou cento e tantos
homens dos seus soldados com um capitão para os caudilhar, que se chamava Francisco
Castrejon que se amassou tão mal com Frutuoso Barbosa não o querendo conhecer por
governador, que foi forçado a deixá"-lo neste forte, só, e ir"-se para Pernambuquo, de onde
se queixou à Sua Majestade para que provesse sobre o caso, como lhe parecesse mais seu
serviço. E sendo ausente Frutuoso Barbosa, veio o gentio por algumas vezes afrontar este
forte e pô"-lo em cerco, o qual sofreu mal o capitão Francisco Castrejon. E, apertado dos
trabalhos, desamparou este forte e o largou aos contrários, passando"-se por terra à
capitania de Tamaracá, que é daí dezoito léguas, e pelo caminho lhe matou o gentio alguma
gente que lhe ficou atrás, como foram mulheres e outra gente fraca. Mas, sabendo os
moradores de Pernambuquo este destroço, se ajuntaram e tornaram a este rio da Parahiba,
com Frutuoso Barbosa e se tornaram a apoderar deste forte, o qual Sua Majestade tem agora
socorrido com gente, munições e mantimentos necessários, a que se juntou uma aldeia do
gentio Tupinamba, que se apartou dos pitiguares, e se veio viver à borda da água, para
ajudar a favorecer este forte. Este rio da Paraiba é muito necessário fortificar"-se, uma
por tirar esta ladroeira dos franceses dele, outra por se povoar, pois é a terra capaz
para isso, onde se podem fazer muitos engenhos de açúcar. E povoado este rio como convém,
ficam seguros os engenhos da capitania de Tamaraqua e alguns da de Pernambuquo que não
lavram com temor dos pitiguares, e outras se tornarão a reformar, que eles queimaram e
destruíram. Dos quais pitiguares é bem que façamos este capítulo, que se segue, antes que
saiamos do seu limite.

\paragraph{[13] Que trata da vida e costumes do gentio pitiguar} \quad
Não é bem que passemos já do rio da Paraiba, onde se acaba o limite por onde reside o
gentio pitiguar, que tanto mal tem feito aos moradores das capitanias de Pernambuquo e
Tamaraqua, e à gente dos navios que se perderam pela costa da Paraiba até o rio do
Maranhão. Este gentio senhoreia esta terra do rio Grande até o da Paraíba,\footnote{ Em
Varnhagen (1851 e 1879), ``senhoreia esta costa''.} onde se confinaram antigamente com
outro gentio, que chamam os caytes, que são seus contrários, e se faziam crudelíssima
guerra uns aos outros, e se fazem ainda agora pela banda do sertão onde agora vivem os
caytés, e pela banda do rio Grande são fronteiros dos Tapuyas, que é a gente mais
doméstica, com quem estão às vezes de guerra e às vezes de paz, e se ajudam uns aos outros
contra os tabajaras, que vizinham com eles pela parte do sertão. Costumam estes pitiguares
não perdoarem a ninguém dos seus contrários que cativam,\footnote{ Em Varnhagen (1851 e
1879), ``a nenhum dos contrários que''.} porque os matam e comem logo. E este gentio é de
má estatura, baços de cor, como todo o outro gentio; não deixam criar nenhuns cabelos no
corpo senão os da cabeça,\footnote{ Em Varnhagen (1851 e 1879), ``não deixam criar''.}
porque em eles nascendo os arrancam logo. Falam a língua dos topinambas\footnote{ Em
Varnhagen (1851 e 1879), ``Tupinambás''.} e caytes; têm os mesmos costumes e
gentilidades,\footnote{ Paganismo.} o que declararemos ao diante no título dos
Topinambas.\footnote{ No manuscrito da \textsc{bgjm}, ``o que declaramos ao diante no
título dos Tapuyas, digo Topinambás''.} E este gentio é muito belicoso, guerreiro e
atraiçoado, e amigo dos franceses, a quem fazem sempre boa companhia, e, industriados
deles, inimigos dos portugueses. São grandes lavradores dos seus mantimentos, de que estão
sempre mui providos, e são caçadores bons e tais flecheiros que não erram flechada que
atirem. São grandes pescadores de linha, assim no mar como nos rios de água doce. Cantam,
bailam, comem e bebem pela ordem dos Tupinambas, onde se declarará miudamente sua vida e
costumes, que é quase o geral de todo o gentio da costa do Brasil.

\paragraph{[14] Em que se declara a costa do rio da Paraiba até Tamaraqua, e quem foi o seu
primeiro capitão} \quad
Do rio da Paraiba, que se diz também o rio de São Domingos, ao rio de Jagoarive, são duas
léguas, em o qual entram barcos. Do rio de Jagoarive ao da Aramamá são duas léguas, o qual
está em altura de sete graus, onde entram caravelões dos que navegam entre a terra e o
arrecife. Deste rio ao da Abionabiaja são duas léguas, cuja terra é alagadiça quase toda,
e entre um rio e outro ancoravam nos tempos passados naus francesas, das que entravam dos
arrecifes para dentro.\footnote{ Em Varnhagen (1851 e 1879), ``naus francesas, e daqui
entravam para dentro''.} Deste rio ao da Capivarimirim são seis léguas, o qual está em
altura de seis graus e meio, cuja terra é toda chã. De Capivarimirim a Itamaracá são seis
léguas, o qual está em altura de sete graus e meio,\footnote{ Em Varnhagen (1851 e 1879),
``altura de seis graus e meio''.} cuja terra é toda chã. De Capivarimirim a Tamaraqua são
seis léguas, e está em sete graus e um terço. Tamaraqua é uma ilha de duas léguas onde
está a cabeça dessa capitania e a vila de Nossa Senhora da Conceição. Derredor desta ilha
entram no salgado cinco ribeiras, em três das quais estão três engenhos; onde se fizeram
mais, se não foram os pitiguares, que vêm correndo a terra por cima e assolando tudo. Até
aqui, como já fica dito, tem o rio de Tamaraqua umas barreiras vermelhas na ponta da
barra; e quem houver de entrar por ela adentro ponha"-se nordeste"-sudoeste com as
barreiras, e entrará a barra à vontade, e daí para dentro o rio ensinará como hão de
ir.\footnote{ Em Varnhagen (1851 e 1879), ``ensinará por onde ir''.} Por esta barra entram
navios de cem tonéis, e mais, a qual fica da banda do sul da ilha, e a outra barra da
banda do norte se entra ao sudeste, pela qual se servem caravelões da costa. De Tamaraqua
ao rio de Igarusu\footnote{ Na edição de Varnhagen de 1851, ``Igarosu'' ou ``Igaruçú'', e
na de 1879, ``Igaraçú''.} são duas léguas, onde se extrema esta capitania da de
Pernambuquo. Desta capitania fez el"-rei D.~João, o \textsc{iii} de Portugal, mercê a Pero
Lopes de Sousa, que foi um fidalgo muito honrado, o qual, sendo mancebo, andou por esta
costa com armada à sua custa, em pessoa foi povoar esta capitania com moradores que para
isso levou do porto de Lisboa de onde partiu; no que gastou alguns anos e muitos mil
cruzados com muitos trabalhos e perigos em que se viu, assim no mar pelejando com algumas
naus francesas que encontrava, de que os franceses não saíram nunca bem, como na terra em
brigas que com eles teve de mistura com os pitiguares, de quem foi por vezes cercado e
ofendido, até que os fez afastar desta ilha de Tamaraqua e vizinhança dela. E esta
capitania não tem de costa mais de vinte e cinco ou trinta léguas, por Pero Lopes de Sousa
não tomar as cinquenta léguas de costa que lhe fez mercê Sua Alteza todas juntas, mas
tomou aqui a metade e a outra demasia junto à capitania de São Vicente, onde chamam Santo
Amaro.

\paragraph{[15] Que declara a costa do rio de Igarusu até Pernambuquo} \quad
\mbox{A vila} dos Cosmos está junto ao rio de Igaruçu, que é marco entre a capitania de Tamaraqua
e a de Pernambuquo, a qual vila será de duzentos vizinhos pouco mais ou menos, em cujo
termo há três engenhos de açúcar muito bons. Do rio de Igarusu ao porto da vila de Olinda
são quatro léguas, e está em altura de oito graus. Neste porto de Olinda se entra pela
boca de um arrecife de pedra ao sul"-sudoeste e depois norte"-sul; e, entrando para dentro
ao longo do arrecife fica o rio Morto, pelo qual entram até acima navios de cem tonéis até
duzentos, tomam meia carga em cima e acabam de carregar onde chamam o Poço defronte da
boca do arrecife, onde convém que os navios estejam bem amarrados, porque trabalham aqui
muito por andar neste porto sempre o mar de levadio;\footnote{ Movediço; muito agitado.}
por esta boca entra o salgado pela terra dentro uma légua ao pé da vila; e defronte do
surgidouro dos navios faz este rio outra volta deixando no meio uma ponta de areia onde
está uma ermida do Corpo Santo. Neste lugar vivem alguns pescadores e oficiais da ribeira,
e estão alguns armazéns em que os mercadores agasalham os açúcares e outras mercadorias.
Desta ponta da areia da banda de dentro se navega este rio até o varadouro,\footnote{ Lugar
de pouco fundo junto ao litoral, onde se encalham embarcações.} que está ao pé da vila,
com caravelões e barcos, e do varadouro para cima se navega com barcos de navios obra de
meia légua, onde se faz aguada fresca para os navios na ribeira que vem do engenho de
Jerônimo de Albuquerque.\footnote{ Em Varnhagen (1851 e 1879), ``para as naus da
ribeira''. Jerônimo de Albuquerque era irmão de D. Brites de Albuquerque, esposa do
donatário da capitania de Pernambuco, Duarte Coelho. Após o falecimento do
capitão"-donatário, D. Brites tornou"-se governadora e administradora da capitania, e
confiou ao seu irmão o governo de Pernambuco. Em suas terras, nas proximidades de Olinda,
fundou o primeiro engenho de açúcar de Pernambuco, o Nossa Senhora da Ajuda, depois
denominado de Forno de Cal.} Também se metem neste rio outras ribeiras por onde vão os
barcos dos navios a buscar os açúcares aos paços onde os trazem encaixados em carros; este
esteiro\footnote{ Braço, estuário.} é limite do arrecife, é muito farto de peixe de redes
que por aqui pescam e do marisco. Perto de uma légua da boca deste arrecife está outro
boqueirão a que chamam a Barreta, por onde podem entrar barcos pequenos estando o mar
bonançoso. Desta Barreta por diante corre este arrecife ao longo da terra duas léguas, e
entre ele e ela se navega com barcos pequenos. E quem vem do mar em fora e puser os olhos
na terra em que está situada esta vila,\footnote{ Em Varnhagen (1851 e 1879), ``com barcos
pequenos que navegam quem navegam de mar em fora, e quem puser os olhos na terra''.}
parecer"-lhe"-á que é o cabo de Santo Agostinho, por ser muito semelhante a ele.

\paragraph{[16] Do tamanho da vila de Olinda e da grandeza de seu termo, quem foi o primeiro
povoador dela} \quad
A vila de Olinda é a cabeça da capitania de Pernambuco, a qual povoou Duarte
Coelho,\footnote{ O português Duarte Coelho (c.1485-1554) foi o primeiro donatário da
Capitania de Pernambuco e chegou a essas terras em 1535.} que foi um fidalgo, de cujo
esforço e cavalaria escusamos tratar aqui em particular por não escurecer muito que dele
dizem os livros da Índia, de cujos feitos estão cheios. Depois que Duarte Coelho veio da
Índia a Portugal, a buscar satisfação de seus serviços, pediu a Sua Alteza que lhe fizesse
mercê de uma capitania nesta costa, que lhe logo concedeu, abalizando"-lha da boca do Rio
de São Francisco da banda do nordeste,\footnote{ Em Varnhagen (1851 e 1879),
``noroeste''.} e correndo dela pela costa cinquenta léguas contra Tamaraqua que se acabam
no rio de Igarusu, como já fica dito. E como a este valoroso capitão sobejavam sempre
espíritos para cometer grandes feitos,\footnote{ Em Varnhagen (1851 e 1879), ``sobravam
sempre espíritos''.} não lhe faltaram para vir em pessoa povoar e conquistar esta sua
capitania, onde veio com uma frota de navios que armou à sua custa, na qual trouxe sua
mulher e filhos e muitos parentes de ambos, e outros moradores com as suas
mulheres,\footnote{ Em Varnhagen (1851 e 1879), ``e outros moradores com a qual tomou''.}
com a qual tomou este porto que se diz de Pernambuquo por uma pedra que junto dele está
furada no mar, que quer dizer pela língua do gentio mar furado. Chegando Duarte Coelho a
este porto desembarcou nele e fortificou"-se, onde agora está a vila em um alto livre de
padrastos,\footnote{ Monte; conjunto de obras de defesa militar.} da melhor maneira que
foi possível, onde fez uma torre de pedra e cal, que ainda agora está na praça da vila,
onde muitos anos teve grandes trabalhos de guerra com o gentio e franceses que em sua
companhia andavam, dos quais foi cercado muitas vezes, malferido e mui apertado, onde lhe
mataram muita gente; mas ele, com a constância de seu esforço, não desistiu nunca da sua
pretensão, e não tão somente se defendeu valorosamente, mas ofendeu e resistiu aos
inimigos, de maneira que os fez afastar da povoação e despejar as terras vizinhas aos
moradores delas, onde depois seu filho, do mesmo nome, lhe fez tal guerra, maltratando e
cativando neste gentio, que é o que se chama cayte, que o fez despejar a costa toda, como
está hoje em dia,\footnote{ Em Varnhagen (1851 e 1879), ``como esta o é hoje em dia''.} e
afastar mais de cinquenta léguas pelo sertão. Nestes trabalhos gastou Duarte Coelho, o
velho, muitos mil cruzados que adquiriu na Índia, a qual despesa foi bem empregada, pois
resultou dela ter hoje seu filho Jorge de Albuquerque Coelho dez mil cruzados de renda,
que tanto lhe importa, a sua redízima e dízima do pescado e os foros que lhe pagam os
engenhos, dos quais estão feitos em Pernambuquo cinquenta, que fazem tanto açúcar que
estão os dízimos dele arrendados em dezenove mil cruzados cada ano. Esta vila de Olinda
terá setecentos vizinhos, pouco mais ou menos, mas tem muitos mais no seu termo, porque em
cada um destes engenhos vivem vinte e trinta vizinhos, fora os que vivem nas roças,
afastados deles, que é muita gente; de maneira que, quando for necessário ajuntar"-se esta
gente com armas, pôr"-se"-ão em campo mais de três mil homens de peleja com os moradores da
vila dos Cosmos, entre os quais haverá quatrocentos homens de cavalo. Esta gente pode
trazer de suas fazendas quatro ou cinco mil escravos da Guiné e muitos do gentio da terra.
É tão poderosa esta capitania que há nela mais de cem homens que têm de mil até cinco mil
cruzados de renda, e há alguns de oito, dez mil cruzados de renda. Desta terra saíram
muitos homens ricos para estes reinos que foram a ela muito pobres, com os quais entram
cada ano desta capitania quarenta e cinquenta naus carregadas de açúcar e pau do
brasil,\footnote{ Em Varnhagen (1851 e 1879), ``navios carregados de açúcar e
pau"-brasil''.} o qual é o mais fino que se acha em toda a costa; e importa tanto este pau
a Sua Majestade que o tem agora novamente arrendado por tempo de dez anos por vinte mil
cruzados cada ano. E parece terra tão rica e tão poderosa,\footnote{ Em Varnhagen (1851 e
1879), ``E parece que será tão rica e tão poderosa''.} de onde saem tantos provimentos
para estes reinos, que se devia de ter mais em conta a fortificação dela, e não consentir
que esteja arriscada a um corsário a saquear e destruir, ao que se pode atalhar com pouca
despesa e menos trabalho.

\paragraph{[17] Em que se declara a terra e costa que há do porto de Olinda até o cabo de
Santo Agostinho} \quad
Do porto de Olinda à ponta de Pero Cavarim são quatro léguas. Da ponta de Pero Cavarim ao
rio de Jaboatão é uma légua, em o qual entram barcos. Do rio de Jaboatão ao cabo de Santo
Agostinho são quatro léguas, o qual cabo está em oito graus e meio. Ao socairo\footnote{
No sopé; ao abrigo.} deste cabo da banda do norte podem surgir naus grandes quando
cumprir, onde têm boa abrigada. Do Cabo até Pernambuquo corre"-se a costa norte"-sul.

Quem vem do mar em fora, para conhecer este cabo de Santo Agostinho, verá por cima dele
uma serra selada, que é boa conhecença, porque por aquela parte não há outra serra da sua
altura e feição, a qual está quase leste"-oeste com o Cabo, e toma uma quarta de
nordeste"-sudoeste. E para quem vem ao longo da costa bota o Cabo fora com pouco mato e em
manchas; e ver"-lhe"-ão que tem a banda do sul, cinco léguas afastado dele, a ilha de Santo
Aleixo, que é baixa e pequena. Até este Cabo é a terra povoada de engenhos de açúcar, e
por junto dele passa um rio que se diz do Cabo, onde também estão alguns, o qual sai ao
mar duas léguas do Cabo e mistura"-se ao entrar no salgado com o rio do Ipojuqua,\footnote{
Em Varnhagen (1851 e 1879), ``ao entrar do salgado com o rio do Ipojuca''.} que está duas
léguas da banda do sul; neste rio entram e saem caravelões do serviço dos engenhos, que
estão nos mesmos rios, onde se recolhem com tempo barcos da costa.

\paragraph{[18] Em que se declara a costa do cabo e rio do Ipojuqua até o rio de São
Francisco} \quad
Já fica dito como se mete o rio de Ipojuqua como o do Cabo ao entrar no salgado, agora
digamos como dele ao porto das Galinhas são duas léguas.\footnote{ No manuscrito da
\textsc{bgjm}, ``como dele ao rio das Galinhas''.} A terra que há entre este porto e o rio
de Ipojuqua é toda alagadiça. Neste porto e rio das Galinhas entram barcos da costa. Do
rio das Galinhas à ilha de Santo Aleixo é uma légua, em a qual há surgidouro e abrigo para
as naus, e está afastada da terra firme uma légua; da ilha de Santo Aleixo ao rio de
Maracaípe são seis léguas, onde entram caravelões, o qual tem uns ilhéus na boca. De
Maracaípe ao rio Formoso são duas léguas, o qual tem um arrecife ao mar defronte de si,
que tem um boqueirão por onde entram navios da costa, o qual está em nove graus, cuja
terra é escalvada mas bem provida de caça. Do rio Formoso ao de Una são três léguas, o
qual tem na boca uma ilha de mangues da banda do norte, a qual se alaga com a maré, e mais
adiante, chegadas à terra, tem sete ilhetas de mato. Deste rio Una ao porto das Pedras são
quatro léguas, o qual está em nove graus e meio. Entre este porto e o rio Una se faz uma
enseada muito grande,\footnote{ Em Varnhagen (1851 e 1879), ``Entre este e o rio Una''.}
onde podem surgir e barlaventear naus que nadem em fundo de cinco até sete braças, porque
tanto tem de fundo.

E corre"-se a costa do cabo de Santo Agostinho até este porto das Pedras nordeste"-sudoeste.
Deste porto ao rio Camaragipe são três léguas, cuja fronteira é de um banco de arrecifes
que tem algumas abertas por onde entram barcos da costa, e ficam seguros de todo tempo
entre os arrecifes e a terra. Neste rio de Camaragipe entram navios de honesto porte, e na
ponta da barra dele da banda do sul tem umas barreiras vermelhas, cuja terra ao longo do
mar é escalvada até o rio de Santo Antônio Mirim, que está dele duas léguas, onde também
entram caravelões da costa. Do rio de Santo Antônio Mirim ao porto Velho dos Franceses são
três léguas, onde eles costumam ancorar com as suas naus e resgatar\footnote{ Negociar,
trocar.} com o gentio. Do porto Velho dos Franceses ao rio de São Miguel são quatro
léguas, que está em dez graus, em o qual entram navios da costa, e entre um e outro entra
no mar o rio da Alagoa, onde também entram caravelões, o qual se diz da Alagoa, por nascer
de uma que está afastada da costa, ao qual rio chamam os índios o porto de
Jaraguoa.\footnote{ Em Varnhagen (1851 e 1879), ``chamam os índios o porto Jaragoá''.} Do
rio de São Miguel ao porto Novo dos Franceses são duas léguas, defronte do qual fazem os
arrecifes que vão correndo a costa, uma aberta por onde os franceses costumavam entrar com
suas naus, e ancoravam entre o arrecife e a terra por ter fundo para isso, onde estavam
muito seguros, e daqui faziam seu resgate com o gentio. Do porto Novo dos Franceses ao de
Iapotiba\footnote{ Em Varnhagen (1851 e 1879), ``Sapetiba''.} é uma légua, do qual ao rio
de Cururuipe\footnote{ Em Varnhagen (1851 e 1879), ``Currurupe''.} são três léguas em o
qual entram navios da costa, cuja terra ao longo do mar é fraca, mas para dentro duas
léguas é arresoada.\footnote{ Arrazoada; razoável.} Deste rio de Cururuipe até o rio de
São Francisco são seis léguas.

Da ponta da barra de Cururuipe,\footnote{ Em Varnhagen (1851), ``Currururipe''.} contra o
Rio de São Francisco se vai armando uma enseada de duas léguas, em a qual bem chegados à
terra estão os arrecifes de D.~Rodrigo, onde também se chama o porto dos Franceses por se
eles costumarem recolher aqui com suas naus à abrigada desta enseada, e iam por entre os
arrecifes e a terra, com suas lanchas, tomar carga do pau da tinta no rio de Cururuipe.

Aqui se perdeu o bispo do Brasil, D.~Pero Fernandes Sardinha,\footnote{ Em Varnhagen (1851
e 1879), ``Pedro Fernandes Sardinha''.} com sua nau vinda da Bahia para Lisboa, em a qual
vinha Antônio Cardoso de Barros, provedor"-mor que fora do Brasil, e dois cônegos, duas
mulheres honradas e casadas, muitos homens nobres e outra muita gente, que seriam mais de
cem pessoas brancas, fora escravos, a qual escapou toda deste naufrágio, mas não do gentio
cayte, que neste tempo senhoreava esta costa da boca deste rio de São Francisco até o da
Paraiba; depois que estes caytes roubaram este bispo e toda esta gente de quanto
salvaram,\footnote{ No manuscrito da \textsc{bgjm}, ``este bispo e gente''.} os despiram e
amarraram a bom recado, e pouco a pouco os foram matando e comendo, sem escapar mais que
dois índios da Bahia com um português que sabia a língua, filho do meirinho da correição.
A terra que há por cima desta enseada até perto do rio de São Francisco é toda alagadiça,
cuja água se ajunta toda em uma ribeira que se dela faz, a qual vai sair ou entrar do rio
de São Francisco duas léguas da barra para cima. Corre"-se a costa do rio de São Francisco
até o porto das Pedras nordeste"-sudoeste,\footnote{ Em Varnhagen (1851 e 1879),
``nornordeste susudoeste''.} e toma da quarta de norte"-sul.

\paragraph{[19] Que trata de quem são estes caytes, que foram moradores na costa de
Pernambuco} \quad
Parece que não é bem que passemos adiante do rio de São Francisco sem dizermos que gentio
é este caité, que tanto mal tem feito aos portugueses nesta costa, o que agora cabe dizer
deles.\footnote{ No manuscrito da \textsc{bgjm}, o período termina em ``nesta costa''.}

Este gentio, nos primeiros anos da conquista deste estado do Brasil, senhoreou desta costa
da boca do rio de São Francisco até o rio da Paraiba, onde sempre teve guerra cruel com os
pitiguares,\footnote{ No manuscrito da \textsc{bgjm}, ``guerra cruel com os portugueses'',
possivelmente um erro do copista.} e se matavam e comiam uns aos outros em vingança de
seus ódios, para execução da qual entravam muitas vezes pela terra dos pitiguares e lhes
faziam muito dano. Da banda do rio de São Francisco guerreavam estes pitiguares em suas
embarcações com os Tupinambas, que viviam da outra banda do rio,\footnote{ Em Varnhagen
(1851 e 1879), ``outra parte do rio''.} em cuja terra entravam a fazer seus saltos, onde
cativavam muitos, que comiam sem lhes perdoar.

As embarcações de que este gentio usava eram de uma palha comprida como a das esteiras de
tábua que fazem em Santarém, a que eles chamam periperi, a qual palha fazem em molhos
muito apertados, com umas varas como vime, a que eles chamam timbós, que são muito brandas
e rijas, e com estes molhos atados em umas varas grossas faziam uma feição de embarcações,
em que cabiam dez a doze índios, que se remavam muito bem, e nelas guerreavam com os
Tupinambas neste rio de São Francisco, e se faziam uns aos outros muito dano. E aconteceu
por muitas vezes fazerem os caités desta palha tamanhas embarcações que vinham nelas ao
longo da costa fazer seus saltos aos topinambás junto da Bahia, que são cinquenta léguas.
Pela parte do sertão, confinava este gentio com os Tapuyas e tupinais, e se faziam cruéis
guerras, para cujas aldeias ordinariamente havia fronteiros, que as corriam e salteavam. E
quando os caités matavam ou cativavam alguns contrários destes, tinham"-no por mor honra,
que não quando faziam outro tanto aos pitiguares nem aos topinambás. Este gentio é da
mesma cor baça, e tem a vida e costumes dos pitiguares e a mesma língua, que é em tudo
como a dos Tupinambas, em cujo título se dirá muito de suas gentilidades.

São estes caites mui belicosos e guerreiros, mas mui atraiçoados, sem nenhuma fé nem
verdade, o qual fez os danos que fica declarado à gente da nau do bispo, a Duarte Coelho,
e a muitos navios e caravelões que se perderam nesta costa, dos quais não escapou pessoa
nenhuma, que não matassem e comessem, cujos danos Deus não permitiu que durassem mais
tempo; mas ordenou de os destruir desta maneira. Confederaram"-se os topinambas seus
vizinhos com os tupinays,\footnote{ Em Varnhagen (1851 e 1879), ``tupinaês''.} pelo
sertão, e ajuntaram"-se uns com os outros pela banda de cima, de onde os tapuyas também
apertavam estes caites, e deram"-lhe nas costas, e de tal feição os apertaram, que os
fizeram descer todos para baixo, junto do mar, onde os acabaram de desbaratar; e os que
não puderam fugir para a serra do Aquetiba não escaparam de mortos ou cativos. Destes
cativos iam comendo os vencedores quando queriam fazer suas festas, e venderam deles aos
moradores de Pernambuquo e aos da Bahia infinidade de escravos a troco de qualquer coisa,
ao que iam ordinariamente caravelões de resgate, e todos vinham carregados desta gente, a
qual Duarte Coelho de Albuquerque por sua parte acabou de desbaratar.

E desta maneira se consumiu este gentio, do qual não há agora senão o que se lançou muito
pela terra adentro, ou se misturou com seus contrários sendo seus escravos, ou se aliaram
por ordem de seus casamentos. Por natureza, são estes caites grandes músicos e amigos de
cantar e de bailar,\footnote{ Em Varnhagen (1851 e 1879), ``grandes músicos e amigos de
bailar''.} são grandes pescadores de linha e nadadores; também são mui cruéis uns para os
outros para se venderem, o pai aos filhos, os irmãos e parentes uns aos outros; e de tal
maneira são cruéis, que aconteceu o ano de 1571 no rio de São Francisco estando nele
algumas embarcações da Bahia resgatando com este gentio, em uma de um Rodrigo Martins
estavam alguns escravos resgatados, em que entrava uma índia caité, a qual enfadada de lhe
chorar uma criança, sua filha, a lançou no rio, onde andou de baixo para cima um pedaço
sem se afogar, até que de outra embarcação se lançou um índio a nado, por mandado de seu
senhor, que a foi tirar, onde a batizaram e durou depois alguns dias.

E como no título dos Tupinambas se conta por extenso a vida e costumes, que toca a maior
parte do gentio que vive na costa do Brasil, temos que basta por agora o que está dito dos
Caites.\footnote{ Em Varnhagen (1851 e 1879), ``basta o que está dito até agora dos
\textit{Caités}''.}

\paragraph{[20] Que trata da grandeza do rio de São Francisco e seu nascimento} \quad
Muito havia que dizer do rio de São Francisco, se lhe coubera fazê"-lo neste lugar, do qual
se não pode escrever aqui o que se deve dizer dele,\footnote{ No manuscrito da
\textsc{bgjm}, ``o que se pode dizer dele''.} porque será escrever tudo o que temos dito,
e não se poderá cumprir com o que está dito e prometido, que é tratar toda a costa em
geral, e em particular da Bahia de Todos os Sanctos, a quem é necessário satisfazer com o
devido. E este rio contente"-se por ora de se dizer dele em suma o que for possível neste
capítulo, para com brevidade chegarmos a quem está esperando por toda a costa.

Está o rio de São Francisco em altura de dez graus e um quarto, o qual tem na boca da
barra duas léguas de largo, por onde entra a maré com o salgado para cima duas léguas
somente, e daqui para cima é água doce, que a maré faz recuar outras duas léguas, não
havendo água do monte. A este rio chama o gentio o Pará, o qual é mui nomeado entre todas
as nações, das quais foi sempre muito povoado, e tiveram uns com outros sobre os sítios
grandes guerras, por ser a terra muito fértil pelas suas ribeiras e por acharem nele
grandes pescarias.

Ao longo deste rio vivem agora alguns caetés, de uma banda, e da outra vivem Tupinambas;
mais acima vivem Tapuyas de diferentes castas, Tupinaes, amoupiras,\footnote{ Em Varnhagen
(1851 e 1879), ``\textit{Amoipiras}''.} ubirajaras e amazonas; e além delas vive outro
gentio, não tratado dos que comunicam com os portugueses, que se atavia com joias de ouro,
de que há certas informações. Este gentio se afirma viver à vista da Alagoa Grande, tão
afamada e desejada de descobrir, da qual este rio nasce. E é tão requestado este rio de
todo o gentio, por ser muito farto de pescado e caça, e por a terra dele ser muito fértil
como já fica dito; onde se dão mui bem toda a sorte de mantimentos naturais da terra.

Quem navega por esta costa conhece este rio quatro e cinco léguas ao mar pelas aguagens
que dele saem furiosas e barrentas. Navega"-se este rio com caravelões até a cachoeira que
estará da barra vinte léguas, pouco mais ou menos, até onde tem muitas ilhas, que o fazem
espraiar muito mais que na barra, por onde entram navios de cinquenta tonéis pelo canal do
sudoeste, que é mais fundo que o do nordeste. Da barra deste rio até a primeira cachoeira
há mais de trezentas ilhas;\footnote{ No manuscrito da \textsc{bgjm}, ``há mais de trinta
ilhas, digo de trezentas ilhas''.} no inverno não traz este rio água do monte, como os
outros, nem corre muito; e no verão cresce de dez até quinze palmos. E começa a vir esta
água do monte, de outubro por diante até janeiro, que é a força do verão destas partes; e
neste tempo se alagam a maior parte destas ilhas, pelo que não criam nenhum arvoredo, nem
mais que canas"-bravas de que se fazem flechas.

Por cima desta cachoeira, que é de pedra viva, também se pode navegar este rio em barcos,
se se lá fizerem, até o sumidouro, que pode estar da cachoeira oitenta ou noventa léguas,
por onde também tem muitas ilhas. Este sumidouro se entende no lugar onde este rio sai de
debaixo da terra, por onde vem escondido, dez ou doze léguas, no cabo das quais arrebenta
até onde se pode navegar, e faz seu caminho até o mar. Por cima deste sumidouro está a
terra cheia de mato, sem se sentir que vai o rio por baixo, e deste sumidouro para cima se
pode também navegar em barcos, se os fizerem lá; os Índios se servem por ele em canoas,
que para isso fazem. Está capaz este rio para se perto da barra dele fazer uma povoação
valente de uma banda e outra da outra parte para segurança dos navios da costa,\footnote{
Em Varnhagen (1851 e 1879), ``uma povoação valente de uma banda, e da outra para
segurança''.} e dos que o tempo ali faz chegar, onde se perdem muitas vezes, e podem os
moradores que nele viverem fazer grandes fazendas e engenhos até a cachoeira, derredor da
qual há muito pau do brasil, que com pouco trabalho se pode carregar.

Depois que este Estado se descobriu por ordem dos reis passados, se trabalhou muito por se
acabar de descobrir este rio, por todo o gentio que nele viveu, e por ele andou afirmar
que pelo seu sertão havia serras de ouro e prata, à conta da qual informação se fizeram
muitas entradas de todas as capitanias sem poder ninguém chegar ao cabo; com este
desengano e sobre esta pretensão veio Duarte Coelho de Albuquerque\footnote{ Em Varnhagen
(1851 e 1879), ``Duarte Coelho''.} a Portugal da sua capitania de Pernambuquo a primeira
vez, e a segunda também teve esse desengano;\footnote{ Em Varnhagen (1851 e 1879), ``e da
segunda também teve desenho''.} mas desconcertou"-se com Sua Alteza pelo não fartar das
honras que pedia. E sendo governador deste Estado Luis de Brito de Almeida,\footnote{ O
português Brito de Almeida governou as capitanias do norte do Estado do Brasil, que tinha
Salvador como capital, de 1572 à 1576, tornando"-se governador de toda a colônia a partir
de 1577. Empreendeu diversas bandeiras pelo interior do Brasil, para descoberta e
mineração de pedras e metais preciosos, e fundou a cidade de Santa Luzia, na Bahia, que deu
inicio à Capitania de Sergipe.} mandou entrar por este rio acima a um Bastião Álvares,
que se dizia do Porto Seguro, o qual trabalhou por descobrir quanto pôde, no que gastou
quatro anos e um grande pedaço da Fazenda de el"-rei, sem poder chegar ao sumidouro, e por
derradeiro veio acabar com quinze ou vinte homens entre o gentio topinamba, a cujas mãos
foram mortos, o que lhe aconteceu por não ter cabedal da gente para se fazer temer, e por
querer fazer esta jornada contra água, o que não aconteceu a João Coelho de Sousa, porque
chegou acima do sumidouro mais de cem léguas, como se verá do roteiro que se fez da sua
jornada. A boca da barra deste rio corta o salgado a terra da banda do sudoeste, e faz
ficar aquela ponta de areia e mato em ilha, que será de três léguas de comprido. E quando
este rio enche com água do monte, não entra o salgado com a maré por ele acima, mas até a
barra é água doce, e traz neste tempo grande correnteza.

\paragraph{[21] Em que se declara a costa do rio de São Francisco até o de Seregipe} \quad
Do rio de São Francisco ao de Goaratibe\footnote{ Em Varnhagen (1851 e 1879),
``Guaratiba''.} são duas léguas, em o qual entram barcos da costa e tem este rio na boca
uma ilha, que é a que vem da ponta da barra do rio de São Francisco; este rio se navega
pela terra dentro três léguas, e faz um braço na entrada junto do arrecife, por onde entra
o salgado até entrar no rio de São Francisco uma légua da barra, por onde vão os barcos de
um rio ao outro, o qual braço faz a ilha declarada. Do rio de Goaratibe a sete léguas está
um riacho a que chamam de Aguaboa, pelo ela ser, o qual, como chega perto do salgado, faz
uma volta ao longo dele, fazendo uma língua de terra estreita entre ele e o mar, de uma
légua de comprido, e no cabo desta légua se mete o mar; entre um rio e outro é tudo praia
de areia, onde se chama a enseada de Vazabarris, a qual tem diante de si tudo arrecifes de
pedra, com alguns boqueirões para barcos pequenos, por onde podem entrar com bonança.
Deste riacho de Aguaboa a uma légua está o rio de Ubirapatiba, por cuja barra podem entrar
barcos e caravelões da costa com a proa ao lesnoroeste. A este rio vem o gentio Tupinamba
mariscar, por achar por aqueles arrecifes muitos polvos, lagostins e caranguejos; e a
pescar à linha, onde matam muito peixe, o qual se navega pela terra adentro mais de três
léguas. Deste rio Ubirapatiba a sete léguas está o rio de Seregipe em altura de onze graus
e dois terços, por cuja barra com batéis diante costumavam entrar os franceses com suas
naus do porte de cem tonéis para baixo, mas não tomavam dentro mais que meia carga, e fora
da barra acabavam de carregar com suas lanchas, em que acabavam de acarretar o pau que ali
resgatavam com os Tupinambas, onde também resgatavam com os mesmos algodão e pimenta da
terra. Tem este rio duas léguas por ele acima a terra fraca, mas dali avante é muito boa
para se poder povoar, onde convém muito que se faça uma povoação, assim para atalhar que
não entrem ali franceses, como por segurar aquela costa do gentio que vive por este rio
acima, o qual todos os anos faz muito dano, assim nos barcos que entram nela e no Rio Real
no inverno com tempo, como em homens, que cometem este caminho para Pernambuquo fugindo à
justiça, e no que pelo mesmo respeito fogem de Pernambuco para a Bahia, os quais de
maravilha escapam que os não matem e comam. Tem este rio de Seregipe na barra de baixa"-mar
três braças, e dentro cinco e seis braças, cuja barra se entra lessudeste e oesnoroeste, e
quem quer entrar pelo boqueirão do baixio vai com a proa ao norte; e como está dentro a
loesnoroeste, vai demandar a ponta do sul, e dela para dentro se vai ao norte; e quem vem
do mar em fora verá por cima deste rio um monte mais alto que os outros, da feição de um
ovo, que está afastado da barra algumas seis léguas, pelo qual é a terra bem conhecida. A
este monte chamam os índios Manhana, que quer dizer entre eles espia por se ver de todas
as partes de muito longe. E corre"-se a costa deste rio ao de São Francisco
nordeste"-sudoeste.\footnote{ Em Varnhagen (1851 e 1879), ``nornordeste susudoeste''.}

\paragraph{[22] Em que se declara a costa do rio de Seregipe até o rio Real} \quad
Deste rio de Seregipe de que acima dissemos, a quatro léguas está outro rio, que se diz de
Cotigipe, cuja boca é de meia légua, no meio da qual tem uma ilha em que tem umas moitas
verdes, a qual ilha faz duas barras a este rio; pela do sul podem entrar navios de oitenta
tonéis, porque no mais debaixo tem de fundo duas braças de baixa"-mar, e mais para dentro
tem cinco braças; pela barra do norte entram caravelões da costa. Tem este rio à boca da
barra uns bancos de areia que botam meia légua ao mar. Por este rio se navega três léguas,
que tantas entra a maré por ele acima, o qual é muito farto de peixe e marisco, cuja terra
é sofrível para se poder povoar e no sertão dela tem grandes matas de pau"-brasil.

Deste rio de Cotegipe ao rio de Pereira, a que outros chamam de Canafístola, são quatro
léguas. Do qual até Seregipe faz a terra outra enseada, a que também chamam de Vazabarris,
no seio da qual está o rio Cotegipe, de que já falamos, a que muitos chamam do nome de
enseada. Do rio de Pereira a duas léguas está a ponta do rio Real, donde se corre a costa
até Sergipe nordeste"-sudoeste.\footnote{ Em Varnhagen (1851 e 1879), ``nornordeste
susudoeste''.}

\paragraph{[23] Que trata do rio Real e seus merecimentos} \quad
Parece que quem tem tamanho nome como o rio Real, que deve de ter merecimentos capazes
dele, os quais convém que venham a terreiro, para que cheguem à notícia de todos. E
comecemos na altura, em que está, que são doze graus escassos; a barra deste rio terá de
ponta a ponta meia légua, em a qual tem dois canais, por onde entram navios da costa de
quarenta toneladas, e pela barra do sudoeste podem entrar navios de sessenta tonéis,
estando com as balizas necessárias, porque tem dois mares em flor; da barra para dentro
tem o rio muito fundo, onde se faz uma baía de mais de uma légua, onde os navios têm
grande abrigada com todos os tempos, na qual há grandes pescarias de peixe"-boi, e de toda
a outra sorte de pescado, e muito marisco. Entra a maré por este rio acima seis ou sete
léguas, e divide"-se em três ou quatro esteiros onde se vêm meter outras ribeiras de água
doce. Até onde chega o salgado, é a terra fraca e pouca dela servirá de mais que de
criação de gado; mas donde se acaba a maré para cima é a terra muito boa e capaz para dar
todas as novidades do que lhe plantarem, em a qual se podem fazer engenhos de açúcar, por
se darem nela as canas muito bem.

Pelo sertão deste rio há muito pau do brasil, que com pouco trabalho todo pode vir ao mar,
para se poder carregar para estes reinos. E para que esta costa esteja segura do gentio, e
os franceses desenganados de não poderem vir resgatar com ele entre a Bahia e Pernambuquo,
convém ao serviço de Sua Majestade que mande povoar e fortificar este rio, o que se pode
fazer com pouca despesa de sua Fazenda, do que já el"-rei D. Sebastião, que está em glória,
foi informado, e mandou mui afincadamente a Luis de Brito, que neste tempo governava este
Estado, que ordenasse com muita brevidade como se povoasse este rio, no que ele meteu todo
o cabedal, mandando a isso Garcia d'Ávila, que é um dos principais moradores da Bahia, com
muitos homens das ilhas e da terra, para que assentassem uma povoação onde parecesse
melhor; o que fez pelo rio acima três léguas, onde o mesmo governador foi em pessoa com a
força da gente que havia na Bahia,\footnote{ No manuscrito da \textsc{bgjm}, ``força de
guerra, digo gente que havia''.} quando foi dar guerra ao gentio daquela parte, o qual
passou por esta nova povoação, de cujo sítio ele e toda a companhia se descontentaram, e
com razão, porque estava longe do mar, para se valerem da fartura dele, e longe da terra
boa, que lhe pudesse responder com as novidades acostumadas. Donde se afastarem por
temerem o gentio que por ali vivia, ao qual Luís de Brito deu tal castigo naquele tempo
como se nunca deu naquelas partes,\footnote{ Em Varnhagen (1851 e 1879), ``como se não deu
naquelas partes''.} porque mandou destruir os mais valorosos e maiores dos corsários
capitães daquele gentio, que nunca houve naquela costa, sem lhe custar a vida a mais que a
dois escravos, os quais principais do gentio foram mortos, e os seus que escaparam com
vida ficaram cativos. E quando o governador recolheu, se despovoou este princípio de
povoação, sem se tornar mais a bulir nisto, por se entender ser necessário fazer"-se uma
casa forte à custa de Sua Alteza, a qual Luís de Brito não ordenou por ser chegado o cabo
de seu tempo, e suceder"-lhe Lourenço da Veiga, que não buliu neste negócio pelos respeitos
que não são sabidos para se aqui declararem.

\paragraph{[24] Em que se declara a terra que há do rio Real até o rio de Tapicuru} \quad
Do rio Real ao de Itapicuru\footnote{ Em Varnhagen (1851 e 1879), ``Itapocurú''.} são
quatro léguas, sem de um rio ao outro haver na costa por onde entre um barquinho, por tudo
serem arrecifes ao longo da costa, cuja terra ao longo dela é muito fraca, que não serve
senão para criações de gado. A boca deste rio é muito suja de pedras, mas podem"-se quebrar
umas pontas de baixa"-mar de águas vivas, em que lhe fique canal aberto, para poderem por
ele entrar caravelões da costa de meia água cheia por diante. Da boca deste rio para
dentro faz"-se uma maneira de baía, onde de baixa"-mar podem nadar naus de duzentos tonéis;
entra a maré por este rio acima cinco léguas ou seis, as quais se podem navegar com
barcos; e onde se mistura o salgado com água doce para cima dez ou doze léguas se podem
também navegar com barquinhos pequenos; e por aqui acima é a terra muito boa para se poder
povoar, porque dá muito bem todos os mantimentos que lhe plantam, e dará muito bons
canaviais de açúcar, porque quando Luís de Brito foi dar guerra ao gentio do rio Real, se
acharam pelas roças destes índios, que viviam ao longo deste rio, mui grossas e mui
formosas canas"-de"-açúcar, pelo que, povoando"-se este rio, se podem fazer nele muitos
engenhos de açúcar, porque tem ribeiras que se nele metem muito acomodadas para isto;
neste mesmo tempo se achou entre este rio e o Real cinquenta ou sessenta léguas pelo
sertão, uma lagoa de quinhentas braças de comprido e cem de largo, pouco mais ou menos,
cuja água é mais salgada que a do mar, a qual alagoa estava cercada de um campo todo cheio
de perrexil muito mais viçoso que o que nasce ao longo do mar, e tocado por fora nos
beiços era tão salgado como se lhe dera o rocio do mar; neste mesmo campo afastado desta
alagoa quinhentas ou seiscentas braças estava outra alagoa, ambas em um andar, cuja água
era muito doce, e o peixe que ambas tinham era de uma mesma sorte, e em ambas havia muitos
porcos"-d'água, dos quais o gentio matou muita quantidade deles. Este rio perto do mar é
muito farto de pescado e marisco e, para cima, de peixe de água doce, e pela terra ao
longo dele tem muita caça de toda a sorte, o qual no verão traz mais água que o Mondego, e
está em doze graus, cujo nascimento é para a banda de leste mais de cem léguas do
mar,\footnote{ Na edição de 1851, ``banda de loeste'', e na de 1879, ``banda do loeste''.}
e está povoado do gentio Tupinamba.

\paragraph{[25] Em que se declara a terra que há do Tapicuru até Tatuapara} \quad
Do rio Itapicuru a Tatuapara são oito ou nove léguas, cuja terra ao longo do mar é muito
fresca e baixa, e não serve senão para criação de gado; mas duas léguas pela terra adentro
é sofrível para mantimentos, pela qual atravessam cinco rios e outras muitas ribeiras, que
vêm sair no mar nestas oito léguas, de que não há que tratar, porque se metem no mar por
cima dos arrecifes,\footnote{ Em Varnhagen (1851 e 1879), ``por se meterem no mar''.} sem
fazerem barra por onde possa andar um barquinho; porque toda esta costa do rio Real até
Tatuapara ao longo do mar é cheia de arrecifes de pedra, que se espraiam muito, por onde
não é possível lançar"-se gente em terra, nem chegar nenhum barco se não for no Itapicuru,
como fica dito.

Tatuapara é uma enseada, onde se mete um riacho deste nome, no qual entram caravelões da
costa com preamar; nesta enseada têm os navios muito boa abrigada e surgidouro, do que se
aproveitam os que andam pela costa. Aqui tem Garcia d'Ávila, que é um dos principais e
mais ricos moradores da cidade do Salvador, uma povoação com grandes edifícios de casas de
sua vivenda, e uma igreja de Nossa Senhora, mui ornada, toda de abóbada, na qual tem um
capelão que lhe ministra os sacramentos.

Este Garcia d'Ávila tem toda sua fazenda em criações de vacas e éguas, e terá alguns dez
currais por esta terra e ao diante;\footnote{ Em Varnhagen (1851 e 1879), ``por esta terra
adiante''.} e os padres da companhia\footnote{ Os padres da Companhia de Jesus, ou
jesuítas.} têm neste direito uma aldeia de índios forros Tupinambas, a qual se chama de
Santo Antônio, onde haverá mais de trezentos homens de peleja; e perto dessa aldeia têm os
padres três currais de vacas, que granjeiam, os quais têm na aldeia uma formosa igreja de
Santo Antônio, e um recolhimento onde estão sempre um padre de missa e um irmão, que
doutrinam estes índios na nossa santa fé católica, no que os padres trabalham todo o
possível; mas por demais, porque é este gentio tão bárbaro, que até hoje não há nenhum que
viva como cristão, tanto que se aparta da conversação dos padres oito dias. Esta enseada
de Tatuapara está a altura de doze graus esforçados e corre"-se a costa daqui até o rio
Real nordeste susudoeste.\footnote{ Em Varnhagen (1851 e 1879), ``nornordesde
susudoeste''.}

\paragraph{[26] Em que se declara a terra e costa de Tatuapara até o rio de Joane} \quad
De Tatuapara ao rio Jacoipe são quatro léguas, as quais ao longo do mar são de terra baixa
e fraca que estão ocupadas com currais de gado de Garcia d'Ávila e de outras pessoas
chegadas à sua casa.\footnote{ Em Varnhagen (1851 e 1879), ``as quais ao longo do mar
estão ocupadas por currais de gado, por serem de terra baixa e fraca; os quais currais são
de Garcia D'Ávila''.} De Tatuapara até este rio não há onde possa entrar um barco, senão
neste rio de Jacoipe e aqui com bonanças ainda com trabalho; mas, atrás uma légua, onde se
chama o porto de Brás Afonso, onde os arrecifes que vêm de Tatuapara fazem uma
aberta,\footnote{ No manuscrito da \textsc{bgjm}, ``fazem uma enseada, digo aberta''.}
podem entrar caravelões, e do arrecife para dentro ficam seguros com todo o tempo. Este
rio de Jacoipe se passa de baixa"-mar acima da barra uma légua a vau, ao longo do qual tem
o mesmo Garcia d'Ávila um curral de vacas. Deste rio de Jacuípe até o rio de
Joane\footnote{ Em Varnhagen (1851 e 1879), ``Joanne''.} são cinco léguas, até onde são
tudo arrecifes, sem haver onde possa entrar um barco, senão onde chamam o porto de
Arambepe, onde os arrecifes fazem outra aberta, por onde com bonança podem entrar barcos,
e ficarem dentro dos arrecifes seguros.

De Jacoipe a Arambepe são duas léguas onde se perdeu a nau Santa Clara, que ia para a
Índia, estando sobre amarra, e foi tanto o tempo que sobreveio, que a fez ir à
caceia,\footnote{ Condição de uma embarcação que se encontra descaída do rumo ou à
deriva.} que foi forçado cortarem"-lhe o mastro grande, o que não bastou para se remediar,
e os oficiais da nau, desconfiados da salvação, sendo meia"-noite, deram à vela do
traquete\footnote{ Mastro da parte dianteira da embarcação.} para
ancorarem em terra e salvarem as vidas, o que lhe sucedeu pelo contrário; porque sendo
esta costa toda limpa, afastada dos arrecifes, foram varar por cima de uma laje, não se
sabendo outra de Pernambuco até a Bahia, a qual laje está um tiro de falcão ao mar dos
arrecifes, onde se esta nau fez em pedaços, e morreram neste naufrágio passante
de  trezentos homens, com Luis d'Alter d'Andrade,\footnote{ Em Varnhagen (1851 e
1879), ``Luiz de Alter de Andrade''.} que ia por capitão desta nau para a Índia. Toda esta
terra até o rio de Joane, três léguas do mar para o sertão, está povoada de currais de
vacas de pessoas diversas; e nesta comarca, três léguas do mar, têm os padres da companhia
duas aldeias de índios forros Tupinambas e de outras nações, em as quais terão setecentos
homens de peleja pelo menos; os quais os padres doutrinam, como fica dito, da aldeia de
Santo Antônio. Estas outras se dizem uma de Santo Espírito e a outra de São João, onde têm
grandes igrejas da mesma advocação e recolhimento para os padres que nelas residem e para
outros que muitas vezes se vão lá recrear. E à sombra e circuito destas aldeias têm quatro
ou cinco currais de vacas ou mais, que granjeiam, de que se ajudam a sustentar. Por onde
estas aldeias estão é a terra boa, onde se dão todos os mantimentos da terra muito bem,
por ser muito fresca, com muitas ribeiras de água; neste limite lança o mar fora todos os
anos muito âmbar pelo inverno, que estes índios vão buscar, o qual dão aos padres. E
corre"-se esta costa de Tatuapara até este rio de Joane nordeste susudoeste.\footnote{ Em
Varnhagen (1851 e 1879), ``nornordesde susudoeste''.}

\paragraph{[27] Em que se declara a costa do rio de Joane até a Bahia} \quad
\mbox{O rio} Joane traz tanta água, quando se mete no mar, como o Zezere quando se mete no Tejo,
o qual entra no mar por cima dos arrecifes, onde espraia muito, o qual se passa de maré
vazia a vau por junto da barra; mas não pode entrar por ela nenhuma jangada, por ser tudo
pedra viva, e de preamar não tem sobre si três palmos de água, a qual anda ali sempre mui
levantada. Este rio está em altura de doze graus e dois terços. Deste rio até Tapoam são
três léguas, cuja terra é baixa e fraca, e não serve, ao longo do mar, mais que para gado;
e até quatro léguas pela terra dentro está este limite e a terra dele ocupada com currais
de vacas. Esta terra e outra tanta além do rio de Joanne é do conselho da cidade do
Salvador. A Tapuam é uma ponta saída ao mar, com uma pedra do cabo cercada dele, a que o
gentio chama deste nome, que quer dizer pedra baixa; defronte desta ponta, num alto, está
uma fazenda de Sebastião Luís, com ermida de São Francisco. Este posto é o que na carta de
marear se chama os Lençóis de Areia\footnote{ São altíssimas dunas do litoral norte que
podem ser avistadas do mar e, dessa forma, indicavam aos navegantes a proximidade da
Bahia.} por onde se conhece a entrada da Bahia; e para o sertão, duas léguas, está uma
grossa fazenda de Garcia d'Ávila, com outra ermida de São Francisco, mui concertada e
limpa. Desta ponta de Tapuam a duas léguas está o rio Vermelho, que é uma ribeira assim
chamada, que se aqui vem meter no mar, até onde são tudo arrecifes cerrados, sem entrada
nenhuma. Neste rio Vermelho pode desembarcar gente, com bonança, e estarem barcos da costa
ancorados nesta boca dele, não sendo travessia na costa nem ventos mareiros; até aqui está
toda a terra ao longo do mar ocupada com criações de gado vacum. E pela terra
adentro,\footnote{ No manuscrito da \textsc{bgjm}, ``e pela dentro''.} duas léguas, têm os
padres da companhia uma grossa fazenda, com dois currais de vacas, na qual têm umas casas
de refrigério, onde se vão recrear e convalescer das enfermidades, e levam a folgar os
governadores, onde tem um jardim muito fresco, com um formoso tanque de água e uma ermida
muito concertada, onde os padres, quando lá estão, dizem missa. Deste rio Vermelho até a
ponta do Padrão\footnote{ Atual largo do Farol da Barra, na Barra de Santo Antônio,
situado na entrada da Bahia, em Salvador.} é uma légua e corre"-se a costa do rio de Joane
à ponta do Padrão nordeste"-sudoeste.\footnote{ Em Varnhagen (1851 e 1879), ``nornordesde
susudoeste''.}

\paragraph{[28] Em que se declara como Francisco Pereira Coutinho foi povoar a Bahia de
Todos os Santos e os trabalhos que nisso teve} \quad
Quem quiser saber quem foi Francisco Pereira Coutinho, vá aos livros da Índia,\footnote{
Em Varnhagen (1851 e 1879), ``leia os livros''.} e sabê"-lo"-á; e verá seu grande valor e
heroicos feitos, dignos de diferente descanso do que teve na conquista do Brasil, onde lhe
coube por sorte a capitania da Bahia de Todos os Santos, de que lhe fez mercê el"-rei D.
João o \textsc{iii}, de gloriosa memória, pela primeira vez, da terra que há da ponta do
Padrão até o rio de São Francisco, ao longo do mar; e, para o sertão, de toda a terra que
couber na demarcação deste Estado, depois lhe fez mercê da terra da Bahia com seus
recôncavos. E como este esforçado capitão tinha o ânimo incansável, não receou de ir
povoar a sua capitania em pessoa, e fez"-se prestes com muitos moradores casados e outros
solteiros,\footnote{ No manuscrito da \textsc{bgjm}, ``muitos moradores casados e outros
soldados''.} que embarcou em uma armada, que fez à sua custa, com a qual partiu do porto
de Lisboa. E com bom vento fez a sua viagem até entrar na Bahia e desembarcou na ponta do
Padrão dela para dentro, e fortificou"-se, onde agora chamam a Vila Velha, no qual sítio
fez uma povoação e fortaleza sobre o mar, onde esteve de paz com o gentio os primeiros
anos, no qual tempo os moradores fizeram suas roças e lavouras. Desta povoação para dentro
fizeram uns homens poderosos, que com ele foram, dois engenhos de açúcar, que depois foram
queimados pelo gentio, que se alevantou, e destruiu todas as roças e fazendas, pelas quais
mataram muitos homens, e nos engenhos, quando deram neles. Pôs este alevantamento a
Francisco Pereira em grande aperto, porque lhe cercaram a vila e fortaleza, tomando"-lhe a
água e mais mantimentos, os quais neste tempo lhe vinham por mar da capitania dos Ilheos,
os quais iam buscar da vila as embarcações, com grande risco dos cercados, que estiveram
nestes trabalhos, ora cercados, ora com tréguas, sete ou oito anos, nos quais passaram
grandes fomes, doenças e mil infortúnios,\footnote{ No manuscrito da \textsc{bgjm},
``doenças e mil enfermidades, digo infortúnios''.} a quem este gentio Tupinamba matava
gente cada dia, com o que se ia apoquentando muito; onde mataram um seu filho bastardo e
alguns parentes e outros homens de nome, com o que a gente, que estava com Francisco
Pereira, desesperadas de poder resistir tantos anos a tamanha e tão apertada guerra, se
determinou com ele apertando"-o que ordenasse de os pôr em salvo, antes que se acabasse de
consumir em poder de inimigos tão cruéis, que ainda não acabavam de matar um homem, quando
o espedaçavam e comiam. E vendo este capitão que a sua gente, que era já tão pouca, mui
determinada, ordenou de a pôr em salvo e passou"-se por mar com ela em uns caravelões que
tinha no porto, para a capitania dos Ilheos;\footnote{ Em Varnhagen (1851 e 1879), ``E
vendo este Capitão sua gente, que já era mui pouca, tão determinada, ordenou de a pôr em
salvo e passou"-se por mar com ela em um caravelão que tinha, para a capitania do
ilheos''.} do que se espantou o gentio muito, e arrependido da ruim vizinhança que lhe
tinha feito, movido também de seu interesse, vendo que como se foram os portugueses, lhe
ia faltando os resgates que eles lhes davam a troco de mantimentos, ordenaram de mandar
chamar Francisco Pereira, mandando"-lhes prometer toda a paz e boa amizade, o qual recado
foi dele festejado, e embarcou"-se logo com alguma gente em um caravelão que tinha e em
outro em que vinha Diogo Álvares, de alcunha o Caramuru, grande língua do gentio, e
partiu"-se para a Bahia, e querendo entrar pela barra adentro, lhe sobreveio muito vento e
tormentoso, que o lançou sobre os baixos da ilha de Tapariqua,\footnote{ Em Varnhagen (1851
e 1879), ``Taparica''. Hoje é a Ilha de Itaparica, no Estado da Bahia.} onde deu à costa;
mas salvou"-se a gente toda deste naufrágio, mas não das mãos dos topinambas, que viviam
nesta ilha, os quais se ajuntaram, e à traição mataram a Francisco Pereira e à gente do
seu caravelão, do que escapou Diogo Álvares com os seus com boa linguagem. Desta maneira
acabou às mãos dos Tupinambas o esforçado cavaleiro Francisco Pereira Coutinho, cujo
esforço não puderam render os rumes\footnote{ Designação dada aos turcos pelos europeus
entre os séculos \textsc{xvi} e \textsc{xviii}.} e malabares\footnote{ Natural ou
habitante da região de Malabar, na Ásia.} da Índia, e foi rendido destes bárbaros, o qual
não somente gastou a vida nesta pretensão, mas quanto em muitos anos ganhou na Índia com
tantas lançadas e espingardadas, e o que tinha em Portugal, com o que deixou sua mulher e
filhos postos no hospital.\footnote{ Hospital Real de Lisboa.}

\paragraph{[29] Em que se torna a correr a costa e explicar a terra dela da ponta do Padrão
até o rio de Camamu} \quad
Não tratemos da Bahia mais particularmente por ora, porque lhe não cabe neste lugar dizer
mais, para no seu se dizer o prometido, pois à sua conta se fez outro memorial,\footnote{
No manuscrito da \textsc{bgjm}, ``se fez este memorial''.} de que pegaremos como acabarmos
de correr a costa, e far"-lhe"-emos seu ofício da melhor maneira que soubermos.

E tornando à ponta do Padrão dela, que está em altura de treze graus esforçados, dizemos
que desta ponta à do morro de São Paulo na ilha de Tinhare são nove ou dez léguas, a qual
ponta está em treze graus e meio, e corre"-se com a ponta do Padrão nordeste"-sudoeste.

Faz esta ilha de Tinhare da banda do sul um morro escalvado, que se diz de São Paulo, a
cuja abrigada ancoram naus de todo o porte, e quem quiser entrar desta ponta para dentro
pode ir bem chegado ao morro e achará fundo de cinco e seis braças. Nesta ilha de Tinhare,
junto do morro, esteve a primeira povoação da capitania dos Ilheos, donde despovoaram logo
por não contentar a terra aos primeiros povoadores, a qual ilha está tão chegada à terra
firme que no mais estreito não há mais canal que de um tiro de espingarda de terra a
terra.

De Tinhare à ilha de Boipeba são quatro léguas; esta ilha possuem os padres da companhia
do colégio da Bahia, a qual e a de Tinhare estão povoadas de portugueses, que despejaram a
terra firme com medo dos Aimores, que lhes destruíram as fazendas e mataram muitos
escravos. De Boipeba ao rio de Camamu são três léguas, o qual está em quatorze graus. Tem
esse rio de Camamu uma boca grande e nela uma ilha pequena perto da ponta da banda do
norte, e tem bom canal para poderem entrar nele naus grandes, as quais hão de entrar
chegadas à ponta da banda do sul, onde têm seis e sete braças de fundo. Da barra deste rio
para dentro tem uma formosa baía, com muitas ribeiras que se nela metem, onde se podem
fazer muitos engenhos. Este rio é muito grande e notável, e vem de muito longe, o qual se
navega do salgado para cima ou seis léguas até a cachoeira, que lhe impede não se navegar
muitas léguas, porque pelo sertão se pode navegar, porque traz sempre muita água, cuja
terra com dez léguas de costa possuem os padres da companhia por lhes fazer dela doação
Mem de Sá;\footnote{ Importante administrador colonial português, Mem de Sá (1500-1572)
chegou ao Brasil em 28 de dezembro de 1557, tomando posse como terceiro governador"-geral
do Brasil no ano seguinte. Os catorze anos de seu governo se caracterizaram por
consideráveis realizações, tais como: a fundação da cidade de São Sebastião do Rio de
Janeiro; a expulsão dos franceses, em 1567; e o aldeamento de tribos indígenas em missões.
Também incentivou a produção açucareira e estimulou o tráfico de escravos africanos para o
Brasil, dificultando a escravidão dos indígenas já catequizados.} os quais padres a
começaram a povoar e alguns outros moradores; mas todos despejaram por mandado dos
Aimores,\footnote{ Na edição de 1851, ``Aimorés'', e na de 1879, ``\textit{Aymorés}''.}
que lhes deram tal trato, que os fez passar dali para as ilhas de Boipeba e Tinhare. E
corre"-se a costa desta ilha ao Camamu norte"-sul pouco mais ou menos.

\paragraph{[30] Em que se declara a terra que há do rio de Camamu até os Ilheos} \quad
Este rio de Camamu está em altura de quatorze graus; e dele ao das Contas são seis léguas,
cuja costa se corre norte"-sul. Tem este rio das Contas, a que os índios chamam Jusiape,
para o conhecer quem vem de mar em fora, sobre a boca uns campinhos descobertos do mato, e
ao mar uma pedra como ilhéu que está na mesma boca, pela qual entram navios de honesto
porte, porque tem fundo e canal para isso bem chegado a esta pedra. Este rio vem de muito
longe e traz mais água sempre que o Tejo, o qual se navega da barra para dentro sete ou
oito léguas até a Cachoeira, e dela para cima se pode também navegar, por ter fundo para
isso. E é muito farto de pescado e marisco e de muita caça, cuja terra é grossa e boa, e
tem muitas ribeiras para engenhos que se vêm meter neste rio, os quais se deixam de fazer
por respeito dos Aimores, pelo que não está povoado, o qual está em catorze graus e um
quarto. Deste rio das Contas a duas léguas está outro rio, que se chama Avemoam,\footnote{
Em Varnhagen (1851 e 1879), ``Amemoão''.} e dele a uma légua está outro rio que se chama
Japarape, os quais se passam a vau ao longo do mar, que também estão despovoados. De
Japarape ao rio de Taipe são três léguas; este rio de Taipe\footnote{ Em Varnhagen (1851),
``Taype''.} vem de muito longe, no qual se metem muitas ribeiras que o fazem caudaloso,
cujo nascimento é de uma lagoa que tem em si duas ilhas. Da lagoa para baixo e perto do
mar tem outra ilha e um engenho mui possante de Luís Álvares de Espinha,\footnote{ Em
Varnhagen (1851 e 1879), ``Luís Álvares de Espenha''.} junto do qual engenho está uma
lagoa grande de água doce, em que se tomam muitas arraias e outro peixe do mar e muitos
peixes"-bois, coisa que faz grande espanto, por se não achar peixe do mar em nenhumas
alagoas. De Taipe ao rio de São Jorge, que é o dos Ilheos, são duas léguas, a qual terra é
toda boa, e está muita aproveitada com engenhos de açúcar, ainda que estão mui apertados
com esta praga dos Aimores; e para se conhecer a barra dos Ilheos há de se vir correndo a
costa à vista da praia para se poderem ver os ilhéus, porque são pequenos, e três; e entre
a terra e o ilhéu grande há bom surgidouro, e os navios que houverem de entrar no rio vão
pelo canal que está norte"-sul como o ilhéu grande onde os navios estão seguros com todo o
tempo, e também estão à sombra do ilhéu grande. Este rio tem alguns braços que se navegam
com caravelões e barcos para serviços dos engenhos que tem;\footnote{ Em Varnhagen (1851 e
1879), ``barcas para serviço''.} cuja terra é muito fértil e grossa e de muita caça; e o
rio tem grandes pescarias e muito marisco, o qual está em altura de quinze graus escassos,
e corre"-se a costa dele ao Rio das Contas norte"-sul.

\paragraph{[31] Em que se contém como se começou de povoar a capitania dos Ilheos por ordem
de Jorge de Figueiredo Correa} \quad
Quando el"-rei D. João de Portugal, o terceiro no nome, repartiu parte da terra da costa do
Brasil em capitanias,\footnote{ Em Varnhagen (1851 e 1879), ``Quando el"-rei D. João 3º
repartiu''.} fez mercê de uma delas, com cinquenta léguas de costa, a Jorge de Figueiredo
Correa, escrivão da sua Fazenda, a qual se começa da ponta da baía do Salvador da banda do
sul, que se entende da ilha de Tinhare, como está julgado por sentença que sobre este caso
deu Mem de Sá sendo governador, e Brás Fragoso sendo ouvidor"-geral e provedor"-mor do
Brasil, e vai correndo ao longo da costa cinquenta léguas. E como Jorge de Figueiredo por
respeito de seu cargo não podia ir povoar esta sua capitania em pessoa,\footnote{ Em
Varnhagen (1851 e 1879), ``esta capitania''.} ordenou de o mandar fazer por outrem, para o
que fez prestes à custa de sua fazenda uma frota de navios com muitos moradores, providos
do necessário para a nova povoação. E mandou por seu logo"-tenente a um cavaleiro
castelhano muito esforçado,\footnote{ Em Varnhagen (1851 e 1879), ``a um castelhano muito
esforçado''.} experimentado e prudente, que se chamava Francisco Romeiro, o qual partiu do
porto de Lisboa com sua frota, e fez sua viagem para esta costa do Brasil, e foi ancorar e
desembarcar no porto de Tinhare, e começou a povoar em cima do morro de São Paulo, do qual
sítio se não satisfez. E como foi bem visto e descoberto do rio dos Ilheos, que assim se
chama pelos que tem defronte da barra, de onde a capitania tomou o nome, se passou com
toda a gente para este rio, onde se fortificou e assentou a vila de São Jorge, onde agora
está, em a qual teve, nos primeiros anos, muitos trabalhos de guerra com o gentio; mas
como eram Tupiniquins, gente melhor acondicionada que o outro gentio, fez pazes com eles,
e fez"-lhe tal companhia que com seu favor foi a capitania em grande crescimento, onde
homens ricos de Lisboa mandaram fazer engenhos de açúcar, com o que a terra se enobreceu
muito; a qual capitania Jerônimo de Alarcão, filho segundo de Jorge de Figueiredo, com
licença de Sua Alteza, vendeu a Lucas Giraldes, que nela meteu grande cabedal, com que a
engrandeceu de maneira que veio a ter oito engenhos ou nove. Mas deu nesta terra esta
praga dos Aimores, de feição que não há aí já mais que seis engenhos, e estes não fazem
açúcar, nem há morador que ouse plantar canas, porque em indo os escravos ou homens ao
campo não escapam a estes alarves, com medo dos quais foge a gente dos Ilheos para a
Bahia, e tem a terra quase despovoada, a qual se despovoará de todo, se Sua Majestade com
muita instância não lhe valer. Esta vila foi muito abastada e rica, e teve quatrocentos ou
quinhentos vizinhos; em a qual está um mosteiro dos padres da companhia, e outro que se
agora começa, de São Bento, e não tem nenhuma fortificação nem modo para se defender de
quem a quiser afrontar.

\paragraph{[32] Em que se declara quem são os Aimores, sua vida e costumes} \quad
Parece razão que não passemos avante sem declarar que gentio é este a que chamam Aimores,
que tanto dano têm feito a esta capitania dos Ilheos, segundo fica dito, cuja costa era
povoada dos Tupiniquins, os quais a despovoaram com medo destes brutos, e se foram viver
ao sertão; dos quais Tupiniquins não há já nesta capitania senão duas aldeias, que estão
junto dos engenhos de Henrique Luís,\footnote{ Henrique Luís de Espinha, capitão"-mor dos
Ilhéus.} as quais têm já muito pouca gente.

Descendem estes Aimores de outros gentios a que chamam Tapuyas, dos quais no tempo atrás
se ausentaram certos casais,\footnote{ Em Varnhagen (1851 e 1879), ``nos tempos de
atrás''.} e foram"-se para umas serras mui ásperas, fugindo a um desbarate, em que os
puseram seus contrários, onde residiram muitos anos sem verem outra gente; e os que destes
descenderam, vieram a perder a linguagem e fizeram outra nova que se não entende de
nenhuma outra nação do gentio de todo este Estado do Brasil. E são estes Aimores tão
selvagens que, dos outros bárbaros, são havidos por mais que bárbaros, e alguns se tomaram
já vivos em Porto Seguro e nos Ilheos, que se deixaram morrer de bravos sem quererem
comer. Começou este gentio a sair ao mar no rio das Caravelas, junto de Porto Seguro, e
corre estes matos e praias até o rio Camamu, e daí veio a dar assaltos perto de Tinhare, e
não descem à praia senão quando vêm dar assaltos. Este gentio tem a cor do outro, mas são
de maiores corpos e mais robustos e forçosos; não têm barbas nem cabelos mais no corpo que
os da cabeça, porque os arrancam todos; pelejam com arcos e flechas muito grandes, e são
tamanhos frecheiros, que não erram nunca tiro; são mui ligeiros à maravilha e grandes
corredores. Não vivem estes bárbaros em aldeias, nem casas, como o gentio, nem há quem
lhas visse nem saiba, nem desse com elas pelos matos até hoje; andam sempre de uma para
outra pelos campos e matos, dormem no chão sobre folhas e se lhes chove arrumam"-se ao pé
de uma árvore,\footnote{ Em Varnhagen (1851 e 1879), ``e se lhes chove arrimam"-se''.} onde
engenham umas folhas por cima, quanto os cobre, assentando"-se em cócaras; e não se lhe
achou até agora outro rasto de gasalhado. Não costumam estes alarves fazer roças, nem
plantar nenhuns mantimentos, mantêm"-se das frutas silvestres e da caça que
matam,\footnote{ Em Varnhagen (1851 e 1879), ``alguns mantimentos'' e ``dos frutos
silvestres''.} a qual comem crua ou mal assada, quando têm fogo; machos e fêmeas todos
andam tosquiados e tosquiam"-se com umas canas que cortam muito; a sua fala é rouca da voz,
a qual arrancam da garganta com muita força, e não se poderá escrever, como
vasconço.\footnote{ Inicialmente atribuído a habitante do País Basco, designa indivíduo de
linguagem ininteligível ou confusa.} Vivem estes bárbaros de saltear toda a sorte de
gentio que encontram e nunca se viram juntos mais que vinte até cinquenta
frecheiros;\footnote{ Em Varnhagen (1851 e 1879), ``vinte até trinta flecheiros''.} não
pelejam com ninguém de rosto a rosto; toda a sua briga é atraiçoada, dão assaltos pelas
roças e caminhos por onde andam, esperando o outro gentio e toda a sorte de criatura em
ciladas detrás das árvores, cada um por si, de onde não erram tiro, e todas as flechas
empregam, e se lhe fazem rosto, logo fogem, cada um para sua parte; mas como veem a gente
desmandada, fazem parada e buscam onde fiquem escondidos, até que passem os que seguem e
dão"-lhes nas costas, empregando suas flechas à vontade. Estes bárbaros não sabem nadar, e
qualquer rio que se não passa a vau basta para defensão deles; mas para o passarem vão
buscar a vau muitas léguas pelo rio acima. Comem estes selvagens carne humana por
mantimento, o que não tem o outro gentio que a não come senão por vingança de suas brigas
e antiguidade de seus ódios. A capitania de Porto Seguro e a dos Ilheos estão destruídas e
quase despovoadas com o temor destes bárbaros, cujos engenhos não lavram açúcar por lhe
terem mortos todos os escravos e gente deles, e a das mais fazendas, e os que escaparam de
suas mãos lhes tomaram tamanho medo, que em se dizendo Aimores despejam as fazendas, e
cada um trabalha por se pôr em salvo, o que também fazem os homens brancos, dos quais têm
morto estes alarves de vinte e cinco anos a esta parte, que esta praga persegue estas duas
capitanias, mais de trezentos homens portugueses e de três mil escravos. Costumam"-se
ordinariamente cartearem"-se os moradores da Bahia com os dos Ilheos, e atravessavam os
homens este caminho ao longo da praia, como lhes convinha, sem haver perigo nenhum, o que
estes Aimores vieram a sentir, e determinaram"-se de virem vigiar estas praias e esperar a
gente que por elas passava, onde têm mortos, e com estes muitos homens e muitos mais
escravos; e são estes salteadores tamanhos corredores, que lhes não escapava ninguém por
pés, salvo os que se lhe metiam no mar, onde eles não se atrevem a entrar, mas andam"-nos
esperando que saiam à terra até a noite, que se recolhem; pelo que este caminho está
vedado, e não atravessa ninguém por ele se não com muito risco de sua pessoa; e se se não
busca algum remédio para destruírem estes alarves, eles destruirão as fazendas da Bahia,
para onde vão caminhando de seu vagar. E como eles são tão esquivos, agrestes e inimigos
de todo o gênero humano,\footnote{ Em Varnhagen (1851 e 1879), ``são tão esquivos inimigos
de todo''.} não foi possível saber mais de sua vida e costumes, e o que está dito pode
bastar por agora;\footnote{ Em Varnhagen (1851 e 1879), ``bastar por ora''.} e tornemos a
pegar da costa, começando dos Ilheos por diante.

\paragraph{[33] Em que se declara a costa do rio dos Ilheos até o rio Grande} \quad
Para satisfazermos com o prometido convém que digamos que terra corre do rio de São Jorge
dos Ilheos por diante, do qual a duas léguas está outro rio que se diz
Cururuipe.\footnote{ Em Varnhagen (1851 e 1879), ``está o rio Cururupe''.} Deste rio a
cinco léguas está outro rio, que se chama Patipe, e em nenhum deles podem entrar barcos,
por não terem barra para isso, cuja costa é de praia e limpa, e terra por dentro baixa ao
longo do mar. Deste rio ao rio Grande são sete léguas, o qual está em quinze graus e meio,
e tem na boca três moitas de mato que do mar parecem ilhas, por onde é muito bom de
conhecer. Na ponta da barra da banda do norte da parte de fora tem bom abrigo para
ancorarem navios da costa, os quais entram neste rio se querem; em cujo canal na barra tem
duas braças, depois uma e daí por diante três, e quatro e cinco braças. Este rio se navega
por ele acima em barcos oito ou dez léguas; neste rio será uma povoação muito proveitosa
por ser muito grande e ter grandes pescarias e muito marisco e caça, cuja terra é muito
boa, onde se darão todos os mantimentos que lhe plantarem; e corre"-se a costa deste rio
Grande ao dos Ilheos norte"-sul.

Este rio vem de muito longe e traz sempre muita água e grande correnteza, pelo qual vieram
abaixo alguns homens dos que foram à serra das Esmeraldas com Antônio Dias Adorno, os
quais vieram em suas embarcações, a que chamam canoas, que são de um pau que tem a casca
muito dura e o mais muito mole, o qual cavacam com qualquer ferramenta, de maneira que lhe
deitam todo o miolo fora, e fica somente a casca; e há destas árvores algumas tamanhas que
fazem delas canoas que levam de vinte pessoas para cima.

Bastião Fernandes Tourinho,\footnote{ Em Varnhagen (1851 e 1879), ``Sebastião Fernandes
Tourinho''.} morador em Porto Seguro, com certos companheiros entrou pelo sertão, onde
andou alguns meses à ventura, sem saber por onde caminhava, e meteu"-se tanto pela terra
adentro, que se achou em direito do Rio de Janeiro, o que souberam pela altura do sol, que
este Bastião Fernandes sabia muito bem tomar, e por conhecerem a serra dos Órgãos, que cai
sobre o Rio de Janeiro; e chegando ao campo grande acharam alagoas e riachos que se metiam
neste Rio Grande; e indo com rosto ao noroeste, deram em algumas serras de pedras, por
onde caminharam obra de trinta léguas, e tornando a leste alguns dias deram em uma aldeia
de Tupiniquins, junto de um rio, que se chama Orabo"-agipe;\footnote{ Em Varnhagen (1851 e
1879), ``Razo"-Aguipe''.} e foram por ele abaixo com o rosto ao norte vinte e oito dias em
canoas, em as quais andaram oitenta léguas. Este rio tem grande correnteza, e entram nele
dois rios outros, um da banda do leste, e outro da banda do oeste, com os quais se vem
meter este rio Orabo"-agipe no rio Grande. E depois que entraram nele navegaram nas suas
canoas por ele abaixo vinte e quatro dias, em os quais chegaram ao mar, vindo sempre com a
proa ao oeste. E fazendo esta gente sua viagem, achou no sertão deste rio no mais largo
dele, que será em meio caminho do mar, vinte ilhas afastadas umas das outras uma légua, e
a duas e três, e mais; e acharam quarenta léguas de barra, pouco mais ou menos, um
sumidouro, que vai por baixo da terra mais de uma légua, quando é no verão, que no inverno
traz tanta água que alaga tudo. Do sumidouro para cima tem este rio grande fundo, e a
partes tem poços, que têm seis e sete braças, por onde se pode navegar em grandes
embarcações; quase toda a terra de longo dele é muito boa.

\paragraph{[34] Em que se declara a costa do Rio Grande até o de Santa Cruz} \quad
Do Rio Grande ao seu braço são duas léguas, pelo qual braço entram caravelões, que por ele
vão entrar no mesmo Rio Grande meia légua da barra para cima. Do braço do Rio Grande ao
rio Boiquisape são três léguas, e do Boiquisape à ponta dos baixos de Santo Antônio são
quatro léguas, e da ponta de Santo Antônio ao seu rio é meia légua; do rio de Santo
Antônio ao de Sernamdetibe\footnote{ Em Varnhagen (1851 e 1879), ``Sernanbitibe''.} são
duas léguas; e deste rio de Santo Antônio e da sua ponta até o rio de Sernamdetibe estão
uns baixos com canal entre eles e a costa, por onde entram barcos pequenos. Pela ponta de
Santo Antônio e mais ao mar ficam uns arrecifes do mesmo tamanho, com canal entre uns e
outros. E defronte do rio de Santo Antônio têm estes arrecifes do mar um boqueirão, por
onde pode entrar uma nau e ir ancorar pelo canal que se faz entre um arrecife e o outro,
onde estará segura; no mesmo arrecife do mar está outro boqueirão, por onde podem entrar
caravelões da costa, defronte do rio de Sernamdetibe, pelo qual se pode ir buscar o porto.
Do rio de Sernamdetibe ao de Santa Cruz são duas léguas, onde esteve um engenho de açúcar.
Neste porto de Santa Cruz entram naus da Índia de todo o porte, as quais entram com a proa
a oeste, e surgem em uma enseada como concha, onde estão muito seguras de todo o tempo.
Este rio de Santa Cruz está em dezesseis graus e meio, e corre"-se a costa do Rio Grande
até esta de Santa Cruz nordeste"-sudoeste, o que se há de fazer afastado da terra duas
léguas, por amor dos baixos. Neste Porto de Santa Cruz esteve Pedro Álvares Cabral, quando
ia para a Índia, e descobriu esta terra e aqui tomou posse dela, onde esteve a vila de
Santa Cruz, a qual terra estava povoada então de Tupiniquins, que senhoreavam esta costa
do rio Camamu até o de Cricare, de cuja vida e feitos diremos ao diante. Esta vila de
Santa Cruz se despovoou donde esteve e a passaram para junto do rio de Sernamdetibe, pela
terra ser mais sadia e acomodada para os moradores viverem.

\paragraph{[35] Em que se declara a costa e terra dela do rio de Santa Cruz até Porto
Seguro} \quad
Do rio de Santa Cruz ao de Itacumirim é meia légua, onde esteve o engenho de João da
Rocha. Do rio de Itacumirim ao de Porto Seguro é meia légua; e entre um e outro está um
riacho, que se diz de São Francisco, junto das barreiras vermelhas. Defronte do rio de
Itacumirim até o de Santa Cruz vai uma ordem de arrecifes que tem quatro boqueirões, por
onde entram barcos pequenos; e faz outra ordem de arrecifes baixos mais ao mar, que se
começam defronte do engenho de João da Rocha, e por entre uns arrecifes e os outros é a
barra do Porto Seguro, por onde entram naus de sessenta tonéis;\footnote{ Em Varnhagen
(1851 e 1879), ``entram navios''.} e se é navio grande, toma meia carga em Porto Seguro, e
vai acabar de carregar em Santa Cruz.

Porto Seguro está em dezesseis graus e dois terços, e quem vem de mar em fora vá com boa
vigia, por amor dos baixos. E para conhecer bem a terra, olhe para o pé da vila, que está
em um alto, e verá umas barreiras vermelhas, que é bom alvo, ou baliza, para o
conhecer.\footnote{ Em Varnhagen (1851 e 1879), ``para por ele a conhecer''.} Entra"-se
este rio leste"-oeste com a proa nestas barreiras vermelhas até entrar dentro do arrecife;
e como estiver dentro vá com a proa ao sul, e ficará dentro do rio. Da outra banda dos
baixos e contra o sul está outra barra, por onde entram navios do mesmo porte; quem entrar
por esta barra, como estiver dentro dela, descobrirá um riacho, que se diz de São
Francisco; e como o descobrir, vá andando para dentro, até chegar ao porto. De Porto
Seguro à vila de Santo Amaro é uma légua, onde está um pico mui alto em que está a ermida
de Nossa Senhora da Ajuda, que faz muitos milagres. De Santo Amaro ao rio de
Torarom\footnote{ Em Varnhagen (1851 e 1879), ``Tororam''.} é uma légua, onde está um
engenho, que foi de Manoel Rodrigues Magalhães,\footnote{ Em Varnhagen (1851), ``Manuel
Rodrigues Magalhães''.} e junto a este engenho uma povoação, que se diz de São Tiago do
Alto, no qual rio entram caravelões. Deste rio de Tororom ao de Maniape são duas léguas, e
antes de chegarem a ele estão as barreiras vermelhas, que parecem, a quem vem do mar,
rochas de pedras. Do rio de Maniape ao de Urubuguape é uma légua, onde está o engenho de
Gonçalo Pires. Do rio de Urubuguape ao rio do Frade é uma légua,\footnote{ No manuscrito
da \textsc{bgjm}, ``rio dos Frades''.} onde entram barcos, e chama"-se do Frade por se nele
afogar um, nos tempos atrás. Do rio do Frade ao de Insuasema\footnote{ Em Varnhagen (1851
e 1879), ``Juhuacema''.} são duas léguas, onde esteve uma vila que se despovoou o ano de
1564, pela grande guerra que tinham os moradores dela com os Aimores. Neste lugar esteve
um engenho, onde chamam a ponta do Corurumbabo.\footnote{ Em Varnhagen (1851 e 1879),
``Cururumbabo''.}

\paragraph{[36] Em que se declara quem povoou a capitania de Porto Seguro} \quad
Não é bem que passemos mais avante sem declararmos cuja é esta capitania do Porto Seguro,
e quem foi o povoador dela, da qual fez el"-rei D. João, o \textsc{iii} de Portugal, mercê
a Pero do Campo Tourinho,\footnote{ Em Varnhagen (1851 e 1879), ``Pedro de Campos
Tourinho''. Donatário da Capitania de Porto Seguro, Pero do Campo Tourinho chegou ao
Brasil em 1535 para tomar posse de suas terras. Seus esforços resultaram no desbravamento
do território e na fundação de vilas para alojar e defender os colonos, entre as quais
estão as de Porto Seguro, Santo Amaro e Santa Cruz.} que foi um cavaleiro natural da vila
de Viana da foz de Lima, homem nobre, esforçado, prudente, e muito visto na arte do
navegar;\footnote{ Em Varnhagen (1851 e 1879), ``arte do marear''.} cuja doação foi de
cinquenta léguas de costa, como as mais que ficam declaradas.

Para Pero do Campo poder povoar esta capitania vendeu toda sua fazenda e ordenou à sua
custa uma frota de navios, que fez prestes, em a qual se embarcou com sua mulher, Ignêz
Fernandes Pinto, e filhos,\footnote{ No manuscrito da \textsc{bgjm}, ``se embarcou com sua
mulher e filhos''.} e muitos moradores, casados, seus parentes e amigos, e outra muita
gente, com a qual se partiu do porto de Viana. E com bom tempo foi demandar a terra do
Brasil, e foi tomar porto no rio de Porto Seguro, onde desembarcou com sua gente, e se
fortificou no mesmo lugar, onde agora está a vila cabeça desta capitania, a qual em tempo
de Pero do Campo floresceu e foi mui povoada de gente; o qual edificou mais a vila de
Santa Cruz e a de Santo Amaro, de que já falamos; e em seu tempo se ordenaram alguns
engenhos de açúcar, no que teve nos primeiros anos muito trabalho com a guerra que lhe fez
o gentio Tupiniquim, que vivia naquela terra, o qual lhe foi tão cruel,\footnote{ Em
Varnhagen (1851 e 1879), ``a qual lha fez tão cruel''.} que o teve cercado por muitas
vezes, e posto em grande aperto, com o que lhe mataram muita gente; mas, como assentaram
pazes, ficou o gentio quieto, e daí por diante ajudou aos moradores fazer suas roças e
fazendas, a troco do resgate que por isso lhe davam. Por morte de Pedro do Campo ficou
esta capitania mal governada com seu filho Fernão do C. Tourinho,\footnote{ No manuscrito
da \textsc{bgjm}, ``com seu filho que após ele''.} e após ele durou pouco e se começou
logo a desbaratar, a qual herdou uma filha de Pero do Campo, que se chamou Leonor do
Campo, que nunca casou. Essa Leonor do Campo, com licença del"-rei, vendeu esta capitania a
D. João de Alencastro, primeiro duque de Aveiro, por cem mil"-réis de juro, o qual a
favoreceu muito com gente e capitão que a governasse, e com navios que a ela todos os anos
mandava com mercadorias; onde mandou fazer, à sua custa, engenho de açúcar, e provocou a
outras pessoas da cidade de Lisboa a que fizessem outros engenhos,\footnote{ Em Varnhagen
(1851 e 1879), ``navio que ela todos os anos mandava, e com mercadorias; provocou a outras
pessoas de Lisboa''.} em cujo tempo os padres da companhia edificaram na vila de Porto
Seguro um mosteiro, onde residem sempre dez ou doze religiosos, que governam ainda agora
algumas aldeias de Tupiniquins cristãos, que estão nesta capitania; em a qual houve, em
tempo do duque, sete ou oito engenhos de açúcar, onde se lavrava cada ano muito, que se
trazia a este reino, e muito pau da tinta, do que na terra há muito. Nesta capitania se
não deu nunca gado vacum por respeito de certa erva, que lhe faz câmaras, de que vem a
morrer; mas dá"-se a outra criação, de éguas, jumentos e cabras, muito bem; dos jumentos há
tanta quantidade na terra, que andam bravos pelo mato em bandos, e fazem nojo às
novidades, os quais ficaram no campo dos moradores, que desta capitania se passaram para
as outras, fugindo dos Aimores, no qual tem feito tamanha destruição, que não tem já mais
que um engenho que faça açúcar, por terem mortos todos os escravos dos outros e muitos
portugueses, pelo que estão despovoados e postos por terra, e a vila de Santo Amaro e a de
Santa Cruz quase despovoadas de todo; e a vila de Porto Seguro está mais danificada e
falta de moradores, em a qual se dão as canas"-de"-açúcar muito bem; e muitas uvas, figos,
romãs, e todas as frutas de espinho, onde a água de flor é finíssima, e se leva a vender à
Bahia por tal.\footnote{ Em Varnhagen (1851 e 1879), ``e se leva à Bahia a vender por
tal''.} Esta capitania parte com a dos Ilheos pelo Rio Grande pouco mais ou menos; e pela
outra parte com a do Espírito Santo, de Vasco Fernandes Coutinho, para onde imos
caminhando.

\paragraph{[37] Em que se declara a terra e costa do Porto Seguro até o rio das Caravelas} \quad
Da vila de Porto Seguro à ponta Corurumbabo são oito léguas, cuja costa se corre
norte"-sul; esta ponta é baixa, e de areia, a qual aparece no cabo do arrecife e demora ao
noroeste, e está em altura de dezessete graus e um quarto. Este arrecife é perigoso e
corre afastado da terra légua e meia. Da ponta de Corurumbabo ao cabo das barreiras
brancas são seis léguas, até onde corre este arrecife, que começa da ponta de Cururumbabo,
pelo que até o cabo destas barreiras brancas se corre esta costa por aqui, afastado da
terra légua e meia. Do cabo das barreiras brancas até ao rio das Caravelas são cinco ou
seis léguas, em o qual caminho há alguns baixos, que arrebentam em frol, de que se hão de
guardar com boa vigia os que por aqui passarem. Defronte de Jucuru está uma rodela de
baixos, que não arrebentam, que é necessário que sejam bem vigiados; e corre"-se a costa de
Cururumbabo até o rio das Caravelas norte"-sul, o qual está em dezoito graus.

Tem este rio na boca uma ilha de uma légua, que lhe faz duas barras, a qual está povoada
com fazendas, e criações de vacas, que se dão nela muito bem. Por este rio acima entram
caravelões da costa, mas tem na boca da barra muitas cabeças ruins, pelo qual entra a maré
três ou quatro léguas, que se navegam com barcos.

A terra por este rio acima é muito boa, em que se dão todos os mantimentos que lhe plantam
muito bem, e pode"-se fazer aqui uma povoação, onde os moradores dela estarão muito
providos de pescado e mariscos, e muita caça, que por toda aquela terra há. Este rio vem
de muito longe, e pelo sertão é povoado do gentio bem acondicionado, e não faz mal aos
homens brancos, que vão por ele acima para o sertão. Aqui neste rio foi desembarcar
Antônio Dias Adorno com a gente que trouxe da Bahia, quando por mandado do governador Luís
de Brito de Almeida foi ao sertão, no descobrimento das esmeraldas, e foi por este rio
acima com cento e cinquenta homens, e quatrocentos índios de paz e escravos, e todos foram
bem tratados e recebidos dos gentios, que acharam pelo sertão deste rio das Caravelas.

\paragraph{[38] Em que se declara a terra que há do rio das Caravelas até Cricare} \quad
Do rio das Caravelas até o rio de Peruipe são três léguas, as quais se navegam pelo canal
indo correndo a costa. Neste rio entram caravelões da costa, junto da qual a terra faz uma
ponta grossa ao mar de grande arvoredo, e toda a mais terra é baixa. Do direito desta
ponta se começam os abrolhos e seus baixos, mas entre os baixos e a terra há fundo de seis
e sete braças, uma légua ao mar somente, por onde vai o canal.

Deste rio Peruipe ao de Maruipe\footnote{ Em Varnhagen (1879), ``Mocuripe''.} são cinco
léguas, o qual tem na boca uma barreira branca como lençol, por onde é bom de conhecer, o
qual está dezoito graus e meio. Por este rio Maruipe entram caravelões da costa à vontade,
e há maré por ele acima muito grande espaço, cuja terra é boa e para se fazer conta dela
para se povoar, porque há nela grandes pescarias, muito marisco e caça.

Deste rio de Maruipe ao de Cricare são dez léguas, e corre"-se a costa do rio das Caravelas
até Cricare norte"-sul, e toma da quarta nordeste"-sudoeste, o qual rio Maruipe está em
dezoito graus e três quartos,\footnote{ No manuscrito da \textsc{bgjm}, ``o qual rio está
em''.} pelo qual entram navios de honesto porto, e é muito capaz para se poder povoar, por
a terra ser muito boa e de muita caça, e o rio de muito pescado e marisco, onde se podem
fazer engenhos de açúcar, por se meterem nele muitas ribeiras de água, boas para eles.
Este rio vem de muito longe, e navega"-se quatro ou cinco léguas por ele acima; o qual tem
na barra, da banda do sul quatro abertas, uma légua e mais uma da outra, as quais estão na
terra firme por cima da costa, que é baixa e sem arvoredo, e de campinas. E quem vem do
mar em fora parecem"-lhe estas abertas bocas de rios,\footnote{ No manuscrito da
\textsc{bgjm}, ``estas barreiras, digo abertas bocas''.} por onde a terra é boa de
conhecer. Até aqui senhorearam a costa os Tupiniquins, de que é bem que digamos neste
capítulo que se segue antes que cheguemos à terra dos guoitacases.\footnote{ Em Varnhagen
(1851 e 1879), ``Goaitacazes''.}

\paragraph{[39] Em que se declara quem são os Tupiniquins e sua vida e costumes} \quad
Já fica dito como o gentio Tupiniquim senhoreou e possuiu a terra da costa do Brasil, ao
longo do mar, do rio de Camamu até o rio de Cricare, o qual tem agora despovoado toda esta
comarca, fugindo dos Topinambás, seus contrários, que os apertaram por uma banda, e aos
Aimores, que os ofendiam por outra;\footnote{ No manuscrito da \textsc{bgjm}, ``ofendiam
por todas''.} pelo que se afastaram do mar, e, fugindo ao mau tratamento que lhes alguns
homens brancos faziam, por serem pouco tementes a Deus. Pelo que não vivem agora junto ao
mar mais que os que são cristãos de que já fizemos menção. Com este gentio tiveram os
primeiros povoadores das capitanias dos Ilheos e Porto Seguro e os do Espírito Santo, nos
primeiros anos, grandes guerras e trabalhos, de quem receberam muitos danos; mas, pelo
tempo adiante, vieram a fazer pazes, que se cumpriram e guardaram bem de parte a parte, e
de então para agora foram os Tupiniquins mui fiéis e verdadeiros aos portugueses. Este
gentio e os Tupinaes descendem todos de um tronco, e não se têm por contrários
verdadeiros, ainda que muitas vezes tivessem diferenças e guerras, os quais Tupinaes lhe
ficavam nas cabeceiras pela banda do sertão, com quem a maior parte dos Tupiniquins agora
estão misturados. Este gentio é da mesma cor baça e estatura que o outro gentio de que
falamos, o qual tem a linguagem, vida e costumes e gentilidades dos Tupinambas, ainda que
são seus contrários, em cujo título se declararão mui particularmente tudo o que se pôde
alcançar. E ainda que são contrários os Tupiniquins dos Topinambas, não há entre eles na
língua e costumes mais diferença da que têm os moradores de Lisboa dos da Beira; mas este
gentio é mais doméstico e verdadeiro que todo outro da costa deste Estado. É gente de
grande trabalho e serviço, e sempre nas guerras ajudaram aos portugueses, contra os
Aimores, Tapuyas e Tamoios, como ainda hoje fazem esses poucos que se deixaram ficar junto
ao mar e das nossas povoações, com quem vizinham muito bem, os quais são grandes
pescadores de linha, caçadores e marinheiros, são valentes homens, caçam, pescam, cantam,
bailam como os Tupinambas e nas coisas de guerra são mui industriosos, e homens para
muito, de quem se faz muita conta a seu modo entre o gentio.

\paragraph{[40] Em que se declara a costa de Cricare até o rio Doce, e do que se descobriu
por ele acima, e pelo Açeçi} \quad
Do rio de Cricare até o rio Doce são dezessete léguas, as quais se correm pela costa
norte"-sul; o qual rio Doce está em altura de dezenove graus.

A terra deste rio, ao longo do mar, é baixa e afastada da costa; por ela adentro tem
arrumada uma serra, que parece, a quem vem do mar em fora, que é a mesma costa. A boca
deste rio é esparcelada bem uma légua e meia ao mar, mas tem seu canal, por onde entram
navios de quarenta tonéis, o qual rio se navega pela terra adentro algumas léguas, cuja
terra ao longo do rio por ali acima é muito boa, que dá todos os mantimentos acostumados
muito bem, onde se darão muito bons canaviais de açúcar, se os plantarem, e se podem fazer
alguns engenhos, por ter ribeiras mui acomodadas a eles. Este rio Doce vem de muito longe
e corre até o mar quase leste"-oeste, pelo qual um Bastião Fernandes Tourinho, de quem
falamos, fez uma entrada navegando por ele acima, até onde o ajudou a maré, com certos
companheiros,\footnote{ No manuscrito da \textsc{bgjm}, ``navegando por ele, até onde''.}
e entrando por um braço acima, que se chama Mandi, onde desembarcou, caminhou por terra
obra de vinte léguas, com o rosto a oés"-sudoeste, onde foi dar com uma lagoa,\footnote{ Em
Varnhagen (1851 e 1879), ``a lés"-sudoeste, e que foi dar''.} a que o gentio chama boca do
mar, por ser muito grande e funda, da qual nasce um rio que se mete neste rio Doce, e leva
muita água. Esta lagoa cresce às vezes tanto, que faz grande enchente nesse rio Doce.
Desta lagoa corre este rio a leste, e dela a quarenta léguas tem uma cachoeira; e andando
esta gente ao longo deste rio, que sai da lagoa mais de trinta léguas, se detiveram ali
alguns dias; e tornando a caminhar, andaram quarenta dias com o rosto a oeste, e no cabo
deles chegaram onde se mete este rio no Doce, e andaram nestes quarenta dias setenta
léguas pouco mais ou menos. E como esta gente chegou a este rio Doce, e o acharam tão
possante, fizeram nele canoas de casca, em que se embarcaram, e foram por ali acima, até
onde se mete neste rio outro, a que chamam Açeçi, pelo qual entraram e foram quatro
léguas, e no cabo delas desembarcaram e foram por terra com o rosto ao noroeste onze dias,
e atravessaram o Açeçi, e andaram cinquenta léguas ao longo dele da banda ao sul trinta
léguas. Aqui achou esta gente umas pedreiras, que têm umas pedras verdoengas, e tomam de
azul que parecem turquescas,\footnote{ Na edição de 1851, ``umas pedreiras, umas pedras
verdoengas, e tomam do azul, que parecem turquesquas'', e na de 1879, ``umas pedreiras,
umas pedras verdoengas, e tomam do azul, que parecem turquescoas''.} e afirmou o gentio
aqui vizinho que no cimo deste monte se tiravam pedras muito azuis, e que havia outras
que, segundo sua informação, têm ouro muito descoberto. E quando esta gente passou o Açeçi
a derradeira vez, dali cinco ou seis léguas da banda do norte, achou Bastião Fernandes uma
pedreira de esmeraldas e outra de safiras, as quais estão ao pé de uma serra cheia de
arvoredo do tamanho de uma légua, e quando esta gente ia do mar por este rio Doce acima
sessenta ou setenta léguas da barra, acharam umas serras ao longo do rio de Arvoredo, e
quase todas de pedra, em que também acharam pedras verdes; e indo mais acima quatro ou
cinco léguas da banda do sul, está outra serra, em que afirma o gentio haver pedras verdes
e vermelhas tão compridas como dedos, e outras azuis, todas mui resplandecentes.

Desta serra para a banda de leste pouco mais de uma légua está uma serra, que é quase toda
de cristal muito fino, a qual terá em si muitas esmeraldas,\footnote{ Em Varnhagen (1851 e
1879), ``a qual cria em si''.} e outras pedras azuis. Com estas informações que Bastião
Fernandes deu a Luís de Brito, sendo governador, mandou Antônio Dias Adorno, como já fica
dito atrás, o qual achou ao pé desta serra, da banda de leste as esmeraldas, e da de oeste
as safiras.\footnote{ Em Varnhagen (1851 e 1879), ``da banda do norte as esmeraldas, e da
de leste as safiras''.} E umas e outras nascem no cristal, de onde trouxeram muitas e
algumas muito grandes, mas todas baixas; mas presume"-se que debaixo da terra as deve de
haver finas, porque estas estavam à flor da terra. Em muitas partes achou esta gente
pedras desacostumadas, de grande peso, que afirmam terem ouro e prata, do que não
trouxeram amostras, por não poderem trazer mais que as primeiras e com trabalho; a qual
gente se tornou para o mar pelo rio Grande abaixo, como já fica dito. E Antônio Dias
Adorno, quando foi a estas pedras, as recolheu por terra, atravessando pelos tupinaes e
por entre os Tupinambas, e com uns e outros teve grandes escaramuças,\footnote{ Em
Varnhagen (1851 e 1879), ``grandes encontros''.} e com muito trabalho e risco de sua
pessoa chegou à Bahia e fazenda de Gabriel Soares de Sousa.

\paragraph{[41] Em que se declara a costa do rio Doce até a do Espírito Santo} \quad
Do rio Doce ao dos Reis Magos são oito léguas; e faz a terra de um rio ao outro uma
enseada grande, o qual rio está em dezenove graus e meio, e corre"-se a costa de um a outro
nordeste"-sudoeste. Na boca deste rio dos Reis Magos estão três ilhas redondas, por onde é
bom de conhecer, no qual entram navios da costa, cuja terra é muito fértil, e boa para se
poder povoar, onde se podem fazer alguns engenhos de açúcar, por ter ribeiras que nele se
metem, mui acomodadas para isso. Navega"-se neste rio da barra para dentro quatro ou cinco
léguas, em o qual há grandes pescarias e muito marisco; e no tempo que estava povoado de
gentio, havia nele muitos mantimentos, e aqui iam resgatar os moradores do Espírito Santo,
o que causava grande fertilidade da terra.\footnote{ Em Varnhagen (1851 e 1879), ``grande
fertilidade. Da terra dos Reis Magos ao rio''.}

Dos Reis Magos ao rio das Barreiras são oito léguas, do qual se faz pouca conta. Do rio
das Barreiras à ponta do Tubarão são quatro léguas, sobre o qual está a serra do Mestre
Álvaro; da ponta do Tubarão à ponta do morro de João Moreno são duas léguas, onde está a
vila de Nossa Senhora da Vitória; entre uma ponta e outra está o rio do Espírito Santo, o
qual tem defronte da barra meia légua ao mar uma lájea, de que se hão de
guardar.\footnote{ Em Varnhagen (1851 e 1879), ``defronte da barra meia légua ao mar uma
légua''.} Em direito desta ponta da banda do norte, duas léguas pela terra dentro, está a
serra do Mestre Álvaro, que é grande e redonda, a qual está afastada das outras serras;
esta serra aparece a quem vem do mar em fora muito longe, que é por onde se conhece a
barra; esta barra faz uma enseada grande, a qual tem umas ilhas dentro, e corre"-se esta
barra para dentro nordeste"-sudoeste.\footnote{ Em Varnhagen (1851 e 1879), ``e entra"-se
nordeste sudoeste''.} A primeira ilha, que está nesta barra, se chama de D. Jorge, e mais
para dentro está outra, que se diz de Valentim Nunes. Desta ilha para a Vila Velha estão
quatro penedos grandes descobertos, e mais para cima está a ilha de Anna Vaz; mais avante
está o ilhéu da Viúva; e no cabo desta baía está a ilha de Duarte de Lemos, ``no cabo
desta baía fica a ilha'' onde está assentada a vila do Espírito Santo, a qual se edificou
no tempo da guerra grande pelos goitacazes,\footnote{ Na edição de 1851, ``no tempo da
guerra pelos Goaitacazes'', e na de 1879, ``no tempo da guerra pelos
\textit{Guaitacazes}''.} que apertaram muito com os povoadores da Vila Velha. Defronte da
vila do Espírito Santo, da banda da Vila Velha, está um penedo mui alto a pique sobre o
rio, ao pé do qual se não acha fundo; é capaz este penedo para se edificar sobre ele uma
fortaleza, o que se pode fazer com pouca despesa, da qual se pode defender este rio ao
poder do mundo todo. Este rio do Espírito Santo está em altura de vinte graus e um terço.

\paragraph{[42] Em que se declara como el"-rei fez mercê da capitania do Espírito Santo a
Vasco Fernandes Coutinho, e como a foi povoar em pessoa} \quad
Razão tinha Vasco Fernandes Coutinho de se contentar com os grandes e heroicos feitos que
tinha com as armas acabado nas partes da Índia, onde nos primeiros tempos de sua conquista
se achou, no que gastou o melhor de sua idade; e passando"-se para estes reinos em busca do
galardão de seus trabalhos, pediu em satisfação deles à Sua Alteza licença para entrar em
outros maiores, pedindo que lhe fizesse mercê de uma capitania na costa do Brasil, porque
a queria ir povoar, e conquistar o sertão dela, a cujo requerimento el"-rei D. João, o
\textsc{iii} de Portugal, satisfez fazendo"-lhe mercê de cinquenta léguas de terra ao longo
da costa no dito Estado, com toda a terra para o sertão, que coubesse na sua demarcação,
começando onde acabasse Pero do Campo, capitão de Porto Seguro. Contente este fidalgo com
a mercê que pediu, para satisfazer à grandeza de seus pensamentos, ordenou à sua custa uma
frota de navios mui provida de moradores e das munições de guerra necessárias, com tudo o
que mais convinha a esta empresa, em a qual se embarcaram, entre fidalgos e criados
del"-rei, sessenta pessoas, entre as quais foi D. Jorge de Menezes, o de Maluco, e D. Simão
de Castelo Branco, que por mandado de Sua Alteza iam cumprir suas penitências a estas
partes. Embarcado este valoroso capitão com sua gente na frota que estava prestes, partiu
do porto de Lisboa com bom tempo, e fez sua viagem para o Brasil, onde chegou a
salvamento, à sua capitania, em a qual desembarcou e povoou a vila de Nossa Senhora da
Vitória, a que agora chamam a Vila Velha, onde se logo fortificou, a qual em breve tempo
se fez uma nobre vila para aquelas partes. Derredor desta vila se fizeram logo quatro
engenhos de açúcar mui bem providos e acabados, os quais começaram de lavrar açúcar, como
tiveram canas para isso, que se na terra deram muito bem. Nestes primeiros tempos teve
Vasco Fernandes Coutinho algumas escaramuças com o gentio seu vizinho, com a qual se houve
de feição que, entendendo estes índios que não podiam ficar bem do partido,\footnote{ Ficar
bem do partido: estar melhor.} se afastaram da vizinhança do mar por aquela parte, por
escusarem brigas que da vizinhança se seguiam. A este gentio chamam Goaitazes,\footnote{ Na
edição de 1851, ``Guaitacazes'', e na de 1879, ``\textit{Guaytcazes}''.} de quem diremos
adiante.

Como Vasco Fernandes viu o gentio quieto, e a sua capitania tanto avante, e em termos de
florescer de bem em melhor, ordenou de vir para Portugal para se fazer prestes do
necessário para ir conquistando a terra pelo sertão, até descobrir ouro e prata e a outros
negócios que lhe convinham; e concertando suas coisas, como relevava, se partiu, e deixou
a D. Jorge de Menezes para em sua ausência a governar, ao qual os Tupiniquins, de uma
banda, e os goainases, da outra, fizeram tão crua guerra que lhe queimaram os engenhos e
muitas fazendas, o desbarataram e mataram a flechadas; o que também fizeram depois a D.
Simão de Castelo Branco, que lhe sucedeu na capitania, e a outra muita gente, e puseram a
vila em cerco e em tal aperto que, não podendo os moradores dela resistir ao poder do
gentio, a despovoaram de todo e se passaram à ilha de Duarte de Lemos, onde ainda estão; a
qual ilha se afasta da terra firme um tiro de berço.\footnote{ Distância que percorria a munição do berço, peça curta de artilharia que atirava balas de ferro de pequeno calibre.}

Esta vila se povoou de novo com título do Espírito Santo, e muitos dos moradores, não se
havendo ali por seguros do gentio, se passaram a outras capitanias. E tornando"-se Vasco
Fernandes para a sua capitania, vendo"-a tão desbaratada, trabalhou todo o possível por
tomar satisfação deste gentio, o que não foi em sua mão, por estar impossibilitado de
gente e munições de guerra, e o gentio mui soberbo com as vitórias que tinha alcançado,
antes viveu muitos anos afrontado dele naquela ilha, onde, a seu requerimento, o mandou
socorrer Mem de Sá, que naquele tempo governava este Estado; o qual ordenou na Bahia uma
armada bem fornecida de gente e armas, que era de navios da costa, mareáveis, da qual
mandou por capitão a seu filho Fernão de Sá, que com ela foi entrar no rio de Cricare,
onde ajuntou com ele a gente do Espírito Santo, que lhe mandou Vasco Fernandes Coutinho; e
sendo a gente toda junta, desembarcou Fernão de Sá em terra, e deu sobre o gentio de
maneira, que o pôs logo em desbarate nos primeiros encontros, o qual gentio se reformou e
ajuntou logo, e apertou com Fernão de Sá, de maneira que o fez recolher para o mar, o que
fez com tamanha desordem dos seus que, antes de poder chegar às embarcações, mataram a
Fernão de Sá, com muita da sua gente, ao embarcar; mas, já agora, está esta capitania
reformada, com duas vilas, em uma das quais está um mosteiro dos padres da companhia, e
tem seus engenhos de açúcar e outras muitas fazendas. No povoar desta capitania gastou
Vasco Fernandes Coutinho muitos mil cruzados, que adquiriu na Índia, e todo o patrimônio
que tinha em Portugal, que todo para isso vendeu, o qual acabou nela tão pobremente, que
chegou a darem"-lhe de comer por amor de Deus, e não sei se teve um lençol seu, em que o
amortalhassem. E seu filho, do mesmo nome, vive hoje na mesma capitania, tão necessitado
que não tem mais de seu que o título de capitão e governador dela.

\paragraph{[43] Em que se vai declarando a costa do Espírito Santo até o cabo de São Tomé} \quad
Do rio do Espírito Santo ao Guoarapari\footnote{ Em Varnhagen (1851 e 1879),
``Goarapari''.} são oito léguas, e faz"-se entre um e outro rio uma enseada. Chegando a
este rio de Guoarapari estão as serras, que dizem de Perocão, e corre"-se a costa do morro
de João Moreno até este rio, norte"-sul; e defronte do morro de João Moreno está a ilha
Escalvada. Do rio de Guoarapari à ponta de Liritibe\footnote{ Em Varnhagen (1851 e 1879),
``ponta do Leritibe''.} são sete léguas, e corre"-se a costa nordeste"-sudoeste, cuja terra
é muito alta; essa ponta tem, da banda do norte, três ilhas, obra de duas léguas ao mar e
a primeira está meia légua da terra firme, as quais têm bom surgidouro; e estão essas
ilhas defronte do rio Guoarapari. A terra deste rio até Liritibi é muito grossa e boa para
povoar como a melhor do Brasil, a qual foi povoada dos guaianases. Esta ponta de Liritibi
tem um arrecife ao mar, que boja bem uma légua e meia, a qual ponta é de terra baixa ao
longo do mar. De Liritibi até Tapemirim são quatro ou cinco léguas, cuja costa se corre
nordeste"-sudoeste, a qual está em vinte graus e três quartos. De Tapemirim a Manage são
cinco léguas, a qual está em vinte e um graus; de Manage ao rio de Paraiba são cinco
léguas, e corre"-se a costa nordeste"-sudoeste, e toma da quarta ao norte"-sul, o qual rio de
Paraíba está em vinte e um graus e dois terços. Este rio de Paraíba tem barra e fundo por
onde entram navios de honesto porte, o qual se pode tornar a povoar, porque derredor dele
e ao longo do mar não há gentio que arrecear, porque todo vive afastado do mar.\footnote{
Em Varnhagen (1851 e 1879), ``tornar a povoar, por derredor dele e ao longo do mar. Da
Parahyba ao''.} Da Paraiba ao cabo de São Tomé são sete léguas, cuja costa se corre
nordeste"-sudoeste, o qual cabo está em vinte e dois graus. Pelo nome deste cabo o tomou a
capitania também de São Tomé, até onde corre o limite dos Guaitacazes, de quem diremos em
seu lugar.\footnote{ No manuscrito da \textsc{bgjm}, ``Guoainazes'', e em Varnhagem (1851
e 1879), ``Guaytacazes''. Optamos por Guaitacazes todas as vezes em que ocorrer esta
divergência, pois, ainda que ambas sejam denominações de povos que viveram na costa do
Brasil, os Guaianazes habitavam terras mais ao sul.}

\paragraph{[44] Em que se trata de como Pedro de Góis foi povoar a sua capitania de Paraíba
ou de São Tomé} \quad
Pero de Góis\footnote{ Em Varnhagen (1851 e 1879), ``Pedro de Goes''.} foi um fidalgo
muito honrado, cavaleiro e experimentado, o qual andou na costa do Brasil com Pero Lopes
de Sousa, e se perdeu com ele no Rio da Prata; e pela afeição que tomou deste tempo à
terra do Brasil, pediu a el"-rei D. João, quando repartiu as capitanias da costa, que lhe
fizesse mercê de uma,\footnote{ Em Varnhagen (1851 e 1879), ``as capitanias, que lhe''.}
da qual Sua Alteza lhe fez mercê, dando"-lhe trinta léguas de terra ao longo da costa, que
se começariam onde se acabasse a capitania de Vasco Fernandes Coutinho, e daí até onde
acaba Martim Afonso de Sousa, e que, não as havendo entre uma capitania e outra, que lhe
dava somente o que houvesse, o que não passaria dos baixos dos Pargos. Da qual capitania
foi tomar posse em pessoa em uma frota de navios,\footnote{ Em Varnhagen (1851 e 1879),
``foi tomar posse em uma frota de navios''.} que à sua custa para isso fez, que proveu de
moradores, armas e o mais necessário para tal empresa, com a qual frota se partiu do porto
de Lisboa, e fez sua viagem com próspero tempo, e foi tomar terra e porto na sua
capitania, e desembarcou no rio Paraíba, onde se fortificou, e fez uma povoação em que
esteve pacificamente os primeiros dois anos em paz com os gentios Guaitacazes, seus
vizinhos,\footnote{ Em Varnhagen (1851 e 1879), ``os primeiros dois anos, com os gentios
Guaytacazes'', e no manuscrito da \textsc{bgjm}, ``os primeiros dois anos, em paz com os
gentios Guoainazes''.} com quem teve depois guerra cinco ou seis anos, dos quais se
defendeu com muito trabalho e risco de sua pessoa, por lhe armarem cada dia mil traições,
fazendo pazes, que lhe logo quebravam, com o que lhe foram matando muita gente, assim
nestas traições como em cercos, que lhe puseram, mui prolongados, com o que padeceu cruéis
fomes, o que não podendo os moradores sofrer apertaram com Pero de Góis rijamente, que a
despovoasse, no que ele se determinou obrigado destes requerimentos e das necessidades em
que o tinham posto os trabalhos, e ver que não era socorrido do reino como devera. E
vendo"-se já sem remédio, foi forçado a despejar a terra e passar"-se com toda a gente para
a capitania do Espírito Santo, onde estava a esse tempo Vasco Fernandes Coutinho, que lhe
mandou para isso algumas embarcações. E como Pero de Góes teve embarcação, se tornou para
estes reinos mui desbaratado, dos quais voltou a ir ao Brasil por capitão"-mor do mar com
Tomé de Sousa, que neste Estado foi o primeiro governador"-geral, com quem ajudou a povoar
e fortificar a cidade do Salvador, na Baía de Todos os Santos.\footnote{ Tomé de Sousa
(1493-1573?), militar e político português que se destacara pelos grandes feitos nas
Índias orientais e na África, foi nomeado por D. João \textsc{iii} primeiro Governador
Geral do Estado do Brasil, cargo criado em 1549 pela Coroa, no intuito de centralizar a
administração colonial.}

Nesta povoação que Pero Góes fez na sua capitania gastou toda a sua fazenda que tinha no
reino, e muitos mil cruzados de Martim Ferreira, que o favoreceu muito com pretensão de
fazerem por conta da companhia grandes engenhos, o que não houve efeito pelos respeitos
declarados neste capítulo.

\paragraph{[45] Em que se diz quem são os Guaitacazes e de sua vida e costumes} \quad
Pois que temos declarado quase toda a costa que senhoreavam os guaitacazes, não é bem que
nos despeçamos dela passando por eles,\footnote{ No manuscrito da \textsc{bgjm}, ``que
senhoreavam não é bem que''.} pois temos dito parte dos danos que fizeram aos povoadores
da Capitania do Espírito Santo e aos da Paraiba,\footnote{ Em Varnhagen (1851 e 1879),
``povoadores do Espírito Santo''.} os quais antigamente partiam pela costa do mar da banda
do sul com os Tamoios, e de norte com os Tapanazes,\footnote{ Em Varnhagen (1851 e 1879),
``Papanazes''.} que viviam entre eles, e os Tupiniquins, e como eram seus contrários,
vieram a ter com eles tão cruel guerra que os fizeram despejar a ribeira do mar, e irem"-se
para o sertão, com o que ficaram senhores da costa até confinar com os Tupiniquins, cujos
contrários também são, e se matam e comem uns aos outros, entre os quais estava por marco
o rio de Cricare.

Este gentio foi o que fez despovoar a Pero de Góes, e que deu tantos trabalhos a Vasco
Fernandes Coutinho. Este gentio tem a cor mais baça que os que dissemos atrás,\footnote{
Em Varnhagen (1851 e 1879), ``tem a cor mais branca que''.} e tem diferente linguagem, é
muito bárbaro, o qual não granjeia muita lavoura de mantimentos, plantam somente legumes,
de que se mantêm, e da caça que matam às flechadas, porque são grandes flecheiros. Não
costuma esta gente pelejar no mato, mas em campo descoberto, nem são muito amigos de comer
carne humana, como o gentio atrás; não dormem em redes, mas no chão, com folhas debaixo de
si. Costumavam estes bárbaros, por não terem outro remédio, andarem no mar nadando,
esperando os tubarões com um pau muito agudo na mão, e, em remetendo o tubarão a eles, lhe
davam com o pau, que lhe metiam pela garganta com tanta força que o afogavam, e matavam, e
o traziam à terra, não para o comerem para o que se não punham em tamanho perigo, senão
para lhes tirar os dentes, para os engastarem nas pontas das flechas. Tem este gentio
muita parte dos costumes dos Tupinambas, assim no cantar, bailar, tingir"-se de jenipapo,
na feição do cabelo da cabeça e no arrancar os mais cabelos do corpo e outras gentilidades
muitas que, por escusarem as prolixidades, as guardamos para se dizer uma só vez.

\paragraph{[46] Em que se declara, em suma, quem são os Papanases e seus costumes} \quad
Parece conveniente este lugar para brevemente se dizer quem são os Papanazes, de quem
atrás fizemos menção, e porque passamos o limite de sua vivenda nos tempos antigos, não é
bem que os guardemos para mais longe.

Este gentio, como fica dito, viveu ao longo do mar entre a capitania de Porto Seguro e a
do Espírito Santo, de onde foi lançado, pelos Tupiniquins, seus contrários, e pelos
goaitacazes, que também o eram, e são hoje, seus inimigos, e uns e outros lhe fizeram tão
cruel guerra que os fizeram sair para o sertão, onde agora têm sua vivenda, cuja linguagem
entendem os Tupiniquins e guoaitacazes, ainda que mal. Este gentio dorme no chão, sobre
folhas, como os guoaitacazes; também não se ocupa em grandes lavouras; mantêm"-se estes
selvagens de caça e peixe do rio, que matam; os quais são grandes flecheiros e pelejam com
arcos e flechas, andam nus como o mais gentio, não consentem cabelos nenhuns no corpo,
senão os da cabeça, pintam"-se e enfeitam"-se com penas de cores dos pássaros; cantam e
bailam; têm muitas gentilidades, das que usam os Tupinambas. Mas, entre si, têm um costume
que não é tão bárbaro, como todos os outros que todo o gentio costuma, que é, se um índio
destes mata outro da mesma geração em alguma briga, ou por desastre, são obrigados os
parentes do matador a entregá"-lo aos parentes do morto, que logo o afogam e o enterram,
estando uns e outros presentes, e todos neste ajuntamento fazem grande pranto, comendo e
bebendo todos juntos por muitos dias, e assim ficam todos amigos; e sendo o caso que o
matador fuja, de maneira que os parentes não o possam tomar, lhe tomam um filho ou filha,
se o tem, ou irmão, e se não tem nem um nem outro, entregam pelo matador o parente mais
chegado, ao qual não matam, mas fica cativo do mais próximo parente do morto, e com isso
ficam todos contentes e amigos como o eram antes do acontecimento do morto.

\paragraph{[47] Em que se torna a dizer de como corre a costa do cabo de São Tomé até o cabo
Frio} \quad
Do cabo de São Tomé à ilha de Santa Anna são oito léguas, e corre"-se a costa
nordeste"-sudoeste. A terra firme desta costa é muito fértil e boa. Esta ilha de Santa Anna
fica em vinte e dois graus e um terço, a qual está afastada da terra firme duas léguas
para o mar, e tem duas ilhas junto de si.\footnote{ Em Varnhagen (1851 e 1879), ``tem dois
ilhéus junto de si''.} E quem vem do mar em fora parece"-lhe tudo uma coisa. Tem esta ilha
da banda da costa um bom surgidouro e abrigada por ser limpo tudo, onde tem de fundo cinco
e seis braças; e na terra firme defronte da ilha tem boa aguada, e na mesma ilha há boa
água de uma lagoa. Por aqui não há de que guardar senão do que virem sobre a água.

E quem vem do mar em fora, para saber se está tanto avante como esta ilha, olhe para a
terra firme, e verá em meio das serras um pico, que parece frade com capelo sobre as
costas, o qual demora a oés"-noroeste, e podem os navios entrar por qualquer das bandas da
ilha como lhe mais servir o vento, e ancorar defronte entre ela e a terra firme.

Da ilha de Santa Ana à baía do Salvador são três léguas e dessa baía à baía Formosa são
sete léguas; da baía Formosa ao cabo Frio são duas léguas. E corre"-se a costa norte"-sul.
Até esta baía Formosa corriam os guoaitacazes no seu tempo, mas vivem já mais afastados do
mar, pelo que não há que arrecear para se povoar qualquer parte desta costa do Espírito
Santo até o cabo Frio.

\paragraph{[48] Em que se explicam os recôncavos do cabo Frio} \quad
O cabo Frio está em vinte três graus; o qual parece, a quem vem de mar em fora, ilha
redonda com uma forcada no meio, porque a terra, que está entre o cabo e as serras, é
muito baixa, e quando se vem chegando a ele aparece uma rocha com riscos brancos, por onde
é muito bom de conhecer. E, ainda que, pelo que se julga do mar, a terra do cabo parece
ilha, e o não seja, por onde o parece, na verdade o cabo é ilha, porque o corta o mar por
onde se não enxerga de fora, mas é de maneira que pode passar um navio por entre ele e a
terra firme à vontade. E tem um baixo neste canal, bem no meio, de duas braças de fundo; o
mais é alto, que basta para passar uma nau.

Perto do cabo estão umas ilhas, no meio das quais é limpo e bom porto para surgirem naus
de todo porte, e não há senão guardar do que virem. Duas léguas do cabo, da banda do
norte, está a baía Formosa, e defronte dela ficam as ilhas, e entre esta baía e as ilhas
há bom surgidouro. No fim desta baía para o norte está a Casa da Pedra, perto da qual está
um rio pequeno, que tem de fora bom surgidouro, e de dez até quinze braças de fundo,
afastado um pouco de uma ilha que está na boca da baía. E perto desta ilha é alto para
ancorarem naus, mas perigoso, porque se venta sudoeste e oeste, faz aqui dano no primeiro
ímpeto, porque vem com muita fúria, como trovoada de Guiné, a qual trovoada é de vento
seco e claro. Costumavam os franceses entrar por este rio pequeno e carregar o pau do
brasil, que traziam para as naus que estavam surtas\footnote{ Surto: ancorado, fundeado.}
na baía, ao longo das ilhas.\footnote{ Em Varnhagen (1851 e 1879), ``ao abrigo das
ilhas''.} Por esta baía entra a maré muito pela terra adentro, que é muito baixa, onde de
20 de janeiro até todo o fevereiro se coalha a água muito depressa, e sem haver marinhas,
tiram os índios o sal coalhado e duro, muito alvo, às mãos cheias, de debaixo da água,
chegando"-se sempre a maré, sem ficar nunca em seco.

\paragraph{[49] Em que se declara a terra que há do cabo Frio até o Rio de Janeiro} \quad
Do Cabo Frio ao Rio de Janeiro são dezoito léguas, que se repartem desta maneira: do Cabo
Frio ao rio de Sacorema são oito léguas; de Sacorema às ilhas de Maricá são quatro léguas,
e de Maricá ao Rio de Janeiro são seis léguas, cuja costa se corre leste"-oeste; o qual Rio
está em vinte e três graus, e tem sobre si umas serras mui altas, que se veem de muito
longe, vindo do mar em fora, a que chamam os Órgãos, e uma destas serras parece do mar
gávea de nau, por onde se conhece bem a terra. Este Rio tem de boca, de ponta a ponta,
perto de meia légua, e na de lés"-sudoeste tem um pico de pedra muito alto e mui a pique
sobre a barra. Na outra ponta tem outro padrasto, mas não é tão alto nem tão áspero, e de
um ao outro se defenderá a barra valorosamente. No meio desta barra, entre ponta e ponta,
criou a natureza uma lájea de cinquenta braças de comprido e vinte e cinco de largo, onde
se pode fazer uma fortaleza, que seja uma das melhores do mundo, o que se fará com pouca
despesa, com o que se defenderá este Rio a todo o poder que o quiser entrar; porque o
fundo da barra é, por junto dessa lájea, a tiro de espingarda dela, e forçado as naus que
quiserem entrar dentro hão de ir à fala dela, e não lhe ficará outro padrasto mais que o
do pico de pedra, donde lhe podem chegar com artilharia grossa; mas é este pico tão áspero
que parece impossível poder"-se levar acima artilharia grossa, e segurando"-se este pico
ficará a fortaleza da lájea inexpugnável. E uma coisa e outra se pode fortificar com pouca
despesa, pela muita pedra que para isso tem ao longo do mar, bem defronte, assim para
cantaria como para alvenaria, e grande aparelho para se fazer muita cal de ostras, de que
neste Rio há infinidade.

\paragraph{[50] Em que se declara a entrada do Rio de Janeiro e as ilhas que tem defronte} \quad
Defronte da barra do Rio de Janeiro, ao sul dela quatro ou cinco léguas, estão duas ilhas
baixas, e ao noroeste delas está um porto de areia bem chegado à terra, onde há abrigada
do vento sul, sudeste, leste e noroeste, e como for outro vento convém fugir na volta de
leste ou do norte, que serve para quem vem para o reino; e quem houver de ancorar aqui,
pode"-se chegar à terra até quatro ou cinco braças de fundo para ficar bem; e quem houver
de entrar no Rio, dando"-lhe o vento lugar, entre pela banda do leste, e sendo o vento
oeste, vá pela barra de oeste, pelo meio do canal que está entre a ponta de Cara de Cão e
a lájea; mas a barra de leste é melhor, por ser mais larga; e por cada uma delas tem fundo
oito até doze braças até a ilha de Viragalham;\footnote{ Ilha de Villegaignon, cujo nome é
em homenagem ao almirante francês Nicolas Durand de Villegagnon, que a ocupou em 1555 na
tentativa de estabelecer a França Antártica. Nela construiu o Forte Coligny, arrasado
pelos portugueses liderados por Mem de Sá em 1560.} e quanto mais forem a oeste, tanto
menos fundo acharão, depois que passarem a ilha, e para a banda de leste acharão mais
fundo em passando a ilha de Viragalham, que se chama assim, por ser este o nome do capitão
francês, que esteve com uma fortaleza nesta ilha, que é a que Mem de Sá tomou e arrasou.

Defronte da barra deste Rio ao mar dela está uma ilha, a que chamam ilha Redonda; e
afastado dela para a banda de leste, está outra ilha, a que chamam a ilha Rasa; e defronte
destas ilhetas, entre elas e a ponta da lájea, estão três ilhéus em meio, e chegando à
terra e a ponta da lájea está outra ilheta, a que chamam Jeribituba,\footnote{ Em
Varnhagen (1851 e 1879), ``ilha Raza: e defronte desta ilha e a ponta da lagoa estão três
ilhas no meio, e chegando à terra está outro ilhote, a que chamam Jeribátuba''.} em
derredor da qual estão quatro ilhotes.

\paragraph{[51] Em que particularmente se explica a baía do Rio de Janeiro da ponta do Pão
de Açúcar para dentro} \quad
É tamanha coisa o Rio de Janeiro da boca para dentro, que nos obriga a gastar o tempo em o
declarar neste lugar, para que se veja como é capaz de se fazer mais conta dele do que se
faz. E comecemos do Pão de Açúcar, que está da banda de fora da barra, que é um pico de
pedra mui alto, da feição do nome que tem, do qual, à ponta da barra, que se diz de Cara
de Cão, há pouco espaço; e a terra, que fica entre esta ponta e o Pão de Açúcar, é baixa e
chã; e virando"-se desta ponta para dentro da barra se chama Cidade Velha, onde se ela
fundou primeiro.\footnote{ A Cidade Velha, considerada o início da cidade do Rio de
Janeiro, foi fundada por Estácio de Sá, em 1º de março de 1565, e situava"-se entre o Pão
de Açúcar e a Ponta de São João, chamada naquela época de Cara de Cão. No dia primeiro de
março de 1567, o governador"-geral Mem de Sá transferiu a cidade da área da Urca para o
Morro do Castelo, por questões estratégicas, conservando"-lhe o nome de São Sebastião do
Rio de Janeiro.} Aqui se faz uma enseada, em que podem surgir navios, se quiserem, porque
o fundo é de vasa, e tem cinco, seis, até sete braças. Esta enseada se chama de Francisco
Velho, por ter aqui sua vivenda e granjearia, a qual é afeiçoada em compasso até outra
ponta adiante, que se chama da Carioqua, junto da qual entra uma ribeira, que se chama do
mesmo nome, donde bebe a cidade. Da ponta da Cara de Cão à cidade pode ser meia légua;
esta ponta de Cara de Cão fica quase em padrasto da lagoa,\footnote{ Em Varnhagen (1851 e
1879), ``padrasto da lájea''.} mas não é muito grande por ela não ser muito alta.

A cidade se chama São Sebastião, a qual edificou Mem de Sá, em um alto, em uma ponta de
terra que está defronte da ilha de Viragalham, a qual está lançada deste alto por uma
ladeira abaixo; e tem em cima, no alto, um nobre mosteiro e colégio de padres da
companhia, e ao pé dela está uma estância com artilharia para uma banda e para outra, um
modo de fortaleza em uma ponta, que defende o porto, mas não a barra, por lá não chegar
bem a artilharia.

Ao pé desta cidade, defronte da ponta do arrecife dela, tem bom surgidouro, que tem de
fundo cinco e seis braças, e chegando"-se mais à terra tem três e quatro braças, onde os
navios têm abrigo para os ventos gerais do inverno, que são sul e su"-sudeste. E quem
quiser ir para dentro há de passar por um banco, que tem de preamar até vinte palmos de
água; e passando este banco, virando para detrás da ponta da cidade, acharão bom fundo,
onde os navios estão seguros de todo tempo, por a terra fazer aqui uma enseada. E quando
os navios quiserem sair deste porto carregados, hão de botar fora por entre a ilha e a
ponta da terra firme, pela banda do norte, e hão de rodear a ilha em redondo para tornarem
a surgir defronte da cidade, e surgirão junto da ilha de Viragalham, entre ela e a cidade;
no qual lugar acharão de fundo três braças, e três e meia, onde tem porto morto e defronte
deste porto é o desembarcadouro da cidade, onde se diz as casas de Manoel de Brito.

\paragraph{[52] Em que se explica a terra da baía do Rio de Janeiro da ponta da cidade para
dentro até tornar à barra} \quad
Na ponta desta cidade o ancoradouro dos navios, que está detrás da cidade, está uma ilheta
que se diz a da Madeira,\footnote{ A antiga Ilha da Madeira hoje é conhecida como Ilha das
Cobras.} por se tirar dela muita, a qual serve aos navios que aqui se recolhem de
consertar as velas. E desta ponta a uma légua está outra ponta, fazendo a terra em meio
uma enseada, onde está o porto, que se diz de Martim Afonso, onde entra nesta baía um
riacho, que se diz Jabiburaçiqua;\footnote{ Em Varnhagen (1851 e 1879), ``Yabubiracica''.}
defronte deste porto de Martim Afonso estão espalhados seis ilhéus de arvoredo. E desta
ponta por diante se torna a terra a recolher, à maneira de enseada, e dali a meia légua
faz outra ponta, e antes dela entra outro riacho no salgado, que se chama Unhaúma; e à
ponta se chama Braço pequeno. Dessa ponta que se diz Braço pequeno por diante, foge a
terra para trás muito, onde se faz um esteiro, por onde entra a maré três léguas; e fica a
terra na boca deste esteiro, de ponta a ponta, um tiro de berço, donde começa a terra a
fazer outra enseada, que de ponta a ponta são duas léguas, a qual terra é alta até a
ponta. Defronte desta enseada está a ilha de Salvador Correa,\footnote{ Atual Ilha do
Governador, na cidade do Rio de Janeiro. Seu nome é em homenagem a Salvador Correa de Sá,
\textit{o Velho}, que a recebeu como sesmaria em 1567 e onde construiu um engenho de
açúcar. Militar português, ocupou por duas vezes o cargo de governador"-geral do Rio de
Janeiro entre 1578 a 1598, destacando"-se na luta pela expulsão dos franceses e na
administração do governo.} que se chama Pernapicu,\footnote{ Em Varnhagen (1851 e 1879),
``Parnápicú''.} que tem três léguas de comprido, e uma de largo, na qual está um engenho
de açúcar, que lavra com bois, que ele fez. Atravessando esta ilha por mar à cidade são
duas léguas, a qual ilha tem em redor de si oito ou nove ilhas, que dão pau do brasil. Do
cabo desta enseada grande e da ponta da terra alta, se faz outra enseada apertada na boca,
em a qual se mete um rio que nasce ao pé da serra dos Órgãos, que está cinco léguas pela
terra adentro, o qual se chama Magipe, e mais adiante légua e meia entra outro riacho
nesta baía, que se chama Sururuí. Deste rio Sururuí a duas léguas, entra outro nesta baía,
que se chama Macucu, que se navega pela terra adentro quatro léguas, em o qual se mete
outro rio, que se chama dos Guoaiatacazes, que vem de muito longe. Defronte do rio Macucu
está uma ilha, que se chama Curiata,\footnote{ Em Varnhagen (1851 e 1879), ``Caiaiba''.} e
desta ilha a uma légua está outra,\footnote{ Em Varnhagen (1879), ``desta ilha a uma está
outra''.} que se chama Pacata; e desta à de Salvador Correia é légua e meia; e estão estas
ilhas todas três em direito leste"-oeste umas das outras. E desta ilha Pacata direito ao
sul estão seis ilhéus, e para o sudeste estão cinco, em duas carreiras. Da ponta do rio
Macucu para a banda de leste se recolhe a terra e faz uma enseada até outra ponta da terra
saída ao mar, em que entra um riacho, que se chama Baxindiba, e da ponta deste riacho à de
Macucu é légua e meia. Defronte de Baxindiba, está outra ilha, cheia de arvoredo; de
Baxindiba se torna a afastar a terra para dentro, fazendo outra enseada, com muitos
mangues no meio, em a qual se mete outro rio, que se diz Suasunha,\footnote{ Em Varnhagen
(1851 e 1879), ``Suaçuna''.} e haverá de ponta a ponta duas léguas. E no meio, bem em
direito das pontas, está outra ilha cheia de arvoredo, e a outra ponta desta enseada se
diz Mutungabo. Da ponta de Mutungabo se esconde a terra para dentro bem dois terços de
légua, onde se mete um rio, que se chama Pau Doce, e faz uma volta, tornando a terra a
sair para fora bem meia légua, onde faz outra ponta, que se chama Urumaré. Desta ponta à
de Mutungabo é uma légua, e, bem em direito destas pontas, em meio desta enseada está
outra ilha de arvoredo. Desta ponta de Mutungabo à de Macucu são quatro léguas; da ponta
de Urumaré a dois terços de légua está outra ponta, onde se começam as barreiras
vermelhas, que ficam defronte da cidade, onde bate o mar da baía; e defronte desta ponta,
para o norte está uma ilha, que se diz de João Fernandes, diante da qual está outra mais
pequena. Das barreiras vermelhas se vai afeiçoando a terra ao longo da água, como cabeça
de cajado, onde se faz uma enseada que se chama de Piratininga, e a ponta e língua da
terra dele vêm quase em direito de Viragalham, a qual ponta se chama de Lery, e o cotovelo
desta língua de terra faz uma ponta defronte da de Cara de Cão, que fica em padrasto sobre
a lájea da barra, em a qual ponta está outra lájea, que o salgado aparta de terra qualquer
coisa, a qual fica ao pé do pico do padrasto, que está sobre a barra. Entram por esta
barra do Rio de Janeiro naus de todo o porte, as quais podem estar neste rio seguras, como
fica dito, de maneira que terá esta baía do Rio de Janeiro, em redondo, da ponta de Cara
de Cão, andando por dentro até o mar, à outra ponta da lájea, vinte léguas pouco mais ou
menos, que se navega em barcos, e pelo mais largo haverá de terra a terra seis léguas.

\paragraph{[53] Que trata de como o governador Mem de Sá foi ao Rio de Janeiro} \quad
Não é bem que passemos avante sem primeiro se dar conta da muita que os anos passados se
teve com o Rio de Janeiro. E como el"-rei D. João, o \textsc{iii} de Portugal, fosse
informado como os franceses tinham feito neste Rio uma fortaleza na ilha de Viragalham,
que foi o capitão que nela residia, que se assim chamava, mandou a D. Duarte da Costa, que
neste tempo era governador deste Estado, que ordenasse de espiar esta fortaleza e Barra do
Rio, o que D. Duarte fez com muita diligência,\footnote{ Em Varnhagen (1851 e 1879),
``neste tempo era governador deste Estado, que D. Duarte fez com muita diligência''.} e
avisou disso a Sua Alteza a tempo, que tinha eleito para governador"-geral deste Estado a
Mem de Sá, a quem encomendou particularmente muito que trabalhasse por lançar esta
ladroeira fora deste Rio. E falecendo el"-rei neste conflito, sucedeu no governo a rainha
D. Catarina, sua mulher, que está em glória, sabendo da vontade de Sua Alteza, escreveu ao
mesmo Mem de Sá, que com a brevidade possível fosse a este Rio e lançasse os franceses
dele, ao que, obedecendo o governador, fez prestes a armada, que do reino para isso lhe
fora, de que ia por capitão"-mor Bartolomeu de Vasconcelos; à qual ajuntou outros navios de
el"-rei, que na Bahia havia, e dez ou doze caravelões; e feita a frota prestes, mandou
embarcar nela as armas e munições de guerra e os mantimentos necessários, em a qual se
embarcou a maior parte da gente nobre da Bahia, e os homens de armas que se puderam
juntar, com muitos escravos e índios forros. E indo o governador com esta armada correndo
a costa, de todas as capitanias levou gente que por sua vontade o quiseram acompanhar
nesta empresa; e, seguindo sua viagem, chegou ao Rio de Janeiro com toda a armada junta,
onde o vieram ajudar muitos dos moradores de São Vicente, onde foi recebido da fortaleza
de Viragalham,\footnote{ Em Varnhagen (1851 e 1879), ``muitos moradores de São Vicente. E
foi recebido''.} que neste tempo era ido à França, com muitas bombardadas, o que não foi
bastante para Mem de Sá deixar de se chegar à fortaleza com os navios de maior porte a
varejar com artilharia grossa; e com os navios pequenos mandou desembarcar a gente em uma
ponta da ilha, onde mandou assestar artilharia, donde bateram a fortaleza rijamente. E
como os franceses se viram apertados, despejaram o castelo e fortaleza uma noite, e
lançaram"-se na terra firme com o gentio Tamoio, que os favorecia muito; e entrada a
fortaleza, mandou o governador recolher a artilharia e munições de guerra, que nela havia;
e mandou"-a desfazer e arrasar por terra, e avisou logo do sucedido à Rainha, numa nau
francesa, que neste Rio tomou, e como houve monção se recolheu o governador para a Bahia,
visitando as capitanias todas, aonde chegou a salvamento. Mas não alcançou esta vitória
tanto a seu salvo, que lhe não custasse primeiro a vida de muitos portugueses e índios
Tupinambas, que lhe os franceses mataram às bombardadas e espingardadas; mas, como a
Rainha soube desta vitória, e entendendo quanto convinha à Coroa de Portugal povoar"-se e
fortificar"-se o Rio de Janeiro, estranhou muito a Mem de Sá o arrasar a fortaleza que
tomou aos franceses, e não deixar gente nela que a guardasse e defendesse, para se povoar
este Rio, o que ele não fez por não ter gente que bastasse para poder defender esta
fortaleza; e que logo se fizesse prestes e fosse povoar este Rio, e o fortificasse,
edificando nele uma cidade, que se chamasse de São Sebastião; e para que isto pudesse
fazer com mais facilidade, lhe mandou uma armada de três galeões, de que ia por
capitão"-mor Cristóvão de Barros, com a qual, e com dois navios del"-rei que andavam na
costa, e outros seis caravelões, se partiu o governador da Bahia com muitos moradores
dela, que levavam muitos escravos consigo, e partiu"-se para o Rio de Janeiro, onde lhe
sucedeu o que neste capítulo se segue.

\paragraph{[54] Que trata de como Mem de Sá foi povoar o Rio de Janeiro} \quad
Partindo Mem de Sá para o Rio de Janeiro, foi visitando as capitanias dos Ilheos, Porto
Seguro e a do Espírito Santo, das quais levou muitos moradores, que como aventureiros os
foram acompanhando com seus escravos, nesta jornada; e como chegou ao Rio de Janeiro viu
que lhe havia de custar mais do que cuidava, como lhe custou; porque o achou fortificado
dos franceses na terra firme, onde tinham feito cercas mui grandes e fortes de madeira,
com seus baluartes e artilharia, que lhes umas naus que ali foram carregar de pau
deixaram, com muitas espingardas. Nestas cercas estavam recolhidos com os franceses os
índios Tamoios, que estavam já tão adestrados deles, que pelejavam muito bem com suas
espingardas, para o que não lhes faltava pólvora nem o necessário, por de tudo estarem bem
providos das naus acima ditas. Desembarcando o governador em terra, tiveram os portugueses
grandes escaramuças com os franceses e Tamoios; mas uns e outros se recolheram contra sua
vontade para as suas cercas, que logo foram cercadas e postas em grande aperto; mas,
primeiro que fossem entradas, custou a vida a Estácio de Sá, sobrinho do governador, e a
Gaspar Barbosa, pessoa de muito principal e de grande estima,\footnote{ Em Varnhagen (1851
e 1879), ``pessoa de muito principal estima''.} e a outros muitos homens e escravos, e,
contudo, foram as cercas entradas, e muitos dos contrários mortos e os mais cativos. E
como os Tamoios não tiveram entre si franceses, se recolheram pela terra adentro, donde
vinham muitas vezes fazer seus saltos, do que nunca saíram bem. E como Mem de Sá viu que
tinha lançado os inimigos da porta, ordenou de fortificar este Rio, fazendo"-lhe uma
estância ao longo da água para defender a barra, a qual depois reedificou Christóvão de
Barros, sendo capitão deste Rio; e assentou a cidade, que murou com muros de taipas com
suas torres,\footnote{ No manuscrito da \textsc{bgjm}, ``taipas com suas terras''.} em que
pôs artilharia necessária, onde edificou algumas igrejas, com sua casa de Misericórdia e
hospital, e um mosteiro de padres da companhia, que agora é colégio, em que os padres
ensinam latim, para o que lhe faz Sua Majestade mercê cada ano de dois mil
cruzados.\footnote{ Em Varnhagen (1851 e 1879), ``o que lhe faz S.A. mercê''.} E acabada
de fortificar e povoar essa cidade, ordenou o governador de se tornar para a Bahia,
deixando nela por capitão a seu sobrinho Salvador Correia de Sá, com muitos moradores e
oficiais de justiça e de fazenda, convenientes ao serviço del"-rei e ao bem da terra, o
qual Salvador Correia defendeu esta cidade alguns anos mui valorosamente, fazendo guerra
ao gentio, de que alcançou grandes vitórias, e dos franceses, que do Cabo Frio os vinham
ajudar e favorecer, aos quais foi tomar, dentro do Cabo Frio, uma nau que passava de
duzentos tonéis, com canoas que levou do Rio de Janeiro, com as quais a abalroou e tomou à
força de armas. A esta cidade mandou depois el"-rei D. Sebastião por capitão e governador
Cristóvão de Barros, que a acrescentou, fazendo nela em seu tempo muitos serviços a Sua
Alteza, que se não podem particularizar em tão pequeno espaço.

\paragraph{[55] Em que se trata de como foi governador do Rio de Janeiro Antônio Salema} \quad
Informado el"-rei D. Sebastião, que glória haja, do Rio de Janeiro, e do muito para que
estava disposto, ordenou de partir este Estado do Brasil em duas governanças, e deu uma
delas ao Dr. Antônio Salema,\footnote{ Jurista português, formado em Coimbra, que veio ao
Brasil como Desembargador da Casa de Suplicação no ano de 1570. Com a morte de Mem de Sá
em 1572, então governador"-geral, Salema foi nomeado pelo rei D. Sebastião governador das
terras do sul do Estado do Brasil, como explica Gabriel Soares, governando entre 1574 e
1577.} que estava na capitania de Pernambuco por mandado de Sua Alteza, com alçada, a
qual repartição se estendia da capitania de Porto Seguro até São Vicente. Esta repartição
se fez no ano de 1572; começava no limite em que partem as duas capitanias dos Ilheos e do
Porto Seguro, e dali tudo para o sul; e a outra, do dito limite até tudo que há para o
norte, deu a Luíz de Brito de Almeida. E era cabeça desta governança a cidade de São
Sebastião do Rio de Janeiro,\footnote{ No manuscrito da \textsc{bgjm}, ``da capitania de
Porto Seguro até São Vicente. E era cabeça desta''.} onde o governador assistiu, e começou
um engenho, que lhe Sua Alteza mandou fazer, para o que lhe mandou dar quatro mil
cruzados, o qual se não acabou, sendo mui necessário para os moradores fazerem suas
canas,\footnote{ Em Varnhagen (1851 e 1879), ``fazerem suas casas''.} e para a terra ir em
grande crescimento. No tempo que Antônio Salema governou o Rio de Janeiro, iam cada ano
naus francesas resgatar com o gentio ao Cabo Frio, onde ancoravam com suas naus na baía
que atrás fica declarado, e carregavam de pau da tinta à sua vontade; e vendo Antônio
Salema tamanho desaforo, determinou de tirar essa ladroeira desse lugar, e fez"-se prestes
para ir fazer guerra ao gentio de Cabo Frio, para o que ajuntou quatrocentos homens
brancos e setecentos índios, com os quais, por conselho de Christóvão de Barros, foram
ambos em pessoa ao Cabo Frio, que está dezoito léguas do Rio, onde acharam os Tamoios com
cercas muito fortes, recolhidos nelas com alguns franceses dentro, onde uns e outros se
defenderam valorosamente às espingardadas e flechadas; e, não podendo os franceses sofrer
o aperto em que estavam, se lançaram com o governador, que lhes desse a vida, com que os
Tamoios foram entrados, mortos infinitos, e cativos oito ou dez mil almas. E com essa
vitória que os portugueses alcançaram, ficaram os Tamoios tão atemorizados, que despejaram
a ribeira do mar, e se foram para o sertão, pelo que não tornaram mais naus francesas a
Cabo Frio a resgatar. E porque deste sucesso fez Antônio Salema um tratado, havemos por
escusado tratar mais deste caso neste capítulo.

\paragraph{[56] Em que se conclui com o Rio de Janeiro com a tornada de Salvador Correa a
ele} \quad
Vendo el"-rei D. Sebastião, que haja glória, o pouco de que lhe servira dividir o Estado do
Brasil em duas governanças, assentou de o tornar a ajuntar, como dantes andava, e o de
mandar por capitão e governador ao Rio de Janeiro somente a Salvador Correa de Sá, e que
viessem as apelações à Bahia, como dantes era; onde o dito Salvador Correa foi e está hoje
em dia, onde tem feito muitos serviços a Sua Majestade, do modo como procede na governança
e defensão desta cidade, e no fazer da guerra ao gentio, de que tem alcançado grandes
vitórias, e também serviu a Sua Majestade em pelejar contra naus francesas,\footnote{ Em
Varnhagen (1851 e 1879), ``pelejar com três naus francesas''.} que queriam entrar pela
barra do Rio de Janeiro, o que lhe defendeu às bombardadas, e não quis consentir que
comunicassem com a gente da terra, por se dizer trazerem cartas do senhor D. Antônio. E
foi esta cidade em tanto crescimento em seu tempo, que pela engrandecer ordenou de fazer
um engenho de açúcar na sua ilha, que faz muito açúcar; e favoreceu a Christóvão de Barros
para mandar fazer outro, que também está moente e corrente, com os quais esta cidade está
muito avante, e com um formoso colégio dos padres da Companhia, cujas obras Salvador
Correa ajudou e favoreceu muito. Neste Rio de Janeiro se podem fazer muitos engenhos por
ter terras e águas para isso, no qual se dão as vacas muito bem, e todo o gado de Espanha;
onde se dá trigo, cevada, vinho, marmelos, romãs, figos e todas as frutas de espinho; é
muito farto de pescado e marisco, e de todos os mantimentos que se dão na costa do Brasil;
onde há muito pau do Brasil, e muito bom.

\paragraph{[57] Em que se declara a costa do Rio de Janeiro até São Vicente} \quad
Da ponta da Cara de Cão do Rio de Janeiro à ponta do rio de Marambaia são nove léguas,
onde se faz uma enseada; e defronte desta enseada está uma ilha de Arvoredo, que se chama
a Ilha Grande, a qual faz de cada banda duas barras com a terra firme, porque tem em cada
boca um penedo no meio, que lhe faz duas abertas, e navega"-se por entre esta ilha e a
terra firme com navios grandes e naus de todo porte. Ao mar desta ilha está um ilhéu, que
se chama Jorge Grego.\footnote{ No manuscrito da \textsc{bgjm}, ``João Grego''. É a atual
Ilha de Jorge Grego, localizada na baía da Ilha Grande, no Estado do Rio de Janeiro.}
Esta Ilha Grande está em vinte e três graus, a qual tem sete ou oito léguas de comprido,
cuja terra é muito boa, toda cheia de arvoredo, com águas boas para engenhos. Quem vem do
mar em fora parece"-lhe esta ilha cabo de terra firme, por estar chegado à terra.

Esta ilha se deu de sesmaria a um desembargador que é falecido, e não a povoou, sendo ela
tanto para se fazer muita conta dela; em a qual há muito bom porto para surgirem navios.
Defronte desta ilha, na ponta dela da banda de oeste está a Angra dos Reis; e corre"-se
esta linha leste"-oeste; e quem navegar por entre ela e a terra firme não tem que recear,
porque tudo é limpo e sem baixo nenhum. Da ponta da Ilha Grande ao morro de Caruçu são
nove léguas, o qual morro está em vinte e três graus e um quarto e tem um ilhéu na ponta,
e entre ela e a Ilha Grande, na enseada junto à terra firme, tem duas ou três ilhetas de
arvoredo. Do morro de Curuçu à Ilha das Couves são quatro léguas, a qual está chegada à
terra; da Ilha das Couves ao porto dos Porcos são duas léguas, o qual porto é muito bom, e
tem defronte uma ilha do mesmo nome. Do Porto dos Porcos à ilha de São Sebastião são cinco
léguas, a qual está em vinte e quatro degraus, e tem cinco ou seis léguas de comprido,
cuja terra é boa para se poder povoar. E para boa navegação há de se navegar entre esta
ilha e a terra firme, mas acostar antes à banda da ilha, por ter mais fundo.

Ao sudoeste desta ilha está outra ilha, que se chama dos Alcatrazes, a qual tem três picos
de pedra, e um deles muito mais comprido que os outros. Por dentro desta ilha de São
Sebastião daí a três léguas ao sudoeste dela estão duas ilhetas; uma se diz da Vitória, e
a outra, dos Búzios. Da ilha de São Sebastião ao Monte do Trigo são quatro léguas; do
Monte do Trigo à barra de São Vicente são quatro léguas. E corre"-se esta costa da Ilha
Grande até São Vicente lés"-nordeste e oés"-sudoeste.

\paragraph{[58] Em que se declara quem é o gentio Tamoio de que tanto falamos} \quad
Ainda que pareça ser já fora do seu lugar tratar aqui do gentio Tamoio, não lhe cabia
outro, por a costa da terra que eles senhorearam passar além do Rio de Janeiro até Angra
dos Reis, pelo que se não podia dizer deles em outra parte mais acomodada. Estes Tamoios,
ao tempo que os portugueses descobriram esta província do Brasil, senhoreavam a costa dele
desde o rio do cabo de São Tomé até a Angra dos Reis; do qual limite foram lançados para o
sertão, onde agora vivem. Este gentio é grande de corpo e muito robusto, são valentes
homens e mui belicosos, e contrários de todo o gentio senão dos Tupinambas, de quem se
fazem parentes, cuja fala se parece muito uma com a outra, e têm as mesmas gentilidades,
vida e costumes, e são amigos uns dos outros. São estes Tamoios mui inimigos dos
guoaitacazes, de quem já falamos, com quem partem, segundo já fica dito, e cada dia se
matam e comem uns aos outros. Por esta outra parte de São Vicente partem com os
Goaynazes,\footnote{ No manuscrito da \textsc{bgjm}, ``Guoaitacazes''.} com quem também
têm contínua guerra, sem se perdoarem. Pelejam estes índios com arcos e flechas, no que
são mui destros e grandes caçadores e pescadores de linha, e grandes mergulhadores, e à
flecha matam também muito peixe, de que se aproveitavam quando não tinham anzóis. As suas
casas são mais fortes que as dos Tupinambas e do outro gentio, e têm as suas aldeias mui
fortificadas com grandes cercas de madeira. São havidos estes Tamoios por grandes músicos
entre todo o gentio e bailadores, os quais são grandes componedores de cantigas de
improviso, pelo que são mui estimados do gentio, por onde quer que vão. Trazem os beiços
debaixo furados e neles umas pontas de osso compridas com uma cabeça como prego, em que
metem esta ponta para que não caia, a qual cabeça lhe fica de dentro do beiço por onde a
metem.\footnote{ Em Varnhagen (1851 e 1879), ``os beiços furados'' e ``em que metem esta
ponta, e para que não caia a tal cabeça''.} Costumam mais nas suas festas enfeitarem"-se
com capas e carapuças de penas de cores de pássaros. Com este gentio tiveram grande
entrada os franceses, de quem foram bem recebidos no Cabo Frio e no Rio de Janeiro, onde
os deixaram fortificar e viver até que o governador Mem de Sá os foi lançar fora; e depois
Antônio Salema, no Cabo Frio. Nestes dois rios costumavam os franceses resgatar cada ano
mil quintais de pau do brasil, onde carregavam muitas naus, que traziam para França.

\paragraph{[59] Em que se declara a barra e povoações da capitania de São Vicente} \quad
Está o rio e barra de São Vicente em altura de vinte e quatro graus e meio, o qual rio tem
a boca grande e muito aberta, onde se diz a barra de Estêvão da Costa. E quem vem de mar
em fora para conhecer a barra, verá sobre ela uma ilha com um monte, da feição de moela de
galinha, com três mamilões. Por esta barra entram naus de todo o porte, as quais ficam
dentro do rio mui seguras de todo o tempo, pelo qual entra a maré cercando a terra de
maneira que fica em ilha muito chegada à terra firme, e faz este braço do rio muitas
voltas. Na ponta desta barra, da banda de leste, está a vila de Nossa Senhora da
Conceição;\footnote{ Nossa Senhora da Conceição de Itanhaém foi a segunda povoação fundada
por Martim Afonso, supostamente entre 1532 e 1534. Elevada à categoria de Vila em 1561,
deu origem à atual cidade litorânea paulista de Itanhaém.} e desta ponta à outra, que se
diz de Estêvão da Costa, se estende a barra de São Vicente; e entrando por este rio acima
está a terra toda povoada de uma banda e da outra de fazendas mui frescas; e antes que
cheguem à vila estão os engenhos dos Esquertes de Frandes e o de José Adorno; e no rio
está uma ilheta, além da qual, à mão direita, está a vila de São Vicente, que é cabeça
desta capitania. Pelo sertão desta capitania nove léguas está a vila de São Paulo, onde
geralmente se diz ``o campo'',\footnote{ Gabriel Soares de Sousa refere"-se, aqui, à Vila de
São Paulo do Campo de Piratininga, fundada em 1554 pelos jesuítas, que deu origem à atual
cidade de São Paulo.} em a qual vila está um mosteiro dos padres da companhia, e derredor
dela quatro ou cinco léguas estão quatro aldeias de índios forros cristãos, que os padres
doutrinam; e servem"-se desta vila para o mar pelo esteiro do Ramalho.\footnote{ Referência
ao colonizador português João Ramalho (1493-1580), que fundou uma povoação junto a sua
casa, a qual foi elevada à categoria de vila em 1553 com o nome de Santo André da Borda do
Campo. Não é possível afirmar que esta vila tenha dado origem às cidades de Santo André e
São Bernardo do Campo, pois a vila colonial foi esvaziada em 1560 por ordem do
governador, que determinou que sua população migrasse para São Paulo de Piratininga.}
Tem vila mais dois ou três engenhos de açúcar na ilha e terra firme; mas todos fazem pouco
açúcar, por não irem lá navios que o tragam. E aparta"-se esta capitania de São Vicente, de
Martim Afonso de Sousa, com a de Santo Amaro, de seu irmão Pero Lopes, pelo esteiro da
vila de Santos, donde se começa a capitania da vila de Santo Amaro.

\paragraph{[60] Em que se declara cuja é a capitania de São Vicente} \quad
Parece que é necessário, antes de passar mais adiante, declarar cuja é a capitania de São
Vicente, e quem foi o povoador dela, da qual fez el"-rei D. João, o \textsc{iii} de
Portugal, mercê a Martim Afonso de Sousa, cuja fidalguia e esforço é tão notório a todos,
que é escusado bulir, neste lugar, nisso, e os que dele não sabem muito vejam os livros da
Índia, e verão os feitos maravilhosos que nela acabou, sendo capitão"-mor do mar e depois
governador. Sendo este fidalgo mancebo, desejoso de cometer grandes empresas, aceitou esta
capitania com cinquenta léguas da costa, como as de que já fizemos menção, a qual
determinou de ir povoar em pessoa, para o que fez prestes uma frota de navios, que proveu
de mantimentos e munições de guerra como convinha; em a qual embarcou muitos moradores
casados que o acompanharam, com os quais se partiu do porto de Lisboa, donde começou a
fazer sua viagem, e com próspero tempo chegou a esta província do Brasil, e no cabo da sua
capitania tomou porto no rio que se agora chama de São Vicente, onde se fortificou e
assentou a primeira vila, que se diz do mesmo nome do rio, que fez cabeça da capitania. E
esta vila foi povoada de muita e honrada gente que nesta armada foi, a qual assentou em
uma ilha, donde lançou os guoaianases, que é o gentio que a possuía e senhoreava aquela
costa até conquistarem com os Tamoios;\footnote{ Em Varnhagen (1851 e 1879), ``costa até
contestarem com''.} a qual vila floresceu muito nestes primeiros anos, por ela ser a
primeira onde se fez açúcar na costa do Brasil, donde se as outras capitanias proveram de
canas"-de"-açúcar para plantarem, e de vacas para criarem e ainda agora floresce e tem em si
um honrado mosteiro de padres da companhia, e alguns engenhos de açúcar, como fica dito.
Com o gentio teve Martim Afonso pouco trabalho, por ser pouco belicoso e fácil de
contentar, e como fez pazes com ele, e acabou de fortificar a vila de São Vicente e a da
Conceição, se embarcou em certos navios que tinha, e foi correndo a costa descobrindo"-a, e
os rios dela até chegar ao Rio da Prata, pelo qual navegou muitos dias com muito trabalho,
aonde perdeu alguns dos navios pelos baixos do mesmo Rio, em que se lhe afogou alguma
gente, donde se tornou a recolher para a sua capitania, que acabou de fortificar como
pôde. E deixando nela quem a governasse e defendesse, se veio para Portugal chamado de Sua
Alteza, que se houve por servido dele naquelas partes, de onde mandou para as da Índia. E
depois de a governar se veio para estes reinos que também ajudou a governar com el"-rei D.
João, que o fez do seu Conselho de Estado; e o mesmo fez reinando el"-rei D. Sebastião, no
tempo em que governava a rainha D. Catarina, sua avó e depois o cardeal D. Henrique, para
o que tinha todas as partes convenientes. Nestes felizes anos de Martim Afonso favoreceu
muito esta sua capitania com navios e gente que a ela mandava, e deu ordem com que
mercadores poderosos fossem e mandassem a ela fazer engenhos de açúcar e grandes fazendas,
como tem até hoje em dia, do que já fizemos menção.

Tem este rio de São Vicente grande comodidade para se fortificar e defender, ao que é
necessário acudir com brevidade, por ser mui importante esta fortificação ao serviço de
Sua Majestade, porque, se se apoderarem desta terra os inimigos,\footnote{ Em Varnhagen
(1851 e 1879), ``apoderarem dela os inimigos''.} serão maus de lançar fora, pelo cômodo
que têm na mesma terra, para se fortificarem nela e defenderem de quem os quiser lançar
fora. Por morte de Martim Afonso herdou esta capitania seu filho primogênito, Pero Lopes
de Sousa,\footnote{ Em Varnhagen (1851 e 1879), ``seu filho, Pero Lopes de Sousa''.} por
cujo falecimento a herdou seu filho Lopo de Sousa.

\paragraph{[61] Em que se declara a capitania de Santo Amaro, e quem a povoou} \quad
Está tão mística \footnote{ Místico: próximo, contíguo, anexo.} a capitania de Santo Amaro
com a de São Vicente,\footnote{ Em Varnhagen (1851 e 1879), ``capitania de São Vicente com
a de Santo Amaro''.} que, se não foram de dois irmãos, amassaram"-se muito mal os moradores
delas, as quais iremos dividindo como pudermos.\footnote{ A palavra ``mística'' refere"-se,
portanto, à mistura ou junção das duas capitanias que, como Soares explica, por
pertencerem a dois irmãos, São Vicente a Martim Afonso de Sousa, e Santo Amaro a Pero Lopes
de Sousa, por vezes pareciam uma só. No início do século \textsc{xvii} Santo Amaro acabou
sendo absorvida, por falta de recursos naturais de importância e por não ter ligações com
o planalto, pela Capitania de São Vicente.} Indo pelo Rio de São Vicente acima, antes que
cheguem à ilha que nele está, à mão direita dele, está a boca do esteiro e perto da vila
de Santos, por onde entra a maré, cercando esta terra até se juntar com estoutro esteiro
de São Vicente; e entrando por este esteiro de Santos, à mão esquerda dele está situada a
vila do mesmo nome, a qual fica também em ilha cercada de água toda, e se navega com
barcos, e lhe dá jurisdição da capitania de Santo Amaro; e tornando à ponta de Estevão da
Costa, que está na boca da barra de São Vicente, dela a três léguas ao longo da costa está
a vila de Santo Amaro, junto da qual está o engenho de Francisco de Barros. De Santo Amaro
fez Pero Lopes de Sousa cabeça desta capitania. Desta vila de Santo Amaro à barra de
Britioga são duas léguas, onde está uma torre com artilharia e bombardeiros,\footnote{ Em
Varnhagen (1851 e 1879), ``barra de Bertioga são duas léguas, onde está um forte com
artilharia''.} que se chama de São Filipe, por esta barra entra a maré cercando esta terra
até se juntar com o esteiro de Santos, por onde fica Santo Amaro também em ilha, e avante
da ponta onde está esta fortaleza,\footnote{ Em Varnhagen (1851 e 1879), ``e da ponta
onde''.} estão no rio duas ilhetas. Defronte da fortaleza de São Filipe faz a terra firme
uma ponta muito chegada a estoutra, onde está outra torre com bombardeiros e artilharia,
que se diz de São Tiago, e por entre uma e outra podem entrar naus grandes por ter fundo
para isso, se destas fortalezas lho não impedirem; e passando avante destas torres pelo
esteiro acima da banda da terra firme estão os rios seguintes, que estão povoados com
engenhos e outras fazendas, os quais se vêm meter aqui no salgado: rio dos Lagartos, o
Piraquê, o de São João, o de São Miguel, o da Trindade, o das Cobras, o do engenho de
Paulo de Proença, o Rio dos Frades, onde está o engenho de Domingos Leitão, que é já da
capitania de São Vicente, o de Santo Antônio,\footnote{ Em Varnhagen (1851 e 1879), ``o de
Santo Amaro''.} o do engenho de Antônio do Vale, o de Manoel de Oliveira, concluindo, é
marco entre a capitania de São Vicente e a de Santo Amaro o esteiro de Santos.

Atrás fica dito como Pero Lopes de Sousa não quis tomar as cinquenta léguas de costa de
que lhe el"-rei fez mercê todas juntas, e de que tomou metade, com Tamaraqua e a outra em
Santo Amaro, de que agora tratamos. Esta capitania foi povoar em pessoa este fidalgo, e
fez para o poder fazer uma frota de navios em que se embarcou com muitos moradores, com os
quais partiu do porto de Lisboa e se foi à província do Brasil, por onde levava sua
derrota, e foi tomar porto no de São Vicente, donde se negociou e fez as povoações e
fortalezas acima ditas, no que passou grandes trabalhos e gastou muitos mil cruzados, a
qual agora possui uma sua neta, por não ficar dele herdeiro varão a quem ela com a de
Tamaraqua houvesse de vir.

\paragraph{[62] Em que se declara parte da fertilidade da terra de São Vicente} \quad
Nestas capitanias de São Vicente e Santo Amaro são os ares frios e temperados, como na
Espanha, cuja terra é mui sadia e de frescas e delgadas águas, em as quais se dá o açúcar
muito bom,\footnote{ Em Varnhagen (1851 e 1879), ``muito bem''.} e se dá trigo e cevada,
do que se não usa na terra por os mantimentos dela serem muito bons e facílimos de
granjear, de que os moradores são mui abastados e de muito pescado e marisco, onde se dão
tamanhas ostras que têm a casca maior que um palmo, e há algumas muito façanhosas. Do
trigo usam somente para fazerem hóstias e alguns mimos. Têm estas capitanias muita caça de
porcos e veados, e outras muitas alimárias e aves, e criam"-se aqui tantos porcos e
tamanhos, que os esfolam para fazerem botas e couros de cadeiras, o que acham os moradores
destas capitanias mais proveitosos e melhor que de couro das vacas, de que nestas
capitanias há muita quantidade por se na terra darem melhor que na Espanha, onde as carnes
são muito gordas e gostosas, e fazem vantagem às das outras capitanias, por a terra ser
mais fria.

Dão"-se nesta terra todas as frutas de espinho melhor que tem Espanha,\footnote{ Em
Varnhagen (1851 e 1879), ``de espinho que tem Espanha''.} às quais a formiga não faz nojo,
nem a outra coisa, por se não criar na terra como nas outras capitanias; dão"-se nestas
capitanias uvas, figos, romãs, maçãs e marmelos, em muita quantidade, e os moradores da
vila de São Paulo têm já muitas vinhas; e há homens nela que colhem já duas pipas de vinho
cada ano, e por causa das plantas é muito verde, e para se não avinagrar lhe dão uma
fervura no fogo; e também há já nesta terra algumas oliveiras, que dão fruto, e muitas
rosas, e os marmelos são tantos que os fazem de conserva, e tanta marmelada que a levam a
vender por as outras capitanias. E não há dúvida se não que há nestas capitanias outra
fruta melhor que é a prata, o que se não acaba de descobrir, por não ir à terra quem a
saiba tirar das minas e fundir.

\paragraph{[63] Que trata de quem são os guoianazes e de seus costumes} \quad
Já fica dito como os Tamoios são fronteiros de outro gentio, que se chamam os guoianazes,
os quais têm sua demarcação ao longo da costa por Angra dos Reis, e daí até o rio de
Cananea, onde ficam vizinhando com outra casta de gentio, que se chama os Carijos. Estes
guoianzes têm continuamente guerra com os Tamoios, de uma banda, e com os carijos da
outra, e matam"-se uns aos outros cruelmente; não são os guoaianazes maliciosos, nem
refalsados, antes simples e bem acondicionados, e facílimos de crer em qualquer coisa. É
gente de pouco trabalho, muito molar, não usam entre si lavoura, vivem da caça que matam e
peixe que tomam nos rios, e das frutas silvestres que o mato dá; são grandes flecheiros e
inimigos de carne humana. Não matam aos que cativam, mas aceitam"-nos por seus escravos; se
encontram com gente branca, não lhe fazem nenhum dano, antes boa companhia, e quem acerta
de ter algum escravo guoianas\footnote{ Em Varnhagen (1851 e 1879), ``Goayná''.} não
espera dele nenhum serviço, porque é gente folgazã de natureza e não sabe
trabalhar.\footnote{ No manuscrito da \textsc{bgjm}, ``guoianas, e não sabe trabalhar,
digo não espera dele nenhum serviço''.} Não costuma este gentio fazer guerra a seus
contrários, fora dos seus limites, nem os vão buscar nas suas vivendas, porque não sabem
pelejar entre o mato, senão no campo onde vivem, e se defendem com seus arcos e flechas
dos Tamoios, quando lhe vêm fazer guerra, com quem pelejam no campo mui valentemente às
flechadas, as quais sabem empregar tão bem como os seus contrários. Não vive este gentio
em aldeias com casas arrumadas, como os Tamoios seus vizinhos, mas em covas pelo campo,
debaixo do chão, onde têm fogo de noite e de dia e fazem suas camas de rama e peles de
alimárias que matam. A linguagem deste gentio é diferente da de seus vizinhos, mas
entendem"-se com os Carijos; são na cor e proporção do corpo como os Tamoios, e têm muitas
gentilidades, como o mais gentio da costa.

\paragraph{[64] Em que se declara a costa do rio de Santo Amaro até a Cananea} \quad
Atrás fica dito como se divide a capitania de São Vicente da de Santo Amaro pelo esteiro
de Santos, e como a vila de Santo Amaro é cabeça desta capitania, da qual o rio da Cananea
são vinte e cinco léguas ou trinta, antes do qual se acaba a capitania de Santo Amaro, e
corre"-se esta costa de Santo Amaro até a Cananea nordeste"-sudoeste, e toma da quarta do
leste"-oeste, a qual terra é toda boa para se poder aproveitar, e tem muitos riachos, que
se vêm meter no mar, entre os quais é um que está onze léguas, antes que cheguem a
Cananea, a qual faz na boca uma enseada, que tem uma ilha junto ao rio, que se diz a ilha
Branca. Este rio da Cananea está em vinte e cinco graus e meio, em o qual rio entram
navios da costa, e se navega por ele acima algumas léguas, e é muito capaz para se poder
povoar, e para se fazer muita conta dele, por ser mui abastado de pescado e marisco, e por
ter muita caça, cuja terra é muito fértil, na qual se dão muitos mantimentos dos naturais,
e se dará tudo o que lhe plantam, toda a criação de gado que lhe lançarem, por ter grande
cômodo para isso. Tem o rio da Cananea na boca uma abra\footnote{ Pequena enseada no mar
ou em rios, própria para abrigar embarcações.} grande, no meio da qual, bem defronte do
rio, tem uma ilha, e nesta abra está grande porto e abrigada para os navios, onde podem
estar seguras naus de todo o porte, porque tem fundo para isso.

\paragraph{[65] Em que se declara a costa da Cananea até o rio de São Francisco} \quad
Do rio da Cananea até o cabo do Padrão são cinco léguas, junto do qual está uma ilheta
chegada à terra e chama"-se este cabo do Padrão, por aqui se assentar um pelos primeiros
descobridores desta costa. Do cabo do Padrão ao rio de Santo Antônio são oito léguas, o
qual está em vinte e seis graus esforçados e dois terços.\footnote{ Em Varnhagen (1851 e
1879), ``está em vinte e seis esforçados e dois terços''.} Neste rio entram barcos da
costa à vontade. Do rio de Santo Antônio ao Alagado são cinco léguas, e entre um e outro
está uma ilheta chegada à terra.

Do rio Alagado ao de São Francisco são cinco léguas, o qual está em vinte seis graus e
dois terços e tem na boca três ilhéus. Neste rio entram navios da costa, onde estão
seguros de todo o tempo; chama"-se este rio de São Francisco, porque afirmam os povoadores
da capitania de São Vicente que se informaram do gentio de onde vinha este rio que entra
no mar desta costa, e que lhe afirmaram ser um braço do Pará, a que os portugueses chamam
de São Francisco, que é o que já dissemos, o que não parece possível, segundo o lugar onde
se vai meter no mar tão distante deste. Por este rio entra a maré muito, por onde se
navega barcos com barcos, no qual se metem muitas ribeiras. Este rio tem grandes pescarias
e muito marisco, e a terra ao longo tem muita caça, e grande cômodo para se poder povoar,
por ser muito fértil, e dará tudo o que lhe plantarem. A terra deste rio é alta e fragosa
e povoada de gentio Carijo.

Corre"-se esta costa da Cananea até o rio de São Francisco nordeste"-sudoeste, e todas estas
ilhas que estão por ela, e as que estão à boca do rio de São Francisco têm bom porte e
surgidouro para os navios ancorarem.

\paragraph{[66] Em que se declara a costa do rio de São Francisco até a de Jumirim ou
Itapucuru} \quad
Do rio de São Francisco ao dos Dragos são cinco léguas, pelo qual entram caravelões, e tem
na boca três ilhéus. Do rio dos Dragos à baía das Seis Ilhas são cinco léguas; e dessa
baía ao rio Itapucuru são quatro léguas, o qual está em vinte e oito graus escassos; e
corre"-se a costa do Itapucuru até o rio de São Francisco norte"-sul.

Este rio acima dito, a que outros chamam Jumirim, tem a boca grande e ao mar dele três
ilhetas, pela qual entram caravelões; e corre"-se por ele acima leste"-oeste, pelo qual
entra a maré muito, onde há boas pescarias e muito marisco. A terra deste rio é alta e
fragosa, e tem mais arvoredos que a terra atrás, especialmente águas vertentes ao mar. A
terra do sertão é de campinas, como a da Espanha, e uma e outra é muito fértil e abastada
de caça e muito acomodada para se poder povoar, porque se navega muito espaço por ela
acima. Este rio está povoado de Carijos, contrários dos guoainazes, de que falamos. Já
estes Carijos estão de paz com os portugueses, que vivem na capitania de São Vicente e
Santo Amaro, os quais vêm por mar resgatar com eles neste rio, onde se contratam, sem
entre uns nem outros haver desavença nenhuma.\footnote{ Em Varnhagen (1851 e 1879), ``sem
entre uns e outros haver desavença alguma''.}

\paragraph{[67] Em que se declara a terra que há de Itapicuru até o rio dos Patos} \quad
Do rio Itapicuru\footnote{ Em Varnhagen (1851 e 1879), ``rio de Itapucuru''.} até o rio
dos Patos são quatro léguas, o qual está em vinte e oito graus. Este rio é muito grande,
cuja boca se serra com a ilha de Santa Catarina, por onde entram os navios da costa, e a
maré muito espaço, por onde se navega. Metem"-se neste rio muitas ribeiras que vêm do
sertão; o qual é muito acomodado para se poder povoar, por a terra ser muito fértil para
tudo que lhe plantarem, a qual tem muita caça de veados, de porcos e de muitas aves, e o
rio é mui provido de marisco, e tem grandes pescarias até onde possuem a terra os Carijos,
daqui por diante é a vivenda dos Tapuyas, e está por marco entre uns e outros este rio dos
Patos.

À boca deste rio está situada a ilha de Santa Catarina, que vai fazendo abrigo à terra até
junto de Itapucuru, que fica à maneira de enseada. Tem esta ilha de comprido oito léguas,
e corre"-se norte"-sul, a qual da banda do mar não tem nenhum surgidouro, salvo um ilhéu,
que está na ponta do sul, e outro que tem na ponta do norte; a qual ilha é coberta de
grande arvoredo, e tem muitas ribeiras de água dentro e tem grande comodidade para se
poder povoar, por ser a terra grossa muito boa e ter grandes portos, em que se podem estar
seguras de todo tempo muitas naus. Mostra esta ilha uma baía grande, que vai por detrás,
entre ela e a terra firme, onde há grande surgidouro e abrigada para naus de todo porte;
nesta enseada que se faz da ilha para terra firme estão muitas ilhetas; está esta boca e
ponta da ilha da banda do norte em vinte e oito graus de altura.

\paragraph{[68] Em que se declara parte dos costumes dos Carijos} \quad
Atrás fica dito como os Carijos são contrários dos guoainazes, e como se matam uns aos
outros. Agora cabe aqui dizer deles o que se pode alcançar e saber de sua vida e costumes.
Este gentio possui esta costa deste rio da Cananea, onde parte, com os guoianazes, na qual
se fazem uns aos outros mui contínua e cruel guerra, pelejando com arcos e flechas, que os
Carijos sabem tão bem manejar como seus vizinhos e contrários. Este gentio é doméstico,
pouco belicoso, de boa razão; segundo seu costume, não come carne humana, nem mata homens
brancos que com eles vão resgatar, sustentam"-se da caça e do peixe que matam, e de suas
lavouras que fazem, onde plantam mandioca e legumes como os Tamoios e Tupiniquins. Vivem
estes índios em casas bem cobertas e tapadas com cascas de árvores, por amor do frio que
há naquelas partes. Esta gente é de bom corpo, cuja linguagem é diferente da de seus
vizinhos, fazem suas brigas com contrários em campo descoberto, especialmente com os
guoainazes, com quem têm suas entradas de guerra; e como os desbaratados se acolhem ao
mato se têm por seguros, porque nem uns nem outros sabem pelejar por entre ele. Costuma
este gentio no inverno lançar sobre si umas peles da caça que matam, uma por diante, outra
por detrás; têm mais muitas gentilidades, manhas e costumes como os Tupinambas, em cujo
título se contam mui particularmente.

\paragraph{[69] Em que se declara a costa do rio dos Patos até o da Alagoa} \quad
Do rio dos Patos ao rio de D. Rodrigo\footnote{ Levava este nome por se ter ali abrigado
D. Rodrigo D’Acuña, após seu naufrágio nesse litoral. É o atual Porto de Imbituba,
localizado em uma enseada aberta no município homônimo, no litoral sul do Estado de Santa
Catarina.} são oito léguas; e corre"-se a costa norte"-sul, até onde a terra é algum tanto
alta, o qual porto está em vinte e oito graus e um quarto. Esse porto está no cabo da ilha
de Santa Catarina, o qual está em uma baía que a terra faz para dentro, onde há grande
abrigada e surgidouro para os navios estarem seguros de todos os ventos, tirando o
nordeste, que cursa no verão e venta igual, com o qual se não encrespa o mar. Do porto de
D. Rodrigo ao porto e rio da Lagoa, são treze léguas, o qual nome tomou por o porto ser
uma calheta grande e redonda e fechada na boca, que parece a lagoa, onde também entram
navios da costa e estão mui seguros. Do rio dos Patos até aqui é esta terra à vista do mar
sem mato, mas está vestida de erva verde, como a de Espanha, onde se dão muito bem todos
os frutos que lhe plantam; em a qual se dará maravilhosamente a criação das vacas e todo o
mais gado que lhe lançarem, por ser a terra fria e ter muitas águas para o gado beber.
Esta terra é possuída dos Tapuyas, ainda que vivem algum tanto afastados do mar, por ser a
terra desabrigada dos ventos; mas o porto de D. Rodrigo é suficiente para se poder povoar,
pela fertilidade da terra e pela comodidade que tem ao longo do mar de pescarias e muito
marisco e por a terra ter muita caça. E o Porto da Alagoa, com que concluímos este
capítulo, tem um ilhéu junto da boca da barra.

\paragraph{[70] Em que se declara a costa do porto da Alagoa até o rio de Martim Afonso} \quad
Do porto da Alagoa ao porto e rio de Martim Afonso são vinte e duas léguas, as quais se
correm pela costa nordeste"-sudoeste e toma da quarta de norte"-sul. Este rio está em trinta
graus e um quarto; e chama"-se de Martim Afonso de Sousa\footnote{ Em Varnhagen (1851 e
1879), ``Martim Afonso''.} por ele o descobrir, quando andou correndo esta costa de São
Vicente até o rio da Prata. Este rio tem muito bom porto de fora para navios grandes e
dentro para os da costa, cuja terra é baixa e da qualidade da de trás. Tem este rio duas
léguas ao mar uma ilha onde há bom porto e abrigada para surgirem navios de todo o porto;
entra a maré por este rio muito, onde há muito marisco, cuja terra é de campinas que estão
sempre cheias de erva verde com algumas reboleiras\footnote{ Reboleira: capão, moita,
touceira.} de mato, onde se dará tudo o que lhe plantarem, e se criará todo o gado que lhe
lançarem; por ser terra fria e ter muitas águas de alagoas, e ribeiras para o gado poder
beber, pelo que este rio se pode povoar, onde os moradores que nele vai viver estarão mui
descansados, o qual é povoado de Tapuyas como a mais terra atrás. Entre o porto da Alagoa
e de Martim Afonso está o porto que se diz de Santa Maria e o que se diz da Terra
Alta,\footnote{ No manuscrito da \textsc{bgjm}, ``está o porto que se diz da Terra
Alta''.} e em um e em outro podem surgir os caravelões da costa.

\paragraph{[71] Em que se declara a costa do rio de Martim Afonso até o porto de São Pedro} \quad
Do rio de Martim Afonso à baía dos arrecifes são dez léguas e da baía ao rio do porto de
São Pedro são quinze léguas, o qual rio está em altura de trinta e um graus e meio, cuja
costa se corre nordeste"-sudoeste; da banda do sudoeste deste porto de São Pedro se faz uma
ponta de areia, que boja ao mar légua e meia.\footnote{ No manuscrito da \textsc{bgjm},
``nordeste"-sudoeste; deste porto de São Paulo''.} Neste porto há um bom surgidouro e
abrigada para os navios entrarem seguros sobre amarra, em o qual se vem meter no salgado
um rio de água doce.

Esta terra é muito baixa e não se vê de mar em fora senão de muito perto, e toda é de
campos cobertos de erva verde, muito boa para mantença de criação de gado vacum e de toda
a sorte, por onde há muitas lagoas e ribeiras de água para o gado beber. E tem esta terra
algumas reboleiras de mato à vista umas das outras, onde há muita caça de veados e porcos
que andam em bandos, e muitas outras alimárias e aves, e ao longo da costa há grandes
pescarias e sítios acomodados para povoações com seus portos, onde entram caravelões, em a
qual se darão todos os frutos que lhe plantarem, assim naturais como de Espanha; e dos
mantimentos de terra se aproveita o gentio Tapuya,\footnote{ No manuscrito da
\textsc{bgjm}, ``caravelões, em a qual terra se aproveita o gentio''.} em suas roças e
lavouras, que fazem afastadas do mar três ou quatro léguas, por estarem lá mais abrigados
dos ventos do mar, que cursam no inverno, onde ao longo dele não têm nenhum abrigo, o
porque lhe fica a lenha muito longe.

\paragraph{[72] Em que se conta como corre a costa do rio de São Pedro até o cabo de Santa
Maria} \quad
Do porto de São Pedro ao cabo de Santa Maria são quarenta e duas léguas, as quais se
correm pela costa nordeste"-sudoeste, o qual está em trinta e quatro graus; e tem, da banda
do sudeste, duas léguas ao mar, três ilhéus altos, que se dizem os Castilhos, entre os
quais e a terra firme há boa abrigada e surgidouro para naus de todo o porte.

Toda esta terra é baixa, sem arvoredo; mas cheia de erva verde em todo o ano, há partes
que têm algumas reboleiras de mato; a erva destes campos é muito boa para criação de gado
de toda sorte, onde se dará muito bem por ser a terra muito temperada no inverno, e no
verão lavada de bons ares frescos e sadios, pela qual há muitas águas frescas para os
gados beberem, assim de lagoas como de ribeira, onde se darão todos os frutos de Espanha
muito bem, como em São Vicente, e pelo Rio da Prata acima das povoações de castelhanos,
onde se dá tanto trigo, que aconteceu o ano de 83 vir ao Rio de Janeiro uma das naus em
que passou D. Alonso, vice"-rei da província de Chile, que desembarcou em Buenos Aires, a
qual carregou neste porto de trigo, que se vendeu no Rio de Janeiro a três reales a
fanega,\footnote{ Fanega: medida de cereais equivalente a 100 quilogramas.} o qual se dará
muito bem do Rio de Janeiro por diante, donde se pode prover toda a costa do Brasil.

Esta costa desde o rio dos Patos até a boca do rio da Prata é povoada de Tapuyas, gente
doméstica e bem acondicionada, que não come carne humana nem faz mal à gente branca que os
comunica, como são os moradores da capitania de São Vicente, que vão em caravelões
resgatar por esta costa com este gentio alguns escravos, cera da terra, porcos, galinhas e
outras coisas, com quem não têm nunca desavenças; e porque a terra é muito rasa e
descoberta aos ventos, e não tem matos nem abrigadas, não vivem estes Tapuyas ao longo do
mar e têm suas povoações afastadas para o sertão, ao abrigo da terra, e vêm pescar e
mariscar pela costa.

Não tratamos aqui da vida e costumes deste gentio, porque se declara ao diante do título
dos Tapuyas, que vivem no sertão da Bahia, e ainda que vivam tão afastados destes, são
todos uns e têm quase uma vida e costumes.

\paragraph{[73] Em que se declara a costa do cabo de Santa Maria até a boca do rio da Prata} \quad
Do cabo de Santa Maria à ilha dos Lobos são quinze léguas, cuja costa se corre
nor"-nordeste su"-sudoeste, a qual está em trinta e quatro graus e dois terços, cuja terra
firme faz defronte da ilha, à maneira de ponta. Entre esta ponta e a ilha há boa abrigada
e porto para navio. Desta ponta se vai recolhendo a terra para dentro até outra ponta, que
esta outra ilha, que se diz a das Flores, que está légua e meia afastada desta ponta, que
se chama do Arrecife, pelo haver daí para dentro até o Monte de Santo Ovidio, está na boca
de um rio que se vem meter aqui no salgado.

Desta ponta da ilha dos Lobos, que está na boca do rio da Prata, à outra banda do rio, que
se diz a ponta de Santo Antônio, são trinta e quatro léguas. Está o meio da boca do rio da
Prata em trinta e cinco graus e dois terços; e ao mar quarenta léguas, bem em direito
desta boca do rio está um ilhéu, cercado de baixos em redor dele, obra de duas léguas,
onde se chama os baixos dos Castelhanos, porque aqui se perdeu uma nau sua, o qual ilhéu
está na mesma altura de trinta e cinco graus e dois terços.

A terra junto da boca deste rio é da qualidade da outra terra do cabo de Santa Maria, onde
se também dará grandemente o gado vacum e tudo o mais que lhe lançarem.

Deste rio da Prata, nem de sua grandeza não temos que dizer neste lugar, porque é tão
nomeado que se não pode tratar dele sem grandes informações, do muito que se pode dizer
dos seus recôncavos, ilhas, rios que nele se metem, fertilidades da terra e povoações que
por ele acima têm feito os castelhanos que escaparam da armada que nele se perdeu em
muitos anos, os quais se casaram com as índias da terra, de que nasceram grande multidão
de mestiços, que agora têm povoado muitos lugares, o qual rio da Prata é povoado muitas
léguas por ele acima dos Tapuyas atrás declarados.\footnote{ No manuscrito da
\textsc{bgjm}, ``é povoado muitos lugares, digo léguas por''.}

\paragraph{[74] Em que se declara a terra e costa da ponta do rio da Prata da banda do sul
até além da Baía de São Matias} \quad
A ponta do rio da Prata que se diz de Santo Antônio, que está da banda do sul, demora em
trinta e seis graus e meio, defronte da qual são baixos uma légua ao mar. Da ponta de
Santo Antônio ao Cabo Branco são vinte e duas léguas e fica"-lhe em meio uma enseada, que
se diz de Santa Apolonia, a qual é cheia de baixos, e toda a costa de ponta a ponta, uma e
duas léguas ao mar, são tudo baixos. Este Cabo Branco está em trinta e sete graus e dois
terços, e corre"-se a costa nordeste su"-sudoeste. Do Cabo Branco ao Cabo das Correntes são
vinte e cinco léguas, e fica entre um cabo e o outro a Angra das Areias, ao mar da qual
sete ou oito léguas são tudo baixos. Este Cabo está em trinta e seis graus,\footnote{ Em
Varnhagen (1851 e 1879), ``trinta e nove graus''.} cuja costa se corre nor"-nordeste
su"-sudoeste.\footnote{ No manuscrito da \textsc{bgjm}, ``se corre nor"-nordeste''.} Do
Cabo das Correntes ao Cabo Aparcelado são oitenta e seis léguas, e corre"-se a costa de
ponta a ponta lés"-nordeste e oés"-sudoeste, o qual Cabo Aparcelado está em quarenta e um
graus, cuja costa é cheia de baixos, e apartes os tem cinco e seis léguas ao
mar;\footnote{ No manuscrito da \textsc{bgjm}, ``apartes os tem como seis léguas''.} é
toda de areia e a terra muito baixa, por onde se metem alguns esteiros no salgado, onde se
podem recolher caravelões da costa, que são navios de uma só coberta que andam em seis e
sete palmos de água. Deste Cabo Aparcelado se torna a recolher a terra para dentro
leste"-oeste, até a ponta da Baía de São Matias, que está na mesma altura de quarenta e um
graus, que serão vinte e sete léguas, e da ponta Aparcelada a quatro léguas em uma enseada
que faz a terra, está uma ilheta, e na ponta desta enseada, da banda de leste, está outra
ilha, uma légua do mar.

Da ponta da Baía de São Matias até a ponta de terra do Marco são trinta e oito léguas,
cuja costa se corre norte"-sul, a qual é toda aparcelada, e antes de se chegar a esta ponta
do Marco está outra ilha. A terra aqui é baixa e pouco proveitosa. Nesta ponta de Marco se
acaba a demarcação da Coroa de Portugal nesta costa do Brasil, que está em quarenta e
quatro graus pouco mais ou menos, segundo a opinião do Dr. Pero Nunes, cosmógrafo del"-rei
D. Sebastião, que está em glória, que nesta arte foi em seu tempo o maior homem de
Espanha.

\end{linenumbers}

\chapter[Segunda parte: Memorial e declaração]{Segunda parte \subtitulo{Memorial e
declaração das grandezas\break da Bahia de Todos os Santos, de sua fertilidade e das notáveis
partes que tem}}
\hedramarkboth{Memorial e declaraçãos}{}

\begin{linenumbers}

\paragraph{[1] Armada de Tomé de Sousa} \quad
Atrás fica dito, passando pela Bahia de Todos os Santos, que se não sofria naquele lugar
tratar"-se das grandezas dela, pois não cabiam ali, o que se faria ao diante mui
largamente,\footnote{ No manuscrito da \textsc{bgjm}, ``ao diante largamente''.} depois
que se acabasse de correr a costa, com que temos já concluído. Da qual podemos agora
tratar e explicar o que se dela não sabe para que venham à notícia de todos os ocultos
desta ilustre terra, por cujos merecimentos deve de ser mais estimada e reverenciada do
que agora é, ao que queremos satisfazer com singelo estilo, pois o não temos grave, mas
fundado tudo na verdade.

Como el"-rei D. João, o \textsc{iii} de Portugal, soube da morte de Francisco Pereira
Coutinho, sabendo já das grandes partes da Bahia, da fertilidade da terra, bons ares,
maravilhosas águas e da bondade dos mantimentos dela, ordenou de a tomar à sua conta para
a fazer povoar, como meio e coração de toda esta costa, e mandar edificar nela uma cidade,
de onde pudesse ajudar e socorrer todas as mais capitanias e povoações dela como a membros
seus; e pondo Sua Alteza em efeito esta determinação tão acertada, mandou fazer prestes
uma armada e provê"-la de todo o necessário para esta empresa, em a qual mandou embarcar
Tomé de Sousa do seu conselho, que elegeu para edificar esta nova cidade, de que o fez
capitão e governador"-geral de todo o Estado do Brasil; ao qual deu grande alçada e poderes
em seu regimento, com que quebrou as doações aos capitães proprietários, por terem
demasiada alçada, assim no crime como no cível; de que eles agravaram à Sua Alteza, que no
caso os não proveu, entendendo convir a si e a seu serviço. E como a dita armada estivesse
prestes, partiu Tomé de Sousa do porto de Lisboa aos dois dias de fevereiro de 1549 anos;
e levando próspero vento chegou à Bahia de Todos os Santos, para onde levava sua derrota,
aos vinte e nove dias de março do dito ano, e desembarcou no porto de Vila
Velha,\footnote{ No manuscrito da \textsc{bgjm}, ``1549 anos, e elevando rota aos vinte e
nove dias de março do dito ano, e desembarcou''.} povoação que Francisco Pereira edificou,
onde pôs mil homens, convém a saber: seiscentos soldados e quatrocentos degradados e
alguns moradores casados, que consigo levou, e outros criados del"-rei, que iam providos de
cargos, que pelo tempo em diante serviram.

\paragraph{[2] Em que se contém quem foi Tomé de Sousa e de suas qualidades} \quad
Tomé de Sousa foi um fidalgo honrado, ainda que bastardo, homem avisado, prudente e mui
experimentado na guerra de África e da Índia, onde se mostrou mui valoroso cavaleiro em
todos os encontros em que se achou; pelos quais serviços e grande experiência que tinha,
mereceu fiar dele el"-rei tamanha empresa como esta que lhe encarregou, confiando de seus
merecimentos e grandes qualidades que daria a conta dela que se dele esperava; a quem deu
por ajudadores ao Dr. Pero Borges,\footnote{ Em Varnhagen (1851 e 1879), ``Pedro
Borges''.} para com ele servir de ouvidor"-geral e pôr o governo da justiça em ordem em
todas as capitanias; e a Antônio Cardoso de Barros para também ordenar neste Estado o
tocante à Fazenda de Sua Alteza, porque até então não havia ordem em uma coisa nem noutra,
e cada um vivia ao som da sua vontade. O qual Tomé de Sousa também levou em sua companhia
padres da Companhia de Jesus, para doutrinarem e converterem o gentio na nossa santa fé
católica, e a outros sacerdotes, para ministrarem os sacramentos nos tempos devidos. E no
tempo que Tomé de Sousa desembarcou, achou na Vila Velha a um Diogo Álvares, de alcunha o
Caramuru, grande língua do gentio, o qual, depois da morte de Francisco Pereira, fez pazes
com o gentio; e, com elas feitas, se veio dos Ilheos a povoar o assento das casas em que
dantes vivia, que era afastado da povoação, em o qual se fortificou e recolheu com cinco
genros que tinha, e outros homens que o acompanharam, dos que escaparam da desventura de
Francisco Pereira, com os quais, ora com armas, ora com boas razões, se foram defendendo e
sustentando até a chegada de Tomé de Sousa, por cujo mandado Diogo Álvares quietou o
gentio e o fez dar obediência ao governador, e oferecer"-se ao servir; o qual gentio em seu
tempo viveu muito quieto e recolhido, andando ordinariamente trabalhando na fortificação
da cidade a troco do resgate que lhe por isso davam.

\paragraph{[3] Em que se declara como se edificou a cidade do Salvador} \quad
Como Tomé de Sousa acabou de desembarcar a gente da armada e a assentou na Vila Velha,
mandou descobrir a baía, e que lhe buscassem mais para dentro alguma abrigada melhor que a
em que estava a armada para a tirarem daquele porto da Vila Velha, onde não estava segura,
por ser muito desabrigado; e por se achar logo o porto e ancoradouro, que agora está
defronte da cidade, mandou passar a frota para lá, por ser muito limpo e abrigado; e como
teve a armada segura, mandou descobrir a terra bem, e achou que defronte do mesmo porto
era o melhor sítio que por ali havia para edificar a cidade, e por respeito do porto
assentou que não convinha fortificar"-se no porto de Vila Velha, por defronte deste porto
estar uma grande fonte, bem à borda do mar que servia para aguada dos navios e serviço da
cidade,\footnote{ Em Varnhagen (1851 e 1879), ``à borda da água que''.} o que pareceu bem
a todas as pessoas do conselho que nisso assinaram. E tomada esta resolução, se pôs em
ordem para este edifício, fazendo primeiro uma cerca muito forte de pau a pique, para os
trabalhadores e soldados poderem estar seguros do gentio. Como foi acabada, arrumou a
cidade dela para dentro, arruando"-a\footnote{ Arruar: projetar ou construir ruas,
caminhos.} por boa ordem com as casas cobertas de palma, ao modo do gentio, em 
as quais por entretanto se agasalharam os moradores e soldados que vieram na
armada.\footnote{ Em Varnhagen (1851 e 1879), ``agasalharam os mancebos e soldados''.} E
como todos foram agasalhados, ordenou de cercar esta cidade de muros de taipa grossa, o
que fez com muita brevidade, com dois baluartes ao longo do mar e quatro da banda da
terra, em cada um deles assentou muito formosa artilharia que para isso levava, com o que
a cidade ficou muito bem fortificada para se segurarem do gentio; em a qual o governador
fundou logo um colégio dos padres da Companhia, e outras igrejas e grandes casas, para
viverem os governadores, casa da câmara, cadeia, alfândega, contos, fazendas, armazéns, e
outras oficinas convenientes ao serviço de Sua Alteza.

\paragraph{[4] Em que se contém como el"-rei mandou outra armada em favor de Tomé de Sousa} \quad
Logo no ano seguinte, de 1550, se ordenou outra armada, com gente e mantimentos, em
socorro desta nova cidade, da qual foi por capitão Simão da Gama de Andrade, com o galeão
velho muito afamado e outros navios marchantes, em a qual foi o bispo D. Pero Fernandes
Sardinha,\footnote{ Em Varnhagen (1851 e 1879), ``Pedro Fernandes Sardinha''. D. Pero
Fernandes (1496-1556) foi ordenado primeiro bispo do Brasil em 22 de junho de 1552.}
pessoa de muita autoridade, grande exemplo e extremado pregador, o qual levou toda a
clerezia, ornamentos, sinos, peças de prata e outras alfaias do serviço da Igreja, e todo
o mais conveniente ao serviço do culto divino; e somou a despesa que se fez no sobredito,
e no cabedal que se meteu na artilharia, munições de guerra, soldos, mantimentos,
ordenados dos oficiais, passante de trezentos mil cruzados.

E logo no ano seguinte, mandou Sua Alteza em favor desta cidade outra armada, e por
capitão dela Antônio de Oliveira, com outros moradores casados e alguns forçados, em qual
mandou a rainha D. Catarina, que está em glória, algumas donzelas de nobre geração, das
que mandou criar e recolher em Lisboa no mosteiro das órfãs, as quais encomendou muito ao
governador por suas cartas, para que as casasse com pessoas principais daquele tempo; a
quem mandava dar em dote de casamento os ofícios do governo da fazenda e
justiça,\footnote{ No manuscrito da \textsc{bgjm}, ``mandava dar de casamento do
ofícios''.} com o que a cidade se foi enobrecendo, e com os escravos de Guiné, vacas e
éguas que Sua Alteza mandou a esta nova cidade, para que se repartissem pelos moradores
dela, e que pagassem o custo por seus soldos e ordenados, e o mais lhe mandava pagar em
mercadorias pelo preço que custavam em Lisboa, por a este tempo não irem a essas partes
mercadores, nem havia para quê, por na terra não haver ainda em que pudessem fazer seus
empregados;\footnote{ No manuscrito da \textsc{bgjm}, ``seus empregos''.} pelo qual
respeito Sua Alteza mandava cada ano em socorro dos moradores desta cidade uma armada com
degredados, moças órfãs, e muita fazenda, com o que a foi enobrecendo e povoando com muita
presteza, do que as mais capitanias se foram também ajudando, as quais foram visitadas
pelo governador e postas na ordem conveniente ao serviço del"-rei, e ao bem de sua justiça
e fazenda.

\paragraph{[5] Em que se trata como D. Duarte da Costa foi governar o Brasil} \quad
Como Tomé de Sousa acabou o seu tempo de governador,\footnote{ No manuscrito da
\textsc{bgjm}, ``acabou de governar''.} que gastou tão bem gastado neste novo Estado do
Brasil, requereu à Sua Alteza que o mandasse vir para este reino,\footnote{ Em Varnhagen
(1851 e 1879), ``mandasse tornar para o reino''.} a cuja petição el"-rei satisfez com
mandar por governador a D. Duarte da Costa, do seu conselho, ao qual deu a armada
conveniente a tal pessoa em que passou a este Estado, com a qual chegou a salvamento à
Bahia de Todos os Santos, e desembarcou na cidade do Salvador, nome que lhe Sua Alteza
mandou pôr; e lhe deu por armas uma pomba branca em campo verde, com um rolo à roda
branco, com letras de ouro que dizem \textit{Sic illa ad Arcam reversa est},\footnote{ No
manuscrito da \textsc{bgjm}, ``de ouro que dizem. E a pomba tem''. A inscrição latina
\textit{Sic illa ad Arcam reversa est}, ``Assim, ela [a pomba] voltou à arca'', hoje lema
da cidade de Salvador, é uma referência à pomba branca, primeiro animal a sair da Arca de
Noé durante o dilúvio, que retornou de seu vôo com um ramo de oliveira no bico, indicando
que havia terra fértil. } e a pomba tem três folhas de oliva no bico; onde lhe foi dada
posse da governança por Tomé de Sousa, que se logo embarcou na dita armada, e se veio para
o reino, onde serviu a el"-rei D. João e a seu neto, el"-rei D. Sebastião, de
veador,\footnote{ Veador da Casa Real: oficial"-mor que fazia serviço junto aos reis no paço
ou fora dele.} e no mesmo cargo serviu depois à rainha D. Catarina enquanto viveu.

E tornando a D. Duarte, como tomou a posse da governança, trabalhou quanto foi possível,
por fortificar e defender esta cidade do gentio que em seu tempo se alevantou, e cometeu
grandes insultos, os quais ele emendou dissimulando alguns com muita prudência, e
castigando outros com armas, fazendo"-lhes crua guerra, a qual caudilhava seu filho, D.
Álvaro da Costa, que nestes trabalhos o acompanhou, e se mostrou neles muito valoroso
capitão.

Em todo o tempo que D. Duarte governou o Brasil, foi todos os anos favorecido e ajudado
com armadas que do reino lhe mandavam, e em que lhe foram muitos moradores e gente forçada
com todo o necessário, ao qual sucedeu Mem de Sá, em cujos feitos já tocamos, o qual foi
também governar este Estado por mandado del"-rei D. João \textsc{iii}, a quem a fortuna
favoreceu de feição em catorze anos, que foi governador do Brasil, que subjugou e
desbaratou todo o gentio Tupinamba da comarca da Bahia e a todo o mais até o Rio de
Janeiro, de cujos feitos se pode fazer um notável tratado; o qual Mem de Sá foi pouco
favorecido destes reinos, por lhe falecer logo el"-rei D. João, que com tanto fervor
trabalhava por acrescentar e engrandecer este seu Estado, a quem a rainha D. Catarina, no
tempo que governou estes reinos, foi imitando; mas como ela desistiu da governança deles,
foram esfriando os favores e socorros que cada ano esta nova cidade recebia, para a qual
não mandaram dali por diante mais que um galeão da armada, em que iam os governadores que
depois a foram governar, pelo que este Estado tornou atrás de como ia florescendo. E se
esta cidade do Salvador cresceu em gente, edifícios e fazenda como agora tem, nasceu"-lhe
da grande fertilidade da terra, que ajudou aos moradores dela, de maneira que hoje tem no
seu termo, da Bahia para dentro, quarenta engenhos de açúcar, mui prósperos edifícios,
escravaria e outra muita fábrica, dos quais houvera muitos mais, se os moradores fossem
favorecidos como convinha, e como eles estão merecendo por seus serviços, com os quais o
governador Mem de Sá destruiu e desbaratou o gentio que vinha de redor da Bahia, a quem
queimou e assolou mais de trezentas aldeias,\footnote{ Em Varnhagen (1851 e 1879),
``gentio que vivia de rededor'' e ``mais de trinta aldeias''.} e os que escaparam de
mortos ou cativos fugiram para o sertão e se afastaram do mar mais de quarenta léguas, e
com os mesmos moradores socorreu e ajudou o dito Mem de Sá as capitanias dos Ilheos, Porto
Seguro e a do Espírito Santo, as quais estavam mui apartadas do gentio daquelas partes, e
com eles foi lançar por duas vezes os franceses fora do Rio de Janeiro, e a povoá"-lo, onde
acabaram muitos destes moradores sem até hoje ser dada nenhuma satisfação a seus filhos. E
todos foram fazer estes e outros muitos serviços à sua custa, sem darem soldo nem
mantimentos, como se costuma na Índia e nas outras partes, e a troco desses serviços e
despesas dos moradores desta cidade não se fez até hoje nenhuma honra nem mercê a nenhum
deles, do que vivem mui escandalizados e descontentes.

\paragraph{[6] Em que se declara o clima da Bahia, como cruzam os ventos na sua costa e
correm as águas nas monções\protect\footnote{ Em Varnhagen (1851 e 1879), ``e correm as
águas''.}} \quad
A Bahia de Todos os Santos está arrumada em treze graus e um terço, como fica dito atrás;
onde os dias em todo o ano são quase iguais com as noites e a diferença que têm os dias de
verão aos do inverno é uma hora até hora e meia. E começa"-se o inverno desta província no
mês de abril, e acaba"-se por todo o julho, em o qual tempo não faz frio que obrigue aos
homens se chegarem ao fogo, senão ao gentio, porque andam despidos. Em todo este tempo do
inverno correm as águas ao longo da costa a cem léguas ao mar dela, das partes do sul para
os rumos do norte, por quatro e cinco meses, e às vezes cursam os ventos do sul, sudoeste
e lés"-sudeste, que há travessia na costa de Porto Seguro até o cabo Santo Agostinho.

Começa"-se o verão em agosto como em Portugal em março, e dura até todo o mês de março, em
o qual tempo reinam os ventos nordestes e lés"-nordestes e correm as águas na costa ao som
dos ventos da parte do norte para os rumos do sul,\footnote{ Em Varnhagen (1851 e 1879),
``norte para o sul''.} pela qual razão se não navega ao longo desta costa senão com as
monções ordinárias. Em todo o tempo do ano, quando chove,\footnote{ No manuscrito da
\textsc{bgjm}, ``quando não chove''.} fazem os céus da Bahia as mais formosas mostras de
nuvens de mil cores e grande resplendor, que se nunca viram noutra parte, o que causa
grande admiração. E há"-se de notar que nesta comarca da Bahia, em rompendo a luz da manhã,
nasce com ela juntamente o sol, assim no inverno como no verão. E em se recolhendo o sol à
tarde, escurece juntamente o dia e cerra"-se a noite logo,\footnote{ Em Varnhagen (1851 e
1879), ``cerra"-se a noite''.} a que matemáticos deem razões suficientes que satisfaçam a
quem quiser saber este segredo, porque os mareantes e filósofos que a esta terra foram,
nem outros homens de bom juízo não têm atinado até agora com a causa por que isso assim
seja.

\paragraph{[7] Em que se declara o sítio da cidade do Salvador} \quad
A cidade do Salvador está situada na Bahia de Todos os Santos uma légua da barra para
dentro, em um alto, com o rosto ao poente, sobre o mar da mesma baía; a qual cidade foi
murada e torreada em tempo do governador Tomé de Sousa, que a edificou, como atrás fica
dito, cujos muros se vieram ao chão por serem de taipa e se não repararem nunca, no que se
descuidaram os governadores, pelo que eles sabem, ou por se a cidade ir estendendo muito
por fora dos muros; e, seja pelo que for, agora não há memória de onde eles estiveram.
Terá esta cidade oitocentos vizinhos, pouco mais ou menos, e por fora dela, em todos os
recôncavos da Bahia, haverá mais de dois mil vizinhos, dentre os quais e os da cidade, se
pode ajuntar, quando cumprir, quinhentos homens de cavalo e mais de dois mil de pé, afora
a gente dos navios que estão sempre no porto. Está no meio desta cidade uma honesta praça,
em que se correm touros quando convém, em a qual estão da banda do sul umas nobres casas,
em que se agasalham os governadores, e da banda do norte tem as casas do negócio da
Fazenda, alfândega e armazéns; e da parte de leste tem a casa da Câmara, cadeia e outras
casas de moradores, com que fica esta praça em quadra e o pelourinho no meio dela, a qual,
da banda do poente, está desabafada com grande vista sobre o mar; onde estão assentadas
algumas peças de artilharia grossa, donde a terra vai muito a pique sobre o mar; ao longo
do qual é tudo rochedo mui áspero; e desta mesma banda da praça, dos cantos dela, descem
dois caminhos em voltas para a praia, um da banda do norte, que é serventia para a fonte
que se diz Pereira, e do desembarcadouro da gente dos navios; o caminho que está da parte
do sul é serventia para Nossa Senhora da Conceição, onde está o desembarcadouro geral das
mercadorias, ao qual desembarcadouro vai ter outro caminho de carro, por onde se estas
mercadorias e outras coisas que aqui se desembarcam levam em carros para a cidade. E
tornando à praça, correndo dela para o norte, vai uma formosa rua de mercadores até a sé,
no cabo da qual, da banda do mar, está situada a casa da Misericórdia e hospital, cuja
igreja não é grande, mas mui bem acabada e ornamentada; e se esta casa não tem grandes
oficinas e enfermarias, é por ser muito pobre e não ter nenhuma renda de Sua Majestade,
nem de pessoas particulares, e sustenta"-se de esmolas que lhe fazem os moradores da terra,
que são muitas, mas são as necessidades mais, por a muita gente do mar e degradados que
destes reinos vão muito pobres, os quais em suas necessidades não têm outro remédio que o
que lhes esta casa dá, cujas esmolas importam cada ano três mil cruzados pouco mais ou
menos, que se gastam com muita ordem na cura dos enfermos e remédio dos necessitados.

\paragraph{[8] Em que se declara o sítio da cidade, da Sé por diante} \quad
A Sé da cidade do Salvador está situada com o rosto sobre o mar da Bahia, defronte do
ancoradouro das naus, com um tabuleiro defronte da porta principal, bem a pique sobre o
desembarcadouro, donde tem grande vista.

A igreja é de três naves, de honesta grandeza, alta e bem assombrada, a qual tem cinco
capelas muito bem feitas e ornamentadas, e dois altares nas ombreiras da capela"-mor. Está
esta Sé em redondo cercada de terreiro, mas não está acabada da torre dos sinos e da do
relógio, o que lhe falta, e outras oficinas muito necessárias, por ser muito pobre e não
ter para fábrica mais do que cem mil"-réis para cada ano, e estes muito mal pagos. Serve"-se
nesta igreja o culto divino com cinco dignidades, seis cônegos, dois meios cônegos, quatro
capelães, um cura e um coadjutor, quatro moços de coro e mestre da capela, e muitos destes
ministros não são sacerdotes; e ainda são tão poucos, fazem"-se nela os ofícios divinos com
muita solenidade, o que custa ao bispo um grande pedaço da sua casa; por contentar os
sacerdotes que prestam para isso, com lhes dar a cada um, um tanto com que queiram servir
de cônegos e dignidades, do que os clérigos fogem, por não ter cada cônego mais de trinta
mil"-réis, e as dignidades a trinta e cinco, tirado o deão, que tem quarenta mil"-réis, o
que lhes não basta para se vestirem. Pelo que querem antes ser capelães da Misericórdia ou
dos engenhos, onde têm de partido sessenta mil"-réis, casas em que morem e o
comer;\footnote{ Em Varnhagen (1851 e 1879), ``casa em que vivam e o de comer''.} e nestes
lugares rendem"-lhes suas ordens e pé de altar outro tanto. Está esta Sé muito necessitada
de ornamentos, e os de que se serve estão mui danificados; e de maneira que, nas festas
principais, se aproveita o cabido dos das confrarias, onde os pedem emprestados; do que
Sua Majestade não deve estar informado, que se o estivera, tivera já mandado prover esta
necessidade em que está o culto divino, pois manda receber os dízimos deste seu Estado,
cuja cabeça está tão danificada, que convém acudir"-lhe com remédio devido com muita
presteza.

\paragraph{[9] Em que se declara como corre a cidade do Salvador da Sé por diante} \quad
Passando além da Sé pelo mesmo rumo do norte, corre outra rua mui larga, também ocupada
com lojas de mercadores, a qual vai dar consigo em um terreiro mui bem assentado e grande,
onde se representam as festas a cavalo, por ser maior que a praça, o qual está cercado em
quadro de nobres casas. E ocupa todo este terreiro e parte da rua da banda do mar um
suntuoso colégio dos padres da Companhia de Jesus,\footnote{ Em Varnhagen (1851 e 1879),
``E ocupa este terreiro e parte''.} com uma formosa e alegre igreja, onde serve o culto
divino com mui ricos ornamentos, a qual os padres têm sempre mui limpa e cheirosa.

Tem este colégio grandes dormitórios e muito bem acabados, parte dos quais fica sobre o
mar, com grande vista; cuja obra é de pedra e cal, com todas as escadas, portas e janelas
de pedrarias, com varandas, e cubículos mui bem forrados, e as clausuras por baixo
lajeadas com muita perfeição;\footnote{ Em Varnhagen (1851 e 1879), ``bem forrados, e por
baixo lajeados''.} o qual colégio tem grandes cercas até o mar, com água muito boa dentro,
e ao longo do mar tem umas terracenas,\footnote{ Terracena ou tercena: espécie de armazém,
construído na beira de rios ou junto a cais para guardar cereais, armamentos, munições.}
onde recolhem o que lhes vai por mar de fora da cidade.\footnote{ Em Varnhagen (1851 e
1879), ``o que vem embarcado de fora''.} Tem este colégio, ordinariamente, oitenta
religiosos, que se ocupam em pregar e confessar alguma parte deles, outros ensinam e
aprendem teologia, artes, latim e casos de consciência,\footnote{ Em Varnhagen (1851 e
1879), ``outros ensinam latim, artes, teologia, e casos de consciência''.} com o que têm
feito muito fruto na terra; o qual está muto rico, porque tem de Sua Majestade, cada ano,
quatro mil cruzados e, d'avantagem, importar"-lhe"-á a outra renda que tem na terra outro
tanto; porque tem muitos currais de vacas, onde se afirma que trazem mais de duas mil
vacas de ventre, que nesta terra parem todos os anos,\footnote{ No manuscrito da
\textsc{bgjm}, ``naquelas terras''.} e tem outra muita granjearia de suas roças e
fazendas, onde tem todas as novidades dos mantimentos, que se na terra dão em muita
abastança.

\paragraph{[10] Em que se declara como corre a cidade por este rumo até o cabo} \quad
Passando avante do colégio vai outra rua muito comprida pelo mesmo rumo do norte, muito
larga e povoada de casas de moradores,\footnote{ Em Varnhagen (1851 e 1879), ``casas e
moradores''.} além da qual, no arrabalde da cidade, em um alto dela, está um mosteiro de
capuchos dos de Santo Antônio,\footnote{ Em Varnhagen (1851 e 1879), ``mosteiro de
capuchinhos de Santo Antônio''.} que há pouco tempo se começou de esmolas do povo, que
lhes comprou este assento, e outros devotos que lhe deram outros chãos junto dele, em que
lhe os moradores fizeram uma igreja, com a qual, e o mais recolhimento que está feito, se
podem acomodar até vinte religiosos,\footnote{ No manuscrito da \textsc{bgjm}, ``mais
recolhimento, se podem''.} e pelo tempo adiante lhe farão outro recolhimento como os
padres quiserem, os quais têm nesse recolhimento sua cerca com água dentro, a qual cerca
vem correndo de cima, onde está o mosteiro, até o mar. E tornando deste mosteiro para a
praça, pela banda da terra, vai a cidade muito bem arruada, com casas de moradores com
seus quintais, os quais estão povoados de palmeiras carregadas de cocos e outras de
tâmaras, e de laranjeiras e outras árvores de espinho, figueiras, romeiras e parreiras,
com o que fica muito fresca; a qual cidade por esta banda da terra está toda cercada com
uma ribeira de água, que serve de lavagem e de se regarem algumas hortas, que ao longo
dela estão.

\paragraph{[11] Em que se declara como corre a cidade da praça para a banda do sul} \quad
Tornando à praça, pondo o rosto no sul, corre outra rua muito formosa de moradores, no
cabo da qual está uma ermida de Santa Luzia, onde está uma estância com artilharia. E ao
longo dessa rua lhe fica muito bem assentada, também toda povoada de lojas de mercadores,
e no topo dela está uma formosa igreja de Nossa Senhora da Ajuda com sua capela de
abóbada;\footnote{ A primeira igreja de N. S. da Ajuda, edificada pelo jesuíta Padre
Manuel da Nóbrega e seus companheiros, vindos com Tomé de Sousa, foi o primeiro templo na
cidade de Salvador que serviu de Sé Catedral, enquanto se construía a definitiva.} no qual
sítio, no princípio desta cidade, esteve a Sé. Passando mais avante com o rosto ao sul, no
outro arrebalde da cidade, em um alto e campo largo, está situado um mosteiro de São
Bento, com sua claustra, e largas oficinas, e seus dormitórios, em que se agasalham vinte
religiosos que naquele mosteiro há, os quais têm sua cerca e horta com uma ribeira de
água, que lhe nasce dentro, que é a que rodeia toda a cidade, como fica atrás dito. Esse
mosteiro de São Bento é muito pobre, o qual se mantém de esmolas que pedem os frades pelas
fazendas dos moradores, e não tem nenhuma renda de Sua Majestade, em quem será bem
empregada, pelas necessidades que tem, cujos religiosos vivem santa e honesta vida, dando
de si grande exemplo, e estão mui bem benquistos e bem recebidos do povo,\footnote{ Em
Varnhagen (1851 e 1879), ``estão benquistos e mui bem recebidos''.} os quais haverá três
anos que foram a esta cidade, com licença de Sua Majestade fundar este mosteiro, que lhes
os moradores dela fizeram à sua custa, com grande fervor e alvoroço.

E não se faz aqui particular menção das outras ruas da cidade, porque são muitas, e fora
nunca acabar querê"-las particularizar.

\paragraph{[12] Em que se declaram outras partes que a cidade tem para notar} \quad
Tem esta cidade grandes desembarcadouros, com três fontes na praia ao pé dela, em as quais
os mareantes fazem sua aguada, bem à borda do mar, das quais se serve também muita parte
da cidade, por serem estas fontes de muito boa água. No principal desembarcadouro está uma
fresca ermida de Nossa Senhora da Conceição,\footnote{ Em Varnhagen (1851 e 1879), ``uma
fraca ermida''.} que foi a primeira casa de oração e obra em que se Tomé de Sousa ocupou.

A vista desta cidade é mui aprazível ao longe, por estarem as casas com os quintais cheios
de árvores, a saber: de palmeiras, que aparecem por cima dos telhados; e de laranjeiras,
que todo o ano estão carregadas de laranjas, cuja vista de longe é mui alegre,
especialmente do mar, por a cidade se estender muito ao longo dele, neste alto. Não tem a
cidade nenhum padrasto, de onde a possam ofender, se a cercarem como ela merece, o que se
pode fazer com lhe ficar dentro uma ribeira de água, que nasce junto dela, que agora a vai
cercando toda, a qual se não bebe agora, por estar o nascimento dela pisado dos bois, que
vão beber, e porcos; mas, limpa, é muito boa água, da qual se não aproveitam os moradores
por haver outras muitas fontes de que bebe cada um, segundo a afeição que lhe tomam, e a
de que lhe fica mais perto se ajuda por serem todas de boa água.

A terra que esta cidade tem, uma e duas léguas à roda, está quase toda ocupada com roças,
que são como os casais de Portugal, onde se lavram muitos mantimentos, frutas e
hortaliças, de onde se remedeia toda a gente da cidade que o não tem de sua lavra, a cuja
praça se vai vender, do que está sempre mui provida, e o mais do tempo o está do pão, que
se faz das farinhas que levam do reino a vender ordinariamente à Bahia, onde também levam
muitos vinhos da ilha da Madeira e das Canárias, onde são mais brandos, e de melhor
cheiro, e cor e suave sabor, que nas mesmas ilhas de onde os levam; os quais se vendem em
lojas abertas, e outros mantimentos de Espanha, e todas as drogas, sedas e panos de toda a
sorte, e as mais mercadorias acostumadas.

\paragraph{[13] Em que se declara o como se tratam os moradores da cidade do Salvador, e
algumas qualidades suas} \quad
Na cidade do Salvador e seu termo há muitos moradores ricos de fazendas de raiz, peças de
prata e ouro, jaezes de cavalos e alfaias de casa, entanto que há muitos homens que têm
dois e três mil cruzados em joias de ouro e prata lavrada. Há na Bahia mais de cem
moradores que têm cada ano de mil cruzados até cinco mil de renda, e outros que têm mais,
cujas fazendas valem vinte mil até cinquenta e sessenta mil cruzados, e d'avantagem, os
quais tratam suas pessoas mui honradamente, com muitos cavalos, criados e escravos, e com
vestidos demasiados, especialmente as mulheres, que não vestem senão sedas, por a terra
não ser fria, no que fazem grandes despesas, mormente entre a gente de menor condição;
porque qualquer peão anda com calções e gibão de cetim ou damasco, e trazem as mulheres
com vasquinhas e gibões do mesmo, os quais, como têm qualquer possibilidade, têm suas
casas mui bem concertadas e na sua mesa serviço de prata, e trazem suas mulheres mui bem
ataviadas de joias de ouro.

Tem esta cidade catorze peças de artilharia grossa, e quarenta, pouco mais ou menos, de
artilharia miúda; a artilharia grossa está assentada nas estâncias atrás declaradas, e em
outra que está na ponta do Padrão para defender a entrada da barra aos navios dos
corsários, se a cometerem, donde não lhe podem fazer mais dano que afastá"-los da carreira,
para que não possam tomar o porto do primeiro bordo, porque é a barra muito grande e podem
afastar as naus que quiserem, sem lhes a artilharia fazer nojo.\footnote{ Nojo: dano.}


\paragraph{[14] Que trata de como se pode defender a Bahia com mais facilidade} \quad
Não parece despropósito dizer neste lugar, que tem el"-rei nosso senhor obrigação de, com
muita instância, mandar acudir ao desamparo em que esta cidade está, mandando"-a cercar de
muros e fortificar, como convém ao seu serviço e segurança dos moradores dela; porque está
arriscada a ser saqueada de quatro corsários, que a forem cometer, por ser a gente
espalhada por fora, e a da cidade não ter onde se possa defender, até que a gente das
fazendas e engenhos a possa vir socorrer. Mas, enquanto não for cercada, não tem remédio
mais fácil para se poder defender dos corsários que na baía entrarem, que pelo mar com
quatro galeotas que com pouca despesa se podem fazer, e estarem sempre armadas; à sombra
das quais podem pelejar muitas barcas dos engenhos, e outros barcos, em que se pode
cavalgar artilharia, para poderem pelejar; e esta armada se pode favorecer com as naus do
reino que de contínuo estão no porto oito e dez,\footnote{ No manuscrito da \textsc{bgjm},
``e com esta armada do reino se podem favorecer as naus que''.} e daqui para cima até
quinze e vinte, que estão tomando carga de açúcar e algodão, em as quais se pode meter
gente da terra para a defender, e alguma artilharia com que ofender aos contrários, os
quais, se não levarem a cidade no primeiro encontro, não a entrarão depois, porque pode
ser socorrida por mar e por terra de muita gente portuguesa até a quantia de dois mil
homens, de entre os quais podem sair dez mil escravos de peleja a saber: quatro mil pretos
da Guiné, e seis mil índios da terra, mui bons flecheiros, que juntos com a gente da
cidade, se fará mui arrazoado exército, com a qual gente,\footnote{ Em Varnhagen (1851 e
1879), ``com o qual corpo de gente''.} sendo bem caudilhada, se pode fazer muito dano a
muitos homens de armas que saírem a terras onde se hão de achar mui embaraçados e
pelejados por entre o mato, que é mui cego, e ser"-lhes"-á forçado recolher"-se com muita
pressa, o que Deus não permita que aconteça, pelo desapercebimento que esta cidade tem; do
que sabem à certeza os ingleses, que a ela foram já, de onde podem tirar grande presa, da
maneira que agora está, se a cometerem com qualquer armada, porque acharão no porto muitos
navios carregados de açúcar e algodão, e muita soma dele recolhido pelas terracenas que
estão na praia dos mercadores, e pela cidade se acham as lájeas cheias de mercadorias e de
muito dinheiro de contado,\footnote{ Em Varnhagen (1851 e 1879), ``dos mercadores, tanto
das mercadorias como de muito dinheiro de contato''.} muitas peças de ouro e prata e
muitas alfaias de casa.

\paragraph{[15] Em que se declaram as grandes qualidades que tem a Bahia de Todos os Santos} \quad
El"-rei D. João \textsc{iii} de Portugal, que está em glória, estava tão afeiçoado ao
Estado do Brasil, especialmente à Bahia de Todos os Santos, que, se vivera mais alguns
anos, edificaria nele um dos mais notáveis reinos do mundo, e engrandecera a cidade do
Salvador de feição que se pudera contar entre as mais notáveis de seus reinos, para o que
ela estava mui capaz, e agora o está ainda mais em poder e aparelho para isso, porque é
senhora desta baía, que é a maior e mais formosa que se sabe pelo mundo, assim em grandeza
como em fertilidade e riqueza. Porque esta baía é grande e de bons ares, mui delgados e
sadios, de muito frescas e delgadas águas, e mui abastada de mantimentos naturais da
terra, de muita caça, e muitos e mui saborosos pescados e frutas, a qual está arrumada
pela maneira seguinte.

A baía se estende da ponta do Padrão ao morro de Tinhare, que demora um do outro nove ou
dez léguas, ainda que o capitão da capitania dos Ilheos não quer consentir que se estenda
senão da ponta da ilha de Tapariqua à do Padrão; mas está já averiguada por sentença, que
se estende a baía da ponta do Padrão até Tinhare, como já fica dito; a qual sentença se
deu por haver dúvida entre os rendeiros da capitania dos Ilheos e da Bahia, sobre a quem
pertenciam os dízimos do pescado, que se pescava junto a este morro de Tinhare, o qual
dízimo se sentenciou ao rendeiro da Bahia, por se averiguar estender"-se a baía do morro
para dentro, como na verdade se deve de entender.

\paragraph{[16] Em que se declaram as barras que tem a Bahia de Todos os Santos, e como está
arrumada a ilha de Taparica, entre uma barra e a outra} \quad
Acima fica dito como dista a ponta de Tinhare da do Padrão nove ou dez léguas, entre as
quais pontas da banda de dentro delas está lançada uma ilha de sete léguas de comprido que
se chama Taparica, a qual Tomé de Sousa, sendo governador"-geral do Brasil, deu de sesmaria
a D. Antônio de Ataíde, primeiro conde de Castanheira, o que lhe Sua Alteza depois
confirmou, e lhe fez nova doação dela, com título de capitão e governador; ao que veio com
embargos a Câmara da cidade do Salvador, sobre o que contendem há mais de trinta anos, e
lhe impediu sempre a jurisdição, sem até agora se averiguar esta causa. Deixa esta ilha
entre si e o morro de Tinhare outra baía mui grande, com fundo e porto, em que podem
entrar naus de todo o porte, e tem grande ancoradouro e abrigada à sombra do morro, de que
se aproveitam muitas vezes as naus que vêm do reino, quando lhes escasseia o vento, e não
podem entrar na baía da ilha para dentro. Da ponta desta ilha de Taparica à ponta do
Padrão está a barra do leste, e entre a outra ponta da ilha e a 
ponta de Jaguaripe está a barra do loeste, por cada uma destas barras se 
nas duas linhas entra na baía com a proa ao norte. A barra do loeste se chama de Jaguoaripe
por se meter nela um rio do mesmo nome. Haverá da terra firme a esta ponta da ilha, perto
de uma légua de terra a terra, a qual barra é aparcelada por ser cheia de baixos de areia,
mas tem um canal estreito por onde navegam, pelo qual entram caravelões da costa e barcas
dos engenhos; mas há de ser com tempos bonançosos, porque com marulho\footnote{ Marulho:
agitação das águas do mar.} não se enxerga o canal. E corre grande perigo quem se aventura
a cometer esta barra de Jaguoaripe com tempo fresco e tormentoso.

\paragraph{[17] Em que se declara como se navega pela barra de Santo Antônio para entrar na
baía} \quad
A barra principal da baía é a banda de leste, a que uns chamam a barra da cidade e
outros de Santo Antônio, por estar junto dela, da banda de dentro em um alto, sua ermida;
a qual barra tem de terra a terra duas léguas, e tanto dista da ponta do Padrão à terra de
Taparica, e à ponta onde está o curral de Cosme Garção,\footnote{ Era o locotenente
(lugar"-tenente) do 1º Conde de Castanheira, D. Antônio de Ataíde, a quem Tomé de Sousa
deu, em sesmaria, a ilha de Itaparica. O curral de Cosme Garção devia ocupar a região da
atual Ponta do Garcês, próxima a Jaguaripe.} que é mais saída ao mar. Da banda da ilha tem
esta barra uma légua de baixos de pedra, onde o mar anda o mais do tempo em flor. Por
entre estes baixos há um canal por onde entram com bonança navios de quarenta tonéis, e
fica a barra por onde as naus costumam entrar e sair da parte do Padrão, a qual tem uma
légua de largo, que toda tem fundo, por onde entram naus da Índia de todo o porte, em o
qual espaço não há baixo nenhum. Por esta barra podem entrar as naus de noite e dia com
todo o tempo, sem haver de que se guardar, e os pilotos, que sabem bem esta costa, se não
podem alcançar esta barra com de dia, e conhecem a terra, quando a veem do mar em fora,
marcando"-se com a ponta do Padrão,\footnote{ Em Varnhagen (1851 e 1879), ``em fora,
mareiam"-se com a ponta''.} e como ficam a barlavento dela, navegam com a proa ao norte e
vão dar consigo no ancoradouro da cidade, onde ficam seguros sobre amarra de todos os
ventos, tirado o sudoeste, que, quando venta, ainda que é muito rijo, no inverno,
nunca passa a sua tormenta de vinte e quatro horas, em as quais se
amarram os navios muito bem, e ficam seguros desta tormenta, que de maravilha acontece, em
o qual tempo se ajudam os navios uns aos outros, de maneira que não corre perigo, e deste
porto da cidade, onde os navios ancoram, à ponta do Padrão, pode ser uma légua.

\paragraph{[18] Em que se declara o tamanho do mar da baía, em que podem andar naus à vela,
e de algumas ilhas}\quad
Da banda da cidade à terra firme da outra banda, que chamam do Paraguoasu, são nove ou dez
léguas de travessia, e fica neste meio uma ilha, que chamam a dos Frades, que tem duas
léguas de comprido, e uma de largo. Ao norte desta ilha está outra, que chamam de Maré,
que tem uma légua de comprido e meia de largo; e dista uma ilha da outra três
léguas.\footnote{ A Ilha dos Frades e a Ilha de Maré, ambas na Baía de Todos os Santos,
permanecem com o mesmo nome e pertencem atualmente ao município de Salvador.} Da ilha da
Maré à terra firme da banda do poente haverá espaço de meia légua. Da ilha dos Frades à de
Taparica são quatro léguas. Da cidade à ilha de Maré são seis léguas, e haverá outro tanto
da mesma cidade à ilha dos Frades, de maneira que, da ponta da ilha de Taparica até a dos
Frades, e à ilha de Maré, e dela à terra firme contra o rio de Matoim, e desta corda para
a cidade, por todo este mar até a boca da barra, se pode barlaventear com naus de todo o
porte, sem acharem baixos nenhuns, como se afastarem de terra um tiro de berço. Esta ilha
dos Frades é de um João Nogueira, lavrador, o qual está de assento nela com seis ou sete
lavradores, que nela têm da sua mão,\footnote{ Ter alguém de sua mão: ter debaixo de sua
proteção, auxiliar; alimentar.} onde têm suas granjearias de roças de mantimentos, com
criações de vacas e porcos; a qual ilha tem muitas águas, mas pequenas para engenhos, cuja
terra é fraca para canaviais de açúcar. A ilha de Maré é muito boa terra para canaviais e
algodoais e todos os mantimentos,\footnote{ Em Varnhagen (1851 e 1879), ``para canaviais,
e algodões, e todos''.} onde está um engenho de açúcar que lavra com bois, que é de
Bartolomeu Pires, mestre da capela da sé, onde são assentados sua mão passante
de\footnote{ Passante de: mais de, mais do que.} vinte moradores, os quais têm aqui uma
igreja de Nossa Senhora das Neves, muito bem concertada, com seu cura, que administra os
sacramentos a estes moradores.

\paragraph{[19] Em que se declara a terra da Bahia, da cidade até a ponta de Tapagipe, e as
suas ilhas}\quad
Atrás fica dito como da cidade até a ponta do Padrão é uma légua; agora convém que vamos
correndo toda a redondeza da Bahia e recôncavos dela, para se mostrar o muito que tem para
ver, e que notar.

Começando da cidade para a ponta de Tapagipe, que é uma légua, no meio deste caminho se
faz um engenho de água em uma ribeira chamada Água dos Meninos, o qual não será muito
proveitoso por ser tão perto da cidade. Este engenho faz um morador dos principais da
terra, que se diz Cristovão de Aguiar d'Altro,\footnote{ Na edição de 1851, ``Christovam
de Aguiar de Alto'', e na de 1879, ``Christovão de Aguiar de Altero''.} e nesta ponta de
Tapagipe estão umas olarias de Garcia de Ávila e um curral de vacas do mesmo, a qual
ponta, bem chegada ao cabo dela, tem uma aberta pelos arrecifes, por onde entram
caravelões, que com tempos se recolhem aqui, e da boca da barra para dentro em uma calheta
onde estes caravelões e barcos estão seguros. Nesta ponta, quando se fundou a cidade,
houve pareceres que ela se edificasse, por ficar mais segura e melhor assentada e muito
forte, a qual está norte e sul com a ponta do Padrão.

Virando desta ponta sobre a mão direita está um esteiro mui fundo, por onde entram naus de
quatrocentos tonéis, ao qual chamam Pirajá, o qual faz para dentro grandes voltas; em uma
delas tem uma praia onde se põem os navios a monte muito à vontade, e se calafetam muito
bem às marés,\footnote{ No manuscrito da \textsc{bgjm}, ``se calafetam as marés''.} porque
com as águas vivas descobrem até a quilha, onde se queimam e calafetam bem.

Deste esteiro para dentro, ao longo desta ponta, estão três ilhetas povoadas e lavradas
com canaviais e roças, e na terra desta ponta estão outras duas olarias de muita fábrica,
por haver aqui muito e bom barro, donde se provêm dele os mais dos engenhos de açúcar da
barra, porque se purga o açúcar com este barro.\footnote{ Em Varnhagen (1851 e 1879), ``os
mais dos engenhos, pois''.}

\paragraph{[20] Em que se relata os engenhos de açúcar que há neste rio de Pirajá e sua
terra\protect\footnote{ Em Varnhagen (1851 e 1879), ``Em que se declaram os engenhos de
açúcar que há neste rio de Pirajá''.}}\quad
Entrando por este esteiro, pondo os olhos na terra firme, tem uma formosa vista de três
engenhos de açúcar, e outras muitas fazendas mui formosas da vista do mar, e no cabo do
salgado se mete nele uma formosa ribeira de água, com que mói um engenho de açúcar de Sua
Majestade, que ali está feito com uma igreja de S. Bartolomeu, freguesia daquele limite, o
qual engenho anda arrendado em seiscentas e cinquenta arrobas de açúcar branco cada ano.
Pelo sertão deste engenho, meia légua dele, está outro de Diogo da Rocha de Sá, que mói
com outra ribeira, o qual está muito ornado de edifícios com uma igreja de S. Sebastião,
muito bem concertada. À mão esquerda deste engenho de Sua Majestade está outro de João de
Barros Cardoso, meia légua para a banda da cidade até onde este esteiro faz um braço por
onde se serve com suas barcas; o qual engenho tem grande aferida\footnote{ Aferida: regato
ou calha por onde a água cai para mover uma roda hidráulica.} e fábrica de escravos,
grandes edifícios e outra muita granjearia de roças, canaviais e currais de vacas, onde
também está uma ermida de Nossa Senhora de Encarnação, muito bem concertada de todo o
necessário.\footnote{ No manuscrito da \textsc{bgjm}, ``muito bem concertada''.} E entre
um engenho e outro está uma casa de cozer meles\footnote{ Casa de meles: local onde
provavelmente se realizava uma das etapas do fabrico do açúcar, sem chegar ao produto
final; por isso a distinção que faz Gabriel Soares entre a casa de meles e o engenho, no
qual se podia realizar todo o processo.} com muita fábrica, a qual é de Antônio Nunes
Reimão. À mão direita deste engenho de Sua Majestade está outro de D. Leonor Soares,
mulher que foi de Simão da Gama de Andrade, o qual mói com uma ribeira de água com grande
aferida e está bem fabricado. Este rio de Pirajá é muito farto de pescado e marisco, de
que se mantêm a cidade e fazendas de sua vizinhança, em o qual andam sempre sete ou oito
barcos de pescar com redes, onde se toma muito peixe, e no inverno, em tempo de tormenta,
pescam dentro dele os pescadores de jangadas dos moradores da cidade e os das fazendas
duas léguas à roda, e sempre tem peixe, de que todos se remedeiam.

\paragraph{[21] Em que se declara a terra e sítio das fazendas que há da boca de Pirajá até
o rio de Matoim}\quad
Por este rio de Pirajá abaixo, e da boca dele para fora ao longo do mar da baía, por ela
acima, vai tudo povoado de formosas fazendas e tão alegres da vista do mar, que não cansam
os olhos de olhar para elas.

E no princípio está uma de Antônio de Oliveira de Carvalhal, que foi alcaide"-mor de Vila
Velha, com uma ermida de São Brás; e vai correndo esta ribeira do mar da baía com esta
formosura até Nossa \EP[1] Senhora da Escada, que é muito formosa igreja dos padres da
Companhia, que a têm mui concertada; onde vão às vezes convalescer alguns padres de suas
enfermidades, por ser o lugar para isso; a qual igreja está uma légua do rio de Pirajá e
duas da cidade. De Nossa Senhora da Escada para cima se recolhe a terra para dentro até o
porto de Paripe, que é daí uma légua, cujo espaço se chama Praia Grande, pelo ela ser e
muito formosa, ao longo da qual está tudo povoado de mui alegres fazendas, e de um engenho
de açúcar que mói com bois e está muito bem acabado, cujo senhorio se chama Francisco de
Aguilar, homem principal, castelhano de nação. Deste porto de Paripe obra de quinhentas
braças pela terra dentro está outro engenho de bois que foi de Vasco Rodrigues Lobato,
todo cercado de canaviais de açúcar, de que se faz muitas arrobas.

Do porto de Paripe se vai à terra afeiçoando à maneira de ponta lançada ao mar, e corre
assim obra de uma légua, onde está uma ermida de São Tomé em um alto, ao pé do qual ao
longo do mar estão umas pegadas assinaladas em uma lájea, que diz o gentio, diziam seus
antepassados, que andara por ali, havia muito tempo, um santo, que fizera aqueles sinais
com os pés. Toda a terra por aqui é mui fresca, povoada de canaviais e pomares de árvores
de espinho, e outras frutas da Espanha e da terra, de onde se ela torna a recolher para
dentro, fazendo outra praia mui formosa e povoada de mui frescas fazendas, por cima das
quais aparece a igreja de Nossa Senhora do Ó, freguesia da povoação de Paripe, que está
junto dela, arruada e povoada de moradores, que é a mais antiga povoação e
julgado\footnote{ Julgado: divisão territorial sobre a qual tem jurisdição o juiz
ordinário.} da Bahia.\footnote{ São Tomé do Paripe é atualmente um bairro da cidade de
Salvador, voltado para a Baía de Todos os Santos.}

Desta praia se torna a terra a afeiçoar à maneira de ponta para o mar, e na mais saída a
ele se chama a ponte do Toquetoque, de onde a terra torna a recuar para trás até a boca do
rio de Matoim, tudo povoado de alegres fazendas. Do porto de Paripe ao rio Matoim são duas
léguas, e de Matoim à cidade são cinco léguas.

\paragraph{[22] Em que se declara o tamanho do rio de Matoim e os engenhos que tem}\quad
Entra a maré pelo rio de Matoim acima quatro léguas, o qual tem de boca, de terra a terra,
um tiro de berço uma da outra, e, entrando por ele acima mais de uma légua, vai povoado de
muitas e mui frescas fazendas, fazendo algumas voltas, esteiros e enseadas, e no cabo
desta légua se alarga o rio muito de terra à terra; e à mão direita por um braço acima
está o famoso engenho de Paripe, que foi de Afonso de Torres e agora é de Baltasar
Pereira, mercador. A este engenho pagam foro todas as fazendas que há no porto de Paripe,
a que também chamam do Tubarão, até a boca de Matoim, e pelo rio acima duas léguas.

E virando deste engenho para cima sobre a mão direita, vai tudo povoado de fazendas, e em
uma de Francisco Barbuda está uma ermida de São Bento e, mais adiante, noutra fazenda, de
Cristóvão de Aguiar, está outra ermida de Nossa Senhora; e assim vai correndo esta terra
até o cabo do Salgado mui povoada de nobres fazendas, mui ornadas de aposentos, e no cabo
deste está um engenho de bois de duas moendas de Gaspar Dias Barbosa, peça de muito preço,
o qual tem nele uma igreja de Santa Catarina. Junto deste engenho está uma ribeira em que
se pode fazer um engenho de água muito bom, o qual se não faz por haver demanda sobre esta
água, entre partes que a pretendem.

Da outra banda deste engenho está assentado outro que se diz de Bastião de
Ponte,\footnote{ Em Varnhagen (1851 e 1879), ``Sebastião da Ponte''.} que mói com uma
ribeira que chamam Cotigipe, o qual engenho está muito adornado de edifícios mui
aperfeiçoados; e tornando por este rio abaixo, sobre a mão direita obra de meia légua,
está uma ilha de Jorge de Magalhães, mui formosa por estar toda lavrada de canaviais, e no
meio dela em um alto tem umas nobres casas cercadas de laranjeiras arruadas, e outras
árvores, coisa muito para ver; e descendo uma légua abaixo do engenho de Cotigipe está uma
ribeira que se chama do Aratu, em a qual Sebastião de Faria tem feito um soberbo engenho
de água, com grandes edifícios de casas de purgar e de vivenda, e uma igreja de São
Jerônimo, tudo de pedra e cal, no que gastou mais de treze mil cruzados.\footnote{ Em
Varnhagen (1851 e 1879), ``mais de doze mil cruzados''.}

Meia légua deste engenho pelo rio abaixo está outra ribeira a que chamam de
Carnuibusu,\footnote{ Em Varnhagen (1851 e 1879), ``Carnaibuçu''.} onde não está engenho
feito por haver litígio sobre esta água. Na boca desta ribeira está uma ilha muito fresca,
que é de Nuno Fernandes. De Carnuibusu a uma légua está um engenho de bois,\footnote{ Em
Varnhagen (1851 e 1879), ``Nuno Fernandes; a uma légua está''.} de que é senhorio Jorge
Antunes, o qual está mui petrechado de edifícios de casas, e tem uma igreja de Nossa
Senhora do Rosário.

Deste engenho até a boca do rio será uma légua pouco mais ou menos, o qual está povoado de
mui grandes fazendas, cujos edifícios e canaviais estão à vista deste rio, que é mui
formoso e largo de alto até baixo.

Defronte da boca deste rio de Matuim está a ilha de Maré, que começa a correr dele para
cima no comprimento dela, da qual fica dito atrás o que se podia dizer.

\paragraph{[23] Em que se declara a feição da terra da boca de Matoim até o esteiro de
Metaripe e os engenhos que tem em si}\quad
Saindo pela boca de Matuim fora, virando sobre a mão direita, vai a terra fabricada com
fazendas e canaviais dali a meia légua, onde está outro engenho de Sebastião de Faria, de
duas moendas que lavram com bois, o qual tem grandes edifícios, assim do engenho como de
casas de purgar, de vivenda e de outras oficinas e tem uma formosa igreja de Nossa Senhora
da Piedade, que é freguesia deste limite, a qual fazenda mostra tanto aparato da vista do
mar que parece uma vila.

E indo correndo a ribeira do Salgado deste engenho a meia légua, está tudo povoado de
fazendas, e no cabo está uma que foi do deão da sé, com uma ermida de Nossa Senhora, bem
concertada, a qual está em uma ponta da terra. Defronte desta ponta, bem chegada à terra
firme, está uma ilha, que se diz de Pero Fernandes,\footnote{ Em Varnhagen (1851 e 1879),
``Pedro Fernandes''.} onde ele vivia com sua família e tem sua granjearia de canaviais e
roças com água dentro.

Da fazenda do deão se começa ir armando a enseada que dizem de Jacarecanga, no meio da
qual está um formoso engenho de bois de Cristóvão de Barros, até onde está tudo povoado de
fazendas e lavradores de canaviais; este engenho tem mui grandes edifícios e uma igreja de
Santo Antônio. Esta enseada está em feição de meia lua e terá, segundo a feição da terra,
duas léguas, em a qual está uma ribeira de água em que se pode fazer um engenho, o qual se
deixa de fazer por se não averiguar o litígio que sobre ela há;\footnote{ Em Varnhagen
(1851 e 1879), ``se deixa de fundar''.} e toda esta enseada à roda, sobre a vista da água,
está povoada de fazendas e formosos canaviais.

E saindo desta enseada, virando sobre a ponta da mão direita, vai correndo a terra fazendo
um canto em espaço de meia légua, em a qual estão dois engenhos de bois, um de Tristão
Ribeiro\footnote{ Em Varnhagen (1851 e 1879), ``Tristão Rodrigo''.} junto da ponta da
enseada, defronte da qual à ilha de Maré está um ilhéu que se chama de Pacé,\footnote{
Atual Ilha do Topete na Baía de Todos os Santos.} de onde tomou o nome a terra firme deste
limite. Este engenho de Tristão Ribeiro tem uma fresca ermida de Santa Ana. O outro
engenho está no cabo desta terra que é de Dinis Gonçalves Varejão,\footnote{ Em Varnhagen
(1851 e 1879), ``Luís Golçalves Varejão''.} em o qual tem outra igreja de Nossa Senhora do
Rosário, que é freguesia desse limite.

Deste engenho se torna a afeiçoar a terra fazendo ponta para o mar, que terá comprimento
de meia légua, e no cabo dela se chama a ponta de Thomas Alegre, até onde está tudo
povoado de fazendas e canaviais, em que entra uma casa de meles de Marcos da Costa.
Defronte desta ponta está o fim da ilha de Maré, daqui torna a fugir a terra para dentro,
fazendo um modo de enseada em espaço de uma légua, que toda está povoada de nobres
fazendas e grandes canaviais, no cabo da qual está um formoso engenho de água de Thomas
Alegre, que tem uma ermida de Santo Antônio mui bem concertada. Deste engenho a uma légua
é o cabo Petinga, até onde está tudo povoado e plantado de canaviais mui formosos. Esta
Pitanga é uma ribeira assim chamada, onde se pode fazer um formoso engenho de água, o que
se não faz por haver contenda sobre a dita ribeira.

Por aqui se serve o engenho de Miguel Batista, que está pela terra dentro meia légua, o
qual tem mui ornados edifícios e uma ermida de Nossa Senhora mui concertada. E tornando
atrás ao esteiro e porto de Pitanga, torna a terra a correr para o mar obra de meia légua,
onde faz uma ponta em redondo, onde está uma formosa fazenda de André Monteiro, da qual
torna a terra a recuar para trás meia légua por um esteiro acima, que se diz de Metaripe,
onde está uma casa de meles de João Adrião, mercador; por este esteiro se serve a igreja,
e julgado do lugar de Tayaçupina, que está meia légua pela terra dentro em um alto à vista
do mar, povoação em que vivem muitos moradores que lavram neste sertão algodoais e
mantimentos,\footnote{ Em Varnhagen (1851 e 1879), ``sertão algodões e mantimentos''.} e a
igreja é da invocação de Nossa Senhora do Ó.\footnote{ No manuscrito da \textsc{bgjm}, ``de
Nossa Senhora''.}

\paragraph{[24] Em que se declara o sítio da terra da boca do esteiro de Metaripe até a
ponta de Mairape e dos engenhos que em si tem}\quad
Deste esteiro de Metaripe\footnote{ Em Varnhagen (1851 e 1879), ``Mataripe''.} ao de Caipe
será meia légua, ou menos, a qual está toda lavrada e aproveitada de muitos canaviais que
os moradores, que por esta terra vivem, têm feito. Neste esteiro de Caipe está um engenho
de bois de duas moendas, peça de muita estima, o qual é de Martim Carvalho, onde tem uma
ermida da Santíssima Trindade mui concertada com as mais oficinas necessárias.

Defronte deste esteiro de Caipe está um ilhéu de pedra meia légua no mar, que se diz
Itapitanga, do qual esteiro corre a terra quase direita obra de uma légua ou mais, no cabo
da qual está outro engenho de bois, fazenda muito grossa de escravos e canaviais, com
nobres edifícios de casas, com uma fresca igreja de Nossa Senhora das Neves, muito bem
acabada, o qual engenho é de André Fernandes Margalho, que o herdou de seu pai com muita
fazenda. Ao longo desta terra, um tiro de berço, está estendida a ilha de Cururupeba, que
é de meia légua de comprido, a qual é dos padres da Companhia, que a têm arrendada a sete
ou oito moradores que nela vivem.\footnote{ A Ilha de Cururupeba, assim nomeada por causa
do índio Cururupeba que ali habitava e que resistiu por anos aos conquistadores
portugueses, é atualmente a Ilha Madre de Deus, que faz parte do município de mesmo nome,
localizado na região metropolitana de Salvador.}

Entre esta ilha e a dos Frades estão duas ilhetas,\footnote{ Ilha de Bom Jesus dos Passos
e Ilha de Santo Antônio, que pertencem ao atual município de Salvador.} em cada uma das
quais está um morador, que as lavram, e são de Antônio da Costa. Deste engenho de André
Fernandes para cima vai fazendo a terra uma enseada de uma légua, no cabo da qual está o
esteiro de Parnamirim; e defronte desta enseada, bem chegadas à terra firme, estão três
ilhas; a primeira defronte do engenho, que é do mesmo André Fernandes, que tem perto de
meia légua, onde tem alguns moradores, que lavram canas e mantimentos; e junto desta ilha
está outra mais pequena, que é do mesmo, de onde tira lenha para o engenho; e mais avante
de Parnamirim está outra ilha, que se diz a das Fontes, que é de João Nogueira, a qual é
de meia légua, onde também vivem sete ou oito moradores. A terra de todas estas três ilhas
é alta e muito boa.\footnote{ No manuscrito da \textsc{bgjm}, ``de todas estas ilhas''.}
Na boca do esteiro de Parnamirim está um engenho de bois de Belchior Dias Porcalho, que
tem uma ermida de Santa Catarina. Por este esteiro de Parnamirim entra a maré uma légua,
no cabo da qual está outro engenho de bois de Antônio da Costa, que está muito bem
acabado. Este esteiro de uma parte e da outra está todo lavrado de canaviais e povoado de
formosas fazendas, no meio do qual está uma ilha de Vicente Monteiro, toda lavrada com uma
formosa fazenda. E tornado à boca deste esteiro, andando sobre a mão direita daí a uma
légua, está tudo povoado de moradores, onde tem muito boas fazendas de canaviais e
algodoais,\footnote{ Em Varnhagen (1851 e 1879), ``canaviais e algodões''.} a qual terra
se chama Tamarari, no meio da qual está uma igreja de Nossa Senhora, que é freguesia deste
limite. Esta terra faz no cabo uma ponta, e virando dela sobre a mão direita vai fugindo a
terra para trás, até dar em outro esteiro que chamam Mairape,\footnote{ Em Varnhagem (1851
e 1879), ``Marapé''.} onde se começam as terras de Mem de Sá, que agora são de seu genro,
o conde de Linhares.

\paragraph{[25] Em que se declara o rio de Serigipe, e terra dele à boca do Paragoasu}\quad
Partindo com a terra de Tamarari começa a do engenho do conde de Linhares, a qual está
muito metida para dentro fazendo uma maneira de enseada, a que chamam Mairape, a qual vai
correndo até a boca do rio de Sirigipe, e terá a grandura de duas léguas que estão
povoadas de mui grossas fazendas. Entra a maré por este rio de Sirigipe\footnote{ Em
Varnhagem (1851 e 1879), ``Seregipe''.} passante de três léguas, onde se mete uma ribeira
que se diz Tariri,\footnote{ Na edição de 1851, ``Tareiry'', e na de 1879, ``Traripe''.}
onde esteve já um engenho, que fez Antônio Dias Adorno, o qual se despovoou por lhe
arrebentar um açude, que lhe custou muito a fazer, pelo que está em mortuário;
mas não estará assim muito tempo, por ser a terra muito boa e
para se meter nela muito cabedal.

Descendo por este esteiro abaixo, légua e meia sobre a mão direita, está situado o afamado
engenho de Mem de Sá, que agora é do conde de Linhares, seu genro, o qual está mui
fabricado de casa forte e de purgar, com grande máquina de escravos e outras benfeitorias,
com uma igreja de Nossa Senhora da Piedade. Desta banda do engenho até a barra do rio que
podem ser duas léguas, não vive nenhum morador; por ser necessária a terra para o meneio
do engenho, e por ter perto da barra uma ribeira, onde se pode fazer outro engenho muito
bom; mas, da outra banda do rio, de cima até abaixo, está tudo povoado de muitas fazendas,
com mui formosos canaviais, entre os quais está uma, que foi
de Gonçalo Annes, que se meteu frade de São Bento, onde
os frades têm feito uma igreja do mesmo santo com seu recolhimento, onde dizem missa aos
vizinhos. Na boca deste rio, fora da barra dele, está uma ilha que chamam
Cajuaiba,\footnote{ Em Varnhagem (1851 e 1879), ``Cajaíba''. A Ilha de Caraíba, localizada
na Baía de Todos os Santos, pertence atualmente ao município de São Francisco do Conde.}
que será de uma légua de comprido e meia de largo, onde estão assentados dez ou doze
moradores, que nela têm bons canaviais e roças de mantimentos, a qual é do conde de
Linhares. Junto desta ilha está outra, pequena, despovoada, de muito boa terra. E, bem
chegado à terra firme, na boca do rio da banda do engenho, está outra ilha,\footnote{ Em
Varnhagem (1851 e 1879), ``no cabo do rio''.} de meia légua em quadra, por entre a qual e
a terra firme escassamente pode passar um barco, a qual também, com as duas atrás, são do
conde de Linhares. Da boca deste rio de Serigipe, virando ao sair dela sobre a mão
direita, vai fazendo a terra grandes enseadas, em espaço de quatro léguas, até onde chamam
o Açu,\footnote{ Em Varnhagem (1851 e 1879), ``Acúm''. Atual rio Açu.} por ter o mesmo
nome uma ribeira que ali se vem meter no salgado, na qual se podem fazer dois engenhos, os
quais não estão feitos por ser esta terra do engenho do conde de Linhares e não a querer
vender nem aforar, pelo que vivem poucos moradores nela, onde o
conde tem um formoso curral de vacas. Do cabo desta terra do conde à boca do rio Paragoaçu
são três ou quatro léguas, despovoadas de fazendas, por a terra ser fraca e não servir
para mais que para criação de vacas, onde estão alguns currais delas.

Esta terra foi dada a Brás Fragoso de sesmaria e pelo rio de Paragoaçu acima quatro
léguas; a qual se vendeu a Francisco de Araújo, que agora a possui com algumas fazendas
que nela fez, onde a terra é boa, que é pelo rio acima.

\paragraph{[26] Em que se declara a grandeza do rio Paraguaçu e os seus engenhos na terra
del"-rei}\quad
Este rio de Paraguaçu é mui caudaloso e terá na boca de terra à terra um tiro de falcão,
pelo qual entra a maré, que sobe por ele acima seis léguas; e de uma banda e da outra até
a ilha dos Franceses, que são duas léguas, é a terra alta e fraca e mal povoada, salvo de
alguns currais de vacas. Da barra deste rio para dentro está uma ilha de meia légua de
comprido e de quinhentas braças de largo e há partes de menos, a qual se chama de Gaspar
Dias Barbosa,\footnote{ Atual Ilha de Monte Cristo, na Baía de Todos os Santos.} cuja
terra é baixa e fraca. E tornando acima no cabo destas duas léguas está uma ilha, que
chamam dos Franceses, mui alterosa, que terá em roda seiscentas braças, onde eles nos anos
atrás chegavam com suas naus por ter fundo para isso,\footnote{ Em Varnhagem (1851 e
1879), ``onde eles chegaram em tempo atrás''.} e estavam nesta ilha seguros do gentio, com
o qual faziam dela seus resgates à vontade. Desta ilha para cima se abre uma formosa baía,
até a boca do rio da Água Doce, que serão duas léguas; e defronte desta ilha dos Franceses
está uma casa de meles de Antônio Peneda. E saindo desta ilha para fora, pondo a vista
sobre a mão direita, faz este rio um recôncavo de três léguas, coisa mui formosa, a que
chamam Uguape; e olhando para a mão esquerda se estende perto de duas léguas, parte das
quais estão ocupadas com três ilhéus despovoados, mas cheios de arvoredo, que se podem
povoar, e de uma ilha de Antônio de Paiva, que está aproveitada com canaviais, onde a
terra firme se vai apertando, que ficará acima desta ilha o rio de terra a terra uma meia
légua. Mas, tornando à casa de meles de Antônio Peneda, virando dela para a enseada de
Uguape, sobre a mão direita, daqui a duas léguas, é a terra fraca e não serve senão para
currais de vacas. No meio deste caminho está uma ilha rasa, que Antônio Dias Adorno teve
já cheia de mantimentos; além da qual está outra ilha, a que chamam da Ostra; de onde se
tem tirado tanta quantidade que se fizeram de ostra mais de dez mil moios\footnote{ Moio:
medida de capacidade equivalente a sessenta alqueires ou a aproximadamente oitocentos
litros em Portugal.} de cal, e vai"-se cada dia tirando tanta que faz espanto, sem se
acabar. No cabo destas léguas começa a terra boa, que está povoada até o engenho de
Antônio Lopes Ulhoa, de muitos canaviais e formosas fazendas, no que haverá espaço de uma
légua. Este engenho mói com grande aferida, e está mui ornado com edifícios de pedra e
cal, e a ribeira com que mói se chama Ubirapitanga. E indo deste engenho para cima, sobre
a mão direita ao longo do salgado, vai povoada a terra de fazendas e canaviais, em que
entra uma casa de meles de Antônio Rodrigues, e andando assim até junto do rio da Água
Doce do Paraguaçu, que podem ser duas léguas, vão dar com o notável e bem assentado
engenho de João de Brito de Almeida, que está senhoreando esta baía com a vista, o qual
engenho é de pedra e cal, e tem grandes edifícios de casas, e muito formosa igreja de São
João, de pedra e cal, o qual engenho tem mui grande aferida e mói com uma ribeira que vem
a este sítio por uma levada\footnote{ Levada: elevação de terreno.} de uma légua, feita
toda por pedra viva ao picão, com suas açudadas, com muros e botaréus\footnote{ Botaréu:
pilastra, escora.} de pedra e cal, coisa muito forte. E antes de se chegar a este engenho,
junto da terra dele estão três ilhéus de areia pequenos, cheios de mangues, onde se vai
mariscar. Acima deste engenho um tiro de berço dele, entra nesta baía, que este rio aqui
faz, o rio da Água Doce do Paraguoasu,\footnote{ Em Varnhagem (1851 e 1879),
``Paraguaçú''.} o qual terá na boca de terra a terra um tiro de falcão de espaço, e
navega"-se por ele acima até a cachoeira que pode ser a três léguas, com barcos grandes; e
indo por ele acima sobre a mão direita tem poucas fazendas, por ser a terra do engenho de
João de Brito. E antes de chegarem à cachoeira, à vista dele, está outro engenho de água
mui bem acabado, o qual fez um Rodrigo Martins, mameluco, por sua conta, e de Luís de
Brito de Almeida, junto do qual vivem muitos mamelucos com suas fazendas.

\paragraph{[27] Em que se declara a terra do rio de Paragoaçu, tocante à capitania de Dom
Álvaro}\quad
Até agora tratamos neste capítulo atrás da grandeza do rio de Paragoasu, no tocante à
terra del"-rei, e daqui por diante convém tratar do mesmo rio, e declarar a terra da outra
banda, que é da capitania de Dom Álvaro da Costa, que tem da boca da barra deste rio por
ele acima dez léguas de terra, e ao longo do mar da baía até o rio de Jaguoaripe, e por
ele acima, outras dez léguas; de que el"-rei D. João lhe fez mercê, com título de capitão e
governador desta terra, de que diremos neste capítulo.

Começando da cachoeira deste rio de Paragoasu para baixo, descendo sobre a mão direita, o
qual rio está povoado de muitos moradores por onde faz muitos esteiros, em que se metem
outras ribeiras, sem haver ainda nenhum engenho; e saindo pela boca fora deste rio à baía
que o salgado nele faz, e virando sobre a mão direita obra de uma légua ao longo das ilhas
de que já dissemos, se vai dar no braço que se diz de Igarusu; e por ele acima espaço de
duas léguas vai o rio mui largo, cuja terra da parte esquerda é fraca, de campinas, e mal
povoada de fazendas, e da banda direita é a terra boa,\footnote{ No manuscrito da
\textsc{bgjm}, ``mal povoada, e da banda''.} mas muito fragosa e povoada de fazendas. No
cabo destas duas léguas se aparta este rio em três braços, por onde entra a maré. E no
braço da mão direita está o engenho de Lopo Fernandes, obra mui forte, e de pedra e cal
assim o engenho como os mais edifícios, e a igreja, que é de Nossa Senhora da Graça, obra
mui bem acabada, com seus canaviais ao redor do engenho, de que faz muito açúcar. Pelo
braço do meio vai subindo a maré duas léguas, ao cabo das quais se mete nele uma formosa
ribeira de água que se diz Igaraçu, onde pode fazer um engenho; e de uma banda e da outra
é tudo povoado de roças e canaviais. Na ponta desta terra entre um esteiro e outro está
uma ermida de São João; e pelo outro esteiro, que está à mão esquerda, está um próspero
engenho de pedra e cal, com grandes edifícios de casas de vivenda e de purgar, e uma
formosa igreja. Este engenho é copioso como os mais do rio, o qual edificou Antônio
Adorno, cujos herdeiros o possuem agora. Neste rio de Paragoaçu e em todos os seus
recôncavos, por onde entra o salgado, há muito marisco de toda a sorte, especialmente
ostras muito grandes, onde em uma maré vazia quatro negros carregam um barco delas, e tem
grandes pescarias, assim de rede como de linha, especialmente na baía que faz abaixo; que
por uma banda tem duas léguas de comprido e por outra duas de largo, pouco mais ou menos,
e em toda a terra deste rio há muita caça.

\paragraph{[28] Em que se declara o como corre a terra do rio de Paragoaçu ao longo do mar
da Bahia até a boca de Jagoaripe e por este rio acima}\quad
Da boca do rio Paragoaçu,\footnote{ Em Varnhagem (1851 e 1879), ``Do cabo do rio
Paraguaçu''.} onde se ele mete na baía grande, vai fazendo a terra umas enseadas de areia,
obra de duas léguas, que estão povoadas de currais de vacas e de pescadores, e no cabo
destas duas léguas faz a terra uma ponta de areia muito saída ao mar da baía a qual corta
a maré a passos; e quando é cheia, fica parte desta ponta em ilha e passada da outra banda
tem sete ou oito ilhéus de areia cheios de mangues; e tornando a correr a costa contra
Jagoaripe,\footnote{ Em Varnhagem (1851 e 1879), ``Jaguaripe''.} se vai armando em
enseadas obra de três léguas que estão povoadas, até em direito da ilha da Pedra, de
currais de vacas e fazendas de gente pobre, que não plantam mais que mantimentos, de que
se mantêm. Esta ilha da Pedra é de pouco mais de meia légua de comprido e tem muito menos
de largura; e mais avante está outra ilha que tem mais de légua de comprido,\footnote{ No
manuscrito da \textsc{bgjm}, ``meia légua de comprimento''.} que se diz a de Fernão Vaz.
Por detrás desta ilha vai correndo a costa da terra firme mui chegada e elas, a qual costa
por detrás destas ilhas terá três léguas de espaço até chegar ao rio de Jagoaripe, tudo
terra despovoada por ser fraca de campinas, onde se mete no salgado uma ribeira, que
chamam Pujuqua\footnote{ Em Varnhagem (1851 e 1879), ``Pojuca''.} que servirá para um
engenho, ainda que junto ao porto vem a água baixa, e será necessário fazer"-se o engenho
um pedaço pela terra adentro, por amor da aferida. E virando da boca de Jagoaripe para
cima, daí a duas léguas, é a terra mui fraca, que não serve senão para vacas e roças de
mantimentos; e do cabo destas duas léguas até a cachoeira é a terra sofrível e tem cinco
ribeiras, que se vêm meter neste rio, em que se podem fazer cinco engenhos, os quais não
são já feitos por o capitão desta terra não querer dar as águas menos de dois por cento do
foro, que no cabo de um ano vem a montar oitenta a cem arrobas de açúcar, que valem a
oitocentos réis cada arroba.

Este rio de Jagoaripe é tamanho como o Douro, mas mais aprazível na frescura; navega"-se
até a cachoeira que está cinco léguas da barra e duas léguas abaixo da cachoeira é água
doce, a qual o salgado com a força da maré faz recuar até a cachoeira. Junto da cachoeira
virando sobre a mão direita, para baixo, está um engenho de água de Fernão Cabral de
Ataíde, obra mui formosa e ornada de nobres edifícios de casas de vivenda e de outras
oficinas e de uma igreja de São Bento mui bem acabada, o qual engenho está feito nas
terras del"-rei que estão livres de todo o foro, que costumam pôr os todos os capitães.
Deste engenho para baixo vivem alguns
moradores que têm suas roças e canaviais ao longo do rio, que o aformoseiam muito, em o
qual se vêm meter três ribeiras por esta mesma banda capazes de três engenhos que se nelas
podem mui bem fazer, duas léguas abaixo de Fernão Cabral. A mais terra desta banda até a
Barra é rasa e de areia,\footnote{ Em Varnhagem (1851 e 1879), ``mas a terra desta banda é
rasa''.} que não serve para mais que para lenha dos mesmos engenhos, a qual terra fica no
cabo em língua estreita defronte da ilha de Fernão Vaz, a qual ponta tem uma ilhota no
cabo, onde se vem ajuntar o rio de Irajahe\footnote{ Em Varnhagem (1851 e 1879),
``Irajuhi''.} com o de Jagoaripe.

\paragraph{[29] Em que se explica o tamanho e formosura do rio Irajahe e seus recôncavos}\quad
Correndo por esta ponta de entre ambos os rios acima com a mão direita ao longo da mesma
terra, da ponta duas léguas pelo rio acima, é a terra fraca, que não serve senão para
lenha dos engenhos; daqui para cima uma légua da cachoeira deste rio, é tudo povoado de
canaviais e fazendas de moradores, até onde a água salgada se mete por dois esteiros
acima, onde se ajuntam com ele duas ribeiras de água, em as quais estão dois engenhos, os
quais deixemos estar para dizermos primeiro do rio de Irajahe, que vai por este meio um
quarto de légua para cima, povoado de canaviais e fazendas em que entra uma casa de meles
de muita fábrica de Gaspar de Freitas, além da qual, junto à cachoeira, está situado o
engenho de Diogo Correa de Sande, que é uma das melhores peças da Bahia, porque está mui
bem acabado, com grandes aposentos e outras oficinas, e uma fresca igreja de Vera
Cruz.\footnote{ No manuscrito da \textsc{bgjm}, ``oficinas, e uma igreja de Vera Cruz''.}

E tornando abaixo ao esteiro da mão direita, que se chama Caipe, indo por ele acima, está
um soberbo engenho com grandes casas de purgar e de vivenda, e muitas outras oficinas, com
uma grande e formosa igreja de São Lourenço, onde vivem muitos vizinhos em uma povoação
que se diz a Graciosa. Esta terra é muito fértil e abastada de todos os mantimentos e de
muitos canaviais de açúcar, a qual é de Gabriel Soares de Sousa; e deste engenho ao de
Diogo Correa não há mais distância que quatrocentas braças de caminho de carro, e para
vizinharem se servem os carros de um engenho ao outro por cima de duas pontes, e
atravessam estes rios e ficam os engenhos à vista um do outro.

E tornando ao outro esteiro que fica da outra banda do rio de Irajahe, onde se mete a
ribeira que se diz de Jaceru, com a qual mói outro engenho que agora novamente fez o mesmo
Diogo Correa, o qual está mui bem acabado e aperfeiçoado com as oficinas necessárias; e
todo este esteiro está povoado de fazendas de moradores com formosos canaviais; e descendo
por este rio abaixo ao longo da terra da mão direita, andando mais de uma légua, vai a
terra povoada da mesma maneira, onde este rio é como o Tejo de Vila Franca para cima.

E daqui até em direito da ponta que divide este rio de Jagoaripe é a terra fraca, onde há
três esteiros que entram por ela dentro duas léguas, em os quais se metem ribeiras com que
se podem moer engenhos; mas a terra não é capaz para dar muitos anos canas. E abaixo deste
esteiro está uma ilheta que chamam a do Sal, porque o gentio, quando vivia mais perto do
mar, costumava"-o fazer ali, defronte da qual está outra ilheta no cabo da ponta dentre
ambos os rios.\footnote{ Em Varnhagem (1851 e 1879), ``da ponta de ambos os rios''.}
Desta ilha até a ponta da barra haverá uma légua, tudo terra de pouca substância.

Desta terra à ilha de Fernão Vaz é perto de uma légua, e entre esta ilha e a de Taparica e
a terra firme fica quase em quadra uma baía de uma légua, onde se mete a barra que se
chama de Jagoaripe, de que se fez já menção.

\paragraph{[30] Em que se declara a terra que há da boca da barra de Jagoaripe até
Juquirixaque, e daí até o rio de Una}\quad
Da ponta da barra de Jagoaripe ao rio de Juquirixaque são quatro léguas ao longo do
mar,\footnote{ Em Varnhagen (1851 e 1879), ``barra de Jaguaripe'' e ``rio de
Juquirijape''.} à feição de enseadas, quase pelo rumo de norte"-sul, cuja terra é baixa e
fraca, com pouco mato, pelo qual atravessam das campinas quatro ribeiras de pouco cabedal,
a qual terra não serve para mais que para criações de vacas. Este rio de Juquirixaque tem
a barra pequena e baixa, por onde não podem entrar mais que caravelões da costa por ter
uma légua na boca que a toma toda; da barra para dentro até a cachoeira é muito fundo, por
onde podem navegar navios de cem tonéis e de mais; e de uma parte a outra pode haver
quatro léguas. Este rio é tão formoso como o de Guoadiana, mas tem muito mais fundo; e
tem, indo por ele acima, de uma banda e da outra até duas léguas, a terra fraca e pela
maior parte de campinas, com muitos alagadiços, terra boa para vacas; e tem, indo por ele
acima mais avante dois esteiros,\footnote{ No manuscrito da \textsc{bgjm}, ``por ele mais
avante''.} se podem fazer dois engenhos.

Do esteiro mais do cabo, para a banda da cachoeira uma légua toda de várzea, e terra mui
grossa para canaviais; da outra banda é a terra mais somenos, e junto desta cachoeira se
vem meter uma ribeira com grande aferida, onde Gabriel Soares\footnote{ Em Varnhagem
(1851), ``Gabriel Torres''.} tem começado um engenho, em o qual tem feito grandes
benfeitorias, e assentado uma aldeia de escravos com um feitor que os manda. Na barra
deste rio tem uma roça com mantimentos, e gente com que se granjeia. Este rio é muito
provido de pescado, marisco e muita caça, e frutas silvestres.

Da barra de Juquirixaque ao curral de Bastião da Ponte serão cinco léguas ao longo do mar,
tudo despovoado em feição de enseada, onde se metem três ribeiras que nascem nas campinas
desta terra, que não servem para mais que para criação de vacas. Toda esta baía e costa no
inverno é mui desabrigada até a barra de Jagoaripe,\footnote{ Em Varnhagem (1851 e 1879)
``toda esta praia e costa''.} onde o tempo leste e lés"-sudeste é travessia, e se toma
aqui os caravelões da costa que se servem por esta barra, e eles não acertam com a boca de
Juquirixaque para se recolherem dentro, não têm outro remédio senão varar em terra, onde
não há perigo para as pessoas por ser tudo areia.\footnote{ No manuscrito da
\textsc{bgjm}, ``não há perigo por ser tudo areia''.} Este curral de Bastião da Ponte
está em uma ponta saída ao mar com o rosto no morro de Tinhare, da qual vai fugindo a
terra para dentro, fazendo uma enseada até o rio Una, que será três léguas todas de praia.
Por este rio entra a maré mais de duas léguas, no cabo das quais está situado o engenho de
Bastião da Ponte, que tem duas moendas de água em uma casa que mói ambas com uma ribeira,
o qual engenho é mui grande e forte e está mui bem fabricado de casas de vivenda e de
purgar e outras oficinas, com uma formosa igreja de São Gonçalo,\footnote{ Em Varnhagem
(1851 e 1879), ``igreja de São Gens''.} com três capelas de abóbada; e por este rio Una
vivem alguns moradores que nele têm feito grandes fazendas de canaviais e mantimentos.

\paragraph{[31] Em que se explica a terra do rio Una até Tinhare, e da ilha de Taparica com
outras ilhas}\quad
Da boca do rio Una a uma légua se mete no mar outro rio, que se diz Taraire,\footnote{ Em
Varnhagem (1851 e 1879), ``Tairiri''.} pelo qual entra a maré duas ou três léguas, onde
Fernão Ribeiro de Sousa\footnote{ Em Varnhagem (1851 e 1879), ``Fernão Rodrigues de
Sousa''. É provável que seja Ribeiro, pois em um relatório de 1609 elaborado pelo
sargento"-mor Diogo de Campos aparece, entre os proprietários de engenho dessa mesma
região, Fernão Ribeiro de Sousa. “Relação das Praças Fortes do Brasil de Diogo de Campos
Moreno” (1609), \textit{Revista do Instituto Arqueológico, Histórico e Geográfico
Pernambucano}, vol.57, 1984.} fez uma populosa fazenda com um engenho mui bem acabado e
aperfeiçoado, com as oficinas acostumadas e uma igreja de Nossa Senhora do Rosário mui bem
concertada, onde tem muitos homens de soldo para se defenderem da praga dos Aimores, que
lhe fizeram já muito dano. E tornado à boca deste rio, que
está mui vizinho da ilha de Tinhare, de onde vai correndo até o morro, fazendo uma enseada
de obra de três léguas até a ponta do morro, onde se acaba o que se entende a Bahia de
Todos os Santos. Esta ilha faz abrigada a esta terra até a ponta do Curral, por a sua
terra ser alta, a qual é fraca para canaviais, onde vivem alguns moradores, que nela estão
assentados da mão de Domingos Saraiva, que é o senhor desta ilha, o qual vivia nela e tem
aí sua fazenda com grandes criações e uma ermida onde lhe dizem missa. Da boca deste rio
de Tareiri a esta ilha pode ser um tiro de falcão. No mar que há entre esta ilha e a terra
firme, há grandes pescarias e muito marisco, onde por muitas vezes no inverno lança o mar
fora nesta ilha e nas praias de defronte até o Juquirixaque âmbar gris muito bom.

Tornado à ilha de Taparica, de que atrás se faz menção pela banda de Tinhare, não tem
porto onde se possa desembarcar, por ser cercada de baixos de pedra, onde o mar quebra
ordinariamente, a qual, pela banda de dentro da baía, tem muitos portos, onde os barcos
podem desembarcar com todo tempo. Tem esta ilha, pela banda de dentro, grandes pontas e
enseadas, onde com tormenta se recolhem as embarcações que vêm das outras partes da baía
para a cidade.

Na ponta desta ilha de Taparica defronte da barra de Jagoaripe está uma ilheta junto a
ela, que se diz de Lopo Rebelo, que está cheia de arvoredo, de onde se tira muita madeira.
E daqui para dentro é povoada Taparica de alguns moradores, que vivem ao longo ao
mar,\footnote{ Em Varnhagem (1851 e 1879), ``que vivem junto ao mar''.} que lavram canas e
mantimentos, e criam vacas. E daqui até Tamatarandiba\footnote{ Em Varnhagem (1851 e
1879), ``Tamarantiba''.} serão duas léguas de costa desta ilha, entre a qual e a de
Tamatarandiba haverá espaço de um tiro de falcão. Esta ilha de Tamatarandiba tem uma légua
de comprido, e meia de largo, cuja terra não serve para mais que para mantimentos, onde
vivem seis ou sete moradores, a qual é do conde de Castanheira. Junto de Tamatarandiba, da
banda da terra firme, está uma ilheta de São Gonçalo, cheia de arvoredo,\footnote{ Em
Varnhagem (1851 e 1879), ``está uma ilheta cheia de arvoredo''.} muito rasa, cuja terra é
fraca e de areia, onde o mais do tempo estão diferentes pescadores de rede, por haver ali
muitos lanços;\footnote{ Lanço: porção de peixe recolhida por uma rede.} e diante dela
estão três ilhéus rasos, fazendo uma ponta ao mar contra a outra que vem da banda do
Paragoasu, e pode haver de uns aos outros uma légua; do mar contra a ponta de Taparica
está outro ilhéu raso com arvoredo, que não serve senão a pescadores de redes. No cabo da
ilha Tamatarandiba, entre ela e a de Taparica, estão três ilhéus de areia pequenos, e
junto deles está uma ilheta, que chamam dos Porcos, que será de seiscentas braças em
quadro. Mais avante, junto da terra de Taparica está outra ilheta, que se diz de João
Fidalgo, onde vive um morador. Avante desta ilheta, em uma enseada grande que Taparica
faz, está um engenho de açúcar que lavra com bois, o qual é de Gaspar Pacheco, por cujo
porto se servem os moradores que vivem pelo sertão da ilha, onde tem uma igreja de Santa
Cruz; e deste engenho a duas léguas está a ponta de Taparica, que é mais saída ao mar, que
se chama ponta da Cruz, até onde está povoada a ilha de moradores, que lavram mantimentos
e algumas canas. Desta ponta a uma légua ao norte está uma ilha que se diz a do Medo, cuja
terra é rasa e despovoada por ser de areia e não ter água.

Da ponta de Taparica se torna a recolher a terra fazendo rosto para a cidade, a qual está
toda povoada de moradores que lavram muitos mantimentos e canaviais. E na fazenda de
Fernão de Sousa\footnote{ No manuscrito da \textsc{bgjm}, ``Simão de Sousa''.} está uma
igreja mui bem concertada, da advocação de Nossa Senhora, onde os vizinhos desta banda têm
missa aos domingos e dias santos.

E por aqui temos concluído com a redondeza da Bahia e suas ilhas, que são trinta e nove, a
saber, vinte e duas ilhas e dezessete ilhéus, fora as ilhas que há dentro nos rios, que
são dezesseis entre grandes e pequenas, que juntas todas fazem a soma de cinquenta e
cinco; e tem a baía, da ponta do Padrão, andando"-a por dentro sem entrar nos rios, até
chegar à ponta do Tinhare, cinquenta e três léguas.

\paragraph{[32] Em que se contém quantas igrejas, engenhos e embarcações tem a Bahia}\quad
Pois que acabamos de explicar a grandeza da Bahia e seus recôncavos, convém que lhe
juntemos o seu poder, não tratando da gente, pois o fizemos atrás.

Mas comecemos nos engenhos, nomeando"-os em suma, ainda que particularmente se dissesse de
cada um seu pouco, havendo que dizer deles e de sua máquina muito, os quais são moentes e
correntes trinta e seis, convém a saber: vinte e um que moem com água e quinze que moem
com bois, e quatro que se andam fazendo. Tem mais oito casas de cozer meles de muita
fábrica e mui proveitosas. Saem da Bahia cada ano destes engenhos passante de cento e
vinte mil arrobas de açúcar, e muitas conservas. Tem a Bahia com seus recôncavos sessenta
e duas igrejas, em que entra a Sé e três mosteiros de religiosos, das quais são dezesseis
freguesias curadas, a saber: nove vigararias que paga Sua Majestade e outras sete pagam
aos curas os fregueses, e a maior parte das outras igrejas têm capelães e suas confrarias,
como em Lisboa; e todas essas igrejas estão mui concertadas, limpas e providas de
ornamentos, em as quais, nos dias da festa dos oragos,\footnote{ Em Varnhagem (1851 e
1879), ``nos dias dos oragos''.} se lhe faz muita
festa. Todas as vezes que cumprir ao serviço de Sua Majestade se ajuntarão na Bahia
oitocentas embarcações,\footnote{ Em Varnhagem (1851 e 1879), ``na Bahia mil e
quatrocentas embarcações''.} a saber: de quarenta e cinco para setenta palmos de quilha,
cem embarcações mui fortes, em cada uma das quais podem jogar dois falcões por proa e dois
berços por banda; e de quarenta e quatro palmos de quilha até trinta e cinco se ajuntarão
duzentas embarcações,\footnote{ Em Varnhagem (1851 e 1879), ``se juntarão oitocentas
embarcações''.} em as quais pode jogar pelo menos um berço por proa; e, se cumprir
ajuntarem"-se as mais pequenas embarcações, ajuntar"-se"-ão trezentos barcos de trinta e
quatro palmos de quilha para baixo, e mais de duzentas canoas, e todas estas embarcações
mui bem remadas. E são tantas as embarcações na Bahia, porque se servem todas as fazendas
por mar, e não há pessoa que não tenha seu barco, canoa pelo menos, e não há engenho que
não tenha de quatro embarcações para cima; e ainda com elas não são bem servidos.

\paragraph{[33] Em que se começa a declarar a fertilidade da Bahia e como se nela dá o gado
da Espanha}\quad
Pois se tem dado conta tão particular da grandura da Bahia de Todos os Santos e do seu
poder, é bem que digamos a fertilidade dela um pedaço, e como produz em si as criações das
aves e alimárias de Espanha e os frutos dela, que nesta terra se plantam.

Tratando em suma da fertilidade da terra, digo que acontece muitas vezes valer mais a
novidade de uma fazenda que a propriedade, pelo que os homens se mantêm honradamente com
pouco cabedal, se se querem acomodar com a terra e remediar com os mantimentos dela, do
que é muito abastada e provida.

As primeiras vacas que foram à Bahia levaram"-nas de Cabo Verde e depois de Pernambuquo, as
quais se dão de feição que parem cada ano e não deixam nunca de parir por velhas; as
novilhas, como são de ano, esperam o touro, e aos dois anos vêm paridas, pelo que acontece
muitas vezes mamar o bezerro na novilha e a novilha na vaca, juntamente, o que se também
vê nas éguas, cabras, ovelhas e porcas; e porque as novilhas esperam o touro de tão tenra
idade, se não consentem nos currais os touros velhos, porque são pesados e derreiam as
novilhas, quando as tomam; as vacas são muito gordas e dão muito leite, de que se faz
muita manteiga e as mais coisas de leite que se fazem na Espanha; e depois de velhas criam
algumas no bucho umas maças tamanhas como uma pela e maiores, e quando são ainda novas têm
o carão de fora como o couro da banda do carnaz;\footnote{ Carnaz: lado interno do couro
dos animais, oposto ao pelo.} as peles das mais velhas são pretas e lisas que parecem
vidradas no resplandor e brandura, umas e outras são muito leves e duras, e dizem que têm
virtude.

As éguas foram à Bahia de Cabo Verde, das quais se inçou a terra, de modo que, custando em
princípio a sessenta mil Réis e mais,\footnote{ No manuscrito da \textsc{bgjm}, ``inçou a
terra, em princípio valiam sessenta mil Réis''.} pelo que levaram lá muitas todos os anos
e cavalos, multiplicaram de maneira que valem agora a dez e a doze mil Réis; e há homens
que têm em suas granjearias quarenta e cinquenta, as quais parem cada ano; e esperam o
cavalo poldras de um ano, como as vacas, e algumas vezes parem duas crianças juntas. São
tão formosas as éguas da Bahia como as melhores da Espanha, das quais nascem formosos
cavalos e grandes corredores, os quais, até a idade de cinco anos, são bem acondicionados,
e pela maior parte como passam daqui criam malícia e fazem"-se mui desassossegados, mal
arrendados e ciosos; assim eles como as éguas andam desferrados, mas não faltam por isso
em nada por serem mui duros de cascos. Da Bahia levam os cavalos a Pernambuco por
mercadoria, onde valem duzentos e trezentos cruzados e mais.

Os jumentos se dão da mesma maneira que as éguas, mas são de casta pequena; os cavalos não
querem tomar as burras por nenhum caso, mas os asnos tomam as éguas por invenção e
artifício, por elas serem grandes e eles pequenos que lhe não podem chegar, e as éguas
esperamos bem, pelo que há poucas mulas, mas estas que ainda
que são pequenas, são muito formosas, bem feitas e de muito trabalho.

As ovelhas foram de Portugal e as cabras de Cabo Verde,\footnote{ Em Varnhagem (1851 e
1879), ``As ovelhas e as cabras foram de Portugal e Cabo Verde''.} as quais se dão muito
bem, umas e outras parem, tirada a primeira paridura, duas crianças e muitas vezes três,
as quais emprenham como são de quatro meses e parem cada ano pelo menos duas vezes, cuja
carne é sempre muito gorda, mui sadia e saborosa; e quanto mais velha é melhor, e umas e
outras dão muito e bom leite, de que se fazem queijos e manteiga.

Os cordeiros e cabritos são sempre muito gordos e saborosos; a carne dos bodes é gorda e
muito dura; a dos carneiros é magra enquanto são novos e depois de velhos não tem preço; e
criam sobre o cacho uma carne como ubre de vacas de três dedos de grosso.

As porcas parem infinidade de leitões, os quais são muito tenros e saborosos, e como a
leitoa é de quatro meses espera o macho, pelo que multiplicam coisa de espanto, porque
ordinariamente andam prenhes, de feição que parem três vezes por ano, se lhe não falta o
macho. A carne dos porcos é muito sadia e saborosa, a qual se dá aos doentes como galinha,
e come"-se todo o ano por em nenhum tempo ser prejudicial, mas não fazem os toucinhos tão
gordos como em Portugal, salvo os que se criam nas capitanias de São Vicente e nas do Rio
de Janeiro.

As galinhas da Bahia são maiores e mais gordas que as de Portugal, e grandes poedeiras e
muito saborosas; mas é de espantar que, como são de três meses, esperam o galo, e os
frangões da mesma idade tomam as fêmeas, os quais são feitos galos e tão tenros, saborosos
e gordos como se não viu em outra parte.

As pombas da Espanha se dão muito bem na Bahia,\footnote{ Em Varnhagem (1851 e 1879), ``se
dão na Bahia''.} mas fazem"-lhes muito nojo as cobras que lhes comem os ovos e os filhos,
pelo que se não podem criar em pombais.

Os galipavos\footnote{ Peru.} se criam e também fazem tão formosos como na Espanha, e
d'avantagem, cuja carne é muito gorda e saborosa, os quais se criam sem mais cerimônias que
as galinhas. E também se dão muito bem os patos e gansos da Espanha, cuja carne é muito
gorda e saborosa.

\paragraph{[34] Em que se declara as árvores da Espanha que se dão na Bahia, e como se criam
nela}\quad
Parece razão que se ponha em capítulo particular os frutos da Espanha e de outras partes,
que se dão na Bahia de Todos os Santos.

E comecemos nas canas"-de"-açúcar, cuja planta levaram à capitania dos Ilheos das ilhas da
Madeira e de Cabo Verde, as quais recebeu esta terra de maneira em si, que as dá melhores
e maiores que nas ilhas e parte de onde vieram a ela, e que em nenhuma outra parte que se
saiba que crie canas"-de"-açúcar, porque na ilha da Madeira, Cabo Verde, São Tomé, Trudente,
Canárias, Valência e na Índia não se dão as canas se se não regam os canaviais como as
hortas e se lhes não estercam as terras, e na Bahia plantam"-se pelos altos e pelos baixos,
sem se estercar a terra, nem se regar; e como as canas são de seis meses, logo acamam e é
necessário cortá"-las para plantar em outra parte,\footnote{ Em Varnhagem (1851 e 1879),
``e é forçoso cortá"-las''.} porque aqui se não dão tão compridas como lanças; e na terra
baixa não se faz açúcar da primeira novidade que preste para nada, porque acamam as canas
e estão tão viçosas que não coalha o sumo delas, se as não misturam com canas velhas, e
como são de quinze meses, logo dão novidade às canas de planta; e as de soca,\footnote{
Cana"-planta e cana"-soca, primeiro e segundo cortes da cana"-de"-açúcar, respectivamente.}
como são de ano, logo se cortam. E na ilha da Madeira e nas mais partes onde se faz açúcar
cortam as canas de planta de dois anos por diante e a soca de três anos, e ainda assim são
canas mui curtas, onde a terra não dá mais que duas novidades. E na Bahia há muitos
canaviais que há trinta anos que dão canas; e ordinariamente as terras baixas nunca cessam
e as altas dão quatro e cinco novidades e mais.

Das árvores a principal é a parreira, a qual se dá de maneira nesta terra, que nunca lhe
cai a folha, senão quando a podam que lha lançam fora; e quantas vezes a podem, tantas dá
fruto; e porque duram poucos anos com a fertilidade, se as podam muitas vezes no ano; é a
poda ordinária duas vezes para darem duas novidades, o que se faz em qualquer tempo do ano
conforme ao tempo que cada um quer as uvas, porque em todo o ano madurecem e são muito
doces e saborosas, e não amadurecem todas juntas;\footnote{ No manuscrito da
\textsc{bgjm}, ``e amadurecem todas juntas''.} e há curiosos que têm nos seus jardins pé
de parreira que têm uns braços com uvas maduras, outros com agraços,\footnote{ Agraço: uva
verde.} outros com frutos em flor e outros podados de novo; e assim em todo o
ano têm uvas maduras, em uma só parreira; mas não há naquela terra mais planta que de uvas
ferrais\footnote{ Ferral: da cor do ferro.} e outras uvas pretas, e se não há nessa terra
muitas vinhas é por respeito das formigas,
que em uma noite dão em uma parreira, lhe cortam a folha e fruto e o lançam no chão; pelo
que não há na Bahia tanto vinho como na ilha da Madeira, e como se dá na capitania de São
Vicente porque não tem formiga que lhe faça nojo, onde há homens que colhem já três e
quatro pipas de vinho cada ano, ao qual dão uma fervura no fogo por se lhe não azedar o
que deve de nascer das plantas.

As figueiras se dão de maneira que no primeiro ano que as plantam vêm como novidade e, daí
por diante, dão figos em todo o ano, às quais nunca cai a folha; e as que dão logo
novidade e figos em todo o ano são figueiras pretas, que dão mui grandes e saborosos figos
pretos e as árvores não são muito grandes, nem duram muito tempo, porque como são de
cinco, seis anos, logo se enchem de uns carrapatos que as comem, e lhes fazem cair a folha
e ensoar\footnote{ Ensoar: murchar.} o fruto, os quais figos pretos não criam
bicho como os de Portugal. Também há outras figueiras pretas que dão figos bêbaras mui
saborosos, as quais são maiores árvores e duram perfeitas mais anos que as outras, mas não
dão a novidade tão depressa como elas.

As romeiras que se plantam de quaisquer raminhos, os quais pegam e logo dão fruto aos dois
anos; as árvores não são nunca grandes, mas dão romãs em todo o ano, e não lhes cai nunca
a folha de todo; o fruto delas é maravilhoso no gosto e de bom tamanho, mas não dão muitas
romãs por pesarem muito e caírem no chão estando em frol,\footnote{Frol: flor} com as quais árvores têm as
formigas grande guerra, e não se defendem delas senão com testos\footnote{ Testo: vaso de
barro.} de água ao pé que fica no meio; e se se atravessa uma palha por cima, por ela lhe
dão logo tal assalto que lhe lançam a folha toda no chão; pelo que se sustentam com
trabalho estas árvores e as parreiras, que às figueiras não faz a formiga nojo.

As laranjeiras se plantam de pevide,\footnote{ Pevide: semente de alguns frutos.} e
faz"-lhes a terra tal companhia que em três anos se fazem árvores mais altas que um homem,
e neste terceiro ano dão fruto, o qual é o mais formoso e grande que há no mundo; e as
laranjas doces têm suave sabor, e é o seu doce mui doce, entanto que a camisa branca com
que se vestem os gomos é também muito doce. As laranjeiras se fazem muito grandes e
formosas, e tomam muita frol\footnote{Frol: flor.} de que se faz água muito fina e de
mais suave cheiro que a de Portugal; e como as laranjeiras doces são velhas, dão as
laranjas com uma ponta de azedo muito galante, às quais árvores as formigas em algumas
partes fazem nojo, mas com pouco trabalho se defendem delas. Tomam essas árvores a flor em
agosto, em que se começa naquelas partes a primavera.

As limeiras se dão da mesma maneira, onde há poucas que deem fruto azedo, por se não usar
dele na terra. As limas doces são muito grandes, formosas e muito saborosas, as quais
fazem muita vantagem às de Portugal, assim no grandor como no sabor. As árvores das limas
são tamanhas como as laranjeiras, a quem a formiga faz o mesmo dano, se lhes pode chegar,
e plantam"-se também de pevide.

As cidreiras se plantam de estaca, mas de pevide se dão melhor, porque dão fruto ao
segundo ano; e as cidras são grandíssimas e saborosas, as quais fazem muita vantagem às de
Portugal, assim no grandor como no sabor; e faz"-se delas muita conserva. Algumas têm o
amargo doce, outras azedo, e em todo o ano as cidreiras estão de vez\footnote{ De vez:
quase maduras.} para dar fruto, porque têm cidras maduras, outras verdes, outras pequenas
e muita frol; e quem as formigas não fazem nojo porque tem o pé da folha muito duro.

Dão"-se na Bahia limões franceses tamanhos, como cidras de Portugal, e são mui saborosos; e
outros limões"-de"-perdiz e os galegos, e uns e outros são mui saborosos e se plantam de
pevide,\footnote{ Em Varnhagem (1851 e 1879), ``e os galegos, e uns e outros se plantam
de''.} e todos aos dois anos vêm com novidade, os quais muito depressa se fazem árvores
mui formosas e tomam muito fruto, o qual dão em todo o ano, como está dito das cidreiras;
e alguns destes limoeiros se fazem muito grandes, especialmente os galegos.

Também se dão na Bahia outras árvores de espinho que chamam azamboas, de que não há muitas
na terra, por se não aproveitarem nela deste fruto.

As palmeiras das que dão os cocos, se dão na Bahia melhor que na Índia, porque, metido um
coco debaixo da terra, a palmeira que dele nasce dá coco em cinco e seis anos, e na Índia
não dão estas palmas fruto em vinte anos. Foram os primeiros cocos à Bahia do Cabo Verde,
de onde se enche a terra, e houvera infinidade deles se não se secaram, como são de oito e
dez anos para cima; dizem que lhes nasce um bicho no olho que os faz secar. Os cocos são
maiores que os das outras partes, mas não há quem lhes saiba matar este bicho, e
aproveitar"-se do muito proveito que na Índia se faz dos palmares, pelo que não se faz
nesta terra conta destas árvores.

Tamareiras se dão na Bahia muito formosas, que dão tâmaras mui perfeitas; as primeiras
nasceram dos caroços que foram do Reino e as depois de semeadas e nascidas daí a oito
anos, deram fruto, e dos caroços deste fruto há outras árvores que dão já, mas não faz
ninguém conta delas, e pode"-se contar por estranheza esta brevidade porque se tem que quem
semeia estas tâmaras ele nem seus filhos lhe comem o fruto senão seus netos. Estas
tamareiras não dão frutos se não houver macho entre elas, e a árvore que é macho não dá
fruto e é mui ramalhuda do meio para cima, e as folhas são de cor verde"-escuro; as fêmeas
têm uma copa em cima e a cor dos ramos é de um verde"-claro.

\paragraph{[35] Em que se contém de outros frutos estrangeiros que se dão na Bahia}\quad
Da ilha de São Tomé levaram à Bahia gengibre, e começou"-se de plantar obra de meia arroba
dele, repartindo por muitas pessoas, o qual se deu na terra de maneira que daí a quatro
anos se colheram mais de quatro mil arrobas, o qual é com muita vantagem do que vem da
Índia, em grandeza e fineza, porque se colheu dele penca que pesava dez e doze
arráteis,\footnote{ Arrátel: medida de peso correspondente a 459 gramas ou 16 onças.} mas
não o sabiam curar bem, como o da Índia, porque ficava denegrido, do qual se fazia muita e
boa conserva, do que se não usa já na terra por el"-rei defender que o não tirem para fora.
Como se isto soube o deixaram os homens pelos campos, sem o quererem recolher, e por não
terem nenhuma saída para fora apodreceram na terra muitas lájeas cheias dele.

Arroz se dá na Bahia melhor que em outra nenhuma parte sabida, porque o semeiam em brejos
e em terras enxutas, como for terra baixa é sem dúvida que o ano dê novidade; de cada
alqueire\footnote{ Alqueire: medida de capacidade, de volume variável, que equivalia a
aproximadamente 14 litros em Portugal.} de semeadura se recolhe de quarenta para sessenta
alqueires, o qual é tão graúdo e formoso como o de Valência;\footnote{ Em Varnhagem (1851
e 1879), ``é tão grado e formoso''.} e a terra em que se semeia se a tornam a limpar dá
outra novidade, sem lhe lançarem semente nova, senão a que lhe caiu ao colher da novidade.
Levaram a semente do arroz ao Brasil do Cabo Verde, cuja palha se a comem os cavalos lhes
faz muito mormo,\footnote{ Mormo: doença de equídeos.} e, se comem muito dela, morrem disso.

Da ilha do Cabo Verde e da de São Tomé foram à Bahia inhames que se plantaram logo na
terra, onde se deram de maneira que pasmam os negros de Guiné, que são os que usam mais
deles; e colhem inhames que não pode um negro fazer mais que tomar um às costas; o gentio
da terra não usa deles porque os seus, a que chamam carazes, são mais saborosos, de quem
diremos em seu lugar.

\paragraph{[36] Em que se diz as sementes de Espanha que se dão na Bahia, e o como se
procede com elas}\quad
Não é razão que deixemos de tratar das sementes de Espanha que se dão na Bahia, e de como
frutificaram. E peguemos logo dos melões que se dão em algumas partes muito bem, e são mui
arrazoados, mas não chegam todos a maduros, porque lhes corta um bicho o pé, cujas pevides
tornam a nascer se as semeiam.

Pepinos se dão melhor que nas hortas de Lisboa, e duram quatro a cinco meses os
pepineiros, dando novidade que é infinita, sem serem regados, nem estercados.

Abóboras das de conservas se dão mais e maiores que nas hortas de Alvalade, das quais se
faz muita conserva, e as aboboreiras duram todo um ano, sem se secarem, dando sempre
novidades mui perfeitas.

Melancias se dão maiores e melhores que onde se podem dar bem na Espanha, das quais se
fazem latadas que duram todo o verão verdes, dando sempre novidade; e faz"-se delas
conserva mui substancial.

Abóboras"-de"-quaresma, a que chamam de Guiné se dão na Bahia façanhosas de grandes, muitas
e muito gostosas, cujas pevides e das outras abóboras, melancias e pepinos, se tornam a
semear, e nada se rega.\footnote{ No manuscrito da \textsc{bgjm}, ``se tormam a semear.
Mostarda se''.}

Mostarda se semeia ao redor das casas das fazendas uma só vez, da qual ordinariamente
nascem mostardeiras e colhe"-se cada ano muita e boa mostarda.

Nabos e rábanos se dão melhores que entre Douro e Minho; os rábanos queimam muito, e
dão"-se alguns tão grossos como a perna de um homem, mas nem uns nem outros não dão semente
senão falida e pouca e que não torna a servir.

As couves tronchudas e murcianas se dão tão boas como em Alvalade, mas não dão sementes; e
quando as colhem cortam"-nas pelo pé, onde lhes arrebentam muitos filhos, e como são do
tamanho da couvinha, as tiram e plantam como couvinha,\footnote{ Em Varnhagen (1851 e
1879), ``plantam como convinham''.} as quais pegam todas sem secar uma, e criam"-se deles
melhores couves que da couvinha, com o que se escusa semente de couve.

Alfaces se dão à maravilha, de grandes e doces, as quais espigam e dão semente muito boa.

Coentros se dão tamanhos que cobrem um homem, os quais espigam e dão muita semente.

Endros se dão tão altos que parecem funcho, e onde os semeiam uma vez, ainda que sequem,
tornam a nascer outros, se lhe alimpam a terra, ainda que lha não cavem.

Funcho se dará vara tamanha, que parece uma cana de roca muito grossa, e dá muita semente
como os endros, e não há quem os desince\footnote{ Desinçar: livrar; tirar.} da terra onde
se semeiam uma vez.

A salsa se dá muito formosa, e se no verão tem conta com ela, deitando"-lhe uma pouca de
água, nunca se seca, mas não dá semente, nem espiga.

A hortelã tem na Bahia por praga nas hortas, porque onde a plantam lavra toda a terra e
arrebenta por entre a outra hortaliça.

A semente de cebolinha nasce mui bem, e delas se dão muito boas cebolas, as quais espigam,
mas não seca aquela maçaroca em que criam a semente, a qual está em frol e com o peso que
tem, faz vergar o grelo até dar com essa maçaroca no chão, cujas flores se não sequem, mas
quantas são tantas pegam no chão, e nasce de cada uma uma cebolinha, a cujo pé chegam uma
pequena de terra, e cortam o grelo\footnote{ Grelo: haste; broto.} da cebola, para que não
abale a cebolinha, a qual se cria assim e cresce até ter disposição para se transpor.

Alhos não dão cabeça na Bahia, por mais que os deixem estar na terra, mas na capitania de
São Vicente se faz cada dente que plantam tamanho como uma cebola em uma só peça, e
corta"-se em talhadas para se pisarem.

Berinjelas se dão na Bahia maiores e melhores que em nenhuma parte, as quais fazem grandes
árvores, e torna a nascer a sua semente muito bem.

Tanchagem se semeia uma só vez, a qual dá muita semente que se espalha pela terra que se
toda inça\footnote{ Inçar: cobrir.} dela.


Poejos se dão muito e bem, onde quer que os plantam lavram a terra toda, como a hortelã,
mas não espigam nem florescem.

Agriões nascem pelas ruas onde acertou de cair alguma semente, e pelos quintais quando
chove, a qual semente vai às vezes misturada com a da hortaliça, e fazem"-se muito
formosos, e dão tanta semente que não há quem os desince, e também os há naturais da terra
pelas ribeiras sombrias.

Manjericão se dá muito bem de semente, mas não se usa dela na terra, porque com um só pé
enche todo um jardim, dispondo raminhos sem raiz, e por pequenos que sejam, todos prendem,
sem secar nenhum, como se tivessem raízes, o qual se faz mais alto e forte que em
Portugal, e dura todo o ano não o deixando espigar, e espiga com muita semente se lha
querem apanhar, o que se não usa.

Alfavaca se planta da mesma maneira, a qual se dá pelos matos tão alta que cobre um homem,
a quem a formiga não faz dano como ao manjericão.

Beldros nem beldroegas se não semeiam, porque nascem infinidade de uns e de outros, sem os
semearem, nas hortas e quintais e em qualquer terra que está limpa de mato; são naturais
da mesma terra.

As chicórias e os mastruços\footnote{ Em Varnhagen (1851 e 1879), ``maturços''.} se dão
muito bem e dão muita semente e boa para tornar a semear.

As cenouras, celgas,\footnote{ Acelga.} espinafres se dão muito bem mas não espigam, nem
dão sementes; nem os cardos: vai muita semente de Portugal, de que os moradores
aproveitam.

\paragraph{[37] Em que se declara que coisa é a mandioca}\quad
Até agora se disse da fertilidade da terra da Bahia tocante às árvores de fruto da
Espanha, e às outras sementes que se nela dão. E já se sabe como nesta província
frutificam os frutos das alheias,\footnote{ Em Varnhagen (1851 e 1879), ``frutificam as
alheias''.} saibamos dos seus mantimentos naturais; e peguemos primeiro da mandioca, que é
o principal mantimento e de mais substância, que em Portugal chamam farinha"-de"-pau.

Mandioca é uma raiz da feição dos inhames e batatas, tem a grandura conforme a bondade da
terra, e a criação que tem. Há casta de mandioca cuja rama é delgada e da cor como ramos
de sabugueiro, e fofos por dentro; a folha é da feição e da brandura da da parra, mas tem
a cor do verde mais escura; os pés destas folhas são compridos e vermelhos, como os das
mesmas folhas das parreiras. Planta"-se a mandioca em covas redondas como melões, muito bem
cavadas, e em cada cova se metem três ou quatro pauzinhos da rama, de palmo cada um, e não
entram pela terra mais que dois dedos, os quais paus quebram à mão, ou os cortam com faca
ao tempo que os plantam, porque em fresco deitam leite pelo corte, de onde nascem e se
geram as raízes; e fazem"-se estas plantadas mui ordenadas seis palmos de uma cova à outra.
Arrebenta a rama desta mandioca dos nós destes pauzinhos aos três dias até os oito,
segundo a fresquidão do tempo, os quais ramos são muito tenros e todos cheios de nós, que
se fazem ao pé de cada folha, por onde quebram muito; quando a planta rebenta é por estes
nós, e quando os olhos nascem deles são como de parreira. A grandura da raiz e da rama da
mandioca é conforme a terra em que a plantam, e a criação que tem; mas, ordinariamente, é
a rama mais alta que um homem, e as partes cobrem um homem a cavalo;\footnote{ No
manuscrito da \textsc{bgjm}, ``ordinariamente, é a rama mais alta que um homem a
cavalo;''.} mas há uma casta, que de sua natureza dá pequenos ramos, a qual plantam em
lugares sujeitos aos tempos tormentosos, porque a não arranque e quebre o vento. Há casta
de mandioca que, se a deixam criar, dá raízes de cinco e seis palmos de comprido, e tão
grossas como a perna de um homem; querem"-se as roças da mandioca limpas de erva, até que
tenha disposição para criar boa raiz.

Há uma casta de mandioca, que se diz manipocamirim, e outra que chamam manaibuçu, a que se
quer comesta\footnote{ Comesta: comida.} de ano e meio por diante; e há outras
castas, que chamam taiaçu e manaiburu, que se querem comestas de um ano por diante, e
duram estas raízes debaixo da terra sem apodrecerem três, quatro anos.

Há outras castas, que se dizem manitinga\footnote{ Em Varnhagen (1851 e 1879),
``manaitinga''.} e parati, que se começam a comer de oito meses por diante, e se passa de
ano apodrecem muito; esta mandioca manitinga e parati se quer plantada em terras fracas e
de areia.

Planta"-se a mandioca em todo o ano, não sendo no inverno, e quer mais tempo seco que
invernoso, se o inverno é grande apodrece a raiz da mandioca nos lugares baixos. Lança a
rama da mandioca na entrada do verão umas flores brancas como de jasmins,\footnote{ No
manuscrito da \textsc{bgjm}, ``verão umas folhas, digo flores brancas''.} que não têm
nenhum cheiro, e por onde quer que quebram a folha lança leite, a qual folha o gentio come
cozida em tempo de necessidade, com pimenta da terra. A formiga faz muito dano à mandioca,
e se lhe come a folha, mais de uma vez, fá"-la secar; a qual como é comesta dela, nunca dá
boa raiz, e para defenderem as roças desta praga da formiga, buscam"-lhe os formigueiros,
de onde as arrancam com enxadas e as queimam; outros costumam, às tardes, antes que se
recolham, pisarem a terra dos olhos dos formigueiros com picões muito bem, para que de
noite, em que elas dão os seus assaltos, se detenham em tornar a furar a terra para saírem
fora, e lançam"-lhe derredor folhas de árvores que elas comem, e das da mandioca velha, com
o que, quando saem acima se embaraçam até pela manhã, que se recolhem nos formigueiros; e
se as formigas vêm de fora das roças a comer a elas, lançam"-lhes desta folha no caminho,
antes que entrem na roça, o qual caminho fazem muito limpo, por onde vão e vêm à vontade,
e cortam"-lhe a erva com o dente, e desviam"-na do caminho. Neste trabalho andam os
lavradores até que a mandioca é de seis meses, que cobre bem a terra com a rama, que então
não lhe faz a formiga nojo, porque acha sempre pelo chão as folhas que caem de cima, com o
que se contentam, e nas terras novas não há formiga que faça nojo a nada.

\paragraph{[38] Que trata das raízes da mandioca e do para que servem}\quad
As raízes da mandioca comem"-nas as vacas, éguas, ovelhas, cabras, porcos e a caça do
mato,\footnote{ No manuscrito da \textsc{bgjm}, ``as vacas, éguas, ovelhas, cabras e a
caça do mato''.} e todos engordam com elas comendo"-as cruas, e se as comem os índios,
ainda que sejam assadas, morrem disso por serem muito peçonhentas; e para se aproveitarem
os índios e mais gente destas raízes, depois de arrancadas rapam"-nas muito bem até ficarem
alvíssimas, o que fazem com cascas de ostras, e depois de lavadas ralam"-nas em uma pedra
ou ralo, que para isso têm, e, depois de bem raladas, espremem esta massa em um engenho de
palma, a que chamam tupitim,\footnote{ Em Varnhagen (1851 e 1879), ``tapitira''. Tapiti ou
tipiti é um cesto cilíndrico de palha em que se põe a massa de mandioca para ser
espremida.} que lhe faz lançar a água que tem toda fora, e fica esta massa toda muito
enxuta, da qual se faz a farinha que se come, que cozem em um alguidar\footnote{ Alguidar:
vaso cuja borda tem diâmetro muito maior que o fundo.} para isso feito, no qual deitam
esta massa e a enxugam sobre o fogo, onde uma índia a mexe com um meio cabaço, como quem
faz confeitos, até que fica enxuta e sem nenhuma umidade, e fica como cuscuz, mas mais
branca, e desta maneira se come, é muito doce e saborosa. Fazem mais desta massa, depois
de espremida, umas filhós,\footnote{ Filhó: biscoito ou bolinho feito de farinha e ovos.}
a que chamam beijus, estendendo"-a no alguidar sobre o fogo, de maneira que ficam tão
delgadas como filhós mouriscas, que se fazem de massa de trigo, mas ficam tão iguais como
obreias,\footnote{ Obreia: folha fina de massa feita de farinha de trigo, também utilizada
na confecção da hóstia.} as quais se cozem neste alguidar até que ficam muito secas e
torradas.

E estes beijus são mui saborosos, sadios e de boa digestão, que é o mantimento que se usa
entre gente de primor, o que foi inventado pelas mulheres portuguesas, que o gentio não
usava deles. Fazem mais desta mesma massa tapiocas, as quais são grossas como filhós de
polme e moles, e fazem"-se no mesmo alguidar como os beijus, mas não são de tanta digestão,
nem tão sadios; e querem"-se comidas quentes e com leite têm muita graça, e com açúcar
clarificado também.

\paragraph{[39] Em que se declara quão terrível peçonha é a da água da mandioca}\quad
Antes de passarmos avante, convém que declaremos a natural estranheza da água da mandioca
que ela de si deita quando a espremem depois de ralada, porque é a mais terrível peçonha
que há nas partes do Brasil, e quem quer que a bebe não escapa por mais contrapeçonha que
lhe deem; a qual é de qualidade que as galinhas em lhe tocando com o bico, e levando uma
só gota para baixo, caem todas da outra banda mortas,\footnote{ No manuscrito da
\textsc{bgjm}, ``caem logo da outra banda mortas''.} e o mesmo acontece aos patos, perus,
papagaios e a todas as aves, pois os porcos, cabras, ovelhas, em bebendo o primeiro bocado
dão três e quatro voltas em redondo e caem mortos, cuja carne se faz logo negra e nojenta;
o mesmo acontece a todo o gênero de alimária que a bebe; e por esta razão se espreme esta
mandioca por curtir em covas cobertas, e em outras partes, onde não faça nojo às criações,
e se estas alimárias comem a mesma mandioca por espremer, engordam com ela e não lhes faz
dano. Tem esta água tal qualidade que, se metem nela uma espada ou cossolete,\footnote{Cossolete: 
corselete de armadura.} espingarda
ou outra qualquer coisa cheia de ferrugem, lha come em vinte
e quatro horas, de feição que ficam limpas como quando saem da mó, do que se aproveitam
algumas pessoas para limparem algumas peças de armas da ferrugem que na mó se não podem
alimpar sem entrar pelo são. Nos lugares onde se esta mandioca espreme, se criam na água
dela uns bichos brancos como vermes grandes, que são peçonhentíssimos, com os quais muitas
índias mataram seus maridos e senhores, e matam a quem querem, do que também se
aproveitavam, segundo dizem, algumas mulheres brancas contra seus maridos; e basta
lançar"-se um destes bichos no comer para uma pessoa não escapar, sem lhe aproveitar alguma
contrapeçonha, porque não mata com tanta presteza como a água de que se criam, e não se
sente este mal senão quando não tem remédio nenhum.

\paragraph{[40] Que trata da farinha que se faz da mandioca}\quad
O mantimento de mais estima e proveito que faz da mandioca é a farinha fresca, a qual se
faz destas raízes, que se lançam primeiro a curtir, de que se aproveita o gentio; e os
portugueses, que não fazem a farinha da mandioca crua, de que atrás temos dito, senão por
necessidade.

Costumam as índias lançar cada dia destas raízes na água corrente ou na encharcada, quando
não têm perto a corrente, onde está a curtir até que lança a casca de si; e como está
desta maneira, está curtida; da qual traz para casa outra tanta como lança na água para
curtir, as quais raízes escascadas ficam muito alvas e brandas, sem nenhuma peçonha, que
toda se gastou na água, as quais se comem assadas e são muito boas.

E para se fazer a farinha destas raízes se lavam primeiro muito bem, e depois, desfeitas a
mão, se espremem no topeti, cuja água não faz mal; depois de
bem espremidas desmancham esta massa sobre uma urupema,\footnote{ Urupema: peneira feita
de talos de palmeira, que servia para separar o farelo grosso da massa de mandioca ralada,
depois de prensada.} que é como joeira,\footnote{ Joeira: peneira para separar o trigo do
joio e de outras sementes com o que está misturado.} por onde se coa o melhor, e ficam os
caroços em cima e o pó que se coou lançam"-no em um alguidar que está sobre o fogo, onde se
enxuga e coze da maneira que fica dito, e fica como cuscuz, a qual em quente e em fria é
muito boa e assim no sabor como em ser sadia e de boa digestão. Os índios usam destas
raízes tão curtidas que ficam denegridas e a farinha, azeda. Os portugueses não a querem
curtida mais que até dar a casca, com a qual mandam misturar algumas raízes de mandioca
crua, com o que fica a farinha mais alva e doce; e desta maneira se aproveitam da
mandioca, a qual farinha fresca dura sem se danar cinco a seis dias, mas faz"-se seca; e
quem é bem servido em sua casa, come"-a sempre fresca e quente.

Estas raízes da mandioca curtida têm grande virtude para curar postemas,\footnote{ Postema
ou apostema: abscesso.} as quais se pisam muito bem sem se espremerem; e, feito da massa
um emplasto, posto sobre a postema a molefica de maneira que a faz arrebentar por si, se a
não querem furar.

\paragraph{[41] Que trata do muito para que prestam as raízes da carimã}\quad
Muito é para notar que de uma mesma coisa saia peçonha e contrapeçonha, como da mandioca,
cuja água é cruelíssima peçonha, e a mesma raiz seca é contrapeçonha, a qual se chama
carimã,\footnote{ Carimã: amido da mandioca, conhecido como polvilho ou fécula de
mandioca.} que se faz desta maneira: depois que as raízes da mandioca estão curtidas na
água, se põe a enxugar sobre o fogo em cima de umas varas, alevantadas três ou quatro
palmos do chão, e como estão bem secas, ficam muito duras, as quais raízes servem para mil
coisas, e têm outras tantas virtudes; a principal serve de contrapeçonha para os mordidos
de cobra, e que comem bichos peçonhentos, e para os que comem a mesma mandioca por curtir
assada, cuidando que são outras raízes, que chamam aipins, bons de comer, que se parecem
com ela, a qual carimã se dá desta feição: tomam estas raízes secas, e rapam"-lhe o
defumado da parte de fora e ficam alvíssimas e pisam"-nas muito bem, e depois peneiram"-nas
e fica o pó delas tão delgado e mimoso como de farinha muito boa; e tomada uma pouca dessa
farinha e delida\footnote{ Delida: desfeita.} em água fria, que fique como 
amendoada, e dada a beber ao tocado da peçonha, faz"-lhe arremessar quanto tem no bucho,
com o que a peçonha que tem no corpo não vai por diante.

Também serve esta carimã para os meninos que têm lombrigas, aos quais se dá a beber
desfeita na água, como fica dito, e mata"-lhes as lombrigas todas; e uma coisa e outra está
mui experimentada, assim pelos índios, como pelos portugueses.

Da mesma farinha da carimã se faz uma massa que posta sobre feridas velhas que têm carne
podre, lha come toda até que deixe a ferida limpa; e como os índios estão doentes, a sua
dieta é fazerem deste pó de carimã uns caldinhos no fogo como os de poejos, que bebem, com
que se acham mui bem por ser muito leve, e o mesmo usam os brancos no mato, lançando"-lhe
mel ou açúcar, com o que se acham bem; e outras muitas coisas de comer que se fazem desta
carimã que se apontam no capítulo que se segue.

\paragraph{[42] Em que se declara que coisa é farinha"-de"-guerra, e como se faz da carimã, e outras coisas}\quad
Farinha"-de"-guerra se diz, porque o gentio do Brasil costuma chamar"-lhe assim pela sua
língua, porque quando determinam de a ir fazer a seus contrários algumas jornadas fora de
sua casa, se provêm desta farinha, que levam às costas ensacada em uns fardos de folhas
que para isso fazem, da feição de uns de couro, em que da Índia trazem especiaria e arroz;
mas são muito mais pequenos, onde levam esta farinha muito calcada e enfolhada, de maneira
que, ainda que lhe caia em um rio, e que lhe chova em cima, não se molha. Para se fazer
esta farinha se faz prestes muita soma de carimã, a qual, depois de rapada, a pisam em um
pilão que para isso têm, e como é bem pisada a peneiram muito bem, como no capítulo antes
fica dito. E como têm esta carimã prestes, tomam as raízes da mandioca por curtir, e ralam
como convém uma soma delas, e, depois de espremidas, como se faz à primeira farinha que
dissemos atrás, lançam uma pouca desta massa em um alguidar que está sobre o fogo, e por
cima dela uma pouca de farinha da carimã e, embrulhada uma com outra, a vão mexendo sobre
o fogo, e assim como se vai cozendo, lhe vão lançando do pó da carimã, e trazem"-na sobre o
fogo, até que fica muito enxuta e torrada, que a tiram fora.

Desta farinha"-de"-guerra usam os portugueses que não têm roças, e os que estão fora delas
na cidade, com que sustentam seus criados e escravos, e nos engenhos se provêm dela para
sustentarem a gente em tempo de necessidade,\footnote{ No manuscrito \textsc{bgjm},
``sustentarem a gente até Portugal, digo em tempo de necessidade''.} e os navios que vêm
do Brasil para estes reinos não têm outro remédio de matalotagem,\footnote{ Matalotagem:
provisão de mantimentos e víveres em uma embarcação para consumo dos tripulantes e
passageiros.} para se sustentar a gente até Portugal, senão o da farinha"-de"-guerra; e um
alqueire dela da medida da Bahia, que tem dois de Portugal, se dá de regra a cada homem
para um mês, a qual farinha"-de"-guerra é muito sadia e desenfastiada, e molhada no caldo da
carne ou do peixe fica branda e tão saborosa como cuscuz.

Também costumam levar para o mar matalotagem de beijus grossos muito torrados, que dura um
ano, e mais sem se danarem, como a farinha"-de"-guerra. Desta carimã e pó dela bem peneirado
fazem os portugueses muito bom pão, e bolos amassados com leite e gemas de ovos, e desta
mesma massa fazem mil invenções de beilhós, mais saborosos que de farinha de trigo, com os
mesmos materiais, e pelas festas fazem as frutas doces com a
massa desta carimã, em lugar da farinha de trigo, e se a que vai à Bahia do reino não é
muito alva e fresca, querem as mulheres antes a farinha de carimã, que é alvíssima e
lavra"-se melhor com a qual fazem tudo muito primo.\footnote{ Primo: excelente.}

\paragraph{[43] Em que se declara a qualidade dos aipins}\quad
Dá"-se nesta terra outra casta de mandioca, a que o gentio chama aipins, cujas raízes são
da feição da mesma mandioca, a rama e a folha são da mesma maneira, sem haver nenhuma
diferença, e planta"-se de mistura com a mesma mandioca, e para se colherem estas raízes as
conhecem os índios pela cor dos ramos, no que atinam poucos portugueses. E estas raízes
dos aipins são alvíssimas; como estão cruas sabem às castanhas cruas da Espanha; assadas
são muito doces, e têm o mesmo sabor das castanhas assadas, e d'avantagem, as quais se
comem também cozidas, e são muito saborosas; e de uma maneira e de outra são ventosas como
as castanhas. Destes aipins se aproveitam nas povoações novas porque como são de cinco
meses se começam a comer assados, e como passam de seis meses fazem"-se duros, e não se
assam bem; mas servem então para beijus e para farinha fresca, que é mais doce que a da
mandioca, as quais raízes duram pouco debaixo da terra, e como passam de oito meses
apodrecem muito.

Destes aipins há sete ou oito castas; mas os que mais se estimam, por serem mais
saborosos, são uns que chamam gerumus. Os índios se valem dos aipins para nas suas festas
fazerem deles cozidos seus vinhos, para o que os plantam mais que para os comerem assados,
como fazem os portugueses.

E porque tudo é mandioca, concluamos que o mantimento dela é o melhor que se sabe, tirado
o do bom trigo, porque pão de trigo"-do"-mar, de milho, de centeio e de cevada, não presta a
par da\footnote{ A par da: comparados com.} mandioca, arroz, inhames e cocos.

Milho de Guiné se dá na Bahia, como ao diante se verá; mas não se tem lá por mantimento, e
ainda digo que a mandioca é mais sadia e proveitosa que o bom trigo, por ser de melhor
digestão. E por se averiguar por tal, os governadores Tomé de Sousa, D. Duarte e Mem de Sá
não comiam no Brasil pão de trigo, por se não acharem bem como ele, e assim o fazem outras
muitas pessoas.

\paragraph{[44] Em que se apontam alguns mantimentos de raízes que se criam debaixo da terra
na Bahia}\quad
Como fica dito da mandioca o que em breve se pode dizer dela, convém que declaremos daqui
por diante outros mantimentos que se dão na Bahia debaixo da terra.

E peguemos logo nas batatas, que são naturais da terra, e se dão nela de maneira que onde
se plantam uma vez nunca mais se desinçam, as quais tornam a nascer das pontas das raízes,
que ficaram na terra, quando se colheu a novidade delas. As batatas não se plantam da rama
como nas Ilhas, mas de talhadas das mesmas raízes, e em cada enxadada que dão na terra,
sem ser mais cavada, metem uma talhada de batata, as quais se plantam em abril e começam a
colher a novidade em agosto, donde têm que tirar até todo o março, porque colhem umas
batatas grandes e ficam outras pequenas, que se vão criando em quinze e vinte dias.

Há umas batatas grandes e brancas e compridas como as das Ilhas; há outras pequenas e
redondas como túberas da terra, e mui saborosas; há outras batatas que são roxas ao longo
da casca e brancas por dentro; há outras que são todas encarnadas e mui gostosas; há
outras que são cor azul anilada muito fina, as quais tingem as mãos; há outras verdoengas,
muito doces e saborosas; e há outra casta, de cor almecegada, mui saborosas; e outras
todas amarelas, de cor muito tostada, as quais são todas úmidas e ventosas, de que se não
faz muita conta entre gente de primor, senão entre lavradores.

Dão"-se na Bahia outras raízes maiores que batatas, a que os índios chamam carazes, que se
plantam da mesma maneira que as batatas, e como nascem põem"-lhes ao pé uns paus, por onde
atrepam os ramos que lançam, como erva. Estes carazes se plantam em março e colhem"-se em
agosto, os quais se comem cozidos e assados como os inhames, mas têm melhor sabor; os mais
deles são brancos, outros roxos, outros brancos por dentro e roxos por fora junto da
casca, que são os melhores, e de mor sabor; outros são todos negros como pós; e uns e
outros se curam no fumo, e duram de um ano para outro. Da massa destes carazes fazem as
portuguesas muitos manjares com açúcar, e cozidos com carne têm muita graça.

Dão"-se nesta terra outras raízes tamanhas como nozes e avelãs, que se chamam mangarazes; e
quando se colhem arrancam"-nos debaixo da terra em touças como junças e tira"-se de cada pé
duzentos e trezentos juntos; e o que está no meio é como um ovo, e como um punho, que é a
planta de onde nasceram os outros, o qual se guarda para se tornar a plantar; e quando o
plantam se faz em talhadas, como as batatas e carazes; mas plantam"-se tão juntos e pela
ordem com que se dispõe a couvinha, e não se cava a terra toda, mas limpa do mato, a cada
enxadada metem uma talhada. As folhas destes mangarazes nascem em moitas como os
espinafres, e são da mesma cor e feição, mas muito maiores, e assim moles como as dos
espinafres, as quais se chamam taiaobas, que se comem esperregadas como eles; e são mui
medicinais, e também servem cozidas com o peixe. As raízes destes mangarazes se comem
cozidas com água e sal, e dão a casca como tremoços,\footnote{ Tremoço: grão do tremoceiro
(planta leguminosa).} e molhados em azeite e vinagre são mui gostosos; com açúcar fazem as
mulheres deles mil manjares; e colhem"-se duas novidades no ano; os que se plantam em março
se colhem em agosto, e os que se plantam em setembro se colhem em janeiro.

Dão"-se nesta terra outras raízes, que se chamam taizes, que se 
plantam como os mangarazes, e são de feição de maçarocas, mas cintadas com uns
perfilos com barbas, como raízes de canas de roça, as quais se comem cozidas na água, mas
sempre ficam tesas. As folhas são grandes, de feição e cor das dos plátanos que se acham
nos jardins da Espanha, aos quais chamam taiobusu;\footnote{ Em Varnhagen (1851 e 1879),
``taiaobuçú''.} comem"-se estas folhas cozidas com peixe em lugar dos espinafres, e com
favas verdes em lugar das alfaces, e têm mui avantajado sabor; os índios as comem cozidas
na água e sal, e com muita soma de pimenta.

\paragraph{[45] Em que se contém o milho que se dá na Bahia e para o que serve}\quad
Dá"-se outro mantimento em todo o Brasil, natural da mesma terra, a que os índios chamam
ubatim, que é o milho de Guiné, que em Portugal chamam zaburro. As espigas que este milho
dá são de mais de palmo, cuja árvore é mais alta que um homem, e da grossura das canas de
roça, com nós e vãs por dentro; e dá três, quatro e mais espigas destas em cada vara. Este
milho se planta por entre a mandioca e por entre as canas novas de açúcar, e colhe"-se a
novidade aos três meses, uma em agosto e outra em janeiro. Este milho come o gentio assado
por fruto, e fazem seus vinhos com ele cozido, com o qual se embebedam, e os portugueses
que comunicam com o gentio, e os mestiços não se desprezam dele, e bebem"-no mui
valentemente. Costuma este gentio de dar suadouros com este milho cozido aos doentes de
boubas,\footnote{ Bouba: pústula ou tumor de
pele.} os quais tomam com o bafo dele, com o que se acham bem;
dos quais suadouros se acham sãos alguns homens brancos e mestiços que se valem deles; o
que parece mistério porque este milho por natureza é frio. Plantam os portugueses este
milho para mantença dos cavalos e criação das galinhas e cabras, ovelhas e porcos; e aos
negros de Guiné o dão por fruta, os quais o não querem por mantimento, sendo o melhor da
sua terra; a cor geral deste milho é branca; há outro almecegado, outro preto, outro
vermelho, e todo se planta a mão, e têm uma mesma qualidade.

Há outra casta de milho que sempre é mole, do qual fazem os portugueses muito bom pão e
bolos com ovos e açúcar. O mesmo milho quebrado e pilado no pilão é bom para se cozer com
caldo de carne,\footnote{ Em Varnhagen (1851 e 1879), ``quebrado e pisado no pilão''.} ou
pescado, e de galinha, o qual é mais saboroso que o arroz, e de uma casta e outra se curam
ao fumo, onde se conserva para se não danar; e dura de um ano para outro.

\paragraph{[46] Em que se apontam os legumes que se dão na Bahia}\quad
Pois que até aqui tratamos dos mantimentos naturais da terra da Bahia, é bem que digamos
dos legumes que se nela criam. E comecemos pelas favas, que os índios chamam comendá, as
quais são muito alvas, e do tamanho e maiores que as de Évora em Portugal; mas são
delgadas e amassadas, como os figos passados.

Há outras favas, meio brancas e meio pretas da mesma feição e tamanho; outras há todas
pretas, mas são pequenas;\footnote{ Em Varnhagen (1851 e 1879), ``Há outras favas meio
brancas e meio pretas, mas são pequenas''.} e estas favas se plantam a mão na entrada do
inverno, e como nascem põe"-se ao pé de cada uma um pau, por onde atrepam, como fazem em
Portugal às ervilhas; e, se têm por onde atrepar, fazem grande ramada; a folha é como a
dos feijões da Espanha, mas maior; a frol é branca; e começa"-se a dar a novidade no fim do
inverno e dura mais de três meses. Estas favas são, em verdes, mui saborosas, e cozem"-se
com as cerimônias que se costuma em Portugal, e são reimosas como as do reino; e dão em
cada bainha quatro e cinco favas, e depois de secas se cozem muito bem, e não criam
bichos, como as da Espanha, e são melhores de cozer; e de uma maneira e de outra fazem no
sabor muita vantagem às de Portugal, assim as declaradas como a outra casta de favas, que
são brancas e pintadas todas de pontos negros.

Dão"-se nesta terra infinidade de feijões naturais dela, uns são brancos, outros pretos,
outros vermelhos, e outros pintados de branco e preto, os quais se plantam a mão. E como
nascem põe"-se"-lhe a cada pé um pau, por onde atrepam, como se faz às ervilhas, e sobem de
maneira para cima que fazem deles latadas nos quintais, e cada pé dá infinidade de
feijões, os quais são da mesma feição dos da Espanha, mas têm mais compridas bainhas, e a
folha e frol como as ervilhas; cozem"-se estes feijões sendo secos, como em Portugal, e são
mui saborosos, e enquanto são verdes cozem"-se com a casca como fazem às ervilhas, e são
mui desenfastiados.

Chamam os índios gerumus\footnote{ Jerimu ou jerimum.} às abóboras"-da"-quaresma, que são
naturais desta terra, das quais há dez ou doze castas, cada uma de sua feição; e plantam
duas vezes no ano, em terra úmida e solta, os quais se estendem muito pelo chão, e dá cada
aboboreira muita soma; mas não são tamanhas como as da casta de Portugal. Costuma o gentio
cozer e assar estas abóboras inteiras por lhe não entrar água dentro, e depois de cozidas
as cortam como melões, e lhes deitam as pevides fora, e são assim mais saborosas que
cozidas em talhadas, e curam"-se no fumo para durarem todo o ano.

Aos que em Portugal chamamos cabaços, chama o gentio pela sua língua geremuiê, das quais
têm entre si muitas castas de diferentes feições, tirando as abóboras compridas, de que
dissemos atrás. Essas abóboras ou cabaços semeia o gentio para fazer delas vasilhas de seu
uso, as quais não costuma comer, mas deixam"-nas estar nas aboboreiras até se fazerem
duras, e como estão de vez, curam"-nas no fumo, de que fazem depois vasilhas para
acarretarem água, por outras pequenas bebem, outras meias levam às costas cheias de água
quando caminham; e há alguns destes cabaços tamanhos que levam dois almudes\footnote{
Almude: antiga medida romana de vinho. No Brasil, era uma medida de capacidade que variava
conforme a região e equivalia a aproximadamente 32 litros.} e mais, nos quais guardam as
sementes que hão de plantar; e costumam também cortar esses cabaços em verdes, como estão
duros, pelo meio, e depois de curadas estas metades servem"-lhes de gamelas, e outros
despejos, e as metades dos pequenos servem"-lhes de escudelas, e dão"-lhes por dentro uma
tinta preta, por fora outra amarela, que se não tira nunca; e estas são as suas
porcelanas.

\paragraph{[47] Em que se declara a natureza dos amendoins e para que servem}\quad
Dos amendoins temos que dar conta particular,\footnote{ No manuscrito da \textsc{bgjm},
``Dos amendoins não temos que dar conta particular''.} porque é coisa que se não sabe
haver senão no Brasil, os quais nascem debaixo da terra, onde se plantam a mão, um palmo
um do outro; as suas folhas são como as dos feijões da Espanha, e tem os ramos ao longo do
chão. E cada pé dá um grande prato destes amendoins, que nascem nas pontas das raízes, os
quais são tamanhos como bolotas, e têm a casca da mesma grossura e dureza, mas é branca e
crespa, e têm dentro de cada bainha três e quatro amendoins, que são da feição dos pinhões
com casca, e ainda mais grossos. Têm uma tona\footnote{ Tona: casca; pele, tono.} parda,
que se lhes sai logo como a do miolo dos pinhões, o qual miolo é muito alvo. Comestos crus
têm sabor de gravanços\footnote{ Gravanço: grão"-de"-bico.} crus, mas comem"-se assados e
cozidos com a casca, como as castanhas,\footnote{ No manuscrito da \textsc{bgjm},
``gravanços crus são mui saborosos e torrados fora da casca, digo que se comem assados e
cozidos com casca, como as castanhas''.} e são muito saborosos, e torrados fora da casca
são melhores. De uma maneira e de outra é esta fruta muito quente em demasia, e causam dor
de cabeça a quem come muitos, se é doente dela.
Plantam"-se estes amendoins em terra solta e úmida, em a qual planta e
benefício dela não entra homem macho, só as índias os costumam plantar, e as mestiças; e
nesta lavoura não entendem os maridos, e têm para si que se eles ou seus escravos os
plantarem, que não hão de nascer. E as fêmeas os vão
apanhar, e, segundo seu uso, hão de ser as mesmas que os plantarem; e para durarem todo o
ano curam"-nos no fumo, onde os têm até vir outra novidade.

Desta fruta fazem as mulheres portuguesas todas as coisas doces, que fazem das amêndoas, e
cortados os fazem cobertos de açúcar, de mistura com os confeitos. E também os curam em
peças delgadas e compridas, de que fazem pinhoadas; e quem os não conhece por tal os come,
se lhos dão. O próprio tempo em que se os amendoins plantam é em fevereiro, e não estão
debaixo da terra mais que até maio, que é o tempo em que se lhes colhe a novidade, o que
as fêmeas vão fazer com grande festa.

\paragraph{[48] Em que se declara quantas castas de pimenta há na Bahia}\quad
À sombra destes legumes, e na sua vizinhança, podemos ajuntar quantas castas de pimenta há
na Bahia, segundo nossa notícia; e digamos logo da que chamam cuihem, que são tamanhas
como cerejas, as quais se comem em verdes, e, depois de maduras, cozidas inteiras com o
pescado e com os legumes, e de uma maneira e de outra queimam muito, e o gentio come"-a
inteira, misturada com a farinha.

Costumam os portugueses, imitando o costume dos índios, secarem esta pimenta, e depois de
estar bem seca, a pisam de mistura com o sal, ao que chamam juquitai,\footnote{ Em
Varnhagen (1851 e 1879), ``juquiray''.} em a qual molham o peixe e a carne, e entre os
brancos se traz no saleiro, e não descontenta a ninguém. Os índios a comem misturada com a
farinha, quando não têm que comer com ela. Estas pimenteiras fazem árvores de quatro e de
cinco palmos de alto, e duram muitos anos sem se secar.

Há outra pimenta, a que pela língua dos negros se chama cuiemoçu; esta é grande e
comprida, e depois de madura faz"-se vermelha; e usam dela como da de cima; e faz árvores
de altura de um homem, e todo o ano dá novidade; sempre têm pimentas vermelhas, verdes e
em frol, e dura muitos anos sem se secar.

Há outra casta, que chamam cuihepia,\footnote{ Em Varnhagen (1851 e 1879), ``cuiepiá''.} a
qual tem bico, feição e tamanho dos gravanços; come"-se em verde, crua e cozida como a de
cima, e como é madura faz"-se vermelha, a qual queima muito; a quem as galinhas e pássaros
têm grande afeição; e faz árvore meã, que em todo o ano dá novidade.

Há outra casta, que chamam sabãa, que é comprida e delgada, em verde não queima tanto como
quando é madura, que é vermelha; cuja árvore é pequena, dá fruta todo o ano, e também se
usa dela como da mais.

Há outra casta que se chama cuihejurimu, por ser da feição de abóbora, assim amassada;
esta, quando é verde, tem a cor azulada, e como é madura se faz vermelha; da qual se usa
como das mais de que temos dito, cuja árvore é pequena e em todo o ano dá novidade.

Há outra casta, que chamam camari,\footnote{ Em Varnhagen (1851 e 1879), ``cumari''.} que
é bravia e nasce pelos matos e campos e pelas roças, a qual nasce do feitio dos pássaros
que a comem muito, por ser mais pequena que gravanços; mas queima mais que todas as que
dissemos, e é mais gostosa que todas; e quando é madura faz"-se vermelha, e quando se acha
esta não se come da outra; faz"-se árvore pequena, tem as flores brancas como as mais, e dá
novidade em todo o ano.

\paragraph{[49] Daqui por diante se dirá das árvores de fruto, começando nos cajus e cajuís}\quad
Convém tratar daqui por diante das árvores de fruto naturais da Bahia, águas vertentes ao
mar e à vista dele; e demos o primeiro lugar e capítulo por si aos cajueiros, pois é uma
árvore de muita estima, e há tantos ao longo do mar e na vista dele. Estas árvores são
como figueiras grandes, têm a casca da mesma cor, e a madeira branca e mole como figueira,
cujas folhas são da feição da cidreira e mais macias. As folhas dos olhos novos são
vermelhas, muito brandas e frescas, a frol é como a do sabugueiro, de bom cheiro, mas
muito breve. A sombra destas árvores é muito fria e fresca, o fruto é formosíssimo;
algumas árvores dão fruto vermelho e comprido, outras o dão da mesma cor e redondo.

Há outra casta que dá o fruto da mesma feição, mas há partes vermelhas e há outras de cor
almecegada; há outras árvores que dão o fruto amarelo e comprido como peros del"-rei, mas
são em tudo maiores que os peros e da mesma cor.

Há outras árvores que dão este fruto redondo, e uns e outros são muito gostosos,
sumarentos e de suave cheiro, os quais se desfazem todos em água.

A natureza destes cajus é fria, e são medicinais para doentes de febres, e para quem tem
fastio, os quais fazem bom estômago e muitas pessoas lhes tomam o sumo pelas manhãs em
jejum, para conservação do estômago, e fazem bom bafo a quem os come pela manhã, e por
mais que se coma deles não fazem mal a nenhuma hora do dia, e são de tal digestão que em
dois credos se esmoem.\footnote{ Esmoer: mastigar; digerir.}


Os cajus silvestres travam junto do olho que se lhes bota fora, mas os que se criam nas
roças e nos quintais comem"-se todos sem terem que lançar fora por não travarem. Fazem"-se
estes cajus de conserva, que é muito suave, e para se comerem logo cozidos no açúcar
cobertos de canela não têm preço. Do sumo desta fruta faz o gentio vinho, com que se
embebeda, que é de bom cheiro e saboroso.

É para notar que no olho deste pomo tão formoso cria a natureza outra fruta, parda, a que
chamamos castanha, que é da feição e tamanho de um rim de cabrito, a qual castanha tem a
casca muito dura e de natureza quentíssima e o miolo que tem dentro; deita esta casca um
óleo tão forte que onde toca na carne faz empola, o qual óleo é da cor de azeite, e tem o
cheiro mui forte. Tem esta castanha o miolo branco, tamanho como o de uma amêndoa grande,
a qual é muito saborosa, e quer arremedar no sabor aos pinhões, mas é de muita vantagem.
Destas castanhas fazem as mulheres todas as conservas doces que costumam fazer com as
amêndoas,\footnote{ No manuscrito da \textsc{bgjm}, ``todas as frutas doces''.} o que tem
graça na suavidade do sabor; o miolo destas castanhas, se está muitos dias fora da casca,
cria ranço do azeite que tem em si; quando se quebram estas castanhas para lhes tirarem o
miolo, faz o azeite que tem na casca pelar as mãos a quem as quebra.

Estas árvores se dão em areia e terras fracas, e se as cortam tornam logo a rebentar, o
que fazem poucas árvores nestas partes.

Cria"-se nestas árvores uma resina muito alva, da qual as mulheres se aproveitam para
fazerem alcorce\footnote{ Alcorce ou alcorça: massa de açúcar e farinha com a qual se
cobrem ou fazem diversos doces.} de açúcar em lugar de alquitira.\footnote{ Alquitira:
goma retirada da planta \textit{Astracantha gummifera} que pode ser usada na confecção de
doces.} Nascem estas árvores das castanhas, e em dois anos se fazem mais altas que um
homem, e no mesmo tempo dão fruto, o qual, enquanto as árvores são novas, é avantajado no
cheiro e sabor.

Há outra casta desta fruta, a que os índios chamam cajuí, cuja árvore é nem mais nem menos
que a dos cajus, senão quanto é muito mais pequena, que lhe chega um homem do chão ao mais
alto dela a colher"-lhe o fruto, que é amarelo, mas não é maior que as cerejas grandes, e
tem maravilhoso sabor com a pontinha de azedo, e criam também sua castanha na ponta, as
quais árvores se não dão ao longo do mar, mas nas campinas do sertão, além da catinga.

\paragraph{[50] Em que se declara a natureza das pacobas e bananas}\quad
\mbox{Pacoba} é uma fruta natural desta terra, a qual se dá em uma árvore muito mole e fácil de
cortar, cujas folhas são de doze e quinze palmos de comprido e de três e quatro de largo;
as de junto ao olho são menores e muito verdes umas e outras, e a árvore da mesma cor mas
mais escura; na Índia chamam a estas pacobeiras figueiras e ao fruto, figos. Cada árvore
destas não dá mais que um só cacho que pelo menos tem passante de duzentas pacobas, e como
este cacho está de vez, cortam a árvore pelo pé e de um só golpe que lhe dão com uma foice
a cortam e cerceiam, como se fora um nabo, do qual corte corre logo água em fio, e dentro
em vinte e quatro horas torna a lançar do meio do corte um olho mui grosso de onde se gera
outra árvore; e derredor deste pé arrebentam muitos filhos que aos seis meses dão fruto, e
o mesmo faz à mesma árvore. E como se corta esta pacobeira, tiram"-lhe o cacho que tem o
fruto verde e muito teso, e dependuram"-no em parte onde amadureça, e se façam amarelas as
pacobas; e na casa onde se fizer fogo amadurecem mais depressa com a quentura; e como esta
fruta está madura, cheira muito bem. Cada pacoba destas tem um palmo de comprido e a
grossura de um pepino, às quais tiram as cascas, que são de grossura das favas; e
fica"-lhes o miolo inteiro almecegado, muito saboroso. Dão"-se estas pacobas assadas aos
doentes em lugar de maçãs, das quais se faz marmelada muito sofrível, e também as
concertam como berinjelas, e são muito gostosas; e cozidas no açúcar com canela são
estremadas,\footnote{ Estremado: ótimo; notável.} e passadas ao sol sabem a pêssegos
passados. Basta que de toda a maneira são muito boas, e dão em todo o ano; mas no inverno
não há tantas como no verão, e a estas pacobas chama o gentio pacobusu,\footnote{ Em
Varnhagen (1851 e 1879), ``pacobuçu''.} que quer dizer pacoba grande.\footnote{ Pacobuçú é
a banana"-da"-terra.}

Há outra casta que não são tamanhas, mas muito melhores no sabor, e vermelhaças por dentro
quando as cortam, e se dão e criam da mesma maneira das grandes.

Há outra casta, que os índios chamam pacobamirim, que quer dizer pacoba pequena, que são
do comprimento de um dedo, mas mais grossas; essas são tão doces como tâmaras, em tudo mui
excelentes.

As bananeiras têm árvores, folhas e criação como as pacobeiras, e não há nas árvores de
umas às outras nenhuma diferença, as quais foram ao Brasil de São Tomé, onde ao seu fruto
chamam bananas, e na Índia chamam a estes figos de horta, as quais são mais curtas que as
pacobas, mas mais grossas e de três quinas; têm a casca da mesma cor e grossura das das
pacobas, e o miolo mais mole, e cheiram melhor como são de vez, às quais arregoa\footnote{
Arregoar: abrir"-se, fender"-se (a fruta).} a casca como vão amadurecendo e fazendo algumas
fendas ao alto, o que fazem na árvore; e não são tão sadias como as pacobas.

Os negros da Guiné são mais afeiçoados a estas bananas que às pacobas, e delas usam nas
suas roças; e umas e outras se querem plantadas em vales perto da água, ou ao menos em
terra que seja muito úmida para se darem bem as bananas e também se dão em terras secas e
de areia; quem cortar atravessadas as pacobas ou bananas, ver"-lhes"-á no meio uma feição de
crucifixo, sobre o que contemplativos têm muito que dizer.

\paragraph{[51] Em que se diz que fruto é o que se chama mamões e jacarateás}\quad
De Pernambuco veio à Bahia a semente de uma fruta que veio do Peru a que chamam
mamões,\footnote{ Em Varnhagen (1851 e 1879), ``de uma fruta a que chamam mamões''.} os
quais são do tamanho e da feição e cor de grandes pêros camoeses,\footnote{ Camoesa:
espécie de maçã alongada e doce que se originou na Galícia, muito conhecida em toda a
Espanha.} e têm muito bom cheiro como são de vez, se fazem nas árvores, e em casa acabam
de amadurecer; e como são maduros se fazem moles como melão; e para se comerem cortam"-se
em talhadas como maçã, e tiram"-lhes as pevides que têm, envoltas em tripas, como as de
melão, mas são crespas e pretas como grãos de pimenta da Índia, às quais talhadas se apara
a casca, como à maçã, e o que se come é da cor e brandura do melão, o sabor é doce e muito
gostoso. E estas sementes se semearam na Bahia, e nasceram logo; e tal agasalhado lhe fez
a terra que no primeiro ano se fizeram as árvores mais altas que um homem, e ao segundo
começaram de dar fruto, e se fizeram as árvores de mais de vinte palmos de alto, e pelo pé
tão grossas como um homem pela cinta; os seus ramos são as mesmas folhas arrumadas como as
das palmeiras; e cria"-se o fruto no tronco entre as folhas.

Entre estas árvores há machos,\footnote{ Mamão"-macho.} que não dão fruto como as
tamareiras, e umas e outras em poucos anos se fazem pelo pé tão grossas como uma pipa, e
d'avantagem.

Nesta terra da Bahia se cria outra fruta natural dela, que em tudo se parece com estes
mamões de cima, senão que são mais pequenos, à qual os índios chamam jaracatiá, mas têm a
árvore delgada, de cuja madeira se não usa. Esta árvore dá a frol branca, o fruto é
amarelo por fora, da feição e tamanho dos figos bêberas ou longais brancos, que têm a
casca dura e grossa, a que chamam em Portugal longais; desta maneira, tem esta fruta a
casca, que se lhe apara quando se come, tem bom cheiro, e o sabor toca de azedo, e tem
umas sementes pretas que se lançam fora.

\paragraph{[52] Em que se diz de algumas árvores de fruto que se dão na vizinhança do mar da
Bahia}\quad
Na vizinhança do mar da Bahia se dão umas árvores nas campinas e terras fracas, que se
chamam mangabeiras, que são do tamanho de pessegueiros. Têm os troncos delgados, e a folha
miúda, e a frol como a do marmeleiro; o fruto é amarelo corado de vermelho, como pêssego
calvos, ao qual chamam mangabas, que são tamanhas como ameixas e outras maiores, as quais
em verdes são todas cheias de leite, e colhem"-se inchadas para amadurecerem em casa, o que
fazem de um dia para o outro, porque, se amadurecem na árvore, caem no chão. Esta fruta se
come toda sem se deitar nada fora como figos, cuja casca é tão delgada que se lhe pela se
as enxovalham, a qual cheira muito bem e tem suave sabor, é de boa digestão e faz bom
estômago, ainda que comam muitas; cuja natureza é fria, pelo que é muito boa para os
doentes de febres por ser muito leve. Quando estas mangabas não estão bem maduras, travam
na boca como as sorvas verdes em Portugal, e quando estão inchadas são boas para conserva
de açúcar, que é muito medicinal e gostosa.\footnote{ Em Varnhagen (1851 e 1879), os onze
parágrafos seguintes, que dizem respeito aos araçazeiros, araticu, pino, abajiru, amaitim,
murici, api, copiuba, maçaranbiba e mucuri, estão inseridos no Capítulo 54. Parece mais
acertado, contudo, da maneira como estão dispostos no manuscrito da \textsc{bgjm},
permanecerem nesse Capítulo 52, já que são árvores que se localizam nas proximidades do
mar.}

Os araçazeiros são outras árvores que pela maior parte se dão em terra fraca na vizinhança
do mar, as quais são como macieiras na grandura, na cor da casca, no cheiro da folha e na
cor e feição dela. A frol é branca, da feição da da murta, e cheira muito bem. Ao fruto
chamam araçás, que são da feição das nêsperas, mas alguns muito maiores. Quando são verdes
têm a cor verde, e como são maduros têm a cor das perass; têm o olho como nêsperas, e por
dentro caroços, como elas, mas muito mais pequenos.\footnote{ No manuscrito da
\textsc{bgjm}, ``ao fruto se chamam araçás, que são da feição das nêsperas, e por dentro
caroços como elas, mas muito mais pequenos''.} Esta fruta se come toda, e tem ponta de
azedo mui saboroso, da qual se faz marmelada, que é muito boa, e melhor para os doentes de
câmaras.


Perto do salgado há outra casta de araçazeiros, cujas árvores são grandes, e o fruto como
laranja, mas mui saboroso, ao qual aparam a casca por ser muito grossa.

Araticu é uma árvore do tamanho de uma amoreira, cuja folha é muito verde escura, da
feição da da laranjeira, mas maior; a casca da árvore é como de loureiro, a madeira é
muito mole, a frol é fresca, grossa e pouco vistosa, mas o fruto é tamanho como uma pinha,
e em verde é lavrado como pinha, mas o lavor é liso e branco. Como este fruto é maduro,
arregoa"-se todo pelos lavores que ficam então brancos, e o pomo é muito mole, e cheira
muito bem, e tamanho é o seu cheiro que, estando em cima da árvore, se conhece debaixo que
está maduro pelo cheiro. Este fruto por natureza é frio e sadio; para se comer corta"-se em
quartos, lançando"-lhe fora umas pevides que têm, amarelas e compridas, como de cabaços,
das quais nascem estas árvores; e aparam"-lhe a casca de fora, que é muito delgada, e todo
mais se come, que tem muito bom sabor com ponta de azedo, a qual fruta é para a calma mui
desenfastiada.

Pino é uma árvore comprida, delgada e esfarrapada da folha, a qual é da feição e tamanho
da folha da parra. O seu fruto nasce em ouriço cheio de espinhos como os das castanhas, e
tirado este ouriço fora fica uma coisa do tamanho de uma noz, e da mesma cor, feição e
dureza, o qual lhe quebram, e tiram"-lhe de dentro dez ou doze pevides do tamanho de
amêndoas sem casca, mas mais delgadas, às quais tiram uma camisa parda que têm, como as
amêndoas, e fica"-lhes o miolo alvíssimo, que tem o sabor como as amêndoas; de que se fazem
todas as frutas doces que se costumam fazer das amêndoas, os quais pinhos, lançados em
água fria, incham e ficam muito desenfastiados para comer, e são bons para dor de cabeça,
de que se fazem amendoadas. Dão"-se estas árvores em ladeiras sobre o mar, e à vista dele,
em terras dependuradas.

Abagiru\footnote{ Em Varnhagen (1851 e 1879), ``abajerú''. Abajeru ou guajuru.} é uma
árvore baixa como carrasco, natural de onde lhe chega o rocio do mar, pelo que se não dão
estas árvores senão ao longo das praias, cuja folha é áspera, e dá uma frol branca e
pequena. O fruto é do mesmo nome da feição e tamanho das ameixas de cá, e de cor roxa;
come"-se como ameixas, mas tem maior caroço; o sabor é doce e saboroso.

Amaitim\footnote{ Em Varnhagen (1851 e 1879), ``amaytim''.} é uma árvore muito direita,
comprida e delgada; tem a folha como figueira, dá uns cachos maiores que os das uvas
ferrais; tem os bagos redondos, tamanhos como os das uvas mouriscas, e muito esfarrapados,
cuja cor é roxa, e cobertos de um pelo tão macio como veludo; metem"-se estes bagos na boca
e tiram"-lhe fora um caroço como de cereja, e a pele que tem o pelo, entre a qual e o
caroço tem um doce mui saboroso como o sumo das boas uvas.

Api\footnote{ Em Varnhagen (1851 e 1879), ``apé''.} é uma árvore do tamanho e feição das
oliveiras, mas tem a madeira áspera e espinhosa como romeira, a folha é da feição de
pessegueiro e da mesma cor. Esta árvore dá um fruto do mesmo nome, da feição das amoras,
mas nunca são pretas, e têm a cor brancacenta; come"-se como as amoras; tem bom sabor, com
ponta de azedo, mui apetitoso para quem tem fastio; as quais árvores se dão ao longo do
mar e à vista dele.

Murici é uma árvore pequena, muito seca da casca e da folha, cuja madeira não serve para
nada; dá umas frutas amarelas, mais pequenas que cerejas, que nascem em pinhas como elas,
com os pés compridos; a qual fruta é mole e come"-se toda; cheira e sabe a queijo do
Alentejo que requeima. Estas árvores se dão nas campinas perto do mar, em terras fracas.

Copiuba\footnote{ Copiúba, Cupiúba, Copiúva ou Cupiúva.} é uma árvore da feição do
loureiro, assim na cor da casca do tronco como na folha, a qual carrega por todos os ramos
de uma fruta preta do mesmo nome, maior que murtinhos, e toma tantos ordinariamente que
negrejam ao longe. Esta fruta se come como uvas, e têm o sabor delas quando as vindimam,
que estão muito maduras, e tem uma pevide preta que se lhe lança fora. Dão"-se estas
árvores ao longo do mar e dos rios por onde entra a maré.

Maçarandiba\footnote{ Maçarandiba ou maçaranduba.} é uma árvore real, de cuja madeira se
dirá ao diante. Só lhe cabe aqui dizer do seu fruto, que é da cor dos medronhos e do seu
tamanho, cuja casca é tesa e tem duas pevides dentro, que se lhe lançam fora com a casca;
o mais se lhe come, que é doce e muito saboroso; e quem come muita desta fruta que se
chama como a árvore, pegam"-se"-lhe os bigodes com o sumo dela, que é muito doce e pegajoso
e para os índios colherem esta fruta cortam as árvores pelo pé como fazem a todas que são
altas. Estas se dão ao longo do mar ou à vista dele.

Mucuri\footnote{ Em Varnhagen (1851 e 1879), ``apé''.} é uma árvore grande que se dá perto
do mar, a qual dá umas frutas amarelas, tamanhas como abricoques, que cheiram muito bem, e
têm grande caroço; o que se lhe come é de maravilhoso sabor, e aparam"-lhe a casca de fora.

Engá\footnote{ Inga ou Ingá.} é árvore desafeiçoada que se não dá senão em terra boa, de
cuja lenha se faz boa decoada\footnote{ Decoada: água fervida com cinzas das fornalhas
usada para livrar de impurezas o caldo da cana nas caldeiras, deixando o açúcar mais
puro.} para os engenhos. E dá uma fruta da feição das alfarrobas da Espanha, que tem
dentro umas pevides como as das alfarrobas, e não se lhe come senão um doce que tem em
derredor das pevides, que é muito saboroso.

Cajá é uma árvore comprida, com grande copa como pinheiro; tem a casca grossa e áspera, e
se a picam deita um óleo branco como leite em fio, que é muito pegajoso. A madeira é muito
mole e serve para fazer decoada para os engenhos; dá frol branca como a de macieira, e o
fruto é amarelo do tamanho das ameixas, tem grande caroço e pouco que comer, a casca é
como a das ameixas. Esta fruta arregoa, se lhe chove, como é madura, a qual cai com o
vento no chão, e cheiram muito bem o fruto e as flores, que são brancas e formosas; o
sabor é precioso, com ponta de azedo, cuja natureza é fria e sadia; dão esta fruta aos
doentes de febres, por ser fria e apetitosa, e chama"-se como a árvore, que se dá ao longo
do mar.

Bacuripari\footnote{ Bacupari.} é outra árvore de honesta grandura, que se dá perto do
mar, e quando a cortam corre"-lhe um óleo grosso de entre a madeira e a casca, muito
amarelo e pegajoso como visco. Dá esta árvore um fruto tamanho como fruta nova, que é
amarelo e cheira muito bem; e tem a casca grossa como laranja, a qual se lhe tira muito
bem, e tem dentro dois caroços juntos, sobre os quais tem o que se lhe come, que é de
maravilhoso sabor.

Piqui é uma árvore real, de cuja madeira se dirá adiante, a qual árvore dá frutas como as
castanhas, cuja casca é parda e tesa e, tirada, ficam umas castanhas alvíssimas, que sabem
como pinhões crus e cada árvore dá disto muito.

\paragraph{[53] Que trata da árvore dos ambus, que se dá pelo sertão da Bahia}\quad
Ambu\footnote{ Ambu ou umbu, fruto do umbuzeiro.} é uma árvore pouco alegre à vista,
áspera da madeira, e com espinhos como romeira, e do seu tamanho, a qual tem a folha
miúda. Dá esta árvore umas flores brancas, e o fruto, do mesmo nome, do tamanho e feição
das ameixas brandas, e tem a mesma cor e sabor, e o caroço maior. Dá"-se esta fruta
ordinariamente pelo sertão, no mato que se chama a caatinga, que está pelo menos afastado
vinte léguas do mar, que é terra seca, de pouca água, onde a natureza criou a estas
árvores para remédio da sede que os índios por ali passam. Esta árvore lança das raízes
naturais outras raízes tamanhas e da feição das botijas, outras maiores e menores,
redondas e compridas como batatas, e acham"-se algumas afastadas da árvore cinquenta e
sessenta passos, e outras mais ao perto.

E para o gentio saber onde estas raízes estão, anda batendo com um pau pelo chão, por cujo
tom o conhece, onde cava e tira as raízes de três e quatro palmos de alto, e outras se
acham à frol da terra, das quais se tira uma casca parda que tem, como a dos inhames, e
ficam alvíssimas e brandas como maçãs de coco; cujo sabor é mui doce, e tão sumarento que
se desfaz na boca tudo em água frigidíssima e mui desencalmada; com o que a gente que anda
pelo sertão mata a sede onde não acha água para beber, e mata a fome comendo esta raiz,
que é mui sadia, e não fez nunca mal a ninguém que comesse muito dela. Destas árvores há
já algumas nas fazendas dos portugueses, que nasceram dos caroços dos ambus, onde dão o
mesmo fruto e raízes.

\paragraph{[54] Em que se diz de algumas árvores de fruto afastadas do mar}\quad
Afastado do mar da Bahia, e perto dele, se dão umas árvores que chamam Sabucaí,\footnote{
Sapucaia, árvore nativa da Mata Atlântica com semente oleaginosa e comestível.} que são
mui grandes, de cujo fruto tratamos aqui somente. Esta árvore toma tanta frol amarela, que
se lhe não enxerga a folha ao longe, a qual frol é muito formosa, mas não tem nenhum
cheiro. Nasce desta frol uma bola de pau tão dura como ferro, que está por dentro cheia de
fruta. Terá esta bola uma polegada de grosso, e tem a boca tapada com uma tapadoura tão
justa que se não enxerga a junta dela, a qual se não despega senão como a fruta que está
dentro e de vez, que esta bola cai no chão, a qual tem por dentro dez ou doze
repartimentos, e em cada um uma fruta tamanha como uma castanha de Espanha, ou mais
comprida; as quais castanhas são muito alvas e saborosas, assim assadas como cruas; e
despegadas estas bolas das castanhas e bem limpas por dentro, servem de grais\footnote{
Gral: recipiente usado para triturar substâncias sólidas.} ao gentio, onde pisam o sal e a
pimenta.

Piquiá é uma árvore de honesta grandura, tem a madeira amarela e boa de lavrar, a qual dá
um fruto tamanho como marmelos que têm o nome da árvore; este fruto tem a casca dura e
grossa como cabaço, de cor parda por fora, e por dentro é todo cheio de um mel branco
muito doce; e tem misturado umas pevides como de maçãs, o qual mel se lhe come em sorvos,
e refresca muito no verão.

Macugê\footnote{ Mucujê.} é uma árvore comprida, delgada e muito quebradiça e dá"-se em
areias junto dos rios, perto do salgado, e pela terra dentro dez ou doze léguas. Quando
cortam esta árvore, lança de si um leite muito alvo e pegajoso, que lhe corre em fio; a
qual dá umas frutas do mesmo nome, redondas, com os pés compridos e cor verdoenga, e são
tamanhas como maçãs pequenas; e quando são verdes travam muito, e são todas cheias de
leite. Colhem"-se inchadas para amadurecerem em casa, e como são maduras tomam a cor
almecegada; e comem"-se todas como figos, cujo sabor é mui suave, e tal que lhe não ganha
nenhuma fruta da Espanha, nem de outra parte; e tem muito bom cheiro.

Jenipapo é uma árvore que se dá ao longo do mar e pelo sertão, cujo fruto aqui tratamos
somente. A sua folha é como de castanheiro, a frol é branca, da qual lhe nasce muita
fruta, de que toma cada ano muita quantidade; as quais são tamanhas como limas, e da sua
feição; são de cor verdoenga, e como são maduras se fazem de cor pardaça e moles, e têm
honesto sabor e muito que comer, com algumas pevides dentro, de que estas árvores nascem.
Quando esta fruta é pequena, faz"-se dela conserva, e como é grande, antes de amadurecer,
tinge o sumo dela muito, com a qual tinta se tinge toda a nação do gentio do Brasil em
lavores pelo corpo,\footnote{ Em Varnhagen (1851 e 1879), ``a nação do gentio em
lavores''.} e quando põe esta tinta é branca como água, e como se enxuga se faz preta como
azeviche;\footnote{ Azeviche: variedade compacta de carvão fóssil, substância mineral de
cor muito negra.} e quanto mais a lavam, mais preta se faz; e dura nove dias, no cabo dos
quais se vai tirando. Tem virtude esta tinta para fazer secar as bostelas\footnote{
Bostela: crosta de ferida.} das boubas aos índios, e a quem se cura com ela.


Pela terra adentro há outra árvore, a que chamam goti,\footnote{ Em Varnhagen (1851 e
1879), ``a que chamam guti''. Oiti, árvore nativa do Brasil, de fruto comestível.} que é
de honesta grandura; dá uma fruta do mesmo nome, do tamanho e cor das peras pardas, cuja
casca se lhe apara, e come em talhadas; mas tem grande caroço,\footnote{ Em Varnhagen
(1851 e 1879), ``se lhe apara, mas tem grande caroço''.} e o que se lhe come se tira em
talhadas, como às peras, e é muito saboroso; e lançadas essas talhadas em vinho não têm
preço. Faz"-se desta fruta marmelada muito gostosa, a qual tem grande virtude para estancar
câmaras de sangue.

Nas campinas há outra árvore, a que chamam ubucaba,\footnote{ Ubucaba ou Ibabiraba, árvore
nativa do Brasil, de frutos redondos e comestíveis.} cuja madeira é mole, e dá umas frutas
pretas e miúdas como murtinhos, que se comem, e têm sabor mui sofrível.

Mondururu\footnote{ Mundururu ou mandapuçá, arbusto nativo do Brasil.} é outra árvore que
dá umas frutas pretas, tamanhas como avelãs, que se comem todas, lançando"-lhes fora umas
pevides brancas que tem, a qual fruta é muito saborosa.

Há outra árvore como laranjeira, que se chama comichã,\footnote{ Grumixama, guamixã ou
gurumixama é o fruto da grumixameira.} a qual carrega todos os anos de umas frutas
vermelhas, tamanhas e de feição de murtinhos, que se comem todas lançando"-lhes fora uma
pevide preta que têm, que é a semente destas árvores, a qual fruta é muito gostosa.

Mandiba é uma árvore grande que dá fruto do mesmo nome tamanho como cerejas, de cor
vermelha, e muito doce; come"-se como sorva, lançando"-lhe a casca fora e uma pevide que tem
dentro, que é a sua semente.\footnote{ Em Varnhagen (1851 e 1879), ``lançando"-lhe o caroço
fora''.}

Cambuí é uma árvore delgada de cuja madeira se não usa, a qual dá uma frol branca, e o
fruto amarelo do mesmo nome; do tamanho, feição e cor das maçãs de anáfega. Esta fruta é
mui saborosa, e tem ponta de azedo; lança"-se"-lhe fora um carocinho que tem dentro como
coentro.

Dá"-se no mato perto do mar e afastado dele uma fruta que se chama curuanha, cuja árvore é
como vides, e trepa por outra árvore qualquer, a qual tem pouca folha; o fruto que dá é de
uns oito dedos de comprido e de três a quatro de largo, de feição da fava, o qual se parte
pelo meio como fava e fica em duas metades, que têm dentro três e quatro caroços, da
feição das colas de Guiné, da mesma cor e sabor, os quais caroços têm virtude para o
fígado. Estas metades têm a casquinha muito delgada como maçãs, e o mais que se come é da
grossura de uma casca de laranja; tem extremado sabor; comendo"-se esta fruta crua, sabe e
cheira a camoesas, e assada tem o mesmo sabor delas assadas; faz"-se desta fruta marmelada
muito boa, a qual por sua natureza envolta no açúcar cheira a almíscar, e tem o sabor de
perada almiscarada; e quem a não conhece entende e afirma que é perada.

Cambucá é outra árvore de honesta grandura, que dá umas frutas amarelas do mesmo nome,
tamanhas como abricoques, mas têm maior caroço e pouco que comer; é muito doce e de
honesto sabor.

\paragraph{[55] Em que se contêm muitas castas de palmeiras que dão fruto pela terra da
Bahia no sertão e algumas junto ao mar}\quad
Como há tanta diversidade de palmeiras que dão fruto na terra na Bahia, convém que as
arrumemos todas neste capítulo, começando logo em umas a que os índios chamam
perina,\footnote{ Em Varnhagen (1851 e 1879), ``pindoba''. Pindoba ou pindova é a
designação comum para certos tipos de palmeiras.} que são muito altas e grossas, que dão
frol como as tamareiras e o fruto em cachos grandes como os coqueiros, cada um dos quais é
tamanho que não pode um negro mais fazer que levá"-lo às costas; em os quais cachos tem os
cocos tamanhos como perass pardas grandes, e têm a casca de fora como um coco, e outra
dentro de um dedo de grosso, muito dura, e dentro dela um miolo maciço com esta casca, de
onde se tira com trabalho, o qual é tamanho como uma bolota, e muito alvo e duro para quem
tem ruins dentes; e se não é de vez, é muito tenro e saboroso; e de uma maneira e de outra
é bom mantimento para o gentio quando não tem mandioca, o qual faz destes cocos azeite
para as suas mezinhas.

Do olho destas palmeiras se tiram palmitos façanhosos de cinco a seis palmos de comprido,
e tão grossos como a perna de um homem. De junto do olho destas palmeiras tira o gentio
três e quatro folhas cerradas, que se depois abrem a mão, com as quais cobrem as casas, a
que chamam pindobusu,\footnote{ Em Varnhagen (1851 e 1879), ``a que chamam pindobuçu''.}
com o que fica uma casa por dentro, depois de coberta, muito formosa; a qual palma no
verão é fria, e no inverno quente; e se não fora o perigo do fogo, é muito melhor e mais
sadia cobertura que a da telha.

Anajá"-mirim é outra casca de palmeiras bravas que dão muito formosos palmitos, e o fruto
como as palmeiras acima; mas são os cocos mais pequenos e as palmas que se lhe tiram de
junto dos olhos têm a folha mais miúda, com que também cobrem as casas onde se não acham
as palmeiras acima. Os cachos destas palmeiras e das outras acima nascem em uma maçaroca
parda de dois a três palmos de comprido, e como este cacho quer lançar a frol, arrebenta
esta maçaroca ao comprido e sai o cacho para fora, e a maçaroca fica muito lisa por dentro
e dura como pau; da qual se servem os índios como de gamelas,\footnote{ Gamela: vaso de
madeira.} e ficam da feição de almadia.\footnote{ Almadia: embarcação comprida e estreita,
construída a partir de um só tronco de árvore.}

Há outras palmeiras bravas que chamam japarasaba,\footnote{ Em Varnhagen (1851 e 1879),
``que chamam japeraçaba''. Japeraçaba ou piaçaba, palmeira nativa do Brasil.} que também
são grandes árvores; mas não serve a folha para cobrir casas, porque é muito rara e não
cobre bem, mas serve para remédio de quem caminha pelo mato cobrir com elas as choupanas,
as quais palmeiras dão também palmito no olho e seus cachos de cocos, tamanhos como um
punho, com miolo como os mais, que também serve de mantimento ao gentio, e de fazerem
azeite; o qual e o de cima têm o cheiro muito fartum.\footnote{ Fartum ou fortum: cheiro
desagradável ou nauseante decorrente da decomposição ou modificação que sofre uma
substância gordurosa em contato com o ar.}

Pati\footnote{ Em Varnhagen (1851 e 1879), ``paty''.} é outra casta de palmeiras bravas
muito compridas e delgadas; as mais grossas são pelo pé como a coxa de um homem, têm a
rama pequena, mole e verde"-escura. Os palmitos que dão são pequenos, e os cocos tamanhos
como nozes, com o seu miolo pequeno que se come. Destas árvores se usa muito, porque têm a
casca muito dura, que se fende ao machado muito bem, da qual se faz ripa para as casas, a
que chamam pataíba, que é tão dura que com trabalho a passa um prego; e por dentro é
estopenta,\footnote{ Estopento: filamentoso como a estopa.} a qual ripa quando se lavra
por dentro cheira a maçãs maduras.


Há outras palmeiras, que chamam beri,\footnote{ Em Varnhagen (1851 e 1879), ``a que chamam
bory''. Buri ou imburi.} que têm muitos nós, que também dão cocos em cachos, mas são
miúdos; estas têm a folha da parte de fora verde e da de dentro branca, com pelo como
marmelos, as quais também dão palmitos muito bons.

Pinsandós\footnote{ Em Varnhagen (1851 e 1879), ``piçandós''. Piçandó ou guriri, palmeira
nativa da América do Sul.} são umas palmeiras bravas e baixas que se dão em terras fracas;
e dão uns cachos de cocos pequenos e amarelos por fora, que é mantimento para quem anda
pelo sertão, muito bom, porque tem o miolo muito saboroso como avelãs, e também dão
palmitos.

As principais palmeiras bravas da Bahia são as que chamam urucurins,\footnote{ Em
Varnhagen (1851 e 1879), ``ururucuri''.} que não são muito altas, e dão uns cachos de
cocos muito miúdos, do tamanho e cor dos abricoques, aos quais se come o de fora, como os
abricoques, por ser brando e de sofrível sabor; e quebrando"-lhe o caroço, donde se lhe
tira um miolo como o das avelãs, que é alvo e tenro e muito saboroso, os quais coquinhos
são mui estimados de todos. Estas palmeiras têm o tronco fofo, cheio de um miolo alvo e
solto como o cuscuz, e mole; e quem anda pelo sertão tira esse miolo e coze"-o em um
alguidar ou tacho,\footnote{ No manuscrito da \textsc{bgjm}, ``cheio de um miolo e cozido
em um alguidar ou tacho''.} sobre o fogo, onde se lhe gasta a umidade, e é mantimento
muito sadio, substancial e proveitoso para os que andam pelo sertão, a que chamam
farinha"-de"-pau.

Patioba é como palmeira nova no tronco e olho, e dá umas folhas de cinco a seis palmos de
comprido e dois e três de largo; é de cor verde, tesa como pergaminho, e serve para cobrir
as casas no lugar onde se não acha outra, e para as choupanas dos que caminham; quando se
estas folhas secam, fazem"-se em pregas tão lindas como de leques da Índia; e quando
nascem, é feitas em pregas,\footnote{ Em Varnhagen (1851 e 1879), ``quando nascem, saem
feitas em pregas''.} como está um leque estando fechado; dá palmitos pequenos, mas mui
gostosos.

\paragraph{[56] Em que se declaram as ervas que dão fruto na Bahia, que não são árvores}\quad
Como na Bahia se criam algumas frutas que se comem, em ervas que não fazem árvores,
pareceu decente arrumá"-las neste capítulo apartadas das outras árvores. E comecemos logo a
dizer dos maracujás, que é uma rama como hera e tem a folha da mesma feição, a qual atrepa
pelas árvores e as cobre todas, do que se fazem nos quintais ramadas muito frescas, porque
duram sem se secar muitos anos. A folha da erva é muito fria e boa para
desafogar,\footnote{ Desafogar: aliviar.} pondo"-se em cima de qualquer nascida ou chaga e
tem outras muitas virtudes; e dá uma flor branca muito formosa e grande que cheira muito
bem, de onde nascem umas frutas como laranjas pequenas, muito lisas por fora; a casca é da
grossura da das laranjas de cor verde"-claro, e tudo o que tem dentro se come, que além de
ter bom cheiro tem suave sabor.

Esta fruta é fria de sua natureza e boa para doentes de febres, tem ponta de azedo e é mui
desenfastiada; e enquanto é nova, faz"-se dela boa conserva; e enquanto não é bem madura, é
muito azeda.

Camapu\footnote{ Em Varnhagen (1851 e 1879), ``canapu''.} é uma erva que se parece com
erva"-moura, e dá uma fruta como bagos de uvas brancas coradas do sol e moles, a qual se
come, mas não tem sabor senão para os índios.

Modurucu\footnote{ Mandacaru.} é nem mais nem menos que uma figueira das que se plantam
nos jardins de Portugal, que tem as folhas grossas, a que chamam figueiras"-da"-índia; estas
têm as folhas de um palmo de comprido e quatro dedos de largo e um de grosso, e nascem as
folhas nas pontas umas das outras, as quais são todas cheias de espinhos tamanhos e tão
duros como agulhas, e tão agudas como elas, e dão o fruto nas pontas e nas ilhargas das
folhas, que são uns figos tamanhos como os lâmparos, vermelhos por fora, com a casca
grossa que se não come; o miolo é de malhas brancas e pretas; o branco é alvíssimo, e o
preto como azeviche, cujo sabor é mui apetitoso e fresco; o que se cria nas areias ao
longo do mar.

Marujaiba\footnote{ Marajaíba.} são uns ramos espinhosos; mas, limpos dos espinhos, ficam
umas canas pretas que servem de bordões, os quais têm a folha como canas de
roça,\footnote{ Em Varnhagen (1851 e 1879), ``de bordões como cana de rota''.} cujos
espinhos são pretos, e tão agudos como agulha. Nos pés destes ramos se dão uns cachos como
os das tamareiras, feitos os fios em cordões cheios de bagos,\footnote{ No manuscrito da
\textsc{bgjm}, ``cana de roça, cujos cachos como os das tamareirias, feitos os fios''.}
como os de uvas ferrais, e do mesmo tamanho, 
os quais têm a casca dura e roxa por fora, e o caroço dentro como cerejas, o
qual com a casca se lhe lança fora; e gosta"-se de um sumo que tem dentro, doce e suave.

Ao longo do mar se criam umas folhas largas, que dão um fruto a que chamam
caroatá,\footnote{ Em Varnhagen (1851 e 1879), ``carauatá''. Gravatá.} que é da feição de
maçaroca, e amarelo por fora; tem bom cheiro, a casca grossa e tesa, a qual se lança fora
para se comer o miolo, que é mui doce, mas empola"-se a boca a quem come muita fruta desta.

Há uma erva que se chama nambu,\footnote{ Em Varnhagen (1851 e 1879), ``nhamby''. Nhambi.}
que se parece na folha com coentro, e queima como mastruços, a qual os comem índios e os
mestiços crua, e temperam as panelas dos seus manjares com ela, de quem é mui estimada.

\paragraph{[57] Em que se declara a propriedade dos ananases não nomeados}\quad
Não foi descuido deixar os ananases para este lugar por esquecimento; mas deixamo"-los para
ele, porque, se lhe déramos o primeiro, que é o seu, não se puseram os olhos nas frutas
declaradas no capítulo atrás; e para o pormos só, pois se lhe não podia dar companhia
conveniente a seus merecimentos.

Ananás é uma fruta do tamanho de uma cidra grande, mas mais comprida; tem olho da feição
das alcachofras, e o corpo lavrado como alcachofra molar, e com uma ponta e bico em cada
sinal das pencas, mas é todo maciço; e muitos ananases lançam no olho e ao pé do fruto
muitos outros tamanhos como alcachofras. A erva em que se criam os ananases é da feição da
que em Portugal chamam erva"-babosa, e tem as folhas armadas, e do tamanho da erva"-babosa,
mas não são tão grossas; a qual erva ou ananaseiro espiga cada ano no meio como cardo, e
lança um grelo da mesma maneira, e em cima dele lhe nasce o fruto, tamanho como
alcachofra, muito vermelho, o qual assim como vai crescendo, vai perdendo a cor e
fazendo"-se verde; e como vai amadurecendo, se vai fazendo amarelo acataçolado de verde, e
como é maduro conhece"-se pelo cheiro, como o melão. Os ananaseiros se transpõem de uma
parte para outra, e pegam sem se secar nenhum; ainda que estejam com as raízes para o ar
fora da terra ao sol mais de um mês, os quais dão novidades daí a seis meses; e, além dos
filhos que lançam ao pé do fruto e no olho, lançam outros ao pé do ananaseiro, que também
espigam e dão cada seu ananás, como a mãe de onde nasceram, os quais se transpõem, e os
olhos que nascem no pé e no olho do ananás. Os quais ananaseiros duram na terra,\footnote{
Em Varnhagen (1851 e 1879), ``no olho do ananás. Os ananaseiros duram''.} sem se secarem,
toda a vida, e se andam limpos de erva, que entre eles nasce, quanto mais velhos são dão
mais novidades, os quais não dão o fruto todos juntamente, mas em todo o ano uns mais
temporões que os outros, e no inverno dão menos fruto que no verão, em que vem a força da
novidade, que dura oito meses. Para se comerem os ananases hão de se aparar muito bem,
lançando"-lhes a casca toda fora, e a ponta de junto do olho, por não ser tão doce, e,
depois de aparado este fruto, o cortam em talhadas redondas, como de laranja, ou ao
comprido, ficando"-lhe o grelo que tem dentro, que vai correndo do pé até o olho; e quando
se corta fica o prato cheio do sumo que dele cai,\footnote{ Em Varnhagen (1851 e 1879),
``que dele sai''.} e o que se lhe come é da cor dos gomos de laranja, e alguns há de cor
mais amarela; e desfaz"-se tudo em sumo na boca, como o gomo de laranja, mas é muito mais
sumarento; o sabor dos ananases é muito doce, e tão suave que nenhuma fruta da Espanha lhe
chega na formosura, no sabor e no cheiro; porque uns cheiram a melão muito fino, outros a
camoesas; mas no cheiro e no sabor, não há quem se saiba afirmar em nada, porque ora sabe
e cheira a uma coisa, ora a outra. A natureza deste fruto é quente e úmida, e muito danosa
para quem tem ferida ou chaga aberta; os quais ananases sendo verdes são proveitosos para
curar chagas com eles, cujo sumo come todo o câncer e a carne podre, do que se aproveita o
gentio; e com tanta maneira come esta fruta, que alimpam com as suas cascas a ferrugem das
espadas e facas, e tiram com elas as nódoas da roupa ao lavar; de cujo sumo, quando são
maduras, os índios fazem vinho, com que se embebedam; para o que os colhem mal maduros,
para ser mais azedo, do qual vinho todos os mestiços e muitos portugueses são mui
afeiçoados. Desta fruta se faz muita conserva, aparada da casca, a qual é muito formosa e
saborosa, e não tem a quentura e umidade de quando se come em fresco.

\paragraph{[58] Daqui por diante se vão arrumando as árvores e ervas de virtude que há na
Bahia}\quad
Não se podiam arrumar em outra parte que melhor estivessem as árvores de virtude que após
das que dão fruto; e seja a primeira a árvore do bálsamo, que se chama cabureíba, que são
árvores mui grandes de que se fazem eixos para engenhos, cuja madeira é pardaça e
incorruptível. Quando lavram esta madeira cheira a rua toda a bálsamo, e todas as vezes
que se queira cheira muito bem. Desta árvore se tira o bálsamo suavíssimo, dando"-lhe
piques até um certo lugar, de onde começa de chorar este suavíssimo licor na mesma hora, o
qual se recolhe em algodões, que lhe metem nos golpes; e como estão bem molhados do
bálsamo, os espremem em uma prensa, onde tiram este licor, que é grosso e da cor do
arrobe,\footnote{ Arrobe: extrato de uva levado ao fogo.} o qual é milagroso para curar
feridas frescas, e para tirar os sinais delas no rosto. O caruncho deste pau, que se cria
no lugar de onde saiu o bálsamo, é preciosíssimo no cheiro; e amassa"-se com o mesmo
bálsamo, e fazem desta massa contas, que depois de secas ficam de maravilhoso cheiro.

De tão santa árvore como a do bálsamo merece ser companheira e vizinha a que chamam
copaiba, que é árvore grande cuja madeira não é muito dura, e tem a cor pardaça; e faz"-se
dela tabuado; a qual não dá fruto que se coma, mas um óleo
santíssimo em virtude, o qual é da cor e clareza de azeite sem sal; e antes de se saber de
sua virtude servia de 
noite nas candeias. Para se tirar este óleo das árvores lhes dão um talho com um machado
acima do pé, até que lhe chegam à veia, e como lhe chegam, corre este óleo em fio, e lança
tanta quantidade cada árvore que há algumas que dão duas botijas cheias, que tem cada uma
quatro canadas.\footnote{ Canada: antiga medida de capacidade para líquidos que equivalia
a aproximadamente dois litros.} Este óleo tem muito bom cheiro, e é excelente para curar
feridas frescas, e as que levam pontos da primeira curam, soldam se as queimam com ele, e
nas estocadas ou feridas que não levam pontos se curam com ele, sem outras mezinhas; com o
qual se cria a carne até encourar, e não deixa criar nenhuma corrupção nem matéria. Para
frialdades, dores de barriga e pontadas de frio é este óleo santíssimo. E é tão sutil que
se vai de todas as vasilhas, se não são vidradas; e algumas pessoas querem afirmar que até
no vidro míngua; e quem se untar com este óleo há de se guardar do ar, porque é
prejudicial.

\paragraph{[59] Em que se trata da virtude da embaiba, e caraobuçu e caraobamirim}\quad
Embaiba\footnote{ Embaúba ou Imbaúba.} é uma árvore comprida e delgada, que faz uma copa
em cima, de pouca rama; a folha é como de figueira, mas tão áspera que os índios cepilham
com elas os seus arcos e hastes de dardos, com as quais se põe a madeira melhor que com a
pele de lixa. Os frutos desta árvore são umas candeias em cachos como as dos castanheiros,
e como amadurecem as comem os passarinhos e os índios, cujo saibo é adocicado, e tem
dentro uns grãos de milho, como os figos passados, que é a semente de que estas árvores
nascem, as quais se não dão em mato virgem, se não na terra que já foi aproveitada; e,
assim no tronco como nos ramos é toda oca por dentro, onde se criam infinidades de
formigas miúdas. Tem o olho desta árvore grandes virtudes para com ele curarem feridas, o
qual, depois de pisado, se põe sobre feridas mortais, e se curam com ele com muita
brevidade, sem outros unguentos; e o entrecasco deste olho tem ainda virtude, com o que
também se curam feridas e chagas velhas; e tais curas se fazem com o olho desta árvore, e
com o óleo da copaiba, que se não ocupam na Bahia cirurgiões, porque cada um o é em sua
casa.

Caraobusu\footnote{ Em Varnhagen (1851 e 1879), ``caraobuçu''.} é uma árvore como
pessegueiro, mas tem a madeira muito seca e a folha miúda, como a da amendoeira; esta
madeira é muito dura e de cor almecegada, a qual se parece com o pau das Antilhas; cuja
casca é delgada; da folha se aproveitam os índios, e com ela pisada curam as boubas,
pondo"-a com o sumo em cima das bostelas ou chagas, com o que se secam muito depressa; e
quando isto não basta, queimam em uma telha estas folhas, e com o pó delas feitas em
carvão, secam estas bostelas; do que também se aproveitam os portugueses, que têm
necessidade deste remédio para curarem seus males, de que muitos têm muitos.

Caraobamirim é outra árvore da mesma casta, senão quanto é mais pequena, e tem a folha
mais miúda, da qual se aproveitam como da caraoba de cima, e dizem que tem mais virtudes;
com as folhas desta árvore, cozidas, tomam os portugueses doentes destes males suadouros,
tomando o bafo desta água, estando muito quente, do que se acham muito bem; e lhes faz
sair todo o humor para fora e secar as bostelas, tomando destes novos suadores, e o sumo
da mesma folha bebido por xarope.

\paragraph{[60] Que trata da árvore da almécega, e de outras árvores de virtude}\quad
Há outras árvores de muita estima, a que os índios chamam ubiracica; tem honesta grandura,
de cuja madeira se não aproveitam, mas valem"-se de sua resina, de que lança grande
quantidade, e quando a deita é muito mole e pegajosa; a qual é maravilhosa almécega, que
faz muita vantagem à que se vende nas boticas, e para uma árvore lançar muito picam"-na ao
longo da casca com muitos piques, e logo começa a lançar por eles almécega, que lhe os
índios vão apanhando com umas folhas, onde a vão ajuntando e fazem em pães.

Esta almécega é muito quente por natureza, da qual fazem emplastos para defensivo da
frialdade, e para soldar carne quebrada, e para fazer vir a furo postemas, os quais faz
arrebentar por si, e lhes chupa de dentro os carnegões,\footnote{ Carnegão: parte central
de furúnculos e tumores, constituída de pus e tecidos necrosados.} e derretida é boa para
escaldar feridas frescas, e faz muita vantagem à terebintina de bétula;\footnote{
Terebintina: resina com propriedades medicinais extraída, nesse caso, da árvore do gênero
\textit{Bétula}, nativa do hemisfério norte.} com a qual almécega se fazem muitos
unguentos e emplastos para quebraduras de perna, à qual os índios chamam icica.

Corneiba\footnote{ Corneíba ou Aroeira.} é uma árvore que na folha, na frol, na baga e no
cheiro é a aroeira da Espanha, e tem a mesma virtude para os dentes, e é diferente na
grandura das árvores, que são tamanhas como oliveira, de cuja madeira se faz boa cinza
para decoada nos engenhos. Naturalmente se dão estas árvores em terra de areia, debaixo de
cujas raízes se acha muita anime\footnote{ Anime: resina aromática.} que é, no cheiro, na
vista e na virtude como a de Guiné, pelo que se entende que o destila de si, pelo baixo do
tronco da árvore, porque se não acha junto de outras árvores.

Em algumas partes do sertão da Bahia se acham árvores de canafístula, a que o gentio chama
geneuna, mas de agrestes dão a canafístula muito grossa e comprida; e tem a côdea áspera,
mas quebrada, e da mesma feição, assim nas pevides que tem como no preto; que se come e
tem o mesmo saibo, da qual não usa o gentio, porque não sabe o para que ela presta. Em
algumas fazendas há algumas árvores de canafístula, que nasceram das que foram de São Tomé
que dão o fruto mui perfeito como o das Índias.

Cuipeuna\footnote{ Cuipuna.} é uma árvore pontualmente como a murta de Portugal, e não tem
outra diferença que fazer maior árvore e ter a folha maior no viço da terra, a qual se dá
pelos campos da Bahia, cuja frol e o cheiro dela é da murta, mas não dá murtinhos; da qual
murta se usa na Misericórdia para cura dos penitentes e para todos os lavatórios, para que
ela serve, porque tem a mesma virtude dessecativa.

Ao longo do mar da Bahia nascem umas árvores que têm o pé como parras, as quais atrepam
por outras árvores grandes, por onde lançam muitos ramos como vides,\footnote{ No
manuscrito da \textsc{bgjm}, ``Ao longo do mar da Bahia nascem umas árvores grandes por
onde lança muitos ramos como vides''.} as quais se chamam mucunas, cujo fruto são umas
favas redondas e aleonadas na cor, e do tamanho de um tostão, as quais têm um círculo
preto, e na cabeça um olho branco. Estas favas para comer são peçonhentas, mas têm grande
virtude para curarem com elas feridas velhas desta maneira: depois de serem estas favas
bem secas, hão"-se de pisar muito bem, e cobrir as chagas com os pós delas, as quais comem
todo o câncer e carne podre.

Criam"-se nesta terra outras árvores semelhantes às de cima, que atrepam por outras
maiores, que se chamam o cipó das feridas, as quais são umas favas aleonadas pequenas, da
feição das de Portugal, cuja folha pisada e posta nas feridas, sem outros unguentos as
cura muito bem.

Há uns mangues, ao longo do mar, a que o gentio chama apareiba, que têm a madeira vermelha
e rija, de que se faz carvão; cuja casca é muito áspera, e tem tal virtude que serve aos
curtidores para curtir toda a sorte de peles, em lugar de sumagre,\footnote{ Sumagre: pó
grosso, produzido a partir da trituração das folhas secas, das flores e da casca de planta
de mesmo, utilizado em tinturaria e curtume de peles e de couro.} com o que fazem tão bom
curtume como com ele. Estes mangues fazem as árvores
muito direitas, dão umas candeias verdes compridas, que têm
dentro uma semente como lentilhas, de que elas nascem.

\paragraph{[61] Daqui por diante se vai relatando as qualidades das ervas de virtude que se
criam na Bahia, e comecemos logo a dizer da erva"-santa e outras ervas semelhantes}\quad
Petume\footnote{ Petume: fumo, tabaco.} é a erva a que em Portugal chamam santa; onde há
muito dela pelas hortas e quintais, pelas grandes mostras que tem dado da sua virtude, com
a qual se têm feito curas estranhas; pelo que não diremos desta erva se não o que é
notório e todos, como é matarem com o seu sumo os vermes que se criam em feridas e chagas
de gente descuidada; com a qual se curam também as chagas e feridas das vacas e das éguas
sem outra coisa, e com o sumo desta erva lhe encouram. Deu na costa do Brasil uma praga no
gentio, como foi adoecerem do sesso\footnote{ Sesso: nádegas.} e criarem bichos
nele, da qual doença morreu muita soma desta gente, sem se entender de quê; e depois que
se soube o seu mal, se curaram com esta erva"-santa, e se curam hoje em dia os tocados
deste mal, sem terem necessidade de outra mezinha.

A folha desta erva, como é seca e curada, é muito estimada dos índios e mamelucos e dos
portugueses, que bebem o fumo dela, ajuntando muitas folhas destas torcidas umas às
outras, e metidas em um canudo de folha de palma, e põe"-se"-lhe o fogo por uma banda, e
como faz brasa metem este canudo pela outra banda na boca, e sorvem"-lhe o fumo para dentro
até que lhe sai pelas ventas fora. Todo o homem que se toma do vinho, bebe muito deste
fumo, e dizem que lhe faz esmoer o vinho. Afirmam os índios que quando andam pelo mato e
lhes falta o mantimento, matam a fome e a sede com este fumo, pelo que o trazem sempre
consigo, e não há dúvida senão que este fumo tem virtude contra a asma, e os que são
doentes dela se acham bem com ele, cuja natureza é muito quente.

Pino é pontualmente na folha, como as que em Portugal chamam figueira"-do"-inferno. Esta
erva dá o fruto em cachos cheios de bagos, tamanhos como avelãs, todos cheios de bicos;
cada um destes bagos tem dentro um grão pardo, tamanho como um feijão, o qual pisado se
desfaz todo em azeite, que serve na candeia; bebido serve tanto como purga de canafístula;
e para os doentes de cólica, bebido este azeite, se lhe passa o acidente logo; as folhas
desta erva são muito boas para desafogarem chagas e postemas.

Giticusu\footnote{ Em Varnhagen (1851 e 1879), ``jeticuçu''. Jeticuçú é uma trepadeira, de
cujas raízes se extrai fécula purgativa.} é uma erva que nasce pelos campos e lança por
cima da terra uns ramos como batatas, os quais dão umas sementes pretas como
ervilhacas\footnote{ Ervilhaca: erva usada como forragem e adubo verde.} grandes; deitam
estas ervas uma raízes por baixo da terra como batatas, que são maravilhosas para purgar,
do que se usa muito na Bahia; as quais raízes se cortam em talhadas, em verdes, que são
por dentro alvíssimas e secam"-nas muito bem ao sol; e tomam dessas talhadas, depois de
secas, para cada purga o peso de dois reais de prata, e lançando em vinho ou em água muito
bem pisado se dá a beber ao doente de madrugada e faz maravilhas. Destas raízes se faz
conserva em açúcar raladas muito bem, como cidrada, e tomada pela manhã uma colher desta
conserva faz"-se com ela mais obra, que com açúcar rosado de Alexandria.

Pecacuem\footnote{ Ipecacuanha ou poaia.} são uns ramos que atrepam como parra, cuja folha
é pequena, redonda e brancacenta; as suas raízes são como de junça brava, mas mais
grossas, as quais têm grande virtude para estancar câmaras; do que se usa tomando uma
pequena desta raiz pisada e lançada em água; posta a serenar e dada a beber ao doente de
câmaras de sangue lhas faz estancar logo.

\paragraph{[62] Em que se declara o modo com que se cria o algodão, e de sua virtude, e de
outras ervas que fazem árvore}\quad
Maniim chamam os índios ao algodão, cujas árvores parecem marmeleiros arruados em pomares;
mas a madeira dele é como de sabugueiro, mole e oca por dentro; a folha parece de
parreira, com o pé comprido, vermelho, com o sumo da qual se curam feridas espremendo
nelas. A frol do algodão é uma campainha amarela muito formosa, de onde nasce um
capucho,\footnote{ Capucho ou capulho: cápsula que envolve o algodão. 
Em Varnhagen (1851 e 1879), ``nasce um capulho''.} que ao longe parece
uma noz verde, o qual se fecha com três folhas grossas e duras, da feição das com que se
fecham os botões das rosas; e como o algodão está de vez, que é de agosto por diante,
abrem"-se estas folhas, com que se fecham estes capuchos, e vão"-se secando e mostrando o
algodão que têm dentro, muito alvo, e se não o apanham logo, cai no chão; e em cada
capulho destes estão quatro de algodão, cada um do tamanho de um capulho de seda; e cada
capucho destes tem dentro um caroço preto, com quatro ordens de carocinhos pretos, e cada
carocinho é tamanho e da feição do feitio dos ratos, que é a semente de onde o algodão
nasce, o qual no mesmo ano que se semeia dá a novidade.

Estes caroços de algodão come o gentio pisados, e depois cozidos, com o que se faz em
papas que chamam mingau.

As árvores destes algodoeiros duram sete a oito anos e mais, quebrando"-lhe cada ano as
pontas grandes a mão, porque se secam, para que lancem outros filhos novos, em que tomam
mais novidade, os quais algodões se alimpam a enxada, duas e três vezes cada ano, para que
a erva os não acanhe.

Camará é uma erva que nasce pelos campos, que cheira a erva"-cidreira, a qual faz árvore
com muitos ramos como de roseira"-de"-alexandria; cuja madeira é seca e quebradiça, a folha
é como de erva"-cidreira; as flores são como cravos"-de"-túnis, amarelos e da mesma feição,
mas de feitio mais artificioso. Cozidas as folhas e flores desta erva, tem a sua água
muito bom cheiro e virtude para sarar sarna e comichão, e para secar chagas de
boubas,\footnote{ No manuscrito da \textsc{bgjm}, ``virtude para secar a carne e
comichão''.} lavando"-as com esta água quente, do que se usa muito naquelas partes; onde há
outra casta deste camará, que dá flores brancas da mesma feição, a qual tem a mesma
virtude; e como cai a frol, assim a uma como a outra, ficam"-lhe umas camarinhas
denegridas, que comem os meninos e os passarinhos, que é a semente, de que esta erva
nasce.

Nas campinas da Bahia se dão urzes de Portugal, da mesma feição, assim nos ramos como na
frol, mas não dão camarinhas; dos quais ramos, cozidos na água, se aproveitam os índios
para secar qualquer humor ruim.

Às canas da Bahia chama o gentio ubá, as quais têm folhas como as da Espanha, e as raízes
da mesma maneira que lavram a terra muito; as quais, cozidas em água, têm a mesma virtude
dessecativa que as da Espanha. Estas canas são compridas, cheias de nós por fora e maciças
por dentro, ainda que têm o miolo mole e estopento. Espigam estas canas cada ano, cujas
espigas são de quinze e vinte palmos de comprido, de que os índios fazem flechas com que
atiram. E também se dão na Bahia as canas da Espanha, mas não crescem tanto como as da
terra.

Jaborandi é uma erva que faz árvore da altura de um homem, e lança umas varas em nós, como
canas, por onde estalam muito como as apertam; a folha será de palmo de comprido, e da
largura da folha da cidreira, a qual cheira a hortelã francesa, e tem a aspereza da
hortelã ordinária; a água cozida com estas folhas é loura e muito cheirosa e boa para
lavar o rosto ao barbear; quem tem a boca danada, ou chagas nela, mastigando as folhas
desta erva duas ou três vezes cada dia, e trazendo"-a na boca, a cura muito depressa;
queimadas estas folhas, os pós delas alimpam o câncer das feridas, sem dar nenhuma pena, e
tem outras muitas virtudes. Esta erva dá umas candeias como castanheiros, onde se cria a
semente de que nasce.

Nascem outras ervas pelo campo, a que chamam os índios caapiam,\footnote{ Caapiá.} que têm
flores brancas da feição dos bem"-me"-queres, onde há umas sementes como gravanços; das
quais e das flores se faz tinta amarela como açafrão muito fino, de que usam os índios no
seu modo de tintas. A árvore que faz esta erva é como a do alecrim, e tem a folha mole, e
a cor verde"-claro, como alface.

Dão"-se ao longo da ribeira da Bahia umas ervas, a que os índios chamam jaborandiba; e dão
o mesmo nome da de cima, por se parecer nos ramos com ela; e os homens que andaram na
Índia lhe chamam bétele, por se parecer em tudo com ele. A folha desta erva, metida na
boca, requeima como folhas de louro, a qual é muito macia, e tem o verde muito escuro. A
árvore que faz esta erva é tão alta como um homem, os ramos têm muitos nós, por onde
estala muito. Quem se lava com estas ervas cozidas nas partes eivadas do fígado,\footnote{
Em Varnhagen (1851 e 1879), ``Quem se lava com ela cozida nas partes''.} lhas cura em
poucos dias; e cozidos os olhos e comestos, são saníssimos para este mal do fígado; e
mastigadas estas folhas e trazidas na boca, tiram a dor de dentes.

\paragraph{[63] Em que se declara a virtude de outras ervas menores}\quad
Há outras ervas menores pelos campos de muita virtude, de que se aproveitam os índios e os
portugueses, das quais faremos menção brevemente neste capítulo, começando na que o gentio
chama tararucu, e os portugueses fedegosos. Esta erva faz árvore do tamanho das
mostardeiras, e tem as folhas em ramos, arrumadas como folhas de árvores, as quais são
muito macias, da feição das folhas de pessegueiro, mas têm o verde muito escuro, e o
cheiro da fortidão da arruda.

Estas folhas deitam muito sumo, se as pisam; o qual de natureza é muito frio, e serve para
desafogar chagas; com este fumo curam o sesso dos índios e das galinhas, porque criam nele
muitas vezes bichos de que morrem, se lhe não acodem com tempo.

Estas ervas dão umas flores amarelas como as da páscoa, das quais lhes nascem umas bainhas
com semente como ervilhacas, de que nascem; pelos campos da Bahia se dão algumas ervas que
lançam grandes braços como meloeiros,\footnote{ Em Varnhagen (1851 e 1879), ``como
ervilhacas. Pelos campos da Bahia''.} que atrepam se acham por onde, as quais dão umas
flores brancas que se parecem até no cheiro com a frol de legação em Portugal; cujos olhos
comem os índios doentes de boubas, e outras pessoas; e dizem acharem"-se bem com eles, e
afirma"-se que esta é a salsaparrilha das Antilhas.

Caapeba\footnote{ Em Varnhagen (1851 e 1879), ``capeba''. Capeba ou Pariparobe.} é uma
erva que nasce em boa terra perto da água, e faz árvore como couve espigada; mas tem a
folha redonda, muito grande, com pé comprido, a qual é muito macia; a árvore faz um grelo
oco por dentro, e muito tenro e, depois de bem espigada, lança umas candeias crespas em
que dá a semente, de que nasce. Esta erva é de natureza frigidíssima, com cujas folhas
passadas pelo ar do fogo se desafoga toda a chaga e inchação que está esquentada,
pondo"-lhe estas folhas em cima; e se a fogagem é grande, seca"-se esta folha; de maneira
que fica áspera, e como está seca se lhe põe outras até que o fogo abrande.

Criam"-se outras ervas pelos campos da Bahia, que se chama guaxima, da feição de
tanchagem;\footnote{ No manuscrito da \textsc{bgjm}, ``campos da Bahia, da feição da
tanchagem''.} mas tem as tolhas mais pequenas, da feição de escudete, e tem o pé comprido,
as quais são brancas da banda de baixo, cuja natureza é fria; e posta sobre chagas e
coçaduras das pernas que têm fogagem, as desafoga, e encouram com elas, sem outros
unguentos.

Pelos mesmos campos se criam outras ervas, a que o gentio chama caapiá, e os portugueses
malvaísco, porque não tem outra diferença do de Portugal que ser muito viçoso, mas tem a
mesma virtude; da qual usam os médicos da Bahia, quando é necessário, para fazerem vir a
furo as postemas e inchações.

Peipeseba\footnote{ Em Varnhagen (1851 e 1879), ``peipeçaba''. Peipeçaba.} é uma erva que
se parece com belverde, que se dá nos jardins de Portugal, da qual fazem as vassouras na
Bahia, com que varrem as casas; cuja natureza é fria, a qual pisam os índios e curam com
elas feridas frescas; e também entre os portugueses se cura com o sumo desta erva o mal do
sesso, para o que tem grande virtude; a qual não dá frol, mas semente muito miúda, de que
nasce.

Por estes campos se cria outra erva, a que os índios chamam cuanapoaná,\footnote{ Em
Varnhagen (1851 e 1879), ``chamam campuava''.} que são mentrastos, nem mais nem menos que
os da Espanha, e têm a mesma virtude, cuja água cozida é boa para lavar os pés; e são
tantos que juncam com eles as igrejas pelas endoenças, em lugar de rosmaninhos.

Nas campinas da Bahia se cria outra erva, a que o gentio chama caácuá,\footnote{ Em
Varnhagen (1851 e 1879), ``caamcuam''.} que tem as folhas de três em três juntas, e são da
cor da salva; dá a frol roxa, de que nasce uma bainha como de tremoços, que tem dentro
umas sementes como lentilhas grandes, a qual erva tem o cheiro muito fortum, que causa dor
de cabeça a quem a colhe; o gado que come esta erva engorda muito no primeiro ano com ela,
e depois dá"-lhe como câmaras, de que morre; pelo qual respeito houve quem quis desinçar
esta erva de sua fazenda, e pôs um dia mais de duzentos escravos a arrancá"-la do campo, os
quais não puderam aturar o trabalho mais que até o meio"-dia, porque todos adoeceram com o
cheiro dela da dor de cabeça, o que fez espanto; e os homens que têm conhecimento da
erva"-besteira da Espanha, e a viram nesta terra, afirmam que é esta mesma erva a besteira.

\paragraph{[64] Daqui por diante se vai dizendo das árvores reais e o para que servem,
começando neste capítulo 64, que trata do vinhático e cedro}\quad
Como temos dito das árvores de fruto, e das que têm virtude para curar enfermidades,
convém que se declare as árvores reais que se dão na Bahia, de que se fazem os engenhos de
açúcar e outras obras, de cuja grandeza há tanta fama.

E parece razão que se dê o primeiro lugar ao vinhático, a que o gentio chama sabijejuba,
cuja madeira é amarela e doce de lavrar, a qual é incorruptível, assim sobre a terra como
debaixo dela, e serve para as rodas dos engenhos, para outras obras deles, e para casas e
outras obras"-primas. Há também façanhosos paus desta casta, que se acham muitos de cem
palmos de roda, e outros daqui para baixo, mui grandes; mas os muitos grandes pela maior
parte são ocos por dentro, dos quais se fazem canoas tão compridas
como galeotas; e acham"-se muitos paus maciços, de que se tira tabuado de três, quatro e
cinco palmos de largo. Esta madeira se não se dá senão em terra boa e afastada do mar.


Os cedros da Bahia não têm diferença dos das Ilhas senão na folha, que a cor da madeira e
o cheiro e brandura ao lavrar é todo um; a estas árvores chama o gentio
acujucatinga,\footnote{ Em Varnhagen (1851 e 1879), ``acajacatinga''. Acajacatinga.} cuja
madeira se não corrompe nunca; da qual se acham mui grandes paus que pela maior parte são
ocos, mas acham"-se alguns maciços, de que se tira tabuado de três e quatro palmos de
largo.

Pelo rio dos Ilheos trouxe a cheia um pau de cedro ao mar tamanho que se tirou dele a
madeira e tabuado com que se madeirou e forrou a igreja da Misericórdia, e sobejou
madeira; a qual é branda de lavrar e proveitosa para obras"-primas e outras obras dos
engenhos, de que se faz muito tabuado para o forro das casas e para barcos; e faz uma
vantagem o cedro da Bahia ao das Ilhas, que logo perde a fortidão do cheiro, e o
fato\footnote{ Fato: roupa, vestuário.} que se mete nas caixas de cedro não toma nenhum
cheiro delas, e as obras do cedro das Ilhas nunca jamais perderam o cheiro, e
danam com ele o fato que se nelas agasalha.

\paragraph{[65] Que trata das qualidades do pequihi e de outras madeiras reais}\quad
Pequihi\footnote{ Pequi.} é uma árvore grande que se dá perto do mar, em terras baixas,
úmidas e fracas; acham"-se muitas dessas árvores de quarenta e cinquenta palmos de roda,
cuja madeira é parda, estopenta, muito pesada, de que se fazem gangorras, mesas,
virgens\footnote{ Virgem: viga de madeira fixada à prensa de mandioca.} e esteios para
engenhos, a qual dura sem apodrecer para fim dos fins, ainda que esteja lançada sobre a
terra ao sol e à chuva. Quando lavram esta madeira cheira a vinagre, e sempre que se tiram
dela os cavacos molhados, ainda que esteja cortada de cem anos, e já viu meter um prego
por uma gangorra, que havia dezesseis anos que estava debaixo da telha de um engenho, e
tanto que o prego começou a entrar para dentro, começou a rebentar pelo mesmo furo um
torno de água em fio que correu até o chão, o qual cheirava a vinagre; e se metem os
cavacos desta madeira no fogo, em quatro horas não pega neles, e já quando pega não fazem
brasa, nem levantam labareda. É esta madeira tão pesada que em a deitando na água se vai
ao fundo, da qual se fazem bons liames e outras obras para barcas grandes e navios.

Guoaparajú\footnote{ Em Varnhagen (1851 e 1879), ``Quaparaiva''. Guaparaíba ou guaparaíva,
conhecida também como mangue"-vermelho.} é outra árvore real muito grande, de que se acham
muitas de trinta e quarenta palmos de roda, cuja madeira é vermelha e mui fixa, que nunca
se viu podre; de que se fazem gangorras, mesas, virgens e esteios para engenhos e outras
obras; e acham"-se muitas árvores tão compridas desta casta que, cortadas direito, o grosso
dá vigas de oitenta a cem palmos de comprido, fora o delgado que fica no mato de que se
fazem frechais e tirantes dos engenhos. Estas árvores são naturais de várzeas de areia,
vizinhas do salgado; e são tão pesadas que, em lançando a madeira na água, se vai logo ao
fundo.

Há outras árvores também naturais de várzeas de areia, a que o gentio chama
jutiapeba,\footnote{ Em Varnhagen (1851 e 1879), ``jutaypeba''. Jataipeba, jataipeva ou
itu.} cuja madeira é vermelhaça, e muito fixa, que nunca apodrece; e é muito dura ao
lavrar; acham"-se muitas árvores desta casta de cinquenta a sessenta palmos de roda; e pela
maior parte estas grandes são ocas por dentro; mas há outras de honesta grandeza, maciças,
de que se fazem gangorras, mesas, virgens, esteios e outras obras dos engenhos, como são
os eixos. Não são estas árvores muito altas, por se desordenarem pelo alto, lançando
grandes troncos; mas tiram"-se delas gangorras de cinquenta a sessenta palmos de comprido,
e a madeira é boa de lavrar, ainda que é muito dura e tão pesada que se vai na água ao
fundo.

Sabucai\footnote{ Sapucaia.} é outra árvore real que nunca apodreceu, assim debaixo da
terra como sobre ela, de cujo fruto tratamos atrás, cuja madeira é vermelhaça, dura e tão
pesada que se vai ao fundo; da qual se acham grandes árvores, de que se fazem gangorras,
mesas, eixos, fusos, virgens, esteios e outras obras dos engenhos. Quando se cortam estas
árvores tinem nelas os machados como se dessem por ferro, onde se quebram muito.

\paragraph{[66] Em que se acaba de concluir a informação das árvores reais que se criam na
Bahia}\quad
Masarandiba\footnote{ Em Varnhagen (1851 e 1879), ``maçarandiba''. Maçarandiba ou
maçaranduba.} é outra árvore real, de cujo fruto já fica dito atrás; são naturais estas
árvores da vizinhança do mar; e acham"-se muitas de trinta a quarenta palmos de roda, de
que se fazem gangorras, mesas, eixos, fusos, virgem, esteios e outras obras dos engenhos,
cuja madeira é de cor de carne de presunto, e tão dura de lavrar que não há ferramenta que
lhe espere, e tão pesada que se vai ao fundo. Estas árvores são tão compridas e direitas
que se aproveitam do grosso delas de cem palmos para cima, e nunca se corrompem.

Há outra árvore real que se chama jataí"-nandi,\footnote{ Em Varnhagen (1851 e 1879),
``jatay"-mondé''. Jataí"-mondé.} que não é tamanha como as de cima, mas de honesta grandura,
de que se fazem eixos, fusos, virgens, esteios e outras obras dos engenhos, cuja madeira é
amarela, de cor formosa, muito rija e doce de lavrar e incorruptível; e é tão pesada que
se vai ao fundo; e não se dá em ruim terra.

Nas várzeas de areia se dão outras árvores reais, a que os índios chamam cunhá,\footnote{
Em Varnhagen (1851 e 1879), ``curuá''.} as quais se parecem na feição, na folha, na cor da
madeira, com carvalhos; e acham"-se alguns de vinte e cinco a trinta palmos de roda, de que
se fazem gangorras, mesas, eixos, virgens, esteios e outras obras miúdas; mas não é muito
fixo ao longo da terra; a qual também serve para liames de navios e barcos e para tabuado;
e de pesado se vai ao fundo.

Há outras árvores reais, a que os portugueses chamam angelim, e os índios
andurababajari,\footnote{ Em Varnhagen (1851 e 1879), ``andurababapari''. Andira, ibiariba
ou argelim.} as quais são muito grandes e acham"-se muitas de mais de vinte palmos de roda,
de que fazem gangorras, mesas, eixos, virgens, esteios e outras obras dos engenhos e das
casas de vivenda, e boas caixas por ser madeira leve e boa de lavrar, e honesta cor.

Juquitiba é outra árvore real, façanhosa na grossura e comprimento, de que se fazem
gangorras, mesas dos engenhos e outras obras, e muito tabuado; e já se cortou árvore
destas tão compridas e grossas, que deu no comprimento e grossura duas gangorras, que cada
uma pelo menos há de ter cinquenta palmos de comprido, quatro de assento e cinco de alto.
Esta madeira tem a cor brancacenta, é leve e pouco durável, onde lhe chove; não se dão
estas árvores em ruim terra.

Ubirahem\footnote{ Em Varnhagen (1851 e 1879), ``ubiraem''.} é outra árvore real de que se
acham muitas de vinte palmos de roda para cima, de que se fazem gangorras, mesas, virgens,
esteios dos engenhos e tabuado para navios, e outras obras, cuja cor é amarelaça; não
muito pesada, e boa de lavrar.

Pelas campinas e terra fraca se criam muitas árvores, que se chamam sepepiras,\footnote{
Sebipira.} que em certo tempo se enchem de flor como de pessegueiro; não são árvores muito
façanhosas na grandura, por serem desordenadas nos troncos, mas tiram"-se delas virgens,
esteios e fusos para os engenhos; a madeira é parda e muito rija, e tão liada que nunca
fende; e para liação de navios e barcos é a melhor que há no mundo, que sofre melhor o
prego e nunca apodrece; de que se também fazem carros muito bons; e é tão pesada esta
madeira que se vai ao fundo.

Mutumusu\footnote{ Em Varnhagen (1851 e 1879), ``putumujú''. Putumuju.} é uma árvore real,
e não se dá senão em terra muito boa; não são árvores muito grandes, mas dão três palmos
de testa. Esta é das mais fixas madeiras que há no Brasil, porque nunca se corrompe, da
qual se fazem eixos, virgens, fusos, esteios para os engenhos, e toda a obra de casas e de
primor; a cor desta madeira é amarela com umas veias vermelhas; é pesada e dura, mas muito
doce de lavrar.

Há outras árvores, que se chamam urucuranas, que são muito compridas e de grossura, que
fazem delas virgens e esteios para os engenhos, e outras muitas obras de casas, e tabuado
para navios, a quem o gusano\footnote{ Gusano: moluscos bivalves que cavam buracos na
madeira submersa dos cais e embarcações.} não faz mal; a qual madeira é pesada, e vai"-se
ao fundo; tem a cor de carne de fumo, e é boa de lavrar e serrar.

\paragraph{[67] Daqui por diante se trata das madeiras meãs}\quad
Madeiras meãs, e de toda a sorte, há tantas na Bahia, que se não podem contar, das quais
diremos alguma parte das que chegaram à nossa notícia.

E comecemos no camasari\footnote{ Em Varnhagen (1851 e 1879), ``camaçari''. Camaçari.} que
são árvores naturais de areias e terras fracas. São estas árvores muito compridas e
direitas, das quais se tiram frechas e tirantes para engenhos de cem palmos, e de cento e
vinte de comprido e dois de largo, palmo e meio afora o delgado da ponta, que serve para
outras coisas; a qual madeira serve para toda a obra das casas, do que se faz muito
tabuado para elas e para os navios. Esta madeira tem a cor vermelhaça, boa de lavrar e
melhor de serrar. Destas árvores se fazem mastros para os navios, e se foram mais leves
eram melhores que os de pinho, por serem mais fortes; as quais árvores são tão roliças,
que parecem torneadas. Cria"-se entre a casca e o âmago desta árvore uma matéria grossa e
alva, que pega como terebintina; e é da mesma cor, ainda que mais alva; o que lança
dando"-lhe piques na casca em fio, e o mesmo lança ao lavrar e ao serrar,\footnote{ No
manuscrito da \textsc{bgjm}, ``ao lavrar e ao tarar''.} e lançam muita quantidade; e se
toca nas mãos, não se tira senão com azeite; e se isto não é terebintina, parece que
fazendo"-lhe algum cozimento, que engrossará e coalhará como resina, que servirá para brear
os navios, de que se fará muita quantidade, por haver muita soma destas árvores à borda da
água, e cada uma deita muita matéria desta.

Guanandi\footnote{ Guanandi ou anani.} é uma árvore comprida, e não muito grossa, cuja
madeira é amarelada, que serve para obra de casas em parte onde lhe não toque a água; a
casca desta árvore é muito amarela por dentro e entre ela e o pau lança um leite grosso, e
de cor amarela muito fina, o qual pega como visco; e com ele armam os moços aos pássaros;
da qual madeira se não faz conta, nem se aproveitam dela senão em obras de pouca dura; as
quais são muito compridas, direitas e roliças, de que se fazem mastros para navios.

\paragraph{[68] Que trata das árvores que dão a embira, de que fazem cordas e estopa para
calafetar os navios}\quad
Acham"-se pelos matos muitas árvores de que se tira a embira para calafetar,\footnote{ No
manuscrito da \textsc{bgjm}, ``se tira a embira, e''.} e comecemos a dizer das que se
chamam invirisu,\footnote{ Em Varnhagen (1851 e 1879), ``enviroçu''. Embiruçu ou
imbiruçu.} que são árvores grandes, cuja madeira é mole, e não se faz conta dela senão
para o fogo; as quais têm a casca áspera por fora, a qual se esfolha das árvores, e se
pisam muito bem, faz"-se branda como estopa, que serve para calafetar. Dão essas árvores
umas flores brancas como cebola"-cecém muito formosas, e da mesma feição, que estão
fechadas da mesma maneira, as quais se abrem como se põe o sol; e estão abertas até pela
manhã, enquanto lhe não dá o sol, e como lhe chega se tornam a fechar, e as que são mais
velhas caem no chão, cujo cheiro é suave, mas muito mimoso; e como apertam com elas não
cheiram.

Há uma árvore meã, que se chama ibiriba, de que se fazem esteios para os engenhos,
tirantes e flechais,\footnote{ Flechal: Viga que vai em cima das paredes, onde se pregam
os barrotes e caibros do teto de uma casa.} e outra obra de casa, tirando tabuado, por ser
má de serrar. Esta madeira é muito dura e má de lavrar, é muito forte para todo o
trabalho, e não há machado com que se possa torar, que não quebre ou se trate mal, é muito
boa de fender; a qual os índios fazem em fios para fachos com que vão mariscar, e para
andarem de noite; e ainda que seja verde cortada daquela hora, pega fogo nela como em
alcatrão; e não apaga o vento os fachos nela; e em casa servem"-se os índios das achas
desta madeira, como candeias, com que se servem de noite à falta delas. Estas árvores se
esfolham e abrem"-se à mão, as quais se fazem todas em fios muito compridos, que se fiam
como cânhamo, de que se fazem amarras e toda a sorte de cordoalha, que é tão forte como de
cairo;\footnote{ Cairo: filamento da entrecasca de certos tipos de cocos usado para fazer
cabos ou cordas.} e pisada esta casca muito bem, se faz tão branda e mais que estopa, com
o que se calafetam os navios e barcos; e para debaixo da água é muito melhor que estopa,
porque não apodrece na água, e incha muito.

Imbirete\footnote{ Em Varnhagen (1851 e 1879), ``embiriti''.} é outra árvore meã, cuja
madeira é mole, e do entrecasco dela se tira embira branca, com que fazem cordas tão alvas
como de algodão e morrões de espingarda muito bons, que se não apagam nunca, e fazem muito
boa brasa; o qual entrecasco se tira tão facilmente, que fazem os negros de Guiné dele
panos de cinco a seis palmos de largo, e do comprimento que querem; os quais amassam e
pisam com uns paus com que os fazem estender, e ficam tão delgados como lona, mas muito
macios, com os quais se cingem e cobrem.

Guoayambira\footnote{ Em Varnhagen (1851 e 1879), ``goayambira''.} é uma árvore pequena,
que não é mais grossa que a perna de um homem; cortam"-na os índios em rolos de dez, doze
palmos, e esfolam"-na inteira para baixo como coelho, e saem os entrecascos inteiros; de
que os índios fazem aljavas em que metem os arcos e flechas, a qual embira é muito alva;
de que fazem cordas e morrões de espingarda.

\paragraph{[69] Que trata de algumas árvores muito duras}\quad
O conduru\footnote{ Em Varnhagen (1851 e 1879) ``condurú''. Conduru ou cunduru.} é árvore
de honesta grandura, e acham"-se algumas que têm três palmos de testa, e não dão um palmo
de âmago vermelho, que todo o mais é branco que apodrece logo, e o vermelho é
incorruptível; de que se fazem leitos, cadeiras e outras obras delicadas. Destes condurus
novos se fazem espeques para os engenhos, porque não quebram, por darem muito de si quando
lhe fazem força.

Suasucanga\footnote{ Em Varnhagen (1851 e 1879), ``suaçucanga''.} é uma árvore pequena,
cujo tronco não é mais grosso que a perna de um homem, a madeira é alvíssima como marfim,
e com as mesmas águas, a qual é muito dura; e serve para marchetar em lugar de marfim.

Há outras árvores grandes de que se fazem esteios para os engenhos, a que os índios chamam
abiraetá,\footnote{ Em Varnhagen (1851 e 1879) ``ubiraetá''.} e os portugueses pau"-ferro,
por serem muito duras e trabalhosas de cortar, cuja madeira é pardaça e incorruptível, as
quais árvores se dão em terra de pedras e lugares ásperos.

Ubirapariba\footnote{ Ibirapariba.} é árvore grande, muito dura, de que os índios fazem os
seus arcos, a madeira tem a cor parda, e é muito dura de lavrar e cortar; que pelo ser se
não aproveitam destas árvores, por quebrarem os machados nelas; cuja madeira se não
corrompe, nem estalam os arcos que se dela fazem; nos quais se faz aleonada depois de
cortada; e é tão pesada que, em tocando nágua, se vai logo ao fundo.

Ubiraúna são árvores grandes de que se fazem esteios para os engenhos por se não corromper
nunca; cuja madeira é preta, muito dura de lavrar, e tão pesada que se vai ao fundo se a
lançam nágua.

Mandiocahi\footnote{ Mandiocaí.} é uma árvore assim chamada pelo gentio, de honesta
grossura e comprimento, de que se fazem esteios dos engenhos e virgens, por ser a madeira
muita dura, a qual é pesada e boa de lavrar, e de cor amarelaça.


Há outras árvores, a que o gentio chama ubirapiroqua;\footnote{ Em Varnhagen (1851 e
1879), ``ubirapiroca''. Ibirapiroca.} são árvores compridas, muito direitas, de que se
tira grossura até palmo e meio de testa, de que se fazem tirantes e flechais de casas.
Esta madeira é pesada e vai"-se ao fundo, e é muito rija e boa de lavrar; têm estas árvores
a casca lisa, a qual pela cada ano, e vem criando outra nova por baixo daquela que
pela.\footnote{ Em Varnhagen (1851 e 1879), ``daquela pele''.}

\paragraph{[70] Que trata das árvores que se dão ao longo do mar}\quad
Ao longo do mar se criam umas árvores, a que os portugueses chamam espinheiros, e os
índios tatagiba,\footnote{ Tataíba ou tatajuba.} que têm as folhas como romeira, e os
ramos cheios de espinhos; a madeira por fora é muito áspera, e por dentro amarela de cor
fina; a qual se lavra muito bem, sem embargo de ser dura; e é tão fixa que não há quem
visse nunca um pau deste podre, de que se fazem muitas obras boas.

Pelo salgado há uma casta de mangues, a que os índios chamam sereiba,\footnote{ Em
Varnhagen (1851 e 1879), ``sereîba''. Sereíba.} que se criam onde descobre a maré, os
quais lançam muitos filhos ao pé todos de uma grossura, delgados, direitos, de grossura
que servem para encaibrar as casas de mato, e os mais grossos servem para as casas dos
engenhos, por serem muito compridos e rijos, e de grossura bastante. Destes mangues se faz
também lenha para os engenhos, aos quais caem algumas folhas, que se fazem amarelas, de
que se mantêm os caranguejos, que por entre elas se criam; e dão estas árvores umas
espigas de um palmo, de feição das dos feijões, e têm dentro um fruto à maneira de favas,
de que tornam a nascer ao pé da mesma árvore, e por derredor dela.

Canapauba\footnote{ Em Varnhagen (1851 e 1879), ``canapaúba''. Canapaúba ou
mangue"-branco.} é outra casta de mangues, cujas árvores são muito tortas e desordenadas,
muito ásperas da casca, cujas pontas tornam para baixo em ramos muito lisos, enquanto
novos e direitos, e vêm assim crescendo para baixo, até que chega a maré; e como esta
chega, a eles logo criam ostras, com o peso das quais vêm obedecendo ao chão até que pega
dele, e como pega logo lança ramos para cima, que vão crescendo muito desafeiçoados e
lançam mil filhos ao longo da água, que têm tão juntos que se afogam uns aos outros.

\paragraph{[71] Em que se trata de algumas árvores moles}\quad
Há umas árvores muito grandes, a que o gentio chama copaubuçu,\footnote{ Em Varnhagen
(1851 e 1879), ``copaubuçú''.} cuja madeira é mole, e não serve senão para cinza, para os
engenhos fazerem decoada. Estas árvores têm umas raízes sobre a terra, feitas por tal
artifício, que parecem tábuas postas ali a mão, as quais lhe cortam ao machado; de que se
tiram tabuões, de que se fazem gamelas de cinco, seis palmos de largo, e sete e oito de
comprido, de onde se fazem também muitas rodelas, que são como as de adargueiro, e
d'avantagem na levidão, cuja madeira é estopenta e muito branda, que não fende.

Paraparaiba é uma árvore que se dá em boa terra que já foi lavrada, a qual em poucos anos
se faz muito alta e grossa, e tem a casca brancacenta, a qual ao longe parece na brancura
e grandura o álamo. Tem esta árvore a folha como figueira, mas os pés mais compridos, a
madeira é muito mole e oca por dentro, de que fazem bombas aos caravelões da costa; e por
dentro tem muitas infindas formigas.

Apeiba\footnote{ Em Varnhagen (1851 e 1879), ``apeyba''. Apeíba.} é uma árvore comprida
muito direita, tem a casca muito verde e lisa, a qual árvore se corta de dois golpes de
machado, por ser muito mole; cuja madeira é muito branca, e a que se esfola a casca muito
bem; e é tão leve esta madeira, que traz um índio do mato às costas paus destes de vinte e
cinco palmos de comprido e da grossura da sua coxa, para fazer dele uma jangada para
pescar no mar a linha; as quais árvores se não dão senão em terra muito boa.

Penaiba é uma árvore comprida e delgada, muito direita, cuja madeira é leve e de cor de
pinho, que serve para mastros e vergas das embarcações da terra, a qual dá de si muito e
não estala; mas não dura muitos anos, porque a corrompe a chuva.

Gerumari\footnote{ Em Varnhagen (1851 e 1879), ``geremari''.} é outra árvore que se dá
pela terra dentro, a qual é delgada no pé e muito grossa em cima; e dá umas favas brancas;
cuja madeira não serve mais que para o fogo.

Dão"-se nas campinas perto do mar umas árvores, que se parecem com cajueiros, de que já
falamos, que não dão fruto, que se chamam caiuapebu.\footnote{ Em Varnhagen (1851 e 1879),
``cajupeba''. Cajupeba ou cajueiro"-bravo.} Têm estas árvores a folha brancacenta, crespa e
áspera como de amoreira, e a casca dessas árvores é seca como de sobreiro. A madeira é
leve, mas muito liada, e não fende, de que se tiram curvas para barcos, e que se fazem
vasos de selas, e destas folhas podem manter bichos"-de"-seda, e os levarem a estas partes.

Pelo sertão da Bahia se criam umas árvores muito grandes em comprimento e grossura, a que
os índios chamam ubiragara, das quais fazem umas embarcações para pescarem pelo rio e
navegarem, de sessenta e setenta palmos de comprido, que são facílimas de fazer, porque se
cortam estas árvores muito depressa, por não ter dura mais que a casca e o âmago; é muito
mole e tanto que dois índios em três dias tiram com suas foices o miolo todo a estas
árvores, e fica a casca só, que lhes serve de canoas, tapadas as cabeças, em que se
embarcam vinte e trinta pessoas.

\paragraph{[72] Em que se apontam algumas árvores de cheiro}\quad
Entre as árvores de cheiro que se acham na Bahia, há uma a que os índios chamam carunje,
que se parece na folha, na casca e no cheiro aos loureiros da Espanha, mas não na baga;
cuja madeira é sobre o mole\footnote{ Mole: molhe.} que se gasta, no fogo dos engenhos.


Anhuibatá\footnote{ Em Varnhagen (1851 e 1879), ``anhaybatãa''. Inhaíba.} é uma árvore que
se dá em várzeas úmidas e de areia, a qual na grandeza e feição é como o louro, cuja
madeira é muito mole e de cor almecegada; o entrecasco desta árvore é da cor de canela; e
cheira, queima e sabe como canela; mas tem a quentura mais branda, e sem dúvida que parece
canela, e parece que se a beneficiarem, que será muito fina, porque o entrecasco dos ramos
queima mais do que o do tronco da árvore.

Jacaranda\footnote{ Jacarandá.} é uma árvore de bom tamanho, que se dá nas campinas, em
terras fracas, cuja madeira é preta com algumas águas; e é muito dura, e boa de lavrar
para obras"-primas; e é muito pesada, e não se corrompe nunca sobre a terra, ainda que lhe
dê o sol e a chuva, a qual tem muito bom cheiro.

Jucuriasu\footnote{ Em Varnhagen (1851 e 1879), ``jucuriaçu''. Jucuriaçu.} é uma árvore
que se dá em terras fracas, e não é demasiada na grandeza, mas, contudo, se acham algumas,
que dão três palmos de testa; a madeira desta árvore não se corrompe nunca, é dura, pesada
e muito boa de lavrar para obras"-primas. Há uma casta de cor parda, com águas pretas, e
outra vermelhaça, com águas também pretas, umas e outras da feição do chamalote; e umas e
outras têm o cheiro suavíssimo, e na casa onde se lavra sai o cheiro por toda a rua, e os
seus cavacos no fogo cheiram muito bem; a qual madeira é muito estimada em toda a parte
pelo cheiro e formosura.

Musutaiba\footnote{ Em Varnhagen (1851 e 1879), ``mucetaiba''. Mucetaíba.} é uma árvore
que se dá em terras boas e não é de demasiada grandeza, a que chamam em Pernambuco
pau"-santo;\footnote{ No manuscrito da \textsc{bgjm}, ``árvore que se dá em boa terra a que
chamam em Pernambuco pau"-santo''.} cuja madeira é de honesta grossura, muito rija e
pesada, mas boa de lavrar e melhor de tornear, e tem boas águas, para se dela fazer obras
de estima; nunca se corrompe do tempo, e cheira muito bem.

Ubirataya\footnote{ Ubirataía.} é uma árvore que não é grande, cuja madeira é mole, de cor
parda, que cheira muito bem; e na casa onde se queima recende o cheiro por toda a rua.

Intagapema\footnote{ Em Varnhagen (1851 e 1879), ``entagapema''. Entagapena.} é uma árvore
que tem a madeira dura, com água sobre aleonado, que cheira muito bem, de que se fazem
contas muito cortesãs, e o gentio as suas espadas.

\paragraph{[73] Em que se trata de árvores de que se fazem remos e hastes de lanças}\quad
Atrás tratamos do jenipapo no tocante ao fruto, agora lhe cabe tratar no tocante à
madeira; cujas árvores são altas e de honesta grossura, tem a folha como castanheiro; a
madeira é de cor branca, como buxo, de que se fazem muitos e bons remos, que duram mais
que os de faia; enquanto são verdes são pesados, mas depois de secos são muito leves; esta
madeira não fende nem estala, de que se faz também toda a sorte de poleame,\footnote{
Poleame: peça de madeira usada para passagem de cabos fixos ou de laborar em uma
embarcação.} por ser doce de lavrar e não fender; e cabos e cepos para toda a ferramenta
de toda a sorte.\footnote{ Em Varnhagen (1851 e 1879), ``por ser doce de lavrar; e
cabos''.}

Hoaquaã\footnote{ Em Varnhagen (1851 e 1879), ``Huacã''.} é outra árvore de que se fazem
remos para os barcos, a qual se dá em terras úmidas e de areia. São estas árvores de meã
grossura, e quando se lavram fazem um roxo claro muito formoso, mas dura"-lhe pouco a cor;
as quais, depois de derrubadas, as fendem os índios de alto a baixo em quartos, para
fazerem os remos, que não duram tanto como os de jenipapo.

Há outras árvores, a que os índios chamam ubiratinga, que não são grossas, mas compridas e
direitas, e têm a casca áspera; a cor da madeira é açafroada e boa de fender, o que se lhe
faz para fazerem hastes de lanças e arremessões, que se fazem muito formosos, e de dardos
que são mais pesados que os de Biscaia; mas mais duros e formosos. Dão"-se estas árvores em
terras baixas e úmidas perto do salgado.\footnote{ Em Varnhagen (1851 e 1879), ``e se
fazem muito formosos. Dão"-se estas árvores''.}

\paragraph{[74] Em que se diz de algumas árvores que têm ruim cheiro}\quad
Nestes matos se acham umas árvores meãs e direitas, de que se fazem obras de casas; a sua
madeira por fora é almecegada e o âmago por dentro muito preto; mas quando a lavram não há
quem lhe sofra o fedor, porque é pior que o de umas necessárias, e chegar os cavacos aos
narizes é morrer, que tão terrível fedor têm; e metendo"-se no fogo se refina mais o fedor;
a estas árvores chamam os índios ubirarema, que quer dizer ``madeira que fede muito''.

Há outra casta de ubirarema,\footnote{ Também conhecido como pau"-d'álho.} cujas árvores
são grandes e desordenadas nos troncos, como as oliveiras; cujos ramos, folhas, cascas e
madeira fedem a alhos, de feição que quem os aperta com as mãos lhe fica fedendo de
maneira que se lhe não tira em todo o dia o cheiro, e têm árvores as folhas da feição das
ameixeiras.

Ao pé de algumas árvores se criam uns ramos como parreira, da grossura e da feição de uma
corda meã, a que os índios chamam cipós, os quais trepam pelas árvores acima como as
videiras; os quais cipós cheiram a alhos, e quem pega deles não se lhe tira o cheiro em
todo aquele dia, por mais que se lave.

\paragraph{[75] Em que se apontam algumas árvores que dão frutos silvestres que se não
comem}\quad
Nos matos se criam umas árvores de honesta grandura, a que os índios chamam
comedoi,\footnote{ Em Varnhagen (1851 e 1879), ``comedoy''. Cumandá ou apucarana.} de cuja
madeira se não faz conta. Esta árvore dá umas bainhas como feijões, meio vermelhos e meio
pretos mui duros, de finas cores, que é a semente de que as árvores nascem, os quais
servem para tentos, e são para isso mui estimados.


Araticupana\footnote{ Em Varnhagen (1851 e 1879), ``araticurana''. Araticum.} é uma árvore
do tamanho e feição do marmeleiro, as quais se criam nos alagadiços, onde se ajunta a água
doce com a salgada, cuja madeira é mole e lisa que se esfola toda em lhe puxando pela
casca. Dão estas árvores um fruto tamanho como marmelos, lavrado pela casca, como pinha, e
muito liso, o qual arregoa como é maduro, e cheira muito bem. Este fruto comem os índios a
medo, porque têm para si que quando os caranguejos da terra fazem mal, que é por comerem
este fruto naquele tempo.

Anhangaquiabo quer dizer ``pente de diabo'', é árvore de bom tamanho, cujo fruto são umas
bainhas grandes; tem dentro de si uma coisa branca e dura, afeiçoada como pente, de que os
gentios se aproveitavam antes de comunicarem com os portugueses e se valerem dos seus
pentes.

Cuigijba\footnote{ Em Varnhagen (1851 e 1879), ``cuiêyba''. Cuieira ou cabaceira.} é uma
árvore tamanha como nogueira, e tem a folha como nogueira, a qual se não cria em ruim
terra, e dá umas flores brancas grandes. Da madeira se não trata, porque as não cortam os
índios, por estimarem muito o seu fruto, que é como melões, maiores e menores, de feição
redonda e comprida, a qual fruto não se dá entre as folhas, como as outras árvores, senão
pelo tronco da árvore e pelos braços dela, cada um por si; estando esta fruta na árvore, é
da cor dos cabaços verdes, e como os colhem, cortam"-nos pelo meio ao comprido, e
lançam"-lhe fora o miolo, e é como o dos cabaços; e vão curando estas peças até se fazerem
duras,\footnote{ Em Varnhagen (1851 e 1879), ``até se fazerem duas''.} dando"-lhe por
dentro uma tinta preta e por fora amarela, que se não tira nunca; ao que os índios chamam
cuias, que lhes servem de pratos, escudelas, púcaros, taças e de outras coisas.

Há outras árvores meãs, a que os índios chamam jatuaiba, cuja madeira é muito pesada, as
quais cai a folha cada ano, e torna arrebentar de novo. Esta árvore dá umas frutas
brancas, do tamanho e feição de azeitonas cordovesas.

Pelo sertão se criam umas árvores a que os índios chamam beribebas, que dão um fruto do
tamanho e feição de noz"-moscada, o qual amaruja e requeima como ela.

\paragraph{[76] Que trata dos cipós e o para que servem}\quad
Deu a natureza no Brasil, por entre os seus arvoredos, umas cordas muito rijas e muitas
que nascem aos pés das árvores e atrepam por elas acima, a que chamam cipós, com que os
índios atam a madeira das suas casas, e os brancos que não podem mais, com que escusam
pregadura; e em outras partes servem em lugar de cordas, e fazem deles cestos melhores que
de vimes, e serão da mesma grossura, mas têm comprimento de cinco e seis braças.

Nestes mesmos matos se criam outras cordas mais delgadas e primas, que os índios chamam
timbós; que são mais rijos que os cipós acima, que servem do mesmo, aos quais fendem
também em quatro partes, e ficam uns fios mui lindos como de rota da Índia em cadeiras, e
com estes fios atam a palma das casas quando as cobrem com ela, do que fazem também cestos
finos; e far"-se"-á deles tudo o que se faz da rota da Índia.

Há outra casta a que os índios chamam timborana, que é da mesma feição dos timbós, mas não
são tão rijos, do que se aproveitam os índios, quando não acham os timbós.

Criam"-se também nestes matos uns cipós muito grossos, a que os índios chamam
cipó"-embé,\footnote{ Cipó"-imbé.} cujo nascimento é também ao pé das árvores,\footnote{ No
manuscrito da \textsc{bgjm}, ``cipao este cujo''.} por onde atrepam; e são rijos que tiram
com eles as gangorras dos engenhos do mato e as madeiras grossas; pelas quais puxam cem e
duzentos índios, sem quebrarem, e se acertam de quebrar tornam"-se logo a atar, e com eles
varam as barcas em terra, e as deitam ao mar, e acham"-nos tão grossos como são
necessários, com os quais se escusam calabrotes de linho.

\paragraph{[77] Que trata de algumas folhas proveitosas que se criam no mato}\quad
Caáeté\footnote{ Em Varnhagen (1851 e 1879), ``cáeté''. Caeté.} é uma folha que se dá em
terra boa e úmida, que é da feição das folhas das alfaces estendidas, mas de quatro e
cinco palmos de comprido, e são muito tesas; as quais nascem em touças muito juntas, e têm
o pé de quatro e cinco palmos de comprido, e não fazem árvore.\footnote{ No manuscrito da
\textsc{bgjm}, ``mas de quatro e cinco palmos de comprido, e não fazem árvore''.} Servem
estas folhas aos índios para fazerem delas uns vasos, em que metem a farinha, quando vão à
guerra, ou algum outro caminho, onde a farinha vai de feição que ainda que chova muito não
lhe entra água dentro.

Caápara\footnote{ Em Varnhagen (1851 e 1879), ``capara''.} é outra folha, que nasce como a
de cima, mas em cada pé estão pegadas quatro folhas como as atrás, pegadas umas nas
outras; com estas folhas arma o gentio em umas varas uma feição como esteira muito tecida,
e fica cada esteira de trinta palmos de comprimento, e três de largo, e assentam"-nas sobre
o emadeiramento das casas, com o que ficam muito bem cobertas; e dura uma cobertura destas
sete, oito anos e mais.

Tocum\footnote{ Tucum.} é uma erva cujas folhas são como de cana"-do"-reino, mas mais curtas
e brandas; a vara onde se criam é cheia de espinhos pretos, e limpa deles fica como
rota"-da"-índia. Estas folhas quebram os índios a mão, e tiram dela o mais fino linho do
mundo, que parece seda, de que fazem linhas de pescar, torcidas a mão, e são tão rijas que
não quebram com peixe nenhum. Este tocum, ou seda que dele sai, é pontualmente do toque da
erva"-da"-índia, e assim o parece; do qual se farão obras mui delicadas, se quiserem.

E porque se não pode aqui escrever a infinidade das árvores e ervas que há pelos matos e
campos da Bahia, nem as notáveis qualidades e virtudes que têm, achamos que bastava para o
propósito deste compêndio dizer o que se contém em seu título; mas há"-se de notar que aos
arvoredos desta província lhe não cai nunca a folha, e em todo o ano estão verdes e
formosos.

\paragraph{[78] Sumário das aves que se criam na terra da Bahia de Todos os Santos do Estado
do Brasil}\quad
Já que temos satisfação com o que está dito no tocante ao arvoredo que há na Bahia de
Todos os Santos, e com os frutos, grandezas e estranhezas dela, e ainda que o que se disse
é o menos que se pode dizer, por haver muitas mais árvores, convém que se dê conta quais
aves se criam entre estes arvoredos, e se mantêm de seus frutos e frescura deles.

E peguemos logo da águia como da principal ave de todas as criadas. A águia, a que o
gentio chama cabureasu,\footnote{ Em Varnhagen (1851 e 1879), ``cabureaçû''. Caburé"-açu.}
é tamanha como as águias da Espanha, tem o corpo pardaço, e as asas pretas; tem o bico
revolto, as pernas compridas, as unhas grandes e muito voltadas, de que se fazem apitos;
criam em montes altos, onde fazem seus ninhos e põem dois ovos somente; e sustentam os
filhos da caça que tomam, de que se mantêm.

Criam"-se nestes matos emas muito grandes, a que o gentio chama nhundu,\footnote{ Em
Varnhagen (1851 e 1879), ``nhandú''. Nhandu ou nandu.} as quais se criam pela terra
adentro em campinas, e são tamanhas como as da África, e eu vi um quarto de uma depenada
tamanho de um carneiro grande. São estas aves brancas, outras cinzentas, e outras malhadas
de preto, as quais têm as penas muito grandes, mas não tem nelas tanta penagem como as da
Alemanha; os seus ovos não são redondos, nem tamanhos como os da África. Estas aves fazem
os ninhos no chão, onde criam; e mantêm os filhos com cobras e outros bichos que tomam, e
com frutas do campo; as quais não voam levantadas do chão, correm em pulos, com as asas
abertas; tomam"-nas os índios a cosso;\footnote{ Cosso: ato de perseguir.} e tanto as
seguem até que as cansam, e de cansadas as tomam. Têm estas aves as pernas e
pescoço compridos, cuja carne é dura, mas muito gostosa; das penas se aproveita o gentio,
e fazem delas uma roda de penachos, que pelas suas festas trazem nas costas, que têm em
muita estima.

Tabuyayá\footnote{ Em Varnhagen (1851 e 1879), ``tabuiaiá''. Tabujajá ou tubaiaiá.} é uma
ave muito maior que pato; tem as pernas altas, os pés grossos, a cor parda, o bico grosso
e grande; tem sobre o bico, que é branco, uma maneira de crista vermelha, e sobre a cabeça
umas penas levantadas, como poupa. Criam em árvores altas, os ovos são como de patos,
mantêm"-se de frutas do mato; cuja carne é dura, mas boa para comer.

\paragraph{[79] Em que se declara a propriedade do macucagoá, motum e das galinhas"-do"-mato}\quad
Macucagoá\footnote{ Em Varnhagen (1851 e 1879), ``macuagoá''. Macaguã.} é uma ave grande
de cor cinzenta, do tamanho de um grande pato, mas tem no peito mais titelas que dois
galipavos, as quais são tenras como de perdiz, e da mesma cor; a mais carne é sobre dura,
sendo assada, mas cozida é muito boa. Têm estas aves as
pernas compridas, cheias de escamas verdoengas; têm o bico pardaço, da feição da galinha;
voam pouco e ao longo do chão, por onde correm muito, e as tomam com cães a cosso, e às
vezes as matam às flechadas; criam no chão, onde põem muitos ovos, em ninhos como de
galinhas; mas têm a casca verde, de cor muito fina, e mantêm"-se das frutas do mato.

Mutuns\footnote{ Em Varnhagen (1851 e 1879), ``motúm''. Mutum.} são umas aves pretas nas
costas, asas e barriga brancas; são do tamanho dos galipavos, têm as pernas compridas e
pretas, e sobre a cabeça umas penas levantadas como pavão, e voam pouco e baixo, correm
muito pelo chão, onde as matam a flechadas e as tomam a cosso com cães. Criam no chão, os
seus ovos são tamanhos como de pata, muito alvos, e tão crespos da casca como confeitos, e
a clara deles é como manteiga de porco derretida, a qual enfastia muito. Têm estas aves o
bico preto como de corvo e tocados ao redor de vermelho, à maneira de crista; a carne
destas aves é muito boa, pontualmente como a de galipavos, e têm no peito muitas mais
titelas.

Jacus são umas aves a que os portugueses chamam galinhas"-do"-mato, e são do tamanho das
galinhas e pretas; mas têm as pernas mais compridas, a cabeça e pés como galinhas, o bico
preto, cacarejam como perdizes, criam no chão, e têm o vôo muito curto; mantêm"-se de
frutas, matam"-nas os índios às flechadas; cuja carne é muito boa, e têm o peito cheio de
titelas como perdiz da mesma cor, e muito tenras; a mais carne é dura para assada e,
cozida, é muito boa.

Tuyuyú\footnote{ Em Varnhagen (1851 e 1879), ``tuiuiú''. Tuiuiú.} é uma ave grande de
altura de cinco palmos, tem as asas pretas, e papo vermelho, e o mais branco; tem o
pescoço muito grande, e o bico de dois palmos de comprido; fazem os ninhos no chão, em
montes muito altos, onde fazem grande ninho, em que põem dois ovos, cada um como um grande
punho; mantêm os filhos com peixe dos rios o qual comem primeiro, e recozem"-no no papo, e
depois arrevessam"-no\footnote{ Arrevessar: expelir pela boca; vomitar.} e repartem"-no
pelos filhos.

\paragraph{[80] Em que se declara a natureza dos canindés, araras e tucanos}\quad
Canindé é um pássaro tamanho como um grande galo; tem as penas das pernas, barriga e colo
amarelas, de cor muito fina, e as costas acatassoladas de azul e verde, e as das asas e
rabo azuis, o qual tem muito comprido, e a cabeça por cima azul, e ao redor do bico,
amarelo; tem o bico preto, grande e grosso; e as penas do rabo e as das asas são
vermelhas, pela banda de baixo. Criam em árvores altas, onde tomam os índios novos nos
ninhos, para se criarem nas casas, porque falam e gritam muito, com voz alta e grossa; os
quais mordem mui valentemente, e comem frutas das árvores, e em casa tudo quanto lhes dão;
cuja carne é dura, mas aproveitam"-se dela os que andam pelo mato. Os índios se aproveitam
das suas penas amarelas para as suas carapuças, e as do rabo, que são de três e quatro
palmos, para as embagaduras das suas espadas.

Arara é outro pássaro do mesmo tamanho, e feição do canindé, mas tem as penas do colo,
pernas e barriga vermelhas, e as das costas, das asas e do rabo azuis, e algumas verdes, e
a cabeça e pescoço vermelhos, e o bico branco e muito grande, e tão duro que quebram com
ele uma cadeia de ferro, os quais mordem muito e gritam mais. Criam estas aves em árvores
altas, comem frutas do mato e milho pelas roças, e a mandioca quando está a curtir. Os
índios tomam estes pássaros quando são novos nos ninhos, para os criarem; os quais, depois
de grandes, cortam com o bico por qualquer pau, como se fosse uma enxó. A sua carne é como
a dos canindés, de cujas penas se aproveitam os índios.

Tucanos são outras aves do tamanho de um corvo; têm as pernas curtas e pretas, as penas
das costas azuladas, a das asas e do rabo aniladas, o peito cheio de frouxel muito miúdo
de finíssimo amarelo, o qual os índios esfolam para forro de carapuças. Têm a cabeça
pequena, o bico branco e amarelo, muito grosso, e alguns são tão compridos como um palmo,
e tão pesados que não podem com ele quando comem, porque tomam grande bocado, com o que
viram o bico para cima, porque não pode o pescoço com tamanho peso, como têm. Criam estes
pássaros em árvores altas, e tomam"-nos novos para se criarem em casa; os bravos matam os
índios a flecha, para lhes esfolarem o peito, cuja carne é muito dura e magra.

\paragraph{[81] Em que se diz das aves que se criam nos rios e lagoas da água doce}\quad
Ao longo dos rios da água doce se criam mui formosas garças, a que o gentio chama
uratinga, as quais são brancas e tamanhas como as da Espanha. Têm as pernas longas,
pescoço e bico mui compridos, pernas e pés amarelos, e têm entre os encontros um molho de
plumas, que lhes chegam à ponta do rabo, que são mui alvas e formosas, e para estimar; e
são estas garças muito magras e criam no chão, junto da água; mantêm"-se do peixe, que
tomam nos rios, e esperam mal que lhe atirem.

Criam"-se mais, ao longo destes rios e nas lagoas, muitos adens, a que o gentio chama
upeca, que são da feição dos da Espanha, mas muito maiores, os quais dormem em árvores
altas, e criam no chão, perto da água. Comem peixe, e da mandioca que está a curtir nas
ribeiras; tomam os índios estes adens, quando são novos, e criam"-nos em casa, onde se
fazem muito domésticos.

Agoapeasoca\footnote{ Em Varnhagen (1851 e 1879), ``aguapeaçoca''. Aguapeaçoca ou jaçanã.}
é uma ave do tamanho de um frangão; tem as pernas muito compridas, e o pescoço e o vestido
de pena aleonada, e derredor do bico uma rosa muito amarela; e tem nos encontros das asas
dois esporões de osso amarelo, e nas pontas delas outros dois, com que ofendem aos
pássaros com que pelejam. Andam estas aves nas lagoas, e criam nas junqueiras junto delas,
onde põem três ovos, não mais, e mantêm"-se de caracóis que buscam.

Jabacatim é um pássaro tamanho como um pintão, tem o bico comprido, o peito vermelho, a
barriga branca, as costas azuis; criam em buracos que fazem nas barreiras sobre os rios,
ao longo dos quais andam sempre com os pés pela água a tomar peixinhos, de que se mantêm.

E há outros, mais pequenos, da mesma feição e costumes, a que o gentio chama
guoarirama.\footnote{ Em Varnhagen (1851 e 1879), ``garirama''.}

Jacoaçu\footnote{ Em Varnhagen (1851 e 1879), ``jacuaçu''.} são outras aves da feição das
garças grandes, e do seu tamanho; são pardas e pintadas de branco, andam nos rios e
lagoas, criam ao longo delas e dos rios, no chão; mantêm"-se do peixe que tomam.

\paragraph{[82] Das aves que se parecem com perdizes, rolas e pombas}\quad
Picasu\footnote{ Em Varnhagen (1851 e 1879), ``picaçu''. Picaçu.} é como pomba brava, mais
pequena alguma coisa, tem a cor cinzenta, os pés vermelhos; cria no chão, onde põe dois
ovos; tem o peito e carne mui saborosa.

Payrari\footnote{ Em Varnhagen (1851 e 1879), ``payrary''. Pairari, bairari ou pomba
campestre.} é uma ave do tamanho, cor e feição das rolas, as quais criam no chão, em
ninhos, em que põem dois ovos, e tomam"-nas em redes, e amansam"-nas em casa de maneira que
criam como pombas, as quais têm o peito muito cheio, e boa carne.

Juritis\footnote{ Juriti ou Jeruti.} é outra casta de rolas do mesmo tamanho, mas são
aleonadas, e têm o bico pardo; também criam no chão onde põem dois ovos, e tomam"-nas em
redes, cuja carne é muito tenra e boa.

Nambu\footnote{ Nambu ou inhambu.} é uma ave da cor e tamanho da perdiz, tem os pés e bico
vermelho, voam ao longo do chão, por onde correm muito, e criam em ninhos que fazem no
chão, onde põem muitos ovos. Estas aves têm grande peito, cheio de titelas muito tenras e
saborosas.

Há outras aves, a que os índios chamam pequipebas,\footnote{ Em Varnhagen (1851 e 1879),
``piquepebas''. Picuipeba ou pomba"-de"-espelho.} que são da feição das rolas, e da mesma
cor, mas são mais pequenas, e têm as penas vermelhas e o bico preto; estas andam sempre
pelo chão, onde criam, e põem dois ovos; as quais, o mais do tempo, andam esgaravatando a
terra com o bico, buscando umas pedrinhas brancas de que se mantêm.

\paragraph{[83] Em que se relata a diversidade que há de papagaios}\quad
Ageruasu\footnote{ Em Varnhagen (1851 e 1879), ``ageruaçu''. Ajuru"-açu ou
papagaio"-moleiro.} são uns papagaios grandes todos verdes, que têm tamanho corpo como um
adem, os quais se fazem mui domésticos em casa, onde falam muito bem; estes, no mato,
criam em ninhos, em árvores altas; são muito gordos e de boa carne, e muito saborosos; mas
hão de ser cozidos.

Agerueté\footnote{ Ajuruetê ou papagaio"-verdadeiro.} são uns papagaios verdadeiros, que
falam e se levam à Espanha,\footnote{ Em Varnhagen (1851 e 1879), ``verdadeiros, que se
levam''.} os quais são verdes, e têm os encontros das asas vermelhos, e o toucado da
cabeça amarelo; criam nas árvores, em ninhos, e comem as frutas delas, de que se mantêm;
cuja carne se come; e para se amansarem tomam"-nos novos.

Há outros papagaios, a que chamam curicas,\footnote{ Em Varnhagen (1851 e 1879),
``corica''. Corica ou curica.} que são todos verdes, e não têm mais que o só queixo
amarelo, e algumas penas nas asas encarnadas; os quais criam em ninhos nas árvores, de
onde fazem grande dano nas searas de milho; tomam"-nos novos para se amansarem em casa,
onde falam muito bem; cuja carne comem os que andam pelo mato, mas é dura.

Marcao\footnote{ Em Varnhagen (1851 e 1879), ``marcaná''. Maracanã.} é um pássaro verde
todo, como papagaio, tem a cabeça toucada de amarelo, o bico grosso e sobre o grande, e
voltado para baixo, o rabo comprido e vermelho; criam"-se em árvores altas, em ninhos; e
amansam"-se alguns, porque falam, cuja carne é dura, mas come"-a quem não tem outra melhor.

Há uns passarinhos todos verdes, que têm os pés e bico branco, a que os índios chamam
tuim; têm o bico revolto para baixo, e criam em árvores, em ninhos de palha, perto do mar,
e não os há pelo sertão; os quais andam em bandos; tomam"-nos em novos para se criarem em
casa, onde falam muito claro e bem, e têm graça no que dizem.

Há outros pássaros todos verdes, maiores que os tuins, que têm o bico branco voltado,
toucado de amarelo e azul, que criam em árvores, em ninhos, de onde se tomam em novos,
para se criarem em casa, onde falam também; estes andam em bandos, destruindo as
milharadas.

\paragraph{[84] Em que se conta a natureza de algumas aves de água salgada}\quad
Na Bahia, ao longo da água salgada, nas ilhas que ela tem, se criam garcetas pequenas, a
que os índios chamam nacabaçu;\footnote{ Em Varnhagen (1851 e 1879), ``carabuçu''.}
algumas são brancas e outras pardas, as quais são umas plumas cinzentas pequenas, muito
fidalgas para gorro; todas criam ao longo do mar, onde tomam peixe, de que se mantêm, e
caranguejos novos; e esperam bem a espingarda.

Há outros pássaros, a que os índios chamam ubarateon,\footnote{ Em Varnhagen (1851 e 1879),
``uirateonteon''.} que se criam perto do salgado, que são pardos e têm o pescoço branco,
o bico verde, e são tamanhos como adens, e têm os pés da sua feição. Esses pássaros andam
no mar, perto da terra, e voam ao longo da água tanto, sem descansar, até que caem como
mortos; e assim descansam até que se tornam a levantar, e voam.

Carapira\footnote{ Em Varnhagen (1851 e 1879), ``carapirá''. Carapirá.} é uma ave, a que
os mareantes chamam rabifurcado, os quais se vão cinquenta e sessenta léguas ao mar, de
onde se recolhem para a Bahia, diante de algum navio do reino, ou do vento sul, que lhes
vem nas costas ventando, de onde tornam logo a fazer volta ao mar; mas criam em terra, ao
longo dele.

Jaburu\footnote{ Em Varnhagen (1851 e 1879), ``jaború''. Jabiru ou jaburu.} é outra ave,
tamanha como um grou; tem a cor cinzenta, as pernas compridas, o bico delgado e mais que
de palmo de comprido; estas aves criam em terra ao longo do salgado e comem o peixe que
tomam no mar, perto da terra, por onde andam.

Ao longo do salgado se criam uns pássaros, a que os índios chamam urateon; são pardos,
tamanhos como frangões, têm as pernas vermelhas, o bico preto e comprido; são mui ligeiros
e andam sempre sobre a água salgada, saltando em pulos, espreitando os peixinhos de que se
mantêm.

Ao longo do mar se criam outros pássaros, a que os índios chamam ati;\footnote{ Em
Varnhagen (1851 e 1879), ``aty''.} têm o corpo branco, as asas pretas, o bico de
peralto, com que cortam o peixe como com tesoura; têm as pernas curtas e brancas; andam
sempre nas barras do rio buscando peixe, do que comem.

Matuimasu\footnote{ Em Varnhagen (1851 e 1879), ``matuim"-açú''.} são uns pássaros que
andam sempre sobre os mangues, tamanhos como franganitos, de cor pardaça; têm as penas e
bico preto, e mantêm"-se de peixe que tomam.

Matuim"-mirim são outros pássaros de feição dos de cima, mas mais pequenos e brancacentos;
mantêm"-se do peixe que tomam; e uns e outros criam no chão, ao longo do salgado.

Pitaoão\footnote{ Pitanguá, pitauá ou bem"-ti"-vi.} são passarinhos do tamanho e cor dos
canários, e têm uma coroa branca na cabeça; fazem grandes ninhos nos mangues, ao longo dos
rios salgados, onde põem dois ovos; e mantém"-se dos peixinhos que alcançam por sua lança.

Há umas aves como garcetas, a que os índios chamam socorí,\footnote{ Em Varnhagen (1851 e
1879), ``socóry''. Socori ou socoí.} que têm as pernas compridas e amarelas, o pescoço
longo, o peito pintado de branco e pardo, e todo o mais pardo; criam em terra no chão,
perto da água salgada, onde se mantêm do peixe que nela tomam, e de caranguejos dos
mangues.

Maygue\footnote{ Em Varnhagen (1851 e 1879), ``margui''.} é um pássaro pequeno e pardo,
tem as pernas mui compridas, o bico e pescoço longo; e está sempre olhando para o chão, e
como vê gente foge, dando um grande grito. Estas aves se criam ao longo do salgado, e
mantêm"-se do peixe que tomam no mar.

\paragraph{[85] Em que se trata de algumas aves de rapina que se criam na Bahia}\quad
Urubus são uns pássaros pretos, tamanhos como corvos, mas têm o bico mais grosso, e a
cabeça como galinha cocurutada, e as pernas pretas, mas tão sujos que fazem seu feitio
pelas pernas abaixo, e tornam"-no logo a comer. Estas aves têm grande faro de coisas
mortas, que é o que andam sempre buscando para sua mantença, as quais criam em árvores
altas; algumas há, mansas, em poder dos índios, que as tomaram nos ninhos.

Toatuo é um pássaro que é, na feição, na cor e no tamanho um gavião, e vive de rapina no
mato; e em povoado não lhe escapa pintão que não tome, e criam em árvores altas.

Urao"-asu\footnote{ Em Varnhagen (1851 e 1879), ``uraoaçu''. Uiraçu.} são como os minhotos
de Portugal, sem terem nenhuma diferença; são pretos e têm grandes asas, cujas penas os
índios aproveitam para empenarem as flechas, os quais vivem de rapina no mato, e em
povoado destroem uma fazenda de galinhas e pintões.

Sabiá"-pitanga são uns pássaros pardos como pardais, que andam pelos monturos e correm pelo
chão com muita ligeireza; e mantêm"-se da mandioca que furtam dos índios quando está a
curtir; os quais criam em ninhos em árvores.

Caracara\footnote{ Em Varnhagen (1851 e 1879), ``carácará''.} são uns pássaros tamanhos
como gaviões, têm as costas pretas, as asas pintadas de branco e o rabo, o bico revolto
para baixo, os quais se mantêm de carrapatos que trazem as alimárias, e de lagartixas que
tomam; e quando as levam no bico vão após eles uns passarinhos, que chamam suiriri, para
que as larguem; e vão"-nos picando até que, de perseguidos, se põem no chão, com a
lagartixa debaixo dos pés, para a defender.

Macaoam\footnote{ Em Varnhagen (1851 e 1879), ``oacaoam''. Macauã ou acauã.} são pássaros
tamanhos como galinhas, têm a cabeça grande, o bico preto voltado para baixo, a barriga
branca, o peito vermelho, o pescoço branco, as costas pardas, o rabo e asas pretas e
brancas. Estes pássaros comem cobras que tomam, e quando falam se nomeiam pelo seu nome;
em os ouvindo, as cobras lhes fogem, porque lhes não escapam; com as quais mantêm os
filhos. E quando o gentio vai de noite pelo mato que se teme das cobras, vai arremedando
estes pássaros para as cobras fugirem.

Pela terra adentro se criam umas aves, a que os índios chamam urubutinga, que são do
tamanho dos galipavos; e são todos brancos, e têm crista como os galipavos. Estas aves
comem carne que acham pelo campo morta, e ratos que tomam; as quais põem um só ovo, que
metem em um buraco, onde o tiram; e mantêm nele o filho com ratos que lhe trazem para
comer.

\paragraph{[86] Em que se contém a natureza de algumas aves noturnas}\quad
Urucuream é uma ave, pontualmente como as corujas da Espanha; umas são cinzentas e outras
brancas; gritam como corujas; as quais criam no mato em tronco de árvores grossas, e em
povoado nas igrejas, de cujas alâmpadas comem o azeite.

Jucurutu\footnote{ Jacurutu, jucurutu ou corujão"-orelhudo.} é uma ave tamanha como um
frango, que em povoado anda de noite pelos telhados; e no mato cria em tocas de árvores
grandes, e anda ao longo dos caminhos; e onde quer que está, toda a noite está gritando
pelo seu nome. Esta ave é de cor brancacenta, tem as pernas curtas, a cabeça grande com
três listas pardas por ela que parecem cutiladas, e duas penas nela de feição de orelhas.

Há outros pássaros, a que os índios chamam ubuyaus,\footnote{ Em Varnhagen (1851 e 1879),
``ubujaús''. Ubujaú ou ibijaú.} que são tamanhos como pintões, têm a cabeça grande, o
rabo comprido; e são todos pardos e muito cheios de penugem, os quais andam de noite
gritando cuxaiguigui

Há outros pássaros do mesmo nome, mais pequenos, que são pintados, os quais andam de
madrugada dando os mesmos gritos, e uns e outros criam no chão, onde põem dois ovos
somente; e mantêm"-se das frutas do mato.

Há outros pássaros pardos, a que os índios chamam oitibó,\footnote{ Oitibó ou noitibó.}
com que têm grande agouro; os quais andam ordinariamente gritando noitibó e de dia não os
vê ninguém; e mantêm"-se das frutas e folhas de árvore, onde lhes amanhece.

Aos morcegos chamam os índios andura;\footnote{ Andirá.} e há alguns muito grandes, que
têm tamanhos dentes como gatos, com que mordem; criam nos côncavos das árvores, e nas
casas e lugares escuros; as fêmeas parem quatro filhos e trazem"-nos pendurados ao pescoço
com as cabeças para baixo, e pegados com as unhas ao pescoço da mãe; quando estes morcegos
mordem alguém que está dormindo de noite, fazem"-no tão sutilmente que se não sente; mas a
sua mordedura é mui peçonhenta. Nas casas de purgar açúcar se criam infinidade deles, onde
fazem muito dano, sujando o açúcar com o seu feitio, que é como de ratos; e comem muito
dele.

\paragraph{[87] Em que se declara de alguns pássaros de diversas cores e costumes}\quad
Uranhengata\footnote{ Em Varnhagen (1851 e 1879), ``uranhengatá''.} é uma ave do tamanho
de um estorninho, que tem o peito, pescoço, barriga e coxas de fino amarelo, e as costas,
asas e rabo de cor preta mui fina, e a cabeça e derredor do bico um só queixo amarelo, e
as pernas e pés como frouva;\footnote{ No original, ``flouba''. Frouva: ave com as costas
pretas e a barriga branca, que era bastante conhecida em Portugal.} os quais criam em
ninhos, em árvores altas, onde os tomam em novos e os criam em casa, onde se fazem tão
domésticos, que vão comer ao mato e tornam para casa.

Sabiá"-tinga\footnote{ Sabiatinga.} são uns passarinhos brancos que têm as pontas das asas
pretas, e as do rabo que têm compridas, os quais criam em ninhos que fazem nas árvores,
mantêm"-se das pimentas que buscam; de cujo feitio se criam pelo campo muitas pimenteiras.

Tigepiranga\footnote{ Em Varnhagen (1851 e 1879), ``tiépiranga''. Tapiranga ou tepiranga.}
são pássaros vermelhos do corpo, que têm as asas pretas, e são tamanhos como pintarroxos;
criam em árvores, onde fazem seus ninhos; aos quais os índios esfolam os peitos para
forrarem as carapuças por serem muito formosos.

Gainambu\footnote{ Em Varnhagen (1851 e 1879), ``gainambi''. Guanambi ou beija"-flor.} são
uns passarinhos muito pequenos, de cor apavonada, que têm os bicos maiores que o corpo, e
tão delgados como alfinetes; comem aranhas pequenas e fazem os seus ninhos das suas teias;
têm as asas pequenas e andam sempre bailando no ar, espreitando as aranhas; criam em tocas
de árvores.

Há outra ave, a que os índios chamam aiaiá,\footnote{ Em Varnhagen (1851 e 1879),
``ayayá''. Aiaiá.} que é do tamanho de uma franga toda vermelha, tem o bico verde, os pés
pretos e o cabo do bico amassado como pata; fazem seus ninhos em árvores altas, e 
mantêm"-se da fruta delas.

Jasana\footnote{ Em Varnhagen (1851 e 1879), ``jaçanã''. Jaçanã.} são uns pássaros
pequenos, todos encarnados e os pés vermelhos; criam"-se em árvores altas, onde fazem os
ninhos, e mantêm"-se das frutas do mato.

Há outros passarinhos pequenos todos vestidos de azul, cor muito subida, aos quais os
índios chamam sayubu,\footnote{ Em Varnhagen (1851 e 1879), ``sayubui''.} que têm o bico
preto e criam em árvores, e mantêm"-se dos bichinhos da terra.

Tupiana são uns passarinhos que têm o peito vermelho, a barriga branca e o mais azul; e
têm os bicos compridos, muito delgados; e criam nas árvores, em ninhos, e mantêm"-se de
bichinhos.

Tiejuba\footnote{ Em Varnhagen (1851 e 1879), ``tiéjuba''.} são passarinhos pequenos que
têm o corpo amarelo, as asas verdes, o bico preto; criam em tocas de árvores, e mantêm"-se
de pedrinhas que apanham pelo chão.

Macasiqua\footnote{ Em Varnhagen (1851 e 1879), ``macaçica''.} é um pássaro pequeno que
tem as asas verdes, a barriga amarela, as costas e o rabo pardo, e o bico preto; fazem
estes pássaros os ninhos nas pontas das árvores, dependurados por um fio da mesma árvore;
e os ninhos são de barro e palha, com coruchéus por cima, muito agudos, e servem"-se por
uma portinha, onde põem dois ovos; e fazem os ninhos desta feição por fugirem às cobras,
que lhes comem os ovos, se os acham em outra parte.

Há outros pássaros, que os índios chamam sija,\footnote{ Em Varnhagen (1851 e 1879),
``sijá''.} que são tamanhos como papagaios todos verdes, e o bico revolto para baixo, os
quais criam em tocas de árvores, de cuja fruta se mantêm.

\paragraph{[88] Em que se trata de alguns passarinhos que cantam}\quad
Suyriri\footnote{ Em Varnhagen (1851 e 1879), ``suiriri''. Suiriri ou siriri.} são uns
passarinhos como chamarizes, que criam em ninhos nas árvores, os quais se mantêm com
bichinhos e formigas das que têm asas, a que em Portugal chamam agudes; estes se criam em
gaiolas, onde cantam muito bem mas não dobram muito quando cantam.

Há outros pássaros pretos, com os encontros amarelos, a que os índios chamam urandi, que
criam em ninhos de palha, onde põem dois ovos, os quais cantam muito bem.

Há outros passarinhos, a que os índios chamam urainhengata,\footnote{ Em Varnhagen (1851 e
1879), ``uraenhangatá''.} que são quase todos amarelos, que criam em ninhos de palha que
fazem nas árvores, os quais cantam nas gaiolas muito bem.

Criam"-se em árvores baixas em ninhos outros pássaros, a que o gentio chama
sicupoecay,\footnote{ Em Varnhagen (1851 e 1879), ``sabiá"-coca''. Sabiapoca.} que são
todos aleonados muito formosos, os quais cantam muito bem.


Pexarorem são uns passarinhos todo pretos, tamanhos como calhandras, que andam sempre por
cima das árvores, mas comem no chão bichinhos e cantam muito bem.

Querejua\footnote{ Em Varnhagen (1851 e 1879), ``querejuá''. Quiruá, crejuá ou
anambé"-azul.} são uns passarinhos todos azuis de cor finíssima, que andam sempre por cima
das árvores, onde criam e se mantêm com o fruto delas, e cantam muito bem.

Muipereru são uns passarinhos pardos tamanhos como carriças; criam nos buracos das árvores
e das pedras, põem muitos ovos, comem aranhas e minhocas, cantam como rouxinóis, mas não
dobram tanto como eles.

\paragraph{[89] Que trata de outros pássaros diversos}\quad
Nhuapupe\footnote{ Em Varnhagen (1851 e 1879), ``nhapupé''. Enapupê.} é uma ave do tamanho
de uma franga, de cor aleonada, tem os pés como galinha, a qual anda sempre pelo chão,
onde cria e põe muitos ovos de fina cor aleonada, cuja carne é dura, e come"-se cozida.

Saracura é uma ave tamanha como galinha, de cor aleonada, que tem as pernas muito
compridas e o pescoço e bico comprido; cria no chão, onde chega a maré de águas vivas, que
se mistura com água doce; as quais não andam pelo salgado, nem pelo mato grande, mas ao
longo deles; de noite, cacareja como perdiz; e tem o peito cheio de titelas tenras, e a
mais carne é boa também.

Orus\footnote{ Uru.} são umas aves tamanhas como papagaios, de cor preta e o bico revolto;
criam em árvores altas, e quando têm filhos nos ninhos, remetem\footnote{ Remeter:
atacar.} aos índios que lhos querem tomar. Estas aves têm grande peito cheio de
titelas, as quais e a mais carne são muito tenras e saborosas como galinhas.

Anu\footnote{ Anu ou anum.} é outra ave preta, do tamanho e feição de gralha; e andam
sempre em bandos, voando de árvore em árvore ao longo do chão; criam em árvores baixas em
ninhos, e mantêm"-se de uma baga preta como murtinhos, e de outras frutinhas que buscam.

Magoari\footnote{ Maguari.} é outra ave de cor branca, que faz tamanho vulto como uma
garça, e tem as pernas e pés mais compridos que as garças, e o pescoço tão longo que
quando voa o faz em voltas; e tem o bico curto e o peito muito agudo, e nenhuma carne,
porque tudo é pena; e voa muito ao longe, e corre pelo chão por entre o mato, que faz
espanto.

Aracoa\footnote{ Em Varnhagen (1851 e 1879), ``aracoã''. Aracuã ou araquã.} é outro
pássaro tamanho como um frangão, de cor parda; tem as pernas como de frangões, mas os
dedos muito compridos e o rabo longo; e tem duas goelas, ambas por uma banda, que leva ao
longo do peito até abaixo, onde se juntam; criam"-se estas aves em árvores, e comem fruta
delas.

Sabiá"-una são uns passarinhos pretos que andam sempre entre arvoredos; comem frutas e
bichinhos, criam nas árvores em ninhos de palha.

Atiuasu\footnote{ Em Varnhagen (1851 e 1879), ``atiaçú''. Atiuaçú ou atingaçú.} é um
pássaro tamanho como um estorninho, tem as costas pardas, o peito e a barriga branca, o
rabo comprido, as pernas verdoengas, os olhos vermelhos; criam em árvores, comem o fruto
delas, e cantam em assobios.

Há uns passarinhos pequenos, todos pretos, a que os índios chamam timoyna,\footnote{ Em
Varnhagen (1851 e 1879), ``timuna''.} que criam em ninhos de palha; mantêm"-se de frutas e
de minhocas.

Uanandi é um pássaro pequeno, pardo, pintado de preto pelas costas e branco na barriga; e
tem o bico curto, e cria em ninhos de palha que faz nas árvores.

Há outros pássaros, a que o gentio chama uapicu, tamanhos como um tordo, têm o corpo preto
e as asas pintadas de branco, e o bico comprido, tão duro e agudo que fura com ele as
árvores que têm abelheiras, até que chega ao mel, de que se mantém; e quando dão as
picadas no pau, soa a pancada a oitenta passos e mais; os quais pássaros têm na cabeça um
cocuruto vermelho e alevantado, e criam nas tocas das árvores.

\paragraph{[90] Que trata de alguns bichos menores que têm asas e têm alguma semelhança de
aves}\quad
Como foi forçado dizer"-se de todas as aves como fica dito, convém que junto delas se diga
de outros bichos que têm asas e mais aparência de aves que de alimárias, ainda que sejam
imundícies, e pouco proveitosas ao serviço dos homens.

Comecemos logo dos gafanhotos, a que o gentio chama tacura,\footnote{ Tucura.} os quais se
criam na Bahia muito grandes, e andam muitas vezes em bandos, os quais são da cor dos que
há na Espanha, e há outros pintados, outros verdes e de diferentes cores, e têm maiores
asas que os da Espanha, e quando voam abrem"-nas como pássaros e não são muito daninhos.

Há outros bichos, a que os índios chamam tacurianda,,\footnote{ Em Varnhagen (1851 e 1879)
``tucuranda''.} e em Portugal saudes, os quais são muito formosos, pintados e grandes,
mas não fazem mal a nada.

Nas tocas das árvores se criam uns bichinhos como formigas, com asas brancas, que não saem
do ninho senão depois que chove muito, e o primeiro dia de sol, a que os índios chamam
araraha;\footnote{ Em Varnhagen (1851 e 1879), ``arará''. Arará ou aleluia.} e quando saem
fora é voando; e sai tanta multidão que cobre o ar, e não torna ao lugar donde saiu, e
perde"-se com o vento.

As borboletas a que chamam mariposa, chamam os índios sarará; as quais andam de noite
derredor das candeias, maiormente em casas palhoças do mato, e em noites de escuro, e são
tão perluxas às vezes que não há quem se valha com elas, porque se vêm ao rosto e dão
enfadamento às ceias, porque se põem no comer, e não deixam as candeias dar seu lume, o
que acontece em povoado.

Há outra casta de borboletas grandes, umas brancas e outras amarelas, e outras pintadas,
muito formosas à vista, a que os índios chamam panamá,\footnote{ Panamá ou panapaná.} as
quais vêm às vezes de passagem no verão em tanta multidão, que cobrem o ar, e põem logo
todo um dia em passar por cima da cidade do Salvador à outra banda da Bahia, que são nove
ou dez léguas de passagem. Estas borboletas fazem muito dano nos algodões quando estão em
frol.

\paragraph{[91] Em que se conta a propriedade das abelhas da Bahia}\quad
Na Bahia há muitas castas de abelhas. Primeiramente, há umas a que o gentio chama
ueru,\footnote{ Em Varnhagen (1851 e 1879), ``heru''.} que são grandes e pardas; estas
fazem o ninho no ar, por amor das cobras, como os pássaros de que dissemos atrás; onde
fazem seu favo e criam mel muito bom e alvo, que lhes os índios tiram com fogo, do que
elas fogem muito; as quais mordem valentemente.

Há outra casta de abelhas, a que os índios chamam tapiuqua,\footnote{ Em Varnhagen (1851 e
1879), ``tapiuja''. Tapiú.} que também são grandes, e criam em ninhos que fazem nas
pontas dos ramos das árvores com barro, cuja abóbada é tão sutil que não é mais grossa que
papel. Estas abelheiras crestam também com fogo, a quem os índios comem as crianças, e
elas mordem muito.

Há outra casta de abelhas, maiores que as da Espanha, a que os índios chamam taturama;
estas criam nas árvores altas, fazendo seu ninho de barro ao longo do tronco delas, e
dentro criam seu mel em favos, o qual é baço, e elas são pretas e mui cruéis.

Há outra casta de abelhas, a que o gentio chama acabesé,\footnote{ Em Varnhagen (1851 e
1879), ``cabecé''.} que mordem muito, que também fazem o ninho em árvores, onde criam mel
muito alvo e bom; as quais são louras e mordem muito.

Há outra casta de abelhas, a que os índios chamam caapoam, que são pequenas, e mordem
muito a quem lhes vai bulir no seu ninho, que fazem no chão, de barro sobre um torrão; o
qual é redondo do tamanho de uma panela, e tem serventia ao longo do chão, onde criam seu
mel, que não é bom.

Cabatan\footnote{ Cabatã ou vespa"-cabocla.} são outras abelhas que não são grandes, que
fazem seu ninho no ar, dependurado por um fio, que desce da ponta de um raminho; e são tão
bravas que, em sentindo gente, remetem logo aos beiços, olhos e orelhas, onde mordem
cruelmente; e nestes ninhos armam seus favos, onde criam mel branco e bom.

Saracoma são outras abelhas pequenas que fazem seu gasalhado entre folhas das árvores,
onde não criam mais que sete ou oito juntas; e fazem ali seu favo, em que criam mel muito
bom e alvo; estas mordem rijamente, e dobram umas folhas sobre outras, que tecem com uns
fios como aranhas, onde têm os favos.

Há outra casta de abelhas, a que o gentio chama cabaobaju,\footnote{ Em Varnhagen (1851 e
1879), ``cabaojuba''.} que são amarelas, e criam nas tocas das árvores, e são mais cruéis
que todas; e em sentindo gente remetem logo a ela; e convém levar aparelho de fogo
prestes, com o qual lhes tiram os favos cheios de mel muito bom.

Copuerusu\footnote{ Em Varnhagen (1851 e 1879), ``capueruçu''.} é outra casta de abelhas
grandes; criam seus favos em ninhos, que fazem no mais alto das árvores, do tamanho de uma
panela, os quais são de barro; os índios os crestam com fogo, e lhes comem os filhos, que
lhes acham; as quais também mordem onde chegam e quem lhes vai bulir.\footnote{ No
manuscrito da \textsc{bgjm}, ``mordem onde chegam''.}

\paragraph{[92] Que trata das vespas e moscas}\quad
Criam"-se na Bahia muitas vespas, que mordem muito; em especial umas, a que chamam os
índios teriguoa,\footnote{ Em Varnhagen (1851 e 1879), ``terigoá''.} que se criam em ramos
de árvores poucas juntas, e cobrem"-se com uma capa que parece teia de aranha, de onde
fazem seu ofício em sentindo gente.

Amisaguoa\footnote{ Em Varnhagen (1851 e 1879), ``amisagoa''.} é outra casta de vespas,
que são à maneira de moscas, que se criam em um ninho, que fazem nas paredes, e nas
barreiras da terra, tamanhos como uma castanha, com um olho no meio, por onde entram, o
qual ninho é de barro, e elas mordem a quem lhes vai bulir nele.

E porque as moscas se não queixem, convém que digamos de sua pouca virtude; e comecemos
nas que se chamam mutuca, que são as moscas gerais e enfadonhas que há na Espanha; as
quais adivinham a chuva, começando a morder onde chegam, de maneira que, se se sente sua
picada, é que há boa novidade.

Há outra casta de moscas, a que os índios chamam muruanja\footnote{ Muruanha.} que são
mais miúdas que as de cima e azuladas; estas seguem sempre os cães e comem"-lhes as
orelhas; e se tocam em chaga ou sangue, logo lançam varejas.

Merus são outras moscas grandes e azuladas que mordem muito onde chegam, tanto que por
cima de rede passam o gibão a quem está lançado nela, e logo fazem arrebentar o sangue
pela mordedura; aconteceu muitas vezes porem estas varejas a homens que estavam dormindo,
nas orelhas, nas ventas e no céu da boca, e lavrarem de feição por dentro as varejas, sem
se saber o que eram, que morreram alguns disso.

Também há outras como as de cavalo, mas mais pequenas e muito negras, que também mordem
onde chegam.

\paragraph{[93] Que trata dos mosquitos, grilos, besouros e brocas que há na Bahia}\quad
Digamos logo dos mosquitos, a que chamam nhitingua;\footnote{ Em Varnhagen (1851 e 1879),
``nhitinga''.} e são muito pequenos e da feição das moscas; os quais não mordem, mas são
muito enfadonhos, porque se põem nos olhos, nos narizes; e não deixam dormir de dia no
campo, se não faz vento. Estes são amigos de chagas e chupam"-lhe a peçonha que têm; e se
se vão pôr em qualquer coçadura de pessoa sã, deixam"-lhe a peçonha nela, do que se vêm
muitas pessoas a encher de boubas. Estes mosquitos seguem sempre em bandos as índias, que
andam nuas, mormente quando andam sujas do seu costume.

Marguins\footnote{ Em Varnhagen (1851 e 1879), ``marguis''. Mariguí, maruí ou maruim.} são
uns mosquitos que se criam ao longo do salgado, e outros na terra perto da água, e
aparecem quando não há vento; e são tamanhos como um pontinho de pena, os quais onde
chegam são fogo de tamanha comichão e ardor que fazem perder a paciência, mormente quando
as águas são vivas; e crescem em partes despovoadas; e se lhes põem a mão, desfazem"-se
logo em pó.

Há outra casta, que se cria entre os mangues, a que os índios chamam inhatium, que tem as
pernas compridas e zunem de noite, e mordem a quem anda onde os há, que é ao longo do mar;
mas se faz vento não aparece nenhum.

Pium\footnote{ Pium ou borrachudo.} é outra casta de mosquitos tamanhos como pulgas
grandes com asas; e em chegando estes à carne, logo sangram sem se sentir, e em lhes
tocando com a mão, se esborracham; os quais estão cheios de sangue; cuja mordedura causa
muita comichão depois, e quer"-se espremida do sangue por não fazer guadelhão na carne.


Há outra casta de mosquitos, a que os índios chamam nhatimuasu;\footnote{ Em Varnhagen
(1851 e 1879), ``inhatiúm"-açú''.} estes são de pernas compridas e mordem e zunem
pontualmente como os que há na Espanha, que entram nas casas onde há fogo; e de que todos
são inimigos.

Também se cria na Bahia outra imundície, a que chamamos brocas, que são como pulgas, e
voam sem lhe enxergarem as asas; as quais furam as pipas do vinho e do vinagre, de maneira
que fazem muita perda, se as não vigiam; e furam todas as pipas e barris vazios, salvo se
tiveram azeite; e nas terras povoadas de pouco fazem mais dano.

Há também grande cópia\footnote{ Cópia: quantidade.} de grilos na Bahia, que se criam pelo
mato e campos; que andam em bandos, como gafanhotos; e se criam também nas
casas de palha, enquanto são novas, nas quais se recolhem muitos entre a palma que vem do
mato; os quais são muito daninhos, porque roem muito os vestidos a que podem chegar; e
metem"-se muitas vezes nas caixas, onde fazem destruição no fato que acham no chão, o qual
cortam de maneira que parece cortado a tesoura; mas como as casas são defumadas
recolhem"-se todos para o mato; estes são grandes e pequenos e têm asinhas; e saltam como
gafanhotos.

Também se criam nestas partes muitos besouros, a que os índios chamam una"-una; mas não
fazem tão ruim feitio com as maçãs que fazem os da Espanha; andam por lugares sujos, têm
asas, e são negros; com a cabeça, pescoço e pernas muito resplandecentes, e tudo muito
duro, mas são muito maiores que os da Espanha; e têm dois cornos virados com as pontas uns
para os outros; e parecem de azeviche.

\subsection{Apontamentos das alimárias que se criam\break na Bahia e da condição e natureza
delas}

\paragraph{[94] Em que se declara a natureza das antas do Brasil}\quad
Bem podemos dizer neste lugar que alimárias se mantêm e criam com a fertilidade da Bahia,
para se acabar de crer e entender o muito que se diz de suas grandezas.

E comecemos das antas, a que os índios chamam tapirusu,\footnote{ Em Varnhagen (1851 e
1879), ``tapiruçú''. Tapira ou anta.} por ser a maior alimária que esta terra cria; as
quais são pardas, com o cabelo assentado, do tamanho de uma mula mas mais baixas das
pernas; e têm as unhas fendidas como vaca, e o rabo muito curto, sem mais cabelo que nas
ancas; e têm o focinho como mula, e o beiço de cima mais comprido que o de baixo, em que
têm muita força. Não correm muito e são pesadas para saltar; defendem"-se estas alimárias
no mato, com as mãos, das outras alimárias, com o que fazem dano onde chegam; comem frutas
silvestres e ervas; e parem uma só criança; e enquanto são pequenas são raiadas de preto e
amarelo tostado ao comprido do corpo, e são muito formosas; mas, depois de grandes,
tornam"-se pardas; e enquanto os filhos não andam, estão os machos por eles e enquanto a
fêmea vai buscar de comer. Matam"-nas em fojos, em que caem, às flechadas. A carne é muito
gostosa, como a de vaca, mas não tem sebo; e quer"-se bem cozida, porque é dura; e tem o
cacho como maçã do peito da vaca; e no peito não tem nada. Os ossos destas alimárias,
queimados e dados a beber, são bons para estancar câmaras; as suas peles são muito rijas,
e em muitas partes as não passa flecha, ainda que seja de bom braço, as quais os índios
comem cozidas pegadas com a carne. Destas peles, se são bem curtidas, se fazem mui boas
couraças, que as não passa estocada.

Se tomam estas antas pequenas, criam"-se em casa, onde se fazem muito domésticas, e tão
mansas que comem as espinhas e roem os ossos com os cachorros e gatos de mistura; e
brincam todos juntos.\footnote{ Em Varnhagen (1851 e 1879), ``e comem as espinhas, e os
ossos com''.}

\paragraph{[95] Em que se trata de uma alimária que se chama jaguarete}\quad
Têm para si os portugueses que jaguarete\footnote{ Em Varnhagen (1851 e 1879),
``jaguareté''. Jaguaretê.} é onça, e outros dizem que é tigre; cuja grandura é como um
bezerro de seis meses; falo dos machos, porque as fêmeas são maiores. A maior parte destas
alimárias são ruivas, cheias de pintas pretas; e algumas fêmeas são todas pretas; e todos
têm o cabelo nédio, e o rosto a modo de cão e as mãos e unhas muito grandes, o rabo
comprido, e o cabelo nele como nas ancas. Têm presas nos dentes como lebréu, os olhos como
gato, que lhe luzem de noite tanto que se conhecem por isso a meia légua; têm os braços e
pernas muito grossos; parem as fêmeas uma e duas crianças; se lhes matam algum filho andam
tão bravas que dão nas roças dos índios, onde matam todos quantos podem alcançar; comem a
caça que matam, para o que são mui ligeiras, em tanto que lhes não escapa nenhuma alimária
grande por pés; e saltam por cima a pique altura de dez, doze palmos; e trepam pelas
árvores após os índios, quando o tronco é grosso; salteiam o gentio de noite pelos
caminhos, onde os matam e comem; e quando andam esfaimadas entram"-lhes nas casas das
roças, se lhes não sentem fogo, ao que têm grande medo. E na vizinhança das povoações dos
portugueses fazem muito dano nas vacas, e como se começam a encarniçar nelas destroem um
curral; e têm tanta força que com uma unhada que dão em uma vaca lhe derrubam a anca
embaixo.\footnote{ Em Varnhagen (1851 e 1879), ``derrubam a anca no chão''.}

Armam os índios a estas alimárias em mundéus, que são uma tapagem de pau"-a-pique, muito
alta e forte, com uma só porta; onde lhes armam com uma árvore alta e grande levantada do
chão, onde lhes põem um cachorro ou outra alimária presa; e indo para a tomar cai esta
árvore que está deitada sobre esta alimária, onde dá grandes bramidos; ao que os índios
acodem e a matam às flechadas; e comem"-lhe a carne, que é muito dura e não tem nenhum
sebo.

\paragraph{[96] Que trata de outra casta de tigres e de alimárias daninhas}\quad
Criam"-se no rio de São Francisco umas alimárias tamanhas como poldros, às quais os índios
chamam jaguoarosu,\footnote{ Em Varnhagen (1851 e 1879), ``jaguaruçú''. Jaguaruçu.} que
são pintadas de ruivo e preto e malhas grandes; e têm as quatro presas dos dentes do
tamanho de um palmo; criam"-se na água deste rio, no sertão; de onde saem à terra, a fazer
suas presas em antas; e ajuntam"-se três e quatro destas alimárias, para levarem nos dentes
a anta ao rio, onde a comem à sua vontade, e a outras alimárias; e também aos índios que
podem apanhar.

Jaguaracangoçu é outra alimária e casta de tigres ou onça da que tratamos já; e são muito
maiores, cuja cabeça é tão grande como de um boi novilho. Criam"-se estas alimárias pelo
sertão, longe do mar, e têm as feições e mais condições dos tigres de que primeiro
falamos. Quando essas alimárias matam algum índio que se encarniçam nele, fazem despovoar
toda uma aldeia, porque em saindo alguma pessoa dela fora de casa não escapa que a não
matem e comam.

Há outra alimária, a que o gentio chama suasuirana,\footnote{ Em Varnhagen (1851 e 1879),
``suçuarana''. Suçuarana.} que é do tamanho de um rafeiro, tem o cabelo comprido e macio,
o rabo como cão, o rosto carrancudo, as mãos como rafeiro, mas tem maiores unhas e mui
agudas e voltadas; vivem da rapina, têm muita ligeireza para correr e saltar; e são
semelhantes na rapina ao lobo, e matam os índios se os podem alcançar, e pela terra
adentro as há muito maiores que na vizinhança do mar. Para os índios matarem estas
alimárias esperam"-nas em cima das árvores, de onde as flecham, e lhes comem a carne; as
quais não têm mais que uma só tripa.

\paragraph{[97] Em que se declaram as castas dos veados que esta terra cria}\quad
Criam"-se nos matos desta Bahia muitos veados, a que os índios chamam suasu,\footnote{ Em
Varnhagen (1851 e 1879), ``suaçú''. Suaçu ou veado.} que são ruivos e tamanhos como
cabras, os quais não têm cornos nem sebo, como os da Espanha. Correm muito, as fêmeas
parem uma só criança. Tomam"-nos em armadilhas, e com cães; cuja carne é sobre o duro, mas
saborosa; as peles são muito boas para botas, as quais se curtem com casca de mangues; e
fazem"-se mais brandas que as dos veados da Espanha.

Mais pela terra adentro, pelas campinas, se criam outros veados brancos, que têm cornos,
que não são tamanhos como os da Espanha, mas são muito maiores que os primeiros, os quais
andam em bandos, como cabras, e têm a mesma qualidade das que se criam perto do mar.

Entrando pelo mato além das campinas, na terra dos tabajaras, se criam uns veados
ruivaços, maiores que os da Espanha, e de maior cornadura, dos quais se acha armação pelo
mato de cinco e seis palmos de alto, e de muitos galhos; os quais mudam os cornos como os
da Espanha, e têm as peles muito grossas, e não têm nenhum sebo; as fêmeas parem uma só
criança, às quais os índios chamam suagupara,\footnote{ Em Varnhagen (1851 e 1879),
``suaçupara''. Suaçuapara.} cuja carne é muito boa; os quais matam em armadilhas, em que
os tomam, às flechadas.

\paragraph{[98] Em que se trata de algumas alimárias que se mantêm de rapina}\quad
Tamanduá é um animal do tamanho de raposa, que tem o rosto como furão; a cor é preta, rabo
delgado na arreigada, e com o cabelo curto; e daí para a ponta é muito felpudo, e tem nela
os cabelos grossos como cavalo, e tamanhos e tantos que se cobre todo com eles quando
dorme; tem as mãos como cão, com grandes unhas e muito voltadas, e de que se fazem apitos.
Este bicho se mantém de formigas, que toma da maneira seguinte: chega"-se a um formigueiro
deita"-se ao longo dele como morto, e lança"-lhe a língua fora, que tem muito comprida, ao
que acodem as formigas, com muita pressa; e cobrem"-lhe a língua, umas sobre as outras; e
como a sente bem cheia, recolhe"-a para dentro, e engole"-as; o que faz até que não pode
comer mais, cuja carne comem os índios velhos, que os mancebos têm nojo dela.

Jagoapitanga\footnote{ Jaguapitanga, jaguamitinga ou raposa"-do"-campo.} é uma alimária do
tamanho de um cachorro, de cor preta, e tem o rosto de cordeiro; tem pouca carne, as unhas
agudas, e é tão ligeira que se mantém no mato de aves que andam pelo chão, toma a cosso, e
em povoado faz ofício de raposa, despovoa uma fazenda de galinha que furta.

Coati\footnote{ Quati.} é um bicho tamanho como gato, tem o focinho como furão e mais
comprido. São pretos, e alguns ruivos; têm os pés como gato, o rabo grande felpudo, o qual
trazem sempre levantado para o ar; são mui ligeiros, andam pelas árvores, de cujas frutas
se mantêm, e os pássaros que nelas tomam. Tomam"-nos os cães quando os acham fora do mato,
a que ferem com as unhas mui valentemente; os novos se amansam em casa, onde tomam as
galinhas que podem alcançar; as fêmeas parem três e quatro.

Maracajao\footnote{Em Varnhagen (1851 e 1879), ``maracajás''. Maracajá, gato"-do"-mato ou
jaguatirica.} são uns gatos bravos tamanhos como cabritos de seis meses; são muito gordos,
e na feição pontualmente como os outros gatos, mas pintados de amarelo e preto em raias,
coisa muito formosa; e são felpudos, mas têm o rabo muito macio, e as unhas grandes e
muito agudas; parem muitos filhos, e mantêm"-se das aves que tomam pelas árvores, por onde
andam como bugios. Os que se tomam pequenos fazem"-se em casa muito domésticos, mas não
lhes escapa galinha nem papagaio, que não matem.


Serigoi\footnote{ Em Varnhagen (1851 e 1879), ``serigoé''. Serigui ou gambá.} é um bicho
do tamanho de um gato grande, de cor preta e alguns ruivaços; tem o focinho comprido e o
rabo, no qual, nem na cabeça, não tem cabelo; as fêmeas têm na barriga um bolso, em que
trazem os filhos metidos, enquanto são pequenos, e parem quatro e cinco; têm as tetas
junto do bolso, onde os filhos mamam; e quando emprenham geram os filhos neste bolso, que
está fechado, e se abre quando parem; onde trazem os filhos até que podem andar com a mãe;
que se lhe fecha o bolso. Vivem estes de rapina, e andam pelo chão, escondidos espreitando
as aves, e em povoado as galinhas; e são tão ligeiros que lhes não escapam.

\paragraph{[99] Que trata da natureza e estranheza do jaguoarecarecaqua}\quad
Jaguoarecaqua\footnote{ Em Varnhagen (1851 e 1879), ``jaguarecaca''. Jaguarecaca ou
jaguaritaca.} é um animal do tamanho de um gato grande; tem a cor pardaça e o cabelo
comprido, e os pés e mãos da feição dos bugios; o rosto como cão, e o rabo comprido; o
qual se mantém das frutas do mato. Anda sempre pelo chão, onde pare uma só criança, o qual
é estranho e fedorento, que por onde quer que passa deixa tamanho fedor que, um tiro de
pedra afastado de uma banda e da outra, não há quem o possa sofrer, e não há quem por ali
possa passar mais de dois meses, por ficar tudo tão empeçonhentado com o mau cheiro que se
não pode sofrer. Deste animal pegam os cães quando vão à caça, mas vão"-se logo lançar na
água, e esfregam"-se com a terra por tirarem o fedor de si, o que fazem por muitos dias sem
lhes aproveitar, e o caçador fica de maneira que por mais que se lave fica sempre com este
terrível cheiro, que lhe dura três e quatro meses; e como este bicho se vê em pressa
perseguido dos cães, lança de si tanta ventosidade, e tão peçonhenta, que perfuma desta
maneira a quem lhe fica por perto; e com estas armas se defendem das onças e de outros
animais quando se veem perseguidos deles, cuja artilharia tem tanta força que a onça ou
outros inimigos que os buscam se tornam e os deixam; e vão"-se logo lavar e esfregar pela
terra, por tirar de si tão terrível cheiro. E aconteceu a um português que, encontrando
com um destes bichos, que trazia o seu caçador do mato morto para mezinha, ficou tão
fedorento que, não podendo sofrer"-se a si, se fez muito amarelo, e se foi para casa doente
do cheiro que em si trazia, que lhe durou muitos dias. A carne deste bicho é boa para
estancar câmaras de sangue; mas a casa onde está fede toda a vida, pelo que as índias a
têm assada muito embrulhada em folhas, depois de bem seca ao ar do fogo; e a têm no fumo
para se conservar; mas nem isso basta para deixar de feder na rua, enquanto está na casa.

\paragraph{[100] Em que se declara a natureza dos porcos"-do"-mato que há na Bahia}\quad
Criam"-se nos matos da Bahia porcos monteses, a que os índios chamam tajasu,\footnote{ Em
Varnhagen (1851 e 1879), ``tajaçú''. Tajaçu ou queixada.} que são de cor parda e
pequenos; tudo têm semelhante com o porco, senão o rabo, que não têm mais comprido que uma
polegada; e têm umbigo nas costas; as fêmeas parem muitos no mato, por onde andam em
bandos, comendo as frutas dele; onde os matam com cachorros e armadilhas, e às flechadas;
os quais não têm banha, nem toucinho, senão uma pele viscosa; a carne é toda magra, mas
saborosa e carregada para quem não tem boa disposição.

Tajasutiraqua\footnote{ Em Varnhagen (1851 e 1879), ``tajaçutirica''.} é outra casta de
porcos monteses, maiores que os primeiros, que têm os dentes como os monteses da Espanha;
e os índios que os flecham hão de ter, prestes, onde se acolham, porque, se se não põem em
salvo com muita presteza, não lhes escapam; os quais são muito ligeiros e bravos, e têm
também o umbigo nas costas; e não têm banha, nem toucinho, mas a carne mais gostosa que os
outros; e em tudo mais são como eles.

Tajasu"-eté\footnote{ Em Varnhagen (1851 e 1879), ``tajaçuété''.} é outra casta de porcos
monteses que são maiores que os de que fica dito, e têm toucinho como os monteses da
Espanha, e grandes presas e o umbigo nas costas, mas não são tão bravos e perigosos para
os caçadores; os quais os fazem levantar com os cachorros para os flecharem; e estes e os
mais andam em bandos pelo mato, onde as fêmeas parem muitos filhos; e no tempo das frutas
entram pelas aldeias dos índios e pelas casas; os quais fazem muito
dano nas roças e nos canaviais de açúcar. A estes porcos cheira o umbigo muito mal; e se,
quando os matam, lhos não cortam logo, cheira"-lhes a carne muito ao mato; e se lhos cortam
é muito saborosa.

\paragraph{[101] Dos porcos e outros bichos que se criam na água doce}\quad
Nos rios de água doce e nas lagoas se criam muitos porcos, a que os índios chamam
capibaras,\footnote{ Capivara.} que não são tamanhos como os porcos"-do"-mato; os quais têm
pouco cabelo e a cor cinzenta, e o rabo como os outros; e não têm na boca mais que dois
dentes grandes, ambos debaixo, na dianteira, que são do comprimento e grossura de um dedo;
e cada um é fendido pelo meio e fica de duas peças; e têm mais outros dois queixais, todos
no queixo de baixo, que no de cima não têm nada; os quais parem e criam os filhos debaixo
da água, onde tomam peixinhos e camarões que comem; também comem erva ao longo da água, de
onde saem em terra, e fazem muito dano nos canaviais de açúcar e roças que estão perto da
água, onde matam em armadilhas; cuja carne é mole, e o toucinho pegajoso; mas
salpresa\footnote{ Salpreso: conservado em sal; levemente salgado.} é boa de toda a
maneira, mas carregada para quem não tem saúde.

Criam"-se nos rios de água doce outros bichos, que se parecem com lontras de Portugal, a
que o gentio chama jaguoapapeba,\footnote{ Em Varnhagen (1851 e 1879), ``jagoarapeba''.}
que têm o cabelo preto e tão macio como veludo. São do tamanho de um gozo,\footnote{ Gozo:
cão pequeno, de pernas curtas e corpo alongado.} têm a cabeça como de gato, e a boca
muito rasgada e vermelha por dentro, e nos dentes grandes presas, as pernas curtas. Andam
sempre na água, onde criam e parem muitos filhos e onde se mantêm dos peixes que tomam e
dos camarões; não saem nunca fora da água, onde gritam quando veem gente ou outro bicho.

Ireraá\footnote{ Em Varnhagen (1851 e 1879), ``irerã''. Irara.} é outro bicho da água
doce, tamanho como um grande rafeiro, de cor parda, e outros pretos. Têm a feição de cão,
e ladram como cão, e remetem à gente com muita braveza; as fêmeas parem muitos filhos
juntos; e se os tomam novos, criam"-se em casa, onde se fazem domésticos. Mantêm"-se do
peixe e dos camarões que tomam na água; cuja carne comem os índios.

Nos mesmos rios se criam outros bichos, a que os índios chamam vyvya,\footnote{ Em
Varnhagen (1851 e 1879), ``vivia''.} que são do tamanho dos gozos, felpudos do cabelo, e
de cor cinzenta; têm o focinho comprido e agudo, as orelhas pequeninas e redondas, do
tamanho de uma casca de tremoço; têm o rabo muito comprido e grosso pela arreigada, como
carneiro; quando gritam no rio, nomeiam"-se pelo seu nome; têm as mãos e unhas de cão,
andam sempre na água, onde as fêmeas parem muitos filhos; mantêm"-se do peixe e camarões
que tomam, cuja carne comem os índios.

\paragraph{[102] De uns animais a que chamam tatus}\quad
Tatu"-asu\footnote{ Em Varnhagen (1851 e 1879), ``tatuaçú''. Tatuaçu.} é um animal
estranho, cujo corpo é como um bácoro;\footnote{ Bácoro: porco pequeno.} tem as pernas
curtas, cheias de escamas, o focinho comprido cheio de conchas, as orelhas pequenas, e a
cabeça que é toda cheia de conchinhas; os olhos pequeninos, o rabo comprido cheio de
lâminas em redondo, que cavalga uma sobre outra; e tem o corpo todo coberto de conchas,
feitas em lâminas, que atravessam o corpo todo, de que tem armado uma formosa coberta; e
quando se este animal teme de outro, mete"-se todo debaixo destas armas, sem lhe ficar nada
de fora, as quais são muito fortes; têm as unhas grandes, com que fazem as covas debaixo
do chão, onde criam; e parem duas crianças. Mantêm"-se de frutas silvestres e minhocas,
andam devagar, e, se caem de costas, têm trabalho para se virar; e têm a barriga vermelha
toda cheia de verrugas. Matam"-nos os índios em armadilhas onde caem; tiram"-lhes o corpo
inteiro fora destas armas, que estendidas são tamanhas como uma adarga; cuja carne é muito
gorda e saborosa, assim cozida como assada.

Há outra casta de tatus pequenos, da feição dos grandes, os quais têm as mesmas manhas e
condição; mas quando se temem de lhes fazerem mal, fazem"-se uma bola toda coberta em
redondo com suas armas, onde ficam metidos sem lhes aparecer coisa alguma; cuja carne é
muito boa; comem e criam como os grandes. A estes chamam tatu"-mirim.

Há outros tatus meãos, que não são tamanhos como os primeiros, de que se acham muitos no
mato, cujo corpo não é maior que de um leitão; têm as pernas curtas cobertas de conchas, a
cabeça comprida cheia de conchas, os dentes de gato, as unhas de cão, o rabo comprido e
muito agudo coberto de conchas até a ponta, e por cima sua coberta de lâminas, como os
grandes, que são muito rijas; e na barriga não têm nada; cuja carne quando estão gordos é
boa, mas cheira ao mato; mantêm"-se de frutas e minhocas, criam debaixo do chão em covas, e
têm as mais manhas e condições dos outros.

Tatupeba\footnote{ Tatupeba ou tatupeva.} é outra casta de tatus maiores que os comuns,
que ficam nesta adição acima, os quais têm as conchas mais grossas, e são muito baixos das
mãos e pernas, e têm"-nas muito grossas, e são muito carrancudos; e andam sempre debaixo do
chão, como toupeiras, e não comem mais que minhocas; e em tudo o mais são semelhantes aos
de cima; e matam"-nos os índios quando veem bulir a terra; cuja carne é muito boa.

\paragraph{[103] Em que se relata a propriedade das pacas e cotias}\quad
Criam"-se nestes matos uns animais, a que os índios chamam pacas, que são do tamanho de
leitões de seis meses, têm a barriga grande, e os pés e mãos curtos, as unhas como
cachorros, e cabeça como lebre, o pelo muito macio, raiado de preto e branco ao comprido
do corpo; têm o rabo muito comprido, correm pouco. As fêmeas parem duas e três crianças,
comem frutas e ervas, criam em covas. Tomam"-se com cães, e com armadilhas, a que chamam
mandéus; são algumas vezes muito gordos, e têm a banha como porco; cuja carne é muito
sadia e gostosa, assim assada como cozida; pela"-se como leitão sem se esfolar, e assada
faz couros como leitão, e de toda maneira é muito boa carne.

Cotias são uns bichos tamanhos como coelhos grandes, mas são muito barrigudos; têm o
cabelo como lebre, a cabeça com o focinho agudo, e os dentes mui agudos; os dois
dianteiros são compridos e agudíssimos, com o que os índios se sarjam\footnote{ Sarjar:
cortar.} como com uma lanceta;\footnote{ No manuscrito da \textsc{bgjm}, ``com o que os
negros se sarjam''.} têm os pés e as mãos como coelhos, as unhas como cão, criam em covas,
em que parem duas e três crianças; mantêm"-se com frutas; quando correm fazem na anca uma
roda de cabelos, que ali têm compridos; são muito ligeiras, entanto que não há cão que as
tome, senão nas covas, onde se defendem com os dentes; também se tomam em laços; se as
tomam em pequenas, fazem"-se tão domésticas em casa como coelhos;\footnote{ Em Varnhagen
(1851 e 1879), ``tão domésticas como coelhos''.} mas são daninhas, porque roem muito o
fato; cuja carne se não esfola, mas pelam"-nas, como leitão; cozida e assada é muito boa.

Cotimirim\footnote{ Em Varnhagen (1851 e 1879), ``cotimerim''.} é outra casta de cotias,
do tamanho de um láparo; têm o focinho comprido, e são muito felpudas, de cor parda; e têm
o rabo muito felpudo, o qual viram para cima, e passa"-lhes a felpa por cima da cabeça, com
que se cobrem; e trepam muito pelas árvores, onde matam outros bichos, que chamam saguins,
do que se mantêm; criam em covas debaixo do chão, e têm os dentes muito agudos.

\paragraph{[104] Que trata das castas dos bugios e suas condições}\quad
Nos matos da Bahia se criam muitos bugios de diversas maneiras; a uns chamam
gigós,\footnote{ Em Varnhagen (1851 e 1879), ``guigós''. Guigó ou sauá.} que andam em
bandos pelas árvores, e, como sentem gente, dão uns assobios com que se avisam uns aos
outros, de maneira que em um momento corre a nova pelo espaço de uma légua, com que
entendem que é entrada de gente, para se porem em salvo. E se atiram alguma flechada a
algum, e o não acertam, matam"-se todos de riso; estes bugios criam em tocas de árvores, de
cujos frutos e da caça se mantêm.

Guaribas é outra casta de bugios que são grandes e mui entendidos; estes têm barbas como
um homem, e o rabo muito comprido; os quais, como se sentem flechados dos índios, se não
caem da flechada, fogem pela árvore acima, mastigando folhas, e metendo"-as pela flechada,
com que tomam o sangue e o curam; e aconteceu muitas vezes tomarem a flecha que têm em si,
e atirarem com ela ao índio que lha atirou, e ferirem"-no com ela; e outras vezes deixam"-se
cair com a flecha na mão sobre o índio que os flechou. Estes bugios criam também nos
troncos das árvores, de cujas frutas se mantêm, e de pássaros que tomam; e as fêmeas parem
uma só criança.

Saguins\footnote{ Sagui, saguim ou sauí.} são bugios pequeninos mui felpudos, e de cabelo
macio, raiado de pardo e preto e branco; têm o rabo comprido, e muita felpa no pescoço, a
qual trazem sempre arrepiada, o que os faz muito formosos; e criam"-se em casa, se os tomam
novos, onde se fazem muito domésticos; os quais criam nas tocas das árvores, e mantêm"-se
do fruto delas e das aranhas que tomam.

Do Rio de Janeiro vêm outros saguins, da feição destes de cima, que têm o pelo amarelo
muito macio, que cheiram muito bem; os quais e os de trás são muito mimosos, e morrem em
casa, de qualquer frio, e das aranhas de casa, que são mais peçonhentas que as das
árvores, onde andam sempre saltando de ramo em ramo.

Há nos matos da Bahia outros bugios, a que os índios chamam cai"-enhagua,\footnote{ Em
Varnhagen (1851 e 1879), ``saîanhangá''.} que quer dizer ``bugio diabo'' que são muito
grandes, e não andam senão de noite; são da feição dos outros, e criam em côncavos de
árvores; mantêm"-se de frutas silvestres e o gentio tem agouro neles, e como os ouvem
gritar, dizem que há de morrer algum.

\paragraph{[105] Que trata da diversidade dos ratos que se comem, e coelhos e outros ratos
de casa}\quad
Pelo sertão há uns bichos a que os índios chamam savya"-eté,\footnote{ Em Varnhagen (1851 e
1879), ``saviá''. Sauiá ou rato"-de"-espinho.} tamanhos como láparos; têm o rabo comprido,
o cabelo como lebre; criam em covas no chão; mantêm"-se das frutas silvestres; tomam"-nos em
armadilhas cuja carne é muito estimada de toda a pessoa, por ser muito saborosa, e
parece"-se com a dos coelhos.

Aperias\footnote{ Em Varnhagen (1851 e 1879), ``aperiá''. Preá.} são outros bichos
tamanhos como láparos, que não têm rabo; e têm o rosto da feição do leitão, as orelhas
como coelho, e o cabelo como lebre; criam em covas, comem frutas e canas"-de"-açúcar, a que
fazem muito dano; cuja carne é muito saborosa.

Mais pela terra adentro há outros bichos da feição de ratos, mas tamanhos como coelhos,
com o cabelo branco, a que os índios chamam savyatinga,\footnote{ Em Varnhagen (1851 e
1879), ``saviátinga''.} os quais criam em covas, e comem frutas; cuja carne é muito boa,
sadia e saborosa.

No mesmo sertão há outros bichos, da feição de ratos, tamanhos como coelhos, a que os
índios chamam saviacoqua,\footnote{ Em Varnhagen (1851 e 1879), ``saviácoca''.} que têm o
cabelo vermelho, criam em covas, e mantêm"-se da fruta do mato; cuja carne é como de
coelhos.

Em toda a parte dos matos da Bahia se criam coelhos como os da Espanha, mas não são
tamanhos, a que os índios chamam tapotim;\footnote{ Tapiti.} e todas as feições têm de
coelhos, senão o rabo, porque o não têm; os quais criam em covas, e as fêmeas parem muito;
cuja carne é como a dos coelhos, e muito saborosa.

Em algumas partes dos matos da Bahia se criam uns bichos, sobre o grande, com todas as
feições e parecer de ratos, a que os índios chamam jupati,\footnote{ Jupati ou cuíca.} que
não se comem, os quais criam em troncos das árvores velhas; e as fêmeas têm um bolso na
barriga, em que trazem sete e oito filhos, até que são criados, que tanto parem.

Aos ratos das casas chamam os índios saviá,\footnote{ Sauiá.} onde se criam infinidade
deles, os quais são muito daninhos, e de dia andam pelo mato, e de noite vêm"-se meter nas
casas.

\paragraph{[106] Que trata dos cágados da Bahia}\quad
Em qualquer parte dos matos da Bahia se acham muitos cágados, que se criam pelos pés das
árvores, sem irem à água, a que os índios chamam jaboti;\footnote{ Em Varnhagen (1851 e
1879), ``jabuty''. Jabuti.} há uns que são muito maiores que os da Espanha, mais altos e
de mais carne, e têm as conchas lavradas em compartimentos oitavados de muito notável
feitio; os lavores dos compartimentos são pretos, e o meio de cada um é branco e
almecegado. Estes cágados têm as mãos, pés, pernas, pescoço e cabeça, cheios de verrugas
tamanhas, como chícharos, muito vermelhas, e agudas nas pontas; estes põem infinidade de
ovos, de que nascem em terra úmida, onde criam debaixo do arvoredo; mantêm"-se de frutas,
que caem pelo chão; e metidos em casa comem tudo quanto acham pelo chão; cuja carne é
muito gorda, saborosa e sadia para doentes.

Há outros cágados, que também se criam no mato, sem irem à água, a que os índios chamam
jabotiapeba;\footnote{ Em Varnhagen (1851 e 1879), ``jabutiapeba''.} os quais têm os
mesmos lavores nas conchas, mas são muito amassados, e têm as costas muito chãs, e não têm
verrugas; têm pouca carne e mui saborosa; criam e mantêm"-se pela ordem dos de cima.

Há outras castas de cágados da feição dos da Espanha, a que os índios chamam
jabotimirim,\footnote{ Em Varnhagen (1851 e 1879), ``jabotemirim''.} que se criam e andam
sempre na água, que também são mui saborosos e medicinais; e dos que se criam na água há
muitas castas de diversas feições, que têm as mesmas manhas e natureza, mas mui diferentes
na grandura. E pareceu"-me decente arrumar neste capítulo os cágados por serem animais que
se criam na terra, e se mantêm de frutas dela.

\paragraph{[107] Em que se declara que bicho é o que se chama preguiça}\quad
Nestes matos se cria um animal mui estranho, a que os índios chamam ahi,\footnote{ Em
Varnhagen (1851 e 1879), ``ahy''. Aí ou preguiça.} e os portugueses preguiça, nome certo
mui acomodado a este animal, pois não há fome, calma, frio, água, fogo, nem outro nenhum
perigo que veja diante, que o faça mover uma hora mais que outra; o qual é felpudo como
cão d'água, e do mesmo tamanho; e tem a cor cinzenta, os braços e pernas grandes, com
pouca carne, e muita lã; tem as unhas como cão e muito voltadas; a cabeça como gato, mas
coberta de gadelhas que lhe cobrem os olhos; os dentes como gato. As fêmeas parem uma só
criança, e trá"-la, desde que a pare, ao pescoço dependurado pelas mãos, até que é criada e
pode andar por si; e parem em cima das árvores, de cujas folhas se mantêm, e não se descem
nunca ao chão, nem bebem; e são estes animais tão vagarosos que posto um ao pé de uma
árvore, não chega ao meio dela desde pela manhã até as vésperas, ainda que esteja morta de
fome e sinta ladrar os cães que a querem tomar; e andando sempre, mas muda uma mão só
muito devagar, e depois a outra, e faz espaço entre uma e outra, e da mesma maneira faz
aos pés, e depois à cabeça; e tem sempre a barriga chegada à árvore, sem se pôr nunca
sobre os pés e mãos e se não faz vento, por nenhum caso se move do lugar onde está
encolhida até que o vento lhe chegue; os quais dão uns assobios, quando estão comendo de
tarde em tarde, e não remetem a nada, nem fazem resistência a quem quer pegar deles, mais
que pegarem"-se com as unhas à árvore onde estão, com que fazem grande presa; e acontece
muitas vezes tomarem os índios um destes animais, e levarem"-no para casa, onde o têm
quinze e vinte dias, sem comer coisa alguma, até que de piedade o tornam a largar; cuja
carne não comem por terem nojo dela.

\paragraph{[108] Que trata de outros animais diversos}\quad
Nestes matos se cria um animal, a que os índios chamam jupara,\footnote{ Em Varnhagen
(1851 e 1879), ``a que os gentios chamam jupará''. Jupará ou japurá.} que quer dizer
``noite''; e é do tamanho de um bugio, e anda de árvore em árvore, como bugio, por ser
muito ligeiro; cria no côncavo das árvores, onde pare um só filho, e mantém"-se dos frutos
silvestres. Este animal tem a boca por dentro até as goelas, e língua tão negra, que faz
espanto, pelo que lhe chamam noite, cuja carne os índios não comem, por terem nojo dela.

Há outro bicho que no mato se cria a que chamam os índios coandu,\footnote{ Cuandu ou
ouriço"-cacheiro.} que é do tamanho de um gato; não corre muito, por ser pesado no andar;
cria no tronco das árvores, onde está metido de dia; e de noite sai da cova ou ninho a
andar pela árvore, onde faz sua morada, a buscar uma casta de formigas que se cria nela, a
que chamam copi,\footnote{ Em Varnhagen (1851 e 1879), ``copy''. Cupim.} de que se mantêm.
Este bicho pare uma só criança, e tem a cor pardaça; o qual dorme todo o dia, e anda de
noite. E no lugar onde pariu aí vive sempre, e os filhos, e toda a sua geração que dele
procede; e não buscam outro lugar senão quando não cabem no primeiro.

Cuim\footnote{ Cuim ou ouriço.} é outro bicho assim chamado pelos índios, que é do tamanho
de um láparo; tem os pés muito curtos, o rabo comprido, o focinho como doninha; e é todo
cheio de cabelos brancos e tesos, e por entre o cabelo é todo cheio de espinhos até o
rabo, cabeça, pés, os quais são tamanhos como alfinetes; com os quais se defende de quem
lhe quer fazer mal, sacudindo"-os de si com muita fúria, com o que ferem os outros animais;
os quais espinhos são amarelos, e têm as pontas pretas e mui agudas; e por onde estão
pregados, no couro são farpados. Estes bichos correm pouco, criam debaixo do chão, onde
parem uma só criança, e mantêm"-se de minhocas e frutas, que acham pelo chão.

Acham"-se outros bichos pelo mato a que os índios chamam queiroa,\footnote{ Em Varnhagen
(1851 e 1879), ``queiroá''.} que são, nem mais nem menos, como ouriços"-cacheiros de
Portugal, da mesma feição, e com os mesmos espinhos; e criam em covas debaixo do chão;
mantêm"-se de minhocas e de frutas que caem das árvores, cuja carne os índios não comem.

\paragraph{[109] Em que se declara a qualidade das cobras, lagartos e outros bichos}\quad
Agora cabe aqui dizermos que cobras são estas do Brasil, de que tanto se fala em Portugal,
e com razão, por que tantas e tão estranhas, não se sabe onde as haja.

Comecemos logo a dizer das cobras a que os índios chamam giboyas,\footnote{ Em Varnhagen
(1851 e 1879), ``giboias''. Jiboia.} das quais há muitas de cinquenta e sessenta palmos
de comprido, e daqui para baixo. Estas andam nos rios e lagoas, onde tomam muitos porcos
da água, que comem; e dormem em terra, onde tomam muitos porcos, veados e outra muita
caça, o que engolem sem mastigar, nem espedaçar; e não há dúvida senão que engolem uma
anta inteira, e um índio; o que fazem porque não têm dentes, e entre os queixos lhes moem
os ossos para os poderem engolir. E para matar uma anta ou um índio, ou outra qualquer
caça, cingem"-se com ela muito bem, e como têm segura a presa, buscam"-lhe o sesso com a
ponta do rabo, por onde o metem até que matam o que têm abarcado; e como têm morta a caça,
moem"-na entre os queixos para a poder melhor engolir. E como têm a anta, ou outra coisa
grande que não podem digerir, empanturram de maneira que não podem andar. E como se sentem
pesadas lançam"-se ao sol como mortas, até que lhes apodrece a barriga, e o que têm nela;
do que dá o faro logo a uns pássaros que se chamam urubus, e dão sobre elas comendo"-lhes a
barriga com o que têm dentro, e tudo o mais, por estar podre; e não lhes deixam senão o
espinhaço, que está pegado na cabeça e na ponta do rabo, e é muito duro; e como isto fica
limpo da carne toda, vão"-se os pássaros; e torna"-lhes a crescer a carne nova, até ficar a
cobra em sua perfeição; e assim como lhes vai crescendo a carne, começam a bulir com o
rabo, e tornam a reviver, ficando como dantes; o que se tem por verdade, por se ter tomado
disto muitas informações dos índios e dos línguas que andam por entre eles no sertão, os
quais o afirmam assim.

E um Jorge Lopes, almoxarife da capitania de São Vicente, grande língua, e homem de
verdade, afirmava que indo para uma aldeia do gentio no sertão, achara uma cobra destas no
caminho, que tinha liado três índios para os matar, os quais livrara deste perigo ferindo
a cobra com a espada por junto da cabeça e do rabo, com o que ficou sem força para os
apertar, e que os largara; e que acabando de matar esta cobra, ele lhe achara dentro
quatro porcos, a qual tinha mais de sessenta palmos de comprido; e junto do curral de
Garcia d'Ávila, na Bahia, andavam duas cobras que lhe matavam e comiam as vacas, o qual
afirmou que adiante dele lhe saíra um dia uma, que remeteu a um touro, e que lho levou
para dentro de uma lagoa; a que acudiu um grande lebréu, ao qual a cobra arremeteu e
engoliu logo; e não pôde levar o touro para baixo pelo impedimento que lhe tinha feito o
lebréu; o qual touro saiu acima da água depois de afogado; e afirmou que neste mesmo lugar
mataram seus vaqueiros outra cobra que tinha noventa e três palmos, e pesava mais de oito
arrobas; e eu vi uma pele de uma cobra destas que tinha quatro palmos de largo. Estas
cobras têm as peles cheias de escamas verdes, amarelas e azuis, das quais tiram logo uma
arroba de banha da barriga, cuja carne os índios têm em muita estima, e os mamelucos, por
acharem"-na muito saborosa.

\paragraph{[110] Que trata de algumas cobras grandes que se criam nos rios da Bahia}\quad
Sucuriu\footnote{ Em Varnhagen (1851 e 1879), ``sucuriú''. Sucuri.} é outra casta de
cobras, que andam sempre na água, e não saem à terra; são mui grandes, têm as escamas
pardas e brancas, das quais matam os índios muitas de quarenta e cinquenta palmos de
comprimento. Estas engolem um porco d'água, cuja carne os índios e alguns portugueses
comem, e dizem ser muito gostosa.

Biúna\footnote{ Em Varnhagen (1851 e 1879), ``boiuna''. Boiúna.} é outra casta de cobras,
que se criam na água, nos rios do sertão, as quais são descompassadas de grandes e
grossas, cheias de escamas pretas, e têm tamanha garganta que engolem um negro sem o
mastigarem nem matarem,\footnote{ Em Varnhagen (1851 e 1879), ``um negro sem o
tomarem''.} entanto que quando o engolem ou alguma alimária, se metem na água para o
afogarem dentro, e não saem da água senão para remeterem a uma pessoa ou caça, que anda
junto ao rio; e se com a pressa com que engolem a presa se embaraça e peja, com o que não
pode tornar para a água de onde saiu, morre em terra, e sai"-se a pessoa ou alimária de
dentro viva; e afirmam os línguas que houve índios que estas cobras engoliram, que estando
dentro da sua barriga tiveram acordo de as matar com a faca que levavam dependurada ao
pescoço, como costumam.

Nos rios e lagoas se criam umas cobras, a que os índios chamam araboya,\footnote{
Araboia.} que são mui grandes, e têm o corpo verde e a cabeça preta, as quais não saem
nunca à terra, e mantêm"-se dos peixes e bichos que tomam na água, cuja carne os índios
comem.

Há outra casta de cobras que se criam nos rios, sem saírem à terra, a que índios chamam
taraiboya,\footnote{ Em Varnhagen (1851 e 1879), ``taraîboia''. Trairaboia.} que são
amarelas e muito compridas e grossas; as quais se mantêm do peixe que tomam nos rios e são
muito gordas e boas para comer.

\paragraph{[111] Que trata das cobras de coral e das gereracas}\quad
Pelos matos e ao redor das casas se criam umas cobras a que os índios chamam
gereracas;\footnote{ Jararaca.} as maiores são de sete e oito palmos de comprido, e são
pardas e brancacentas nas costas, as quais se põem às tardes ao longo dos caminhos
esperando a gente que passa, e em lhes tocando com o pé lhes dão tal picada que se lhes
não acodem logo com algum defensivo, não dura o mordido vinte e quatro horas. Estas cobras
se põem também em ramos de árvores junto dos caminhos para morderem à gente, o que fazem
muitas vezes aos índios, e quando mordem pela manhã têm a peçonha mais força, como a
víbora; as quais mordem também as éguas e vacas, do que morrem algumas, sem se sentir de
quê, senão depois que não tem mais remédio. Têm estas cobras nos dentes presas, com as
quais mordem de ilhargas; e aconteceu na capitania dos Ilheos morder uma destas cobras um
homem por cima da bota, e não sentir coisa que lhe doesse, e zombou da cobra, mas ele
morreu ao outro dia; e vendendo"-se o seu fato em leilão comprou outro homem as botas e
morreu em vinte e quatro horas com lhe incharem as pernas; pelo que se buscaram as botas, e
acharam nelas a ponta do dente, como de uma agulha, que estava metida na bota; no que se
viu claro que estas gereracas têm a peçonha nos dentes. Essas cobras se criam entre pedras
e paus podres e mudam a pele cada ano; cuja carne os índios comem.

Ubouboca\footnote{ Em Varnhagen (1851 e 1879), ``ububoca''. Ibiboca, ibiboboca ou
cobra"-coral.} são outras cobras assim chamadas do tamanho das gereracas, mas mais
delgadas, a que os portugueses chamam de coral, porque têm cobertas as peles de escamas
grandes vermelhas e quadradas, que parecem coral; e entre uma
escama e outra vermelha, têm uma preta pequena. Estas cobras não remetem à gente, mas se
lhe tocam picam logo com os dentes dianteiros e são as suas mordeduras mais peçonhentas
que as da gereraca, e de maravilha\footnote{ De maravilha: por milagre.} escapa pessoa
mordida delas. E quando estão enroscadas no chão parece um ramal de corais; e houve homem
que tomou uma que estava dormindo, e meteu"-a no seio, cuidando serem corais, e não lhe fez
mal; as quais criam debaixo de penhascos e da rama seca.

\paragraph{[112] Em que se declara que cobras são as de cascavel, e as dos formigueiros, e
as que chamam boyaapoá}\quad
Boicininga\footnote{ Boicininga ou cascavel.} quer dizer ``cobra que tange'', pela língua
do gentio; as quais são pequenas e muito peçonhentas quando mordem; chamam"-lhes os
portugueses cobras de cascavel, porque têm sobre a ponta do rabo uma pele dura,\footnote{
Em Varnhagen (1851 e 1879), ``têm sobre o rabo uma''.} ao modo de reclamo, tamanha como
uma bainha de gravanço, mas é muito aguda na ponta que tem para cima, onde tem dois
dentes, com que mordem, que são agudos. Esta bainha lhes retine muito, quando andam, pelo
que são logo sentidas, e não fazem tanto dano. E afirmam os índios que as cobras desta
casta não mordem com a boca mas com aquele aguilhão farpado que têm neste cascavel, o qual
também retine fora da cobra; e tem tantos reclamos como a cobra tem de anos; e cada ano
lhe nasce um; as quais cobras mordem ou picam com esta ponta de cascavel de salto.

Nos formigueiros velhos se criam outras cobras, que se chamam uboijara;\footnote{ Em
Varnhagen (1851 e 1879), ``úbojára''.} são de três até cinco palmos e têm o rabo rombo na
ponta, da feição da cabeça; e não têm outra diferença um do outro que ter a cabeça boca,
em a qual não têm olhos e são cegas; e saem dos formigueiros quando se eles enchem com a
água da chuva; e como se saem fora, ficam perdidas sem saberem por onde andam; e se chegam
a morder, são também mui peçonhentas. Estas cobras não são ligeiras como as outras, e
andam muito devagar; têm a pele de cor acatassolada pela banda de cima, e pela de baixo
são brancas; mantêm"-se nos formigueiros das formigas quando as podem alcançar, e do seu
mantimento, de onde também se saem apertadas de fome.

Boitiapoa\footnote{ Em Varnhagen (1851 e 1879), ``boitiapóia''. Boitiaboia.} são cobras de
cinquenta e sessenta palmos de comprido e muito delgadas, que não mordem a nada; porque
têm o focinho muito comprido e o queixo de baixo muito curto; onde têm a boca muito
pequena e não podem chegar com os dentes a quem querem fazer mal, porque lho impede o
focinho; mas para matarem uma pessoa ou alimária enroscam"-se com ela, e apertam"-na
rijamente e buscam"-lhe com a ponta do rabo os ouvidos, pelos quais lhe metem com muita
presteza, porque a têm muito dura e aguda; e por este lugar matam a presa, em que se
depois desenfadam à vontade.

\paragraph{[113] Em que se declara a natureza de cobras diversas}\quad
Surucucu são umas cobras muito grandes e brancas na cor, que andam pelas árvores, de onde
remetem à gente, e à caça que passa por junto delas, as quais têm os dentes tamanhos que
quando mordem levam logo bocado de carne fora. Da carne destas cobras são os índios muito
amigos,\footnote{ Em Varnhagen (1851 e 1879), ``Destas cobras são os índios muito
amigos''.} e tomam"-nas em armadilhas que chamam mundéus, e se o macho acha ali a fêmea
presa e morta, espera ali o armador, com que se cinge, e não o larga até que o mata, e
torna a esperar ali até que venha outra pessoa, a quem morde somente, e com esta vingança
se vai daquele lugar.

Há outra casta de cobras, a que os índios chamam tiaparana,\footnote{ Em Varnhagen (1851 e
1879), ``tiopurana''.} que são de quarenta e cinquenta palmos de comprido, que não mordem
nem fazem mal a gente nenhuma, e mantêm"-se da caça que tomam. Estas tomam os índios às
mãos, quando são novas, e prendem"-nas em casa, onde as criam, e se fazem tão domésticas
que vão buscar comer ao mato, e tornam"-se para casa, cuja carne é muito saborosa.

Caninam são outras cobras meãs na grandura, com a pele preta nas costas e amarela na
barriga, as quais criam no côncavo de paus podres, e são muito peçonhentas, e os mordidos
delas morrem muito depressa, se lhes não acodem logo.

Boiubu\footnote{ Boiubi, boiubu ou cobra"-cipó.} quer dizer ``cobra verde'', que não são
grandes, e criam"-se no campo, onde se mantêm com ratos que tomam. Estas também mordem
gente se podem, mas não são muito peçonhentas,\footnote{ Em Varnhagen (1851 e 1879), ``mas
são muito peçonhentas''. Boiubu é uma espécie de serpente não peçonhenta.} as quais se
enroscam com as lagartixas, ratos e com outros bichos com que se atrevem, que também matam
para comer.

Há outra casta de cobras, a que os índios chamam ubiracoa,\footnote{ Em Varnhagen (1851 e
1879), ``ubiracoá''. Ubiraquá.} que são pequenas e de cor ruivaça, as quais andam sempre
pelas árvores, de onde mordem no rosto e pelos lugares altos das pessoas, e não se descem
nunca ao chão; se não acodem a mordedura desta com brevidade, é a sua peçonha tão fina que
faz arrebentar o sangue em três horas por todas as partes, de que o mordido morre logo.

Urapiaguoara\footnote{ Em Varnhagen (1851 e 1879), ``urapiagára''.} são outras cobras, que
andam pelas árvores salteando pássaros, e a comer"-lhes os ovos nos ninhos, de que se
mantêm; as quais não são grandes, mas muito ligeiras.

\paragraph{[114] Que trata dos lagartos e dos camaleões}\quad
Nas lagoas e rios de água doce se criam uns lagartos a que os índios chamam jacaré, dos
quais há alguns tamanhos como um homem; e têm a cabeça como um grande lebréu; estes
lagartos são todos cobertos de conchas muito rijas, os quais não remetem à gente, antes
fogem dela; e mantêm"-se do peixe que matam,\footnote{ Em Varnhagen (1851 e 1879), ``do
peixe que tomam''.} e da erva que comem ao longo da água; e há alguns negros que lhes têm
perdido o medo, e se vão a eles, chamando"-os pelo seu nome; e vão"-se chegando a eles até
que os tomam às mãos e os matam para os comerem; cuja carne é um tanto adocicada, e tão
gorda que tem na barriga banha como porco, a qual é alva e saborosa e cheira bem. Os
testículos dos machos cheiram como os dos gatos"-de"-algália, e às fêmeas cheira"-lhes a
carne de junto do vaso muito bem.

No mato se criam outros lagartos, a que os índios chamam senebuis,\footnote{ Em Varnhagen
(1851 e 1879), ``senembús''. Senembu.} que também são muito grandes, mas não tamanhos
como os jacarés; estes remetem à gente; criam"-se nos troncos das árvores; cuja carne é
muito boa e saborosa.

Criam"-se no mato outros lagartos tamanhos como os de cima, a que os índios chamam
tijuasu,\footnote{ Em Varnhagen (1851 e 1879), ``tijuaçú''.} os quais são mansos, e
criam"-se em cova na terra; mantêm"-se das frutas que buscam pelo mato; cuja carne é havida
por muito boa e saborosa.

Pelos matos se criam outros lagartos pequenos pintados como os da Espanha, a que os índios
chamam jacare"-penima,\footnote{ Em Varnhagen (1851 e 1879), ``jacaré"-pinima''.
Jacarepinima.} os quais criam por entre as pedras, e em tocas de árvores, com os quais
têm as cobras grandes brigas.

Anijuacanga são outros bichos que não têm nenhuma diferença dos camaleões, mas são muito
maiores que os da África, cuja cor naturalmente é verde, a qual mudam como fazem os da
África, e estão logo presos a uma janela um mês sem comerem nem beberem; e estão sempre
virados com o rosto para o vento, de que se mantêm; e não querem comer coisa que lhes
deem, do que comem os outros animais; são muito pesados no andar, e tomam"-nos às mãos, sem
se defenderem; os quais têm o rabo muito comprido, e têm um modo de barbatanas nele como
os cações.

\paragraph{[115] Que trata da diversidade das rãs e sapos que há no Brasil}\quad
Chamam os índios cururus\footnote{ Em Varnhagen (1851 e 1879), ``cururú''. Cururu, ou
sapo"-cururu.} aos sapos da Espanha, que não têm nenhuma diferença, mas não mordem, nem
fazem mal, estando vivos; mortos, sim, porque o seu fel é peçonha mui cruel, e os fígados
e a pele, da qual o gentio usa quando quer matar alguém. Estes sapos se criam pelos
telhados, e em tocas de árvores e buracos das paredes, os quais têm um bolso na barriga em
que trazem os ovos, que são tamanhos como avelãs e amarelos como gema de ovos, de que se
geram os filhos, onde os trazem metidos até que saiam para buscar sua vida; estes sapos
buscam de comer de noite, a que os índios comem, como às rãs; mas tiram"-lhes as tripas e
forçura fora, de maneira que lhe não arrebente o fel; porque se arrebenta fica a carne
toda peçonhenta, e não escapa quem a come, ou alguma coisa da pele e forçura.

E porque as rãs são de diferentes feições e costumes, digamos logo de umas a que os índios
chamam juiponga,\footnote{ Juiponga, ferreiro ou sapo"-martelo.} que são grandes, e quando
cantam parecem caldeireiros que malham nas caldeiras; e estas são pardas, e criam"-se nos
rios onde desovam cada lua; as quais se comem, e são muito alvas e gostosas.

Desta mesma casta se criam nas lagoas, onde desovam enquanto tem água, mas, como se seca,
recolhem"-se para o mato nos troncos das árvores, onde estão até que chove, e como as
lagoas têm qualquer água, logo se tornam para elas, onde desovam; e os seus ovos são
pretos, e de cada um nasce um bichinho com barbatanas e rabo, e as barbatanas se lhes
convertem nos braços, e o rabo se lhes converte nas pernas. Enquanto são bichinhos lhes
chamam os índios juins, do que há sempre infinidade deles, assim nas lagoas como no
remanso dos rios; do que se enchem balaios quando os tomam, e para os alimparem
apertam"-nos entre os dedos, e lançam"-lhes as tripas fora, e embrulham"-nos às mancheias em
folhas, e assam"-nos no borralho; o qual manjar gabam muito os línguas que tratam com o
gentio, e os mestiços.

Juiguoa\footnote{ Em Varnhagen (1851 e 1879), ``juigiá''.} é outra casta de rãs, que são
brancacentas, e andam sempre na água, e quando chove muito falam de maneira que parecem
crianças que choram, as quais se comem esfoladas, como as mais; e são muito alvas, e
gostosas.

Há outra casta de rãs, a que os índios chamam juihi; e são muito grandes, e de cor
pretaça; e desovam na água como as outras, as quais, depois de esfoladas, têm tamanho
corpo como um honesto coelho.

Cria"-se na água outra casta de rãs, a que os índios chamam juiperequa,\footnote{ Em
Varnhagen (1851 e 1879), ``juiperega''. Perereca.} que saltam muito, em tanto que dão
saltos do chão em cima dos telhados, onde andam no inverno, e cantam de cima como chove;
as quais são verdes, e desovam também na água em lugares úmidos; e esfoladas comem"-se como
as outras.

Há outra casta de rãs, a que os índios chamam juiguaran"-guaray,\footnote{ Em Varnhagen
(1851 e 1879), ``juigoaraigarai''.} que são pequenas, e no inverno, quando há de fazer
sol e bom tempo, cantam toda noite no alagadiço, onde se criam, o qual sinal é muito
certo; estas são verdes, e desovam na água que corre entre junco ou rama, também esfoladas
se comem e são muito boas.

Como não há ouro sem fezes, nem tudo é a vontade dos homens, ordenou Deus que entre tantas
coisas proveitosas para o serviço dele, como fez na Bahia, houvesse algumas imundícias que
os enfadassem muito, para que não cuidassem que estavam em outro paraíso terreal, de que
diremos daqui por diante, começando no capítulo que se segue das lagartas.

\paragraph{[116] Que trata das lagartas que se criam na Bahia}\quad
Soqua\footnote{ Em Varnhagen (1851 e 1879), ``soca''.} chamam os índios à lagarta, que é
também como bichos"-da"-seda, quando querem morrer que estão gordos, a qual se cria de
borboletas grandes que vão de passagem. Às vezes se cria essa lagarta com muita água e
morre como faz sol; outras vezes se cria com grande seca e morre como chove. Uma e outra
destrói as novidades de mandioca, algodão, arroz; e faz mal à cana nova de açúcar, e às
vezes é tanta esta lagarta, que vão as estradas cheias delas, e deixam o caminho varrido
da erva, e escaldado. E quando dão nas roças da mandioca chasqueiam de maneira que se ouve
um tiro de pedra, às quais comem os olhinhos novos, e depois as outras folhas; e muitas
vezes é tanta que comem a casca dos ramos da mandioca; e se se não muda o tempo, destroem
as novidades de maneira que causa haver fome na terra, e o chão por onde esta praga passa,
ainda que seja mato, fica escaldado de maneira que não cria erva em dois anos.

Imbua\footnote{ Em Varnhagen (1851 e 1879), ``imbuá''. Emboá ou imbuá.} é outra casta de
lagartas verdes pintadas de preto e a cabeça branca, e outras pintadas de vermelho e
preto, e todas são tão grossas como um dedo, e de meio palmo de comprido, com muitas
pernas, e as quais crestam a terra e árvores por onde passam.

Há outras, mais pequenas que as de trás, que são pretas, de cor muito fina, todas cheias
de pelo tão macio como veludo, e tão peçonhento que faz inchar a carne se lhe tocam, com
cujo pelo os índios fazem crescer a natura; e chamam a estas socauna.

Nos limoeiros e em outras árvores naturais da terra se criam outras lagartas verdes, todas
cobertas de esgalhos verdes, muito sutis e de estranho feitio, tão delgados como cabelos
da cabeça, o que é impossível poder"-se contrafazer com pintura; estas têm os índios por
mais peçonhentas que todas, e fogem muito delas; e afirmam que fazem secar os ramos das
árvores por onde passam com lhes morderem os olhos.

Em que outras árvores, que se chamam cajueiros, se criam umas lagartas ruivaças, tamanhas
como as das couves em Portugal, todas cobertas de pelo, as quais como sentem gente
debaixo, sacodem este pelo de si, e na carne onde chega, se levanta logo tamanha comichão,
que é pior que a das urtigas, o que dura todo um dia; e criam"-se estas nos ramos velhos.

\paragraph{[117] Que trata das lucernas, e de outro bicho estranho}\quad
\mbox{Na Bahia} se criam uns bichos, a que os índios chamam mamoas, aos quais chamam em Portugal
lucernas, e outros cagalume,\footnote{ Caga"-lume ou vaga"-lume.} que andam em noites
escuras, assim em Portugal como na Bahia, em cujos matos os há muito grandes; os quais
entram de noite nas casas às escuras, onde parecem candeias muito claras, porque alumiam
uma casa toda, em tanto que às vezes uma pessoa de súbito vendo a casa clara, deitando"-se
às escuras, de que se espanta, cuidando ser outra coisa; dos quais bichos há muita
quantidade em lugares mal povoados.

Também se criam outros bichos na Bahia mui estranhos, a que os índios chamam buijeja, que
são do tamanho de uma lagarta de couve, o qual é muito resplandecente, em tanto que
estando de noite em qualquer casa, ou lugar fora dela, parece uma candeia acesa, e quando
anda é ainda mais resplandecente. Tem este bicho uma natureza tão estranha que parece
encantamento, e tomando"-o na mão parece um rubi, mui resplandecente, e se o fazem em
pedaços, se torna logo a juntar e andar como dantes; e sobre acinte\footnote{ Sobre
acinte: de propósito, de modo provocador.} se viu por vezes em diferentes partes 
cortar"-se um destes bichos com uma faca em muitos pedaços, e se tornarem logo a
juntar; e depois o embrulharam em um papel por sete ou oito dias,\footnote{ Em Varnhagen
(1851 e 1879), ``em um papel durante oito dias''.} e cada dia o espedaçavam em migalhas, e
tornava"-se logo a juntar e reviver, até que enfadava, e o largavam.

\paragraph{[118] Que trata da diversidade e estranheza das aranhas e dos lacraus}\quad
Na Bahia se cria muita diversidade de aranhas, e tão estranhas que convém declarar a
natureza de algumas. E peguemos logo das que chamam nhanduasu,\footnote{ Em Varnhagen (1851
e 1879), ``nhanduaçu''.} as quais são tamanhas como grandes caranguejos, e muito
cabeludas e peçonhentas; remetem à gente de salto, e têm os dentes tamanhos como ratos,
cujas mordeduras são mui perigosas; e criam"-se em paus podres, no côncavo deles, e no
povoado em paredes velhas.

Há outra casta de aranhas, a que os índios chamam nandus,\footnote{ Em Varnhagen (1851 e
1879), ``nhandui''.} que são acostumadas em toda a parte, de que se criam tantas no
Brasil, com a umidade da terra que, se não alimpam as casas muitas vezes, não há quem se
defenda delas. Estas fazem um bolso na barriga, muito alvo, que parece de longe algodão,
que é do tamanho de dois reais, e de quatro, e de oito reais, em o qual bolso criam mais
de duzentas aranhas; e como podem viver sem a mãe largam o bolso de si com elas e cada uma
vai fazer seu ninho; e como esta sevandija é tão nojenta, escusamos de dizer mais dela.

Suaraju\footnote{ Em Varnhagen (1851 e 1879), ``surajú''.} chamam os índios a um bicho
como os lacraus de Portugal, mas são tamanhos como camarões, e têm duas bocas compridas;
e, se mordem uma pessoa, está atormentada com ardor vinte e quatro horas mas não periga.

Criam"-se na Bahia outros bichos da feição dos lacraus, a que os índios chamam nhanduabiju,
os quais têm o corpo tamanho como um rato, e duas bocas tamanhas como de lagosta; os quais
são todos cheios de pelo, e muito peçonhentos, cujas mordeduras são mui perigosas; e
criam"-se em tocas de árvores velhas, no podre delas.

Não são para lembrar as imundícias de que até aqui tratamos, porque são pouco danosas, e
ao que se pode atalhar com alguns remédios; mas à praga das formigas não se pode
compadecer, porque se elas não foram, a Bahia se poderá chamar outra terra de promissão,
das quais começaremos a dizer daqui por diante.

\paragraph{[119] Que trata das formigas que mais dano fazem, que se chamam saubas}\quad
Muito havia que dizer das formigas do Brasil, o que se deixa de fazer tão copiosamente
como se poderá fazer, por se escusar prolixidade; mas diremos em breve de algumas,
começando nas que mais dano fazem na terra, a que o gentio chama usauba,\footnote{ Em
Varnhagen (1851 e 1879), ``ussaúba''. Saúva.} que é a praga do Brasil, as quais são como
as grandes de Portugal, mas mordem muito e onde chegam destroem as roças de mandioca, as
hortas de árvores da Espanha, as laranjeiras, romeiras e parreiras. Se estas formigas não
foram, houvera na Bahia muitas vinhas e uvas de Portugal; as quais formigas vêm de muito
longe de noite buscar uma roça de mandioca, e trilham o caminho por onde passam, como se
fosse gente por ele muitos dias, e não salteiam senão de noite; e por atalharem a não
comerem as árvores a que fazem nojo, põem"-lhe um cesto de barro ao redor do pé, cheio de
água, e se de dia lhe secou a água, ou lhe caiu uma palha de noite que a atravesse, trazem
tais espias que são logo disso avisadas; e passa logo por aquela palha tamanha multidão
delas que antes que seja manhã, lhe dão com toda a folha no chão; e se as roças e árvores
estão cheias de mato derredor, não lhes fazem mal, mas tanto que as veem limpas, como que
entende que tem gosto a gente disto, saltam nelas de noite, e dão"-lhe com a folha no chão,
para a levarem para os formigueiros; e não há dúvida senão que trazem espias pelo campo,
que levam aviso aos formigueiros, porque se viu muitas vezes irem três e quatro formigas
para os formigueiros, e encontrarem outras no caminho e virarem com elas, e tornarem todas
carregadas, e entrarem assim no formigueiro, e saírem"-se logo dele infinidade delas a
buscarem de comer à roça, onde foram as primeiras; e têm tantos ardis que fazem espanto. E
como se destas formigas não diz o muito que delas há que dizer, é melhor não dizer mais
senão que se elas não foram que o despovoará muita parte da Espanha para irem povoar o
Brasil; pois se dá nele tudo o que se pode desejar, o que esta maldição impede, de maneira
que tira o gosto aos homens de plantarem senão aquilo sem o que não podem viver na terra.

\paragraph{[120] Em que se trata da natureza das formigas"-de"-passagem}\quad
Temos que dizer de outra casta de formigas mui estranha, a que os índios chamam
guoaju,\footnote{ Em Varnhagen (1851 e 1879), ``goajugoaju''. Guaiú.} as quais são
pequenas e ruivas, e mordem muito; estas, de tempos em tempos, se saem da cova, maiormente
depois que chove muito, e torna a fazer bom tempo, que se lhe enche a cova de água; e dão
em uma casa onde lhe não fica caixa em que não entrem nem buraco, nem greta pelo chão e
pelas paredes, onde matam as baratas, as aranhas e os ratos, e todos os bichos que andam;
e são tantas que os cobrem de improviso, e entram"-lhes pelos olhos, orelhas e narizes, e
pelas partes baixas, e assim os levam para os seus aposentos, e a tudo o que matam; e como
correm uma casa toda passam por diante a outra, onde fazem o mesmo e a toda uma aldeia; e
são tantas estas formigas, quando passam, que não há fogo que baste para as queimar, e
põem em passar por um lugar toda uma noite, e se entram de dia, todo um dia; as quais vão
andando em ala de mil em cada fileira; e se as casas em que entram são térreas, e acham a
roupa da cama no chão, por onde elas subam, fazem alevantar mui depressa a quem nela jaz,
e andar por cima das caixas e cadeiras, sapateando, lançando"-as fora e coçando; porque
elas, em chegando, cobrem uma pessoa toda, e se acham cachorros e gatos dormindo, dão
neles de feição, e em outros animais, que os fazem voar; e matam também as cobras que
acham descuidadas; e viu"-se por muitas vezes levarem"-nas estas formigas a rastões
infinidades delas; e matam"-nas primeiro entrando"-lhes pelos olhos e ouvidos, por onde as
tratam e mordem tão mal, e de feição que as acabam.

\paragraph{[121] Que trata da natureza de certas formigas grandes}\quad
Nesta terra se criam umas formigas grandes, a que os índios chamam quibuquibura, que são
as que em Portugal chamam agudes, mas são maiores. Estas saem dos formigueiros depois que
chove muito e vão diversas voando por lugares onde enxameiam grande soma de formigas, e
como lhes toca qualquer coisa, ou lhes dá o vento, logo lhes caem as asas e morrem; e não
pode ser menos destas enxamearem de voo, porque em hortas cercadas de água que ficam em
ilha, lhes arrebentam formigueiros dentro, estando antes a terra limpa delas, e não podem
passar por respeito da água que cerca estas hortas.

Criam"-se na mesma terra outras formigas, a que os índios chamam isans,\footnote{ Em
Varnhagen (1851 e 1879), ``içans''. Içá.} as quais têm o corpo tamanho como passas de
Alicante, e são da mesma cor, as quais têm asas como as agudes, e também se saem dos
formigueiros depois que chove muito, a enxugar"-se ao sol; e têm grande boca, e tão aguda,
que cortam com ela como tesoura o fato a que chegam, e quando na carne de alguma pessoa se
aferram de maneira que não se podem tirar senão cortando"-lhe a cabeça com as unhas; as
quais se mantêm das folhas das árvores e de minhocas, e outros bichinhos que tomam pelo
chão; a estas formigas comem os índios torradas sobre o fogo, e fazem"-lhe muita festa; e
alguns homens brancos que andam entre eles e os mestiços, têm por bom jantar, e o gabam de
saboroso, dizendo que sabem a passas de Alicante; e torradas são brancas por dentro.

Há outras formigas, a que os índios chamam tarasam,\footnote{ Em Varnhagen (1851 e 1879),
``turusã''.} que são ruivas, e têm o corpo tamanho como grão de trigo, e grande boca; as
quais são amigas das caixas, onde roem o fato que está nelas, e o que acham pelo chão; no
qual fazem lavores que parecem feitos a tesoura, e sucedeu muitas vezes terem os
sapateiros o calçado feito, e ficar nas encóspias do chão, onde lhe chegaram de noite, e
quando veio pela manhã as acharam todas lavradas pela banda da frol e a tinham toda
abocanhada.

\paragraph{[122] Que trata de diversas castas de formigas}\quad
Ubiraipu é outra casta de formigas, que se criam nos pés das árvores; são pardas e
pequenas, mas mordem muito; as quais se mantêm das folhas das árvores e da podridão do
côncavo delas.

Há outra casta, a que os índios chamam taricena,\footnote{ Em Varnhagen (1851 e 1879),
``tacicema''.} que se criam nos mangues que estão com a maré cobertos de água até o meio;
as quais são pequenas, e fazem ninho da terra nestas árvores, obrados como favo de mel,
onde criam; a qual terra vão buscar enxuta, quando a maré está vazia; e mantêm"-se dos
olhos dos mangues e de ostrinhas, que se neles criam, e de uns caramujos que se criam nas
folhas destes mangues, e que são da feição e natureza dos caracóis.

Tasibura\footnote{ Em Varnhagen (1851 e 1879), ``tacibura''. Tacibura.} é outra casta de
formigas, que são pequenas de corpo e têm grande cabeça, têm dois cominhos nela; são
pretas e mordem muito, e criam"-se nos paus podres que estão no chão, e mantêm"-se deles e
da umidade que estes paus têm em si.

Tapitangua\footnote{ Em Varnhagen (1851 e 1879), ``tacipitanga''. Tacipitanga.} é outra
casta de formigas pequenas, as quais não mordem, mas não há quem possa defender delas as
coisas doces, nem outras de comer. Estas se criam pelas casas em lugares ocultos, que se
não podem achar, mas como as coisas doces entram em casa, logo lhes dão assalto, com o que
enfadam muito; e são muito certas em casas velhas, que têm as paredes de terra.

Outras formigas chamam os índios tapiahy,\footnote{ Em Varnhagen (1851 e 1879),
``tacîahi''.} que são grandes e pretas, e criam"-se debaixo do chão; também mordem muito,
mas não se afastam muito do seu formigueiro.

\paragraph{[123] Em que se trata que coisa é o copi que há na Bahia, e dos carrapatos}\quad
Copi\footnote{ Cupim.} são uns bichos que são tão prejudiciais como as formigas, os quais
arremedam na feição às formigas, mas são mais curtos, redondos e muito nojentos, e se lhes
tocam com as mãos logo se esborracham, e ficam fedendo a percevejos, e são brancacentos.
Estes bichos se criam nas árvores e na madeira das casas, onde não há quem se defenda
deles; os quais vêm do mato por baixo do chão a entrar nas casas, e trepam pelas paredes
aos forros e em madeiramento delas; e fazem de barro um caminho muito para ver,\footnote{
No manuscrito da \textsc{bgjm}, ``fazem um ninho muito para ver''.} que vai todo coberto
com uma abóbada de barro em volta de berço, coisa sutilíssima e tão delgada a parede dela
como casca de castanha, e servem"-se por dentro por onde sempre caminham, uns para cima e
outros para baixo; e fazem nas partes mais altas das casas seus aposentos, pelas juntas de
madeira em redondo; uns tamanhos como bolas, outros como botijas, e tamanhos como potes; e
se se não tem muito tento nisto, destroem umas casas, e comem"-lhes a madeira, e
apodrentam"-na toda; e o mesmo feito fazem nas árvores, com que as fazem secar; e é
necessário que se alimpem as casas dele, de quando em quando; e quando lhe tiram fora
estes aposentos, estão todos lavrados por dentro como favo de mel, mas têm as casas mais
miúdas, e todas estas cheias deste copi; o qual lançam às galinhas com o que engordam
muito.

Pelas árvores se cria outra casta de copi preto, e do tamanho e feição do gorgulho que na
Espanha se cria no trigo; este morde muito, e é mais ligeiro que o de cima, e faz seus
ninhos pelos ramos das árvores secas; e lavram"-nos todos por dentro.

Há na Bahia muitos carrapatos, dos quais se cria infinidade deles no mato, nas folhas das
árvores, e com o vento caem no chão; e quem anda por baixo destas árvores leva logo seu
quinhão; dos quais nasce grande comichão; mas como se untam com qualquer azeite, logo
morrem. Destes carrapatos se pegam muitos na caça grande, e nas vacas, onde se fazem muito
grandes; mas há uns pássaros de que dissemos atrás, que os matam às alimárias e às vacas,
que os esperam muito bem, e mantêm"-se disto.

Também se criam nas palmeiras uns caracóis do tamanho de oito reais, que são baixos e
enroscada a casca em voltas como a postura de uma cobra, quando está enroscada, os quais
fazem mal aos índios, se comem muitos. Dos caracóis da Espanha se criam muitos nas árvores
e nas ervas.

\paragraph{[124] Que trata das pulgas e piolhos, e dos bichos que se criam nos pés}\quad
Pulgas há poucas no Brasil, a que os índios chamam tungasu,\footnote{ Em Varnhagen (1851 e
1879), ``tungaçu''.} e nenhuns piolhos do corpo entre a gente branca; entre os índios se
criam alguns nas redes em que dormem, como estão sujas, os quais são compridos com feição
de pernas, como os piolhos"-ladros, e fazem grande comichão no corpo.

Para se arrematar esta parte das informações dos bichos prejudiciais, e de nenhum proveito
que se criam na Bahia, convém que se diga que são esses bichos tão temidos em Portugal,
que se metem nos pés da gente, a que os índios chamam tungas, os quais são pretinhos,
pouco maiores que ouções. Criam"-se em casas despovoadas, como as pulgas em Portugal, e em
casas sujas de negros que as não alimpam, e dos brancos que fazem o mesmo, mormente se
estão em terra solta e de muito pó, nos quais lugares estes bichos saltam como pulgas nas
pernas descalças; mas nos pés é a morada a que eles são mais inclinados, mormente junto
das unhas; e como estes bichos entram na carne, logo se sentem como picadas de agulha. Há
alguns que doem ao entrar na carne, e outros fazem comichão como de frieiras; e não andam
nas casas sobradadas, nem nas térreas que andam limpas, nem fazem mal a quem anda calçado;
aos preguiçosos e sujos fazem estes bichos mal, que aos outros homens não; porque em os
sentindo os tiram logo com a ponta de alfinete, como quem tira um oução;\footnote{ Oução:
designação comum, em Portugal, para um tipo de ácaro que vive debaixo da pele causando
coceira, o qual se costumava tirar com a ponta de uma agulha ou de alfinete.} e os que
estão entre as unhas doem muito ao tirar, porque estão metidos pela carne, os quais se
tiram em menos espaço de uma ave"-maria; e de onde saem fica uma covinha, em que põem"-lhe
uns pós de cinza ou nada, e não se sente mais dor nenhuma; mas os preguiçosos e sujos, que
nunca lavam os pés, deixam estar os bichos neles, onde vêm a crescer e fazerem"-se tamanhos
como camarinhas e daquela cor; porque estão por dentro todos cheios de lêndeas e como
arrebentam vão estas lêndeas lavrando os pés, do que se vêm a fazer grandes chagas.

No princípio da povoação do Brasil vieram alguns homens a perder os pés, e outros a
encherem"-se de boubas, o que não acontece agora, porque todos os sabem tirar, e não se
descuidam tanto de si, como faziam os primeiros povoadores.

\subsection{Daqui por diante vão arrumados os peixes que\break se criam no mar da Bahia e nos
rios dela}

Pois queremos manifestar as grandezas da Bahia de Todos os Santos, a fertilidade da terra,
e abastança dos mantimentos, frutos e caça dela, convém que se saiba se tem o mar tão
abundoso de pescado e marisco como tem a terra do muito que se nela cria, como já fica
dito; e porque havemos de satisfazer a esta obrigação, gastando um pedaço em relatar a
diversidade de peixes que este mar e os rios que nele entram criam, comecemos logo no
capítulo seguinte.

\paragraph{[125] Que trata das baleias que se entram no mar da Bahia}\quad
Entendo que cabe a este primeiro capítulo dizermos das baleias que entram na Bahia, como
do maior peixe do mar dela, a que os índios chamam ``pirapoam''\footnote{ Em Varnhagen
(1851 e 1879), ``pirapuã''.} das quais entram na Bahia muitas no mês de maio, que é o
primeiro do inverno naquelas partes, onde andam até o fim de dezembro que se vão; e neste
tempo de inverno, que reina até o mês de agosto, parem as fêmeas abrigadas da terra da
Bahia pela tormenta que faz no mar largo, e trazem aqui os filhos, depois que parem, três
e quatro meses, que eles têm disposição para seguirem as mães pelo mar largo; e neste
tempo tornam as fêmeas a emprenhar, em a qual obra fazem grandes estrondos no mar. E
enquanto estas baleias andam na Bahia, foge o peixe do meio dela para os baixos e
recôncavos, onde elas não podem andar, as quais às vezes pelo irem seguindo dão em seco,
como aconteceu no rio de Pirajá o ano de 1580, que ficaram neste rio duas em seco, macho e
fêmea, as quais foi ver quem quis; e eu mandei medir a fêmea, que estava inteira, e tinha
do rabo até a cabeça setenta e três palmos de comprido, e dezessete de alto, fora o que
tinha metido pela vasa, em que estava assentada; o macho era sem comparação maior, o que
se não pôde medir, por a este tempo estar já despido da carne, que lhe tinham levado para
azeite; a fêmea tinha a boca tamanha que vi estar um negro metido entre um queixo e outro,
cortando com um machado no beiço de baixo com ambas as mãos, sem tocar no beiço de cima; e
a borda do beiço era tão grossa como um barril de seis almudes; e o beiço de baixo saía
para fora mais que o de cima, tanto que se podia arrumar de cada banda dele um quarto de
meação; a qual baleia estava prenhe, e tiraram"-lhe de dentro um filho tamanho como um
barco de trinta palmos de quilha; e se fez em ambas de duas tanto azeite que fartaram a
terra dele dois anos. Quando estas baleias andam na Bahia acompanham"-se em bandos de dez,
doze juntas, e fazem grande temor aos que navegam por ela em barcos, porque andam urrando,
e em saltos, lançando a água mui alta para cima; e já aconteceu por vezes espedaçarem
barcos, em que deram com o rabo, e matarem a gente deles.

\paragraph{[126] Que trata do espadarte e de outro peixe não conhecido que deu à costa}\quad
Entram na Bahia, no tempo das baleias, outros peixes muito grandes, a que os índios chamam
pirapicu, e os portugueses espadartes, os quais têm grandes brigas com as baleias, e fazem
tamanho estrondo quando pelejam, levantando sobre a água tamanho vulto e tanta dela para
cima, que parece de longe um navio a vela; o que se vê de três e quatro léguas de espaço,
e com esta revolta, em que andam, fazem grande espanto ao outro peixe miúdo; com o que
foge para os rios e recôncavos da Bahia.

Aconteceu na Bahia, no verão do ano de 1584, onde chamam Tapoam, vir um grande vulto do
mar fazendo grande marulho de diante, após o peixe miúdo que lhe vinha fugindo para a
terra, até dar em seco; e como vinha com muita força, varou em terra pela praia, de onde
se não pôde tornar ao mar por vazar a maré e lhe faltar a água para nadar; ao que acudiram
os vizinhos daquela comarca a desfazer este peixe, que se desfaz todo, em azeite, como faz
a baleia; o qual tinha trinta e sete palmos de comprido, e não tinha escama, mas couro
muito grosso e gordo como toucinho, de cor verdoenga; o qual peixe era tão alto e grosso
que tolhia a vista do mar, a quem se punha detrás dele; cuja cabeça era grandíssima, e
tinha por natureza um só olho no meio da frontaria do rosto; as espinhas e ossos eram
verdoengos; ao qual peixe não soube ninguém o nome, por não haver entre os índios nem
portugueses quem soubesse dizer que visse nem ouvisse que o mar lançasse outro peixe como
este fora, de que se admiraram muito.

\paragraph{[127] Que trata dos homens marinhos}\quad
Não há dúvida senão que se encontram na Bahia e nos recôncavos dela muitos homens
marinhos, a que os índios chamam pela sua língua upupiara,\footnote{ Upupiara, Ipupiara oi
Igpupiara. Do tupi \textit{îpupi'ara}, aquele que vive nas águas ou homem marinho.} os
quais andam pelo rio de água doce pelo tempo do verão, onde fazem muito dano aos índios
pescadores e mariscadores que andam em jangadas, onde os tomam, e aos que andam pela borda
da água, metidos nela; a uns e outros apanham, e metem"-nos debaixo da água, onde os
afogam; os quais saem à terra com a maré vazia afogados e mordidos na boca, narizes e na
sua natura; e dizem outros índios pescadores que viram tomar estes mortos, que viram sobre
água uma cabeça de homem lançar um braço fora dela e levar o morto; e os que isso viram se
recolheram fugindo à terra assombrados, do que ficaram tão atemorizados que não quiseram
tornar a pescar daí a muitos dias; o que também aconteceu a alguns negros de Guiné; os
quais fantasmas ou homens marinhos mataram por vezes cinco índios meus; e já aconteceu
tomar um monstro destes dois índios pescadores de uma jangada e levarem um, e salvar"-se
outro tão assombrado que esteve para morrer; e alguns morrem disto. E um mestre"-de"-açúcar
do meu engenho afirmou que olhando da janela do engenho que está sobre o rio, e que
gritavam umas negras, uma noite, que estavam lavando umas formas de açúcar, viu um vulto
maior que um homem à borda da água, mas que se lançou logo nela; ao qual mestre"-de"-açúcar
as negras disseram que aquele fantasma vinha para pegar nelas, e que aquele era o homem
marinho, as quais estiveram assombradas muitos dias; e destes acontecimentos acontecem
muitos no verão, que no inverno não falta nunca nenhum negro.

\paragraph{[128] Que trata do peixe"-serra, tubarões, toninhas e lixas}\quad
Araguoagoai\footnote{ Em Varnhagen (1851 e 1879), ``aragoagoay''. Araguaguá.} é chamado
pelos índios o peixe a que os portugueses chamam peixe"-serra; os quais têm o couro e
feição dos tubarões, mas têm no focinho uma espinha de osso muito dura, com dentes de
ambas as bandas mui grandes, uns de meio palmo, e outros de mais, e de menos; segundo o
peixe, é a espinha de seis, sete palmos de comprido, os quais se defendem com elas dos
tubarões e de outros peixes. Estes se tomam com anzóis de cadeia com arpoeiras compridas,
que lhe largam para quebrar a fúria e se vazar do sangue. Este peixe naturalmente é seco,
e fazem"-no em tassalhos\footnote{ Tassalho: grande fatia ou pedaço.} para se secar, que
serve a gente do serviço; e tem tamanhos fígados, que se tomam muitos de cujos
fígados se tiram trinta para quarenta canadas de azeite, que serve para a candeia e para
concertar o breu para os barcos.

Uperu\footnote{ Iperu.} é o peixe a que os portugueses chamam tubarão, que há muita soma
no mar da Bahia; estes comem gente, se lhe chegam a lanço, e andam sempre à caça do peixe
miúdo, aos quais matam com anzóis de cadeia com grandes arpoeiras, como o peixe"-serra; nos
quais acham pegados os peixes romeiros, como nos do mar largo; cuja carne comem os índios,
e em tassalhos secos se gasta com a gente dos engenhos, os quais têm tamanhos fígados que
se tira deles vinte e vinte e quatro canadas de azeite; cujos dentes aproveitam os índios,
que os engastam nas pontas das flechas; e os que os têm são muito estimados deles.

Por tempo de calma aparecem no mar da Bahia toninhas, a que os índios chamam
pojusi,\footnote{ Em Varnhagen (1851 e 1879), ``pojujî''.} dos quais também foge o peixe
miúdo para os recôncavos; mas não se faz conta deles para os matarem em nenhum tempo.

No mar da Bahia se criam muitas lixas\footnote{ Lixa ou cação"-lixa.} maiores que as da
Espanha, que aparecem em certa monção do ano, as quais têm tamanhos fígados que se tira
deles quinze e vinte canadas de azeite; as quais andam ao longo da areia, onde há pouco
fundo, e tomam"-nas com arpéus, o que esperam bem; e secas e escaladas servem para a gente
dos engenhos, e para matalotagem da gente que há de passar o mar.

\paragraph{[129] Que trata da propriedade do peixe"-boi}\quad
Guoaraguoa\footnote{ Em Varnhagen (1851 e 1879), ``goarágoa''. Guarabá ou peixe"-boi.} é o
peixe a que os portugueses chamam boi, que anda na água salgada e nos rios juntos da água
doce, de que eles bebem; e comem de uma erva miúda como milhã que se dá ao longo da água;
o qual peixe tem o corpo tamanho como um novilho de dois anos, e tem dois cotos como
braços, e neles umas mãos sem dedos; não tem pés, mas tem o rabo à feição de peixe, e a
cabeça e focinho como boi; tem o corpo muito maciço, e duas goelas, e uma só tripa; o qual
tem os fígados e bofes e a mais forçura como boi, e tudo muito bom; não tem escama, mas
pele parda e grossa. A estes peixes se mata com arpões muito grandes, atados a grandes
arpoeiras mui fortes, e no cabo delas atado um barril ou outra boia, porque lhe largam com
o arpão a arpoeira, e o arpoador vai em uma jangada seguindo o rasto do barril ou boia,
que o peixe leva atrás de si com muita fúria, até que o peixe se vaza todo de sangue, e se
vem acima da água morto; o qual levam atado à terra ou ao barco, onde o esfolam como
novilho, cuja carne é muito gorda e saborosa; e tem o rabo como toucinho sem ter nele
nenhuma carne magra, o qual derretem como banha de porco, e se desfaz todo em manteiga,
que serve para tudo o para que presta a de porco, e tem muito melhor sabor; a carne deste
peixe em fresco cozida com couves sabe a carne de vaca, e salpresa melhor, e adubada
parece e tem o sabor de carne de porco; e feita em tassalhos posta de fumo faz"-se muito
vermelha, parece e tem o sabor, cozida, de carne de porco muito boa; a qual se faz muito
vermelha e é feita toda em fêveras com sua gordura misturada; e em fresca e salpresa, e de
vinha d'alhos, assada parece lombo de porco, e faz"-lhe vantagem no sabor; as mãos cozidas
deste peixe são como as de porco, mas têm mais que comer; o qual tem os dentes como boi, e
na cabeça entre os miolos tem uma pedra tamanha como um ovo de pata, feita em três peças,
a qual é muito alva e dura como marfim, e tem grandes virtudes contra a dor de pedra. As
fêmeas parem uma só criança, e têm o seu sexo como outra alimária; e os machos têm os
testículos e vergalho como boi; na pele não têm cabelo nem escama.

\paragraph{[130] Que trata dos peixes prezados e grandes}\quad
Beijupira\footnote{ Em Varnhagen (1851 e 1879), ``beijupirá''. Beijupirá ou bijupirá.} é o
mais estimado peixe do Brasil, tamanho e feição do solho,\footnote{ Solho: peixe que habita
as costas atlânticas e desova em rios. Bastante conhecido em Portugal, o solho era
encontrado no Tejo.} e pardo na cor; tem a cabeça grande e gorda como toucinho, cujas
escamas são grandes; quando este peixe é grande, é"-o muito, e tem sabor saborosíssimo; a
sua cabeça é quase maciça, cujos ossos são muito tenros e desfazem"-se na boca em manteiga
todo; as fêmeas têm as ovas amarelas, e cada uma enche um prato grande, as quais são muito
saborosas. Andam estes peixes pelos baixos ao longo da areia, onde esperam bem que os
arpoem; também morrem a linha, mas hão"-lhes ir andando com a linha para comerem a isca, e
assim a vão seguindo até que caem ao anzol, onde não bolem consigo; e porque há poucos
índios que os saibam tomar, morrem poucos.

Tapisa\footnote{ Em Varnhagen (1851 e 1879), ``tapyrsiçá''. Tapireçá ou olho"-de"-boi.} é
outro peixe assim chamado pelos índios, em cuja língua quer dizer ``olho"-de"-boi'', pelo
qual nome o nomeiam os portugueses; este peixe é quase da feição do beijupira, senão
quanto é mais barrigudo, o qual tem também grandes ovas e muito boas; e morrem a linha, e
é muito saboroso, e de grande estima.\footnote{ No manuscrito da \textsc{bgjm}, ``a linha,
e é mui saboroso, tamanho como uma pescada, digo é mui saboroso e de grande estima''.}

Camoropi\footnote{ Em Varnhagen (1851 e 1879), ``camuropi''. Camurupi ou camurupim.} é
outro peixe prezado e saboroso, tamanho como uma pescada muito grande e da mesma feição,
mas cheio de escamas grossas do tamanho da palma da mão, e outras mais pequenas; e cortado
em postas, está arrumado um eito de espinhas grandes, e outro de carne, e no cabo tem
muitas juntas como o sável;\footnote{ Peixe marinho, com apromimadamente sessenta centímetros de
comprimento, encontrado em certas épocas do ano em rios ou estuários, onde se reproduz.}
as fêmeas têm ovas tamanhas que enchem um grande prato cada uma delas; e quando este peixe
é gordo é mui saboroso; o qual morre à linha no verão; e são muitos deles tamanhos que
dois índios não podem com um às costas atado em um pau.

Há outro peixe, a que os índios chamam piraquiroa,\footnote{ Em Varnhagen (1851 e 1879),
``piraquiroâ''.} que são como os corcovados de Portugal, que se tomam a linha, os quais
são muito estimados, porque, como são gordos, são muito saborosos em extremo.

Carapitangua\footnote{ Em Varnhagen (1851 e 1879), ``carapitanga''. Carapitanga.} são uns
peixes que pela língua do gentio querem dizer ``vermelhos'', porque o são na cor; os
grandes são como pargos;\footnote{ Pargo: peixe de águas tropicais e temperadas do
Atlântico e Mediterrâneo, que tem o dorso róseo a avermelhado, e nadadeiras dorsal,
peitoral e caudal rosadas.} e os pequenos como gorazes, mas mais vermelhos uns e outros,
e mais saborosos; os quais morrem em todo o ano; e quando estão gordos não têm preço, e
são mui sadios. Estes peixes morrem a linha em honesto fundo, e ordinariamente em todo o
ano morre muita soma deles, os quais a seu tempo têm ovas grandes e muito gostosas, e
salpreso é estimado.

\paragraph{[131] Que trata das propriedades dos meros, cavalas, pescadas e xaréus}\quad
Cunapu\footnote{ Em Varnhagen (1851 e 1879), ``cunapú''. Canapu.} são uns peixes a que
chamam em Portugal meros, os quais são mui grandes, e muitos morrem tamanhos que lhes
caberia na boca um grande leitão de seis meses; e por façanha se meteu já um negrinho de
três anos dentro da boca de um destes peixes, os quais têm tamanhos fígados como um
carneiro, e sal"-pimentados são muito bons; e têm o bucho tamanho como uma grande cidra, o
qual cozido e recheado dos fígados tem muito bom sabor; o couro deste peixe é tão grosso
como um dedo e muito gordo, o qual se toma com qualquer anzol e linha, sem trabalharem por
se soltarem dele, e no tempo das águas vivas se tomam em umas tapagens de pedras e de
paus, a que os índios chamam camboas, onde morrem muitos, os quais salpresos são muito
bons.

Genoa\footnote{ Em Varnhagen (1851 e 1879), ``cupá''. Cupa.} são uns peixes a que os
portugueses chamam pescadas bicudas, que são pontualmente da feição das das Ilhas
Terceiras, mas muito maiores e mais gostosas, as quais se tomam a linha; e salpresas de um
dia para outro, fazem as postas folhas como as boas pescadas de Lisboa e em extremo são
saborosas.

Guoarapicu\footnote{ Em Varnhagen (1851 e 1879), ``guarapicú''. Guarapicu.} são uns peixes
a que os portugueses chamam cavalas, das quais há muitas que começam a entrar na Bahia no
verão com os nordestes, e recolhem"-se com eles, com a criação que desovaram na Bahia. São
estes peixes maiores que as grandes pescadas, mas de feição e cor dos sáveis; os quais não
comem a isca estando queda, pelo que os pescadores vão andando sempre com as jangadas; e
acodem então à isca, e pegam do anzol, que é grande, por trabalhar muito como se sente
preso. Este peixe é muito saboroso, e quando está gordo sabem as suas ventrechas a sável;
cujo rabo é gordíssimo, e tem grandes ovas, em extremo saborosas; os seus ossos do focinho
se desfazem todos entre os dentes em manteiga; e salpreso este peixe é muito gostoso, e se
faz todo em folhas como pescada, mas é muito avantajado no sabor e levidão.

Chamam os índios guiara\footnote{ Em Varnhagen (1851 e 1879), ``guiará''. Guiará ou
xaréu.} ao que os portugueses chamam xaréu, que é peixe largo, branco, prateado e teso, o
qual quando é gordo é em extremo saboroso; e tem nas pontas das espinhas, nas costas, uns
ossos alvos atonelados, tão grossos no meio como avelãs, mas compridos; o qual peixe morre
a linha e em redes em todo o ano, e além de ser gostoso é muito sadio.

\paragraph{[132] Que se trata dos peixes de couro que há na Bahia}\quad
Panapana\footnote{ Em Varnhagen (1851 e 1879), ``panapanâ''. Panapaná ou panapanã.} é uma
casta de cações que em tudo o parecem, se não quando têm na ponta do focinho uma roda de
meio compasso, de palmo e meio e de dois palmos, o qual peixe tem grandes fígados como
tubarões; e os grandes tomam"-se com anzóis de candeia, e os pequenos a linha ou em redes,
de mistura com o outro peixe; comem"-se os grandes secos, em tassalhos, e os pequenos
frescos, e são muito gostosos e leves, frescos e secos.

Aos cações chamam os índios socori,\footnote{ Sicuri ou cação"-garoupa.} do que há muitos
na Bahia, que se tomam a linha e com redes; e os pequenos são mui leves e saborosos, e uns
e outros não têm na feição nenhuma diferença dos que andam na Espanha.

Há outro peixe, a que os índios chamam guris\footnote{ Em Varnhagen (1851 e 1879),
``curis''. Guri ou bagre marinho.} e os portugueses bagres; têm o couro prateado, sem
escama, tomam"-se à linha; o qual tem a cabeça como enxarroco,\footnote{ Enxarroco: peixe
marinho, cuja cabeça é redonda, áspera e maior que o corpo.} mas muito dura; e tem o
miolo dela duas pedrinhas brancas muito lindas; este peixe se toma em todo o ano, e é
muito leve e gostoso.

Há outra casta de bagres, que têm a mesma feição, mas têm o couro amarelo, a que os índios
chamam urutus, que também morrem em todo o ano à linha, da boca dos rios para dentro até
onde chega a maré, cujas peles se pegam muito nos dedos; e não são tão saborosos como os
bagres brancos.

Chamam os índios às moreias caramuru,\footnote{ Caramuru ou moreia.} das quais há muitas,
mui grandes e muito pintadas como as da Espanha; as quais mordem muito, e têm muitas
espinhas, e são muito gordas e saborosas; não as há senão junto das pedras, onde as tomam
às mãos.

Arraias há na Bahia muitas, as quais chamam os índios jabubira\footnote{ Em Varnhagen
(1851 e 1879), ``jabubirá''.} e são de muitas castas como as de Lisboa; e morrem à linha e
em redes; há umas muito grandes e outras pequenas que são muito saborosas e sadias.

\paragraph{[133] Que trata da natureza das albacoras, bonitos, dourados, corvinas e outros}\quad
Tacupapirema\footnote{ Tacupapirema: corvina.} é um peixe que arremeda as corvinas da
Espanha, o qual morre no verão, da boca dos rios para dentro até onde chega a maré, e tem
uma cor amarelaça em fresco, e tem a carne mole, e salpreso faz"-se em folhas como pescada,
e é muito gostoso. Este peixe tem na cabeça metidas nos miolos duas pedras muito alvas do
tamanho de um vintém, e morre à linha; do que há muito por estes rios.

Bonitos entram também na Bahia no verão muita soma, que morrem à linha; são como os do mar
largo, e têm"-se em pouca estima.

Também entram na Bahia no verão muitas douradas, que são da feição das do mar largo, mas
mais secas; morrem à linha, e não é havido por bom peixe, e têm a espinha verde.

No mesmo tempo entram na Bahia muitas albucoras,\footnote{ Em Varnhagen (1851 e 1879),
``albacora''. Albacora ou atum.} a que os índios chamam caraoata,\footnote{ Em Varnhagen
(1851 e 1879), ``caraoatá''.} que são como as que seguem os navios, mas têm bichos nas
ventrechas que se lhes tiram, que são como os que se criam na carne; o qual peixe é seco e
toma"-se à linha.

Piraçuqua\footnote{ Em Varnhagen (1851 e 1879), ``piracuca''. Piracuca ou garoupa.} chamam
os índios às goroupas, que são como as das Ilhas, mas muito maiores, tomam"-se à linha; têm
o peixe mole, mas em fresco é saboroso e sadio, e seco também.

Camorois\footnote{ Em Varnhagen (1851 e 1879), ``camurîs''. Camuri ou robalo.} são uns
peixes, assim chamados pelos índios, que se parecem com os robalos de Portugal, os quais
são poucas vezes gordos e nenhuns estimados; morrem à linha das bocas dos rios para dentro
até onde chega a maré.

Abroteas\footnote{ Em Varnhagen (1851 e 1879), ``abróteas''. Abrótea.} morrem na Bahia,
que são pontualmente como as das Ilhas Terceiras; pescam"-se onde o fundo seja de pedra; é
peixe mole, mas muito sadio e saboroso.

Há outros peixes na Bahia, a que os índios chamam ubaranas, que se parecem com tainhas; os
quais morrem em todo o ano à linha; têm muitas espinhas farpadas como as do sável, e é
peixe muito saboroso e sadio.

Goaivicoara são uns peixes a que os portugueses chamam roncadores, porque roncam debaixo
da água, dos quais morrem em todo o ano muitos à linha; e é peixe leve e pouco estimado.

Sororoquas\footnote{ Em Varnhagen (1851 e 1879), ``sororocas''. Sororoca.} são outros
peixes da feição e tamanho dos chícharros, que vêm no verão de arribação à Bahia, e após
eles as cavalas, de que dissemos atrás; morrem à linha e são de pouca estima.

Chamam os índios ao peixe"-agulha timuçu,\footnote{ Timucu ou timbucu.} que morrem à linha
no verão; e há alguns de cinco, seis palmos de comprido; são muito gordos e de muitas
espinhas, as quais são muito verdes; e há desta casta muitos peixes pequenos, de que fazem
a isca para as cavalas.

Majucugoara\footnote{ Em Varnhagen (1851 e 1879), ``maracuguara''. Maracuguara ou
peixe"-porco.} é um peixe a que os portugueses chamam porco, porque roncam no mar como
porco; são do tamanho e feição dos sargos, mas muito carnudos e tesos e de bom sabor, e
têm grandes fígados e muito gordos e saborosos, e em todo o ano se toma este peixe à
linha.

Chamam os índios às tartarugas girocoá;\footnote{ Em Varnhagen (1851 e 1879),
``girucoá''.} e tomam"-se muitas na costa brava tamanhas que as suas cascas são do tamanho
de adargas, as quais põem nas areias infinidades de ovos, dos quais se comem somente as
gemas, porque as claras, ainda que estejam no fogo oito dias a cozer ou assar, não se hão
de coalhar nunca; e sempre estão como as dos ovos crus de galinhas.

\paragraph{[134] Em que se contêm diversas castas de peixes que se tomam em redes}\quad
Além dos peixes que morrem nas redes, de que fica dito atrás, se toma nelas o que se
contém neste capítulo, que não morre à linha. E comecemos logo do principal, que são as
tainhas, a que os índios chamam paratis,\footnote{ Parati ou paratibu.} do que há
infinidade delas na Bahia; com as quais secas se mantêm os engenhos, e a gente dos navios
do Reino, de que fazem matalotagem para o mar. Estas tainhas se tomam em redes, porque
andam sempre em cardumes; e andam na Bahia ordinariamente a elas mais de cinquenta redes
de pescar; e são estas tainhas, nem mais nem menos como as da Espanha, mas muito mais
pequenas e gordas,\footnote{ Em Varnhagen (1851 e 1879), ``mais gostosas e gordas''.} das
quais saem logo, em um lanço, três quatro mil tainhas, que também têm boas ovas. E de
noite, com águas vivas, as tomam os índios com umas redinhas de mão, que chamam
pissas,\footnote{ Em Varnhagen (1851 e 1879), ``puçás''. Puçá: rede em forma cônica ou
peneira utizada para apanhar peixes pequenos, crustáceos e camarões.} que vão atadas em
uma vara arcada; e ajuntam"-se muitos índios, e tapam a boca de um esteiro com varas e
ramas, e como a maré está cheia tapam"-lhe a porta; e põem"-lhe as redinhas ao longo da
tapagem, quando a maré vaza, e outros batem no cabo do esteiro, para que se venham todas
abaixo a meter nas redes; e desta maneira carregam uma canoa de tainhas, e de outro peixe
que entra no esteiro.

Há outro peixe que morre nas redes, a que os índios chamam zabucai,\footnote{ Em Varnhagen
(1851 e 1879), ``saviãácoca''. Zabucaí ou peixe"-galo.} e os portugueses galo, o qual é
alvacento, muito delgado e largo, com uma boca pequena, e faz na cabeça uma feição como
crista, e nada de peralto; este peixe é muito saboroso e leve.

Tareira\footnote{ Em Varnhagen (1851 e 1879), ``tareîra''. Tareira ou enxada.} quer dizer
``enxada'', que é o nome que tem outro peixe que morre nas redes, que é quase quadrado,
muito delgado pela banda da barriga e grosso pelo lombo, o qual também nada de peralto, e
é muito saboroso e leve.\footnote{ Não se encontra no manuscrito da \textsc{bgjm} este
parágrafo sobre a tareira.}

Chamam os índios corimas\footnote{ Em Varnhagen (1851 e 1879), ``coirimas''.} a outros
peixes da feição das tainhas, que morrem nas redes e que têm o mesmo sabor, mas são muito
maiores; e quando estão gordos estão cheios de banhas, e são muito gostosos, e têm grandes
ovas; os quais morrem nas enseadas.

Arabori\footnote{ Aravari.} é outro peixe de arribação, da feição das savelhas de Lisboa,
e assim cheias de espinhas; as quais salpresas arremedam às sardinhas de Portugal no
sabor; e tomam"-se em redes.

Carapebas\footnote{ Carapeba ou carapeva.} são uns peixes que morrem nas redes em todo o
ano, que são baixos e largos, do tamanho dos sarguetes, em todo o ano são gordos,
saborosos e leves.

\paragraph{[135] Que trata de algumas castas de peixe medicinal}\quad
Jaguoarasa\footnote{ Em Varnhagen (1851 e 1879), ``jagoaraçá''. Jaguareçá ou jaguriçá.} é
um peixe que morre à linha, tamanho como cachuchos, e tem a cor de peixe"-cabra e feição de
salmonete; tem os fígados vermelhos como lacre; a carne deste peixe é muito tesa, muito
saborosa; e são tão leves que se dão aos doentes.

Tomam"-se na Bahia outros peixes, que são pontualmente na feição, na cor, no sabor os
salmonetes da Espanha, os quais morrem à linha junto das pedras; e são tão leves que se
dão aos doentes.

Pirasaquem\footnote{ Em Varnhagen (1851 e 1879), ``piraçaquem''.} é um peixe da feição dos
safios de Portugal, o qual não tem escama; morre à linha em todo o ano; é peixe saboroso e
muito leve para doentes.

Bodioes\footnote{ Em Varnhagen (1851 e 1879), ``bodiaens''. Bodião.} são uns peixes de
linha, que se dão na costa das Ilhas, dos quais há muitos na Bahia; é peixe mole, mas
muito gostoso e leve.

Atucupa são uns peixes pequenos e largos como choupas, que morrem à linha; e quando é
gordo é muito saboroso. Estes peixes nascem no inverno com água do monte; no céu da boca
têm uns carrapatos, que lhes comem todo o céu da boca; os quais lhes morrem no verão, em
que lhes torna a encourar a chaga que lhes os bichos fazem; este peixe se dá aos doentes.

Guoaibi"-coati\footnote{ Em Varnhagen (1851 e 1879), ``goayibicoati''.} são uns peixes
azulados pequenos, que se tomam à cana, nas pedras, que são em todo o ano muito gordos e
saborosos, e leves para doentes; e outros muitos peixes há, muito medicinais para doentes,
e de muita substância, que por não enfadar não digo deles.

\paragraph{[136] Que trata da natureza de alguns peixes que se criam na lama e andam sempre
no fundo}\quad
Uramasá\footnote{ Em Varnhagen (1851 e 1879), ``uramaçâ''.} é uma casta de peixe da feição
de linguados de Portugal, o qual se toma debaixo da vasa, ou com redes, cujo sabor não é
muito bom; e se cozem ou assam, sem o açoitarem, faz"-se em pedaços.

Nos arrecifes se tomam muitos polvos, que são como os da Espanha, sem nenhuma diferença, a
que os índios chamam caiacanga, os quais não andam nunca em cima da água; e tomam"-se na
baixa"-mar de maré de águas vivas, nas concavidades que têm os arrecifes, onde ficam com
pouca água; e de noite se tomam melhor com fachos de fogo.

Aimore\footnote{ Aimoré.} é um peixe que se cria na vasa dos rios da água salgada, onde se
tomam nas covas da vasa, os quais são da feição e cor dos enxarrocos; e são escorregadios
como eles, e têm a cabeça da mesma maneira; são sobre o mole, mas muito gostosos cozidos e
fritos, e muito leves; as suas ovas são pequenas e gostosas,\footnote{ No manuscrito da
\textsc{bgjm}, ``mas muito gostosos, as suas ovas''.} mas são tão peçonhentas que de
improviso fazem mal a quem as come, e fazem arvoar\footnote{ Arvoar: sentir tontura, ficar
atordoado.} a cabeça, e dor de estômago, e vomitar, e grande fraqueza, mas
passa este mal logo.

Chama o gentio aymirocos\footnote{ Em Varnhagen (1851 e 1879), ``aimoréoçús''.} a outros
peixes, que se criam na vasa dos mesmos rios do salgado, que são da feição dos eirós de
Lisboa, mas mais curtos e assim escorregadios. Estes, quando estão ovados, têm as ovas tão
compridas, que quase lhes chegam à ponta do rabo, e são muito saborosas, e o mesmo peixe;
mas as ovas são peçonhentas, e de improviso se acha mal quem as come, como as dos aimores;
mas o peixe é muito gostoso e sadio.

Bayque\footnote{ Em Varnhagen (1851 e 1879), ``baiacú''. Baiacu.} é um peixe que quer
dizer ``sapo'', da mesma cor e feição, e mui peçonhento, mormente a pele, os fígados e o
fel, ao qual os índios com fome esfolam, e tiram"-lhe o peçonhento fora, e comem"-nos; mas
se lhes derrama o fel, ou lhes fica alguma pele, incha quem o come até rebentar; com os
quais peixes assados os índios matam os ratos, os quais andam sempre no fundo da água.

Piraquiroa\footnote{ Em Varnhagen (1851 e 1879), ``piraquiroâ''.} é um peixe de feição de
um ouriço"-cacheiro, todo cheio de espinhas tamanhas como alfinetes grandes, os quais tem
pegados na pele por duas pontas com que estão arreigados; tomam"-se em redes; os quais
andam sempre ao longo da areia no fundo; a quem os índios esfolam, e comem"-lhes a carne.

Bacupua\footnote{ Em Varnhagen (1851 e 1879), ``bacupuá''. Bacupua ou bagre"-branco.} é um
peixe da feição do enxarroco nos ombros e na cabeça, mas tem a boca muito pequena e
redonda; e é dos ombros para baixo muito estreito, delgado e duro como nervo, e as
barbatanas do rabo são duras e grossas, e na despedida do rabo tem duas pernas como rãs, e
no fim dele duas barbatanas duras como as do rabo; e debaixo, na barriga, tem dois
bracinhos curtos, e neles maneira de dedos; e tem as costas cheias de sarna como ostrinha,
e da cabeça lhe sai um corno de comprimento de um dedo, mas delgado e duro como osso e
muito preto, e o mais é cor vermelhaça; e tem na barriga debaixo das mãos dois buracos.
Este peixe não nada, mas anda sempre pela areia sobre as mãos, onde há pouca água; ao qual
os índios comem esfolado, quando não têm outra coisa.

\paragraph{[137] Que trata da qualidade de alguns peixinhos e dos camarões}\quad
Mirocaya\footnote{ Em Varnhagen (1851 e 1879), ``mirocaia''. Murucaia.} é um peixe, assim
chamado dos índios, da feição de choupinhas, que se tomam à cana nos rios do salgado; são
tesos e de fraco sabor; em cujas bocas se criam no inverno, com as cheias, uns bichos como
minhocas, que lhes morrem no verão.

Piraquiras\footnote{ Piraquiba.} são uns peixinhos como os peixes"-reis de Portugal, e como
as ruivacas de água doce, os quais se tomam na água salgada em camboas, que são umas
cercas de pedra insossa onde se estes peixinhos vêm recolher fugindo do peixe grande, e
ficam com a maré vazia dentro nas poças, onde se enchem balaios deles; e em certo tempo
trazem os índios destes lugares sacos cheios destes peixinhos.

Pequitinins\footnote{ Em Varnhagen (1851 e 1879), ``pequitins''. Piquitinga.} são uns
peixinhos muito pequeninos que se tomam em poças de água, onde ficam com a maré vazia, e
são tamanhos que os índios assam mil juntos, embrulhados em umas folhas debaixo do
borralho, e ficam depois de assados todos pegados, à feição de uma maçaroca.

Carapiaçaba são uns peixinhos que se tomam à cana, os quais são redondos como choupinhas,
e pintados de pardo e amarelo, e são sempre gordos e muito bons para doentes.

E afora estes peixinhos há mil castas de outros de que se não faz menção, por escusar
prolixidade, mas está entendido que onde há tanta diversidade de peixes grandes, haverá
muito mais dos pequenos.

Potipemas chamam os índios aos camarões, que são como os de Vila França, os quais têm as
unhas curtas, as barbas compridas, e são aborrachados na feição; têm a casca branda e são
mui saborosos; criam"-se estes nos esteiros de água salgada, e tomam"-se em redinhas de mão,
nas redes grandes de pescar vêm de mistura com o outro peixe.

\subsection{Lembrança do marisco que se dá na Bahia}

Uma das mais importantes lembranças convenientes à redondeza da Bahia de Todos os Santos é
declarar o muito e diverso marisco que nela se cria, do que convém que digamos agora,
começando no capítulo seguinte.\footnote{ Nas edições de Varnhagen (1851 e 1879), assim
como nas posteriores, não existem o subtítulo ``Lembrança do marisco que se dá na Bahia''
e esse parágrafo introdutório. Do capítulo 137 passa"-se imediatamente ao seguinte.}

\paragraph{[138] Que trata da natureza dos lagostins e uçás}\quad
Aos lagostins chama o gentio potiquequia;\footnote{ Em Varnhagen (1851 e 1879),
``potiquequiâ''.} os quais são da maneira das lagostas, mas mais pequenos alguma coisa e
em tudo o mais têm a mesma feição e feitio; e criam"-se nas concavidades dos arrecifes,
onde se tomam em conjunção das águas vivas muitos; e em seu tempo, que é nas marés da lua
nova, estão melhores que na lua cheia, na qual estão cheios de corais muito grandes as
fêmeas, e os machos muito gordos; e para se tomarem bem estes lagostins, há de ser de
noite, com fachos de fogo.

O marisco mais proveitoso à gente da Bahia são uns caranguejos a que os índios chamam
usás,\footnote{ Em Varnhagen (1851 e 1879), ``ussás''. Uçá é uma espécie de caranguejo que
vive nos mangues.} os quais são grandes e têm muito que comer; e são mui sadios para
mantença dos escravos e gente do serviço; estes caranguejos se criam na vasa, entre os
mangues, de cuja folha se mantêm e têm corais uma só vez no ano; e como desovam pelam a
casca, assim os machos como as fêmeas, e nasce"-lhes outra casca por baixo; e enquanto a
têm mole estão por dentro cheios de leite, e fazem dor de barriga aos que os comem; e
quando as fêmeas estão com corais, os machos estão mui gordos, tanto que parece o seu
casco estar cheio de manteiga; e quando assim estão são mui gostosos, os quais se querem
antes assados que cozidos. Têm estes caranguejos no casco um fel grande, e bucho junto à
boca com que comem, o qual amarga muito, e é necessário tirá"-lo a tento, porque não faça
amargar o mais. Estes usás são infinitos, e faz espanto a quem atenta para isso, e é não
haver quem visse nunca caranguejos desta casta quando são pequenos, que todos aparecem e
saem das covas de lama, onde fazem a sua morada, do tamanho que hão de ser; das quais
covas os tiram os índios mariscadores com o braço nu; e como tiram as fêmeas fora as
tornam logo a largar para que não acabem, e façam criação. Estes caranguejos têm as pernas
grandes, e duas bocas muito maiores com que mordem muito, e nas quais têm tanto que comer
como as das lagostas; e o que se delas come e o mais do caranguejo, é muito gostoso. E não
há morador nas fazendas da Bahia que não mande cada dia um índio mariscar destes
caranguejos; e de cada engenho vão quatro ou cinco destes mariscadores, com os quais dão
de comer a toda a gente de serviço; e não há índios destes que não tome cada dia trezentos
e quatrocentos caranguejos, que trazem vivos em um cesto serrado feito de verga delgada, a
que os índios chamam samura;\footnote{ Samburá: cesto de boca estreita, feito de cipó ou
taquara, utilizado para carregar iscas ou pescado.} e recolhem em cada samura destes um
cento, pouco mais ou menos.

\paragraph{[139] Que trata de diversas castas de caranguejos}\quad
Há outros caranguejos, a que os índios chamam serizes,\footnote{ Siris.} que têm outra
feição mais natural com os caranguejos de Portugal, mas são muito maiores, e têm as duas
bocas muito compridas e grandes, e os braços delas quadrados, no que têm muito que comer.
Estes desovam em cada lua nova, em a qual as fêmeas têm grandes corais vermelhos, e os
machos os têm brancos, e estão muito gordos; os quais uns e outros têm muito que comer, e
em todo o tempo são muito gostosos e sadios; criam"-se na praia de areia dentro na água,
onde os tomam às mãos quando a maré enche, e não têm fel como os usás.

Criam"-se outros caranguejos na água salgada, a que os índios chamam goayas;\footnote{ Em
Varnhagen (1851 e 1879), ``goaiá''. Guaiá.} estes são compridos, e têm as pernas curtas e
pequenas bocas; são muito poucos, mas muito bons.

Aratuns\footnote{ Em Varnhagen (1851 e 1879), ``aratús''. Aratu.} são outros caranguejos
pequenos, como os de Portugal, que se tomam no rio de Sacavém, em Lisboa; criam"-se entre
os mangues, de cuja folha e casca se mantêm e sempre lhes estão roendo nos pés; dos quais
há infinidade, mas têm a casca mole; e em seu tempo, uma vez no ano, têm as fêmeas corais,
e os machos estão muito gordos; e uns e outros são sadios e gostosos.

Há outros caranguejos, a que os índios chamam goaiareru,\footnote{ Em Varnhagen (1851 e
1879), ``goaiarara''.} que se criam nos rios, onde a água doce se mistura com a salgada,
os quais são mui lisos e de cor apavonada, e têm o casco redondo, as pernas curtas, e são
poucos e gostosos.

Guoara"-usa\footnote{ Em Varnhagen (1851 e 1879), ``goaiaussá''.} são outros caranguejos
que se criam dentro da areia que se descobre na vazante da maré, os quais são pequenos e
brancos, e têm as covas mui fundas; e andam sempre pelas praias, enquanto não vem gente, e
como a sentem se metem logo nas covas; e aconteceu já fazer um índio tamanha cova, para
tirar um destes caranguejos, que lhe caiu areia em cima de maneira que não pôde tirar a
cabeça e afogou"-se; no que os índios tomam tanto trabalho, porque lhes serve este
guoaira"-usa de isca, que o peixe come bem; os quais têm a casca muito mole ordinariamente,
e não se comem por pequenos.

\paragraph{[140] Que trata da qualidade das ostras que há na Bahia}\quad
\mbox{As mais} formosas ostras que se viram são as do Brasil; e há infinidade delas, como se vê
na Bahia, onde lhes os índios chamam leri"-u-asu,\footnote{ Em Varnhagen (1851 e 1879),
``lerîuçú''. Leriaçu.} as quais estão sempre cheias, e têm ordinariamente grandes miolos;
e em algumas partes os têm tamanhos que se não podem comer senão cortados em talhadas, as
quais, cruas, assadas e fritas, são muito gostosas; as boas se dão dentro da vasa no
salgado, e pelos rios onde se junta a água doce na salgada se criam muitas na
vasa,\footnote{ Em Varnhagen (1851 e 1879), ``água doce ao salgado''.} e muito grandes,
mas quando há água do monte, estão mui doces e sem sabores. E há tantas ostras na Bahia e
em outras partes, que se carregam barcos delas, muito grandes, para fazerem cal das
cascas, de que se faz muita e muito boa para as obras, a qual é muito alva; e há engenho
que se gastou nas obras dele mais de três mil moios de cal destas ostras; as quais são
muito mais sadias que as da Espanha.

Nos mangues se criam outras ostras pequenas, a que os índios chamam leri"-mirim,\footnote{
Em Varnhagen (1851 e 1879), ``lerîmerim''.} e criam"-se nas raízes e ramos deles até onde
lhes chega a maré de preamar; as quais raízes e ramos estão cobertos destas ostras, que se
não enxerga o pau, e estão umas sobre outras; as quais são pequenas, mas muito gostosas; e
nunca acabam, porque tiradas umas, logo lhes nascem outras; e em todo o tempo são muito
boas e muito leves.

Há outras ostras, a que os índios chamam leripebas, que se criam em baixos de areia de
pouca água, as quais são como as salmoninas que se criam no rio de Lisboa, defronte do
Barreiro, da feição de vieiras. Estas leripebas são um marisco de muito gosto, e estão na
conjunção da lua nova muito cheias, cujo miolo é sobre o teso e muito excelente; em as
quais se acham grãos de aljôfar pequenos, e criam"-se logo serras destas leripebas, umas
sobre as outras, muito grandes; e já aconteceu descer com a maré serra delas até defronte
da cidade, em que a gente dela e do seu limite teve que comer mais de dois anos.

\paragraph{[141] Que trata de outros mariscos que há na Bahia}\quad
Na Bahia se criam outras sortes de marisco miúdo debaixo da areia. Primeiramente,
sarnabis;\footnote{ Em Varnhagen (1851 e 1879), ``sernambis''. Sernambi.} é marisco que se
cria na vasa, que são como as amêijoas grandes de Lisboa; mas têm a casca muito redonda e
grossa, e têm dentro grande miolo de cor pardaça, que se come assado e cozido, mas o
melhor deste marisco é frito, porque se lhe gasta do fogo a muita reima\footnote{ Reima:
secreção aquosa.} que tem, e um cheiro fortum que assado e cozido tem; e de
toda a maneira este marisco é prezado.

Nos baixos de areia que tem a Bahia se cria outro marisco, a que os índios chamam
tarcobas,\footnote{ Tarioba.} que são da feição e tamanho das amêijoas de Lisboa, e têm o
mesmo gosto e sabor, assim cruas como abertas no fogo; as quais se tiram de debaixo da
areia, e têm"-se em casa na água salgada vivas, quinze e vinte dias; as quais, além de
serem maravilhosas no sabor, são muito leves.

Cria"-se na vasa da Bahia infinidade de mexilhões, a que os índios chamam sururus, que são
da mesma feição e tamanho e sabor dos mexilhões de Lisboa, os quais têm caranguejinhos
dentro, e o mais que têm os de Lisboa; e com a minguante da lua estão muito cheios.

Dos berbigões há grande multidão na Bahia, nas praias da areia, a que os índios chamam
saranamitinga,\footnote{ Em Varnhagen (1851 e 1879), ``sarnambitinga''. Sarnambitinga.}
que são da mesma feição dos de Lisboa, mas têm a casca mais grossa, e são mais pequenos;
comem"-se abertos no fogo, e são mui gostosos, e também crus; mas têm um certo sabor, que
requeimam algum tanto na língua.

Nas enseadas da Bahia, na vasa delas, se cria outro marisco, a que os índios chamam
guoarypoapem,\footnote{ Em Varnhagen (1851 e 1879), ``guaripoapem''.} a que os portugueses
chamam longueirões,\footnote{ Em Varnhagen (1851 e 1879), ``portugueses dizem
lingoeirões''. Longueirão é a designação comum de diversos moluscos bivalves de concha
longa e retangular.} os quais são tão compridos como um dedo e mais, da mesma grossura, e
têm um miolo grande e muito gostoso, e se come aberto no fogo; e a casca se abre como a
das amêijoas.

\paragraph{[142] Que trata da diversidade de búzios que se criam na Bahia}\quad
Tapesi\footnote{ Em Varnhagen (1851 e 1879), ``tapuçú''.} são uns búzios tamanhos de palmo
e meio, que têm uma borda estendida para fora no comprimento do búzio, de um coto de
largo, os quais são algum tanto baixos e têm grande miolo; que os índios comem, mas é
muito teso; os quais búzios servem aos índios de buzinas, e criam"-se na areia; e no miolo
têm uma tripa cheia dela, que se lhes tira facilmente.

Há outros búzios, a que os índios chamam oatapesi,\footnote{ Em Varnhagen (1851 e 1879),
``oatapú''.} que são tamanhos como uma grande cidra, e pontiagudos no fundo, e roliços,
com grande boca; estes têm grande miolo bom para comer, e algum tanto teso, o qual tem uma
tripa cheia de areia, que se lhe tira bem. A estes búzios furam os índios pelo pé para
tangerem com eles, e não há barco que não tenha um, nem casa de índios onde não haja três
e quatro, com que tangem, os quais soam muito mais que as buzinas; e criam"-se estes búzios
na areia.

Também se criam na areia outros búzios de três quinas, a que os índios chamam
iatetaoasu,\footnote{ Em Varnhagen (1851 e 1879), ``oapuaçú''.} que são tamanhos como uma
pinha e maiores; e no que a boca abre para fora são mui formosos, cujo miolo é grande e
saboroso, sobre o teso, onde têm uma tripa cheia de areia; também servem de buzina aos
índios.

Piriguoas\footnote{ Em Varnhagen (1851 e 1879), ``ferigoas''.} são outros búzios que se
criam na areia, tamanhos como nozes e maiores; são brancos, cheios de bichos muito bem
afeiçoados, os quais têm um miolo dentro, que cozidos e assados, se lhes tira com a mão
muito bem; e têm uma tripa cheia de areia fácil de se tirar. Este marisco é de muito gosto
e leve, de que há muita soma, e com tormenta lança"-os o mar fora nas enseadas.

Há outros búzios, a que os índios chamam ticoarapoam,\footnote{ Em Varnhagen (1851 e 1879),
``ticoarapuâ''.} tamanhos como um ovo, com um grande bico no fundo, e são muito alvos,
lavrados em caracol por fora; têm miolo grande com tripa, como estes outros, que se lhes
tira; o qual é muito saboroso; que se cria também na areia; do que há muita quantidade.

Sacurauna é outra casta de búzios, que se criam na areia, tamanho como peras pardas, que
são ásperos por fora, e têm grande miolo, mas sobre o duro, e também têm tripa de areia.

Há outros búzios, que se criam na areia, a que os índios chamam oacaré, que são muito
lisos e pintados por fora, os quais têm grande miolo, e sobre o teso. Estes búzios são os
com que as mulheres burnem\footnote{ Burnir: lustrar, polir.} e assentam as costuras.

Ticoerauna são uns búzios pequenos, uns da feição de caramujos pintados por
fora,\footnote{ Em Varnhagen (1851 e 1879), ``búzios pequenos da feição''.} outros
compridos, também pintados, que servem de tentos;\footnote{ Tento: utensílio de barro para
amparar as panelas} os quais se criam nas folhas dos mangues como
caracóis; e cozidos tiram"-se com alfinetes, como caramujos, e são muito bons e saborosos.
Outras muitas castas há destes búzios pequenos, que por atalhar prolixidade se não diz
aqui deles.

\paragraph{[143] Em que se contém algumas estranhezas que o mar cria na Bahia}\quad
Assim como se na terra criam mil imundícias de bichos prejudiciais ao remédio da vida
humana, como atrás no capítulo das alimárias fica declarado, da mesma maneira se criam no
mar, como se verá pelo que neste capítulo se contém.

Pinda\footnote{ Em Varnhagen (1851 e 1879), ``pindá''. Pindá ou ouriço"-do"-mar.} chamam os
índios aos ouriços que se criam no mar da Bahia, que são como os da costa de Portugal, os
quais se criam em pedras; e não usa ninguém deles para se comerem, nem para outra coisa
alguma que aproveite para nada.

Lança este mar fora, muitas vezes, com tormenta, umas estrelas da mesma feição e tamanho
das que lança o mar da Espanha, as quais não servem para nada, a que os índios chamam
tasi.\footnote{ Em Varnhagen (1851 e 1879), ``jacî''. Jaci.}

Também este mar lança fora pelas praias alforrecas ou coroas"-de"-frades, como aquelas que
saem no rio de Lisboa na praia de Belém e em outras partes; e na Bahia saem às vezes
juntas duas e três mil delas, a que os índios chamam musiqui.\footnote{ Em Varnhagen (1851
e 1879), ``muciquî''.}

Muitas vezes se acha pelas praias da Bahia uma coisa preta, mui liada, como fígado de
vaca, com o que se enganaram muitos homens cuidando ser âmbar, e é uma água"-morta, segundo
a opinião dos mareantes.

Também deita o mar por estas praias muitas vezes esponjas, a que os índios chamam
itamambequa,\footnote{ Em Varnhagen (1851 e 1879), ``itamambeca''.} as quais se criam no
fundo do mar, de onde umas saem delgadas e moles, e outras tesas e aperfeiçoadas.

Aos gusanos chamam os índios ubirasoqua,\footnote{ Em Varnhagen (1851 e 1879),
``ubiraçoca''. Ubiraçoca ou gusano.} do qual não é de espantar furar a madeira dos
navios, pois fura as pedras, onde não acha paus, as quais se acham cada hora lavradas
deles e furadas de uma banda e outra; este gusano é um bicho mole e comprido como minhoca,
e da mesma feição; e tem a cabeça e boca dura, o qual se cria em uma casca roliça,
retorcida, alva e dura, como búzio, e com ela faz as obras e dano tão sabidos; e para roer
não lança fora desta casca mais que a boca, com que faz o caminho diante desta sua camisa,
que o corpo do bicho de dentro manda para onde quer; e para este gusano não fazer tanto
dano nas embarcações, permitiu a natureza que o que se cria na água salgada morra entrando
na água doce, e o que se cria na água doce morra na água salgada. Na Bahia houve já muito,
mas já agora não há tanto que faça mal aos navios e outras embarcações.

Nas redes de pescar saem às vezes umas pedras brancas, que fizeram já os homens terem
pensamentos que era coral branco, por se criarem no fundo do mar, soltas, feitas em
casteletes alvíssimos, que são tão delicados, lindos e de tanto artifício, que é coisa
estranha; os quais são muito duros e resplandecentes; e dizem alguns contemplativos que se
criam dos limos do mar, porque se acham alguns muitas vezes enfarinhados de areia
congelada e dura, e eles mui brancos, mas não ainda aperfeiçoados, como coisa que se vai
criando.

\paragraph{[144] Que trata da natureza e feições dos peixes de água doce}\quad
Não menos são de notar os pescados que se criam nos rios de água doce da Bahia, que os que
se criam no mar dela; do que é bem que digamos daqui por diante.

E comecemos dos eirós que há nestes rios, que se criam debaixo das pedras, a que os índios
chamam mosim,\footnote{ Em Varnhagen (1851 e 1879), ``mocim''. Muçum ou muçu é um tipo de
enguia de água doce.} os quais são da feição e sabor das de Portugal.

Tarayras\footnote{ Em Varnhagen (1851 e 1879), ``tareîras''. Tareira.} são peixes tamanhos
como mugens, e maiores; mas são pretos, da cor dos enxarrocos, e têm muitas espinhas; os
quais se tomam à linha, nos rios de água doce; têm boas ovas e nenhuma escama; do que há
grandes pescarias.

Inguyás\footnote{ Em Varnhagen (1851 e 1879), ``juquiás''. Jundiá ou jandiá.} chamam os
índios a outros peixes da feição dos safios da Espanha, mas mais pequenos; os quais se
tomam às mãos, entre as pedras; o qual peixe não tem escama, e é mui saboroso.

Tamoatas\footnote{ Tamuatá ou tambuatá, também chamado de cascudo, em razão da couraça que
o cobre.} são outros peixes destes rios que se não escamam, por terem a casca mui grossa e
dura, e que se lhe tira fora inteira depois de assados ou cozidos; os quais se tomam à
linha; e é peixe miúdo, muito gostosos e sadio.

Piranha quer dizer ``tesoura'', é peixe de rios grandes, e onde o há, é muito; e é da
feição dos sargos, e maior, de cor mui prateada; este peixe é muito gordo e gostoso, e
toma"-se à linha; mas tem tais dentes que corta o anzol cerce;\footnote{ Cerce: rente.}
pelo que os índios não se atrevem a meter na água onde há este peixe, porque remete a eles
muito e morde"-os cruelmente; se lhes alcançam os genitais, levam"-lhos cerce, e o mesmo faz
à caça que atravessa os rios onde este peixe anda.

Queriquo\footnote{ Em Varnhagen (1851 e 1879), ``querico''.} é outro peixe de água doce da
feição das savelhas,\footnote{ Savelha: sardinha.} e tem as mesmas espinhas e muitas, e é
muito estimado e saboroso, o qual peixe se toma à linha.

Cria"-se nestes rios outro peixe, a que os índios chamam oaquari,\footnote{ Uacari ou
acari.} que são do tamanho e feição das choupas de Portugal, mas têm o rabo agudo, a
cabeça metida nos ombros e duas pontas como cornos; e têm a pele grossa, a qual os índios
têm por contrapeçonha para mordeduras de cobras e outros bichos, o qual se toma a cana.

Tomam"-se nestes rios outros peixes, a que os índios chamam piabas, que são pequenos, da
feição dos pachões do rio de Lisboa, o qual é peixe saboroso e de poucas espinhas.

Também se tomam à cana nestes rios outros peixes a que os índios chamam
maturagos,\footnote{ Em Varnhagen (1851 e 1879), ``maturaqué''. Maturaqué ou traíra.} que
são pequenos, largos e muito saborosos.

Há outros peixes nos rios, a que os índios chamam guarara,\footnote{ Em Varnhagen (1851 e
1879), ``goarara''.} que são como ruivacas, e têm a barriga grande, os quais se tomam à
cana.

Acará são outros peixes do rio, tamanhos como bezugos, mas têm o focinho mais comprido,
que é peixe muito saboroso; o qual se toma a cana.

Há outras muitas castas de peixe nos rios de água doce, que para se escrever houvera"-se de
tomar muito de propósito mui largas informações, mas por ora deve de bastar o que está
dito para que possamos dizer de algum marisco que se cria na água doce.

\paragraph{[145] Que trata do marisco que se cria na água doce}\quad
Assim como a natureza criou tanta diversidade de mariscos na água salgada, fez o mesmo nos
rios e lagoas da água doce, como se verá pelos mexilhões que se criam nas pedras destes
rios e no fundo das lagoas, que são da feição e tamanho dos do mar, os quais não são tão
gostosos por serem doces.

Também se criam nas pedras destes rios caramujos maiores que os do mar e compridos, a que
os índios chamam sapicareta.\footnote{ Em Varnhagen (1851 e 1879), ``sapicaretá''.}

No fundo das lagoas, nas lamas delas, se criam amêijoas redondas que têm grande miolo, a
que os índios chamam como as do mar, as quais são, pelo lugar onde nascem, muito insossas.

Mais pelo sertão se criam, nos rios grandes, uns mexilhões de palmo de comprido e quatro
dedos de largo, que são pela banda de dentro da cor e lustro da madrepérola, que servem de
colheres aos índios, os quais têm grandes miolos; por serem de água doce não são mui
gostosos como os do mar.

Também se criam nestes rios muitos e mui diversos camarões, dos quais diremos o que foi
possível chegar à nossa notícia; começando primeiro dos mais gerais, que os índios chamam
potim, que são muitos, do tamanho dos grandes de Lisboa, mas são mais grossos e têm as
barbas curtas, os quais se criam entre as pedras das ribeiras e entre as raízes das
árvores, que vizinham com a água, e em quaisquer ervas que se criam na água; de que os
índios se aproveitam tomando"-os às mãos; e são muito saborosos.

Há outra casta de camarões, a que os índios chamam arataem,\footnote{ Em Varnhagen (1851 e
1879), ``aratuem''.} que são da mesma maneira dos primeiros, mas mais pretos na cor, e
têm a casca mais dura, que se criam e tomam da maneira dos de cima, os quais cozidos são
muito bons.

Nestas ribeiras se criam outros camarões, a que os índios chamam aratare,\footnote{ Em
Varnhagen (1851 e 1879), ``arature''.} que têm pequeno corpo e duas bocas como lacraus e
a cabeça de cada um é tamanha como o corpo, os quais se criam em pedras no côncavo delas,
e da terra das ribeiras, que são muito gostosos e tomam"-se às mãos.

Potiuasu\footnote{ Em Varnhagen (1851 e 1879), ``potiuaçú''. Potiaçu ou
camarão"-d'água"-doce.} são uns camarões que se criam nas cavidades das ribeiras, e têm
tamanho corpo como os lagostins, e o pescoço da mesma maneira; têm a casca nédia e as
pernas curtas, os quais criam corais em certo tempo, e em outro têm o casco gordo como
lagostas, que se também tomam às mãos, e são muito saborosos; e estes e os mais não são
nada carregados.

\paragraph{[146] Em que se declara a natureza dos caranguejos"-do"-mato}\quad
Andei buscando até agora onde agasalhar os caranguejos"-do"-mato, sem lhes achar lugar
cômodo, porque para os arrumar com os caranguejos do mar parecia despropósito, pois se
eles criam na terra, sem verem nem tocarem água do mar; e para os contar com os animais
parece que também não lhes cabia este lugar, pois se parecem com o marisco do mar; e por
não ficarem sem gasalhado nestas lembranças, os aposentei na vizinhança do marisco de
terra, ainda que se não criam na água estes caranguejos, mas em lugares úmidos por todas
as ribeiras.

A estes caranguejos da terra chamam os índios guoanhamus;\footnote{ Guaiamu.} os quais se
criam em várzeas úmidas, não muito longe do mar, mas na vizinhança da água doce, os quais
são muito grandes e azuis, com o casco e pernas mui luzentes; os machos são muito maiores
que as fêmeas, e tamanhos que têm os braços grandes, onde têm as bocas com tamanhos bicos
nelas, e tão compridos e voltados que faz com ele tamanha aparência como faz o dedo
demonstrativo da mão de um homem com o polegar, o que é tão duro como ferro, e onde pegam
com esta boca não largam até os não matarem. Criam"-se estes caranguejos em covas debaixo
da terra, tão fundas que com trabalho se lhe pode chegar com o braço e ombro de um índio
metidos nela, onde os mordem mui valentemente. No mês de fevereiro estão as fêmeas, até
meados de março, todas cheias de coral mui vermelho, e têm tanto no casco como uma
lagosta, o qual e tudo o mais é muito gostoso; tiram"-lhe o fel ou bucho que têm, cheio de
tinta preta muito amargosa; porque se se derrama faz amargar tudo e por onde ele chegou.

No mês de agosto, que é no cabo do inverno, se saem os machos e fêmeas ao sol, com o que
anda a terra coberta deles; em o qual tempo se saem ao sol passeando de uma parte para
outra, e são então bons de tomar; e nesta conjunção andam os machos tão gordos que têm os
cascos cheios de uma amarelidão como gemas de ovos, os quais são mui gostosos à maravilha,
mas são carregados; e para os índios os tirarem das covas sem trabalho, tapam"-nas com um
molho de ervas, com o que eles abafam nas covas, e se vêm para tomar ar, e por não acharem
caminho desimpedido, morrem à boca da cova abafados. Algumas vezes morreram pessoas de
comerem estes caranguejos,\footnote{ Em Varnhagen (1851 e 1879), ``comerem este
guanhamú''.} e dizem os índios que no tempo em que fazem mal comem uma fruta, a que chamam
araticupaná, de que já fizemos menção, a qual é peçonhenta.

\subsection{Daqui por diante se trata da vida e costumes\break do gentio da terra da Bahia}

Já era tempo de dizermos quem foram os povoadores e possuidores desta terra da Bahia, de
que se tem dito tantas maravilhas, e quem são estes Tupinambas tão nomeados, cuja vida e
costumes temos prometido por tantas vezes neste tratado, ao que começamos satisfazer daqui
por diante.

\paragraph{[147] Que trata de quais foram os primeiros povoadores da Bahia}\quad
Os primeiros povoadores que viveram na Bahia de Todos os Santos e sua comarca, segundo as
informações que se têm tomado dos índios muito antigos, foram os Tapuyas, que é uma casta
de gentio muito antigo, de quem diremos ao diante em seu lugar. Estes Tapuyas foram
lançados fora da terra da Bahia e da vizinhança do mar dela por outro gentio seu
contrário, que desceu do sertão, à fama da fartura da terra e mar desta província, que se
chamam Tupinaes, e fizeram guerra um gentio a outro tanto tempo quanto gastou para os
Tupinaes vencerem e desbaratarem aos Tapuyas, e os fazerem despejar a ribeira do mar, e
irem"-se para o sertão, sem poderem tornar a possuir mais esta terra de que eram senhores,
a qual os Tupinaes possuíram e senhorearam muitos anos, tendo guerra ordinariamente pela
banda do sertão com os Tapuyas, primeiros possuidores das faldas do mar; e chegando à
notícia dos Tupinambas a grossura e fertilidade desta terra, se juntaram e vieram de além
do rio de São Francisco, descendo sobre a terra da Bahia que vinham senhoreando, fazendo
guerra aos Tupinaes que a possuíam, destruindo"-lhes suas aldeias e roças, matando aos que
lhe faziam rosto, sem perdoarem a ninguém, até que os lançaram fora das vizinhanças do
mar; os quais se foram para o sertão e despejaram a terra aos Tupinambas, que a ficaram
senhoreando. E estes Tupinaes se foram pôr em frontaria com os Tapuyas, seus contrários,
os quais faziam crua guerra com força, da qual os faziam recuar pela terra adentro, por se
afastarem dos Tupinambas que os apertavam da banda do mar, de que estavam senhores, e
assim foram possuidores desta província da Bahia muitos anos, e tempos fazendo guerra a
seus contrários com muito esforço,\footnote{ Em Varnhagen (1851 e 1879), ``muitos anos,
fazendo guerra''.} até a vinda dos portugueses a ela; dos quais Tupinambas e Tupinaes se
tem tomado esta informação, em cuja memória andam estas histórias de geração em geração.

\paragraph{[148] Em que se declara a proporção e feição dos Tupinambas, e como se dividiram
logo}\quad
Os Tupinambas são homens de meã estatura, de cor muito baça, bem feitos e bem dispostos,
muito alegres do rosto, e bem assombrados; todos têm bons dentes, alvos, miúdos, sem lhes
nunca apodrecerem; têm as pernas bem feitas, os pés pequenos; trazem o cabelo da cabeça
sempre aparado; em todas as outras partes do corpo os não consentem e os arrancam como
lhes nascem; são homens de grandes forças e de muito trabalho; são muito belicosos, e em
sua maneira esforçados, e para muito, ainda que atraiçoados; são muito amigos de
novidades, e demasiadamente luxuriosos, e grandes caçadores e pescadores, e amigos de
lavouras.

Como se este gentio viu senhor da terra da Bahia, dividiu"-se em bandos por certas
diferenças que tiveram uns com os outros, e assentaram suas aldeias apartadas, com o que
se inimizaram; os que se aposentaram entre o rio de São Francisco e o rio Real, se
declararam por inimigos dos que se aposentaram do rio Real até a Bahia, e faziam"-se cada
dia cruel guerra, e comiam"-se uns aos outros; e os que cativavam, e a que davam vida,
ficavam escravos dos vencedores.

E os moradores da Bahia da banda da cidade se declararam por inimigos dos outros
Tupinambas moradores da outra banda da Bahia, no limite do rio de Paraguosu\footnote{ Em
Varnhagen (1851 e 1879), ``Paraguassú''.} e do de Seregipe, e faziam"-se cruel guerra uns
aos outros por mar; onde se davam batalhas navais em canoas, com as quais faziam ciladas
uns aos outros, por entre as ilhas, onde havia grande mortandade de parte a parte, e se
comiam, e faziam escravos uns aos outros, no que continuaram até o tempo dos
portugueses.\footnote{ No manuscrito da \textsc{bgjm}, ``cruel guerra uns aos outros por
entre as ilhas onde havia grande mortandade''.}

\paragraph{[149] Que trata de como se dividiram os Tupinambas, e se passaram à ilha de
Taparica e dela a Jaguoaripe}\quad
Entre os Tupinambas moradores da banda da cidade armaram desavenças uns com os outros
sobre uma moça que um tomou a seu pai por força, sem lha querer tornar; com a qual
desavença se apartou toda a parentela do pai da moça, que eram índios principais, com a
gente de suas aldeias, e passaram"-se à ilha de Taparica, que está no meio da Bahia, com os
quais se lançou outra muita gente, e incorporaram"-se com os vizinhos do rio Paraguoasu, e
fizeram guerra aos da cidade, a cujo limite chamavam Caramure; e salteavam"-se uns aos
outros cada dia, e ainda hoje em dia há memória de uma ilheta, que se chama a do Medo, por
se esconderem detrás dela; onde faziam ciladas uns aos outros com canoas, em que se
matavam cada dia muitos deles,

Destes Tupinambas que se passaram à ilha de Taparica, se povoou o rio Jaguoaripe, Tinhare
e a costa dos Ilheos; e tamanho ódio se criou entre esta gente, sendo toda uma por sua
avoenga,\footnote{ Avoengo: avô; antepassado.} que ainda hoje em dia, entre esses poucos
que há,\footnote{ Em Varnhagen (1851 e 1879), ``avoenga, que ainda hoje, entre''.} se
querem tamanho mal que se matam uns aos outros, se o podem fazer, em tanto que se
encontram alguma sepultura antiga dos contrários, lhe desenterram a caveira, e lha
quebram, com o que tomam nome novo, e de novo se tornam a inimizar.

E em tempo que os portugueses tinham já povoado este rio de Jaguoaripe, houve na sua
povoação grandes ajuntamentos das aldeias dos índios ali vizinhos, para quebrarem caveiras
em terreiros, com grandes festas, para os quebradores das cabeças tomarem novos nomes, as
quais caveiras foram desenterrar a uma aldeia despovoada para vingança de morte dos pais
ou parentes dos quebradores delas, para o que as enfeitavam com penas de pássaros ao seu
modo; em as quais festas houve grandes bebedices, o que ordenaram os portugueses ali
moradores para se escandalizarem os parentes dos defuntos, e se quererem de novo mal;
porque se temiam que se viessem a confederar uns com os outros para lhes virem fazer
guerra, o que foi bastante para não o fazerem, e se assegurarem com isto os portugueses
que viviam neste rio.

\paragraph{[150] Em que se declara o modo e a linguagem dos Tupinambas}\quad
Ainda que os Tupinambas se dividiram em bandos, e se inimizaram uns com outros, todos
falam uma língua que é quase geral pela costa do Brasil, e todos têm uns costumes em seu
modo de viver e gentilidades; os quais não adoram nenhuma coisa, nem têm nenhum
conhecimento da verdade, nem sabem mais senão que há morrer e viver; e qualquer coisa que
lhes digam, se lhes mete na cabeça, e são mais bárbaros que quantas criaturas Deus criou.
Têm muita graça quando falam, mormente as mulheres; são mui compendiosas na forma da
linguagem, e muito copiosos no seu orar; mas faltam"-lhes três letras das do \textsc{abc},
que são \textsc{f, l, r} grande ou dobrado, coisa muito para se notar; porque, se não têm
\textsc{f}, é porque não têm fé em nenhuma coisa que adorem; nem os nascidos entre os
cristãos e doutrinados pelos padres da Companhia têm fé em Deus Nosso Senhor, nem têm
verdade, nem lealdade a nenhuma pessoa que lhes faça bem. E se não têm \textsc{l} na sua
pronunciação, é porque não têm lei alguma que guardar, nem preceitos para se governarem; e
cada um faz lei a seu modo, e ao som da sua vontade; sem haver entre eles leis com que se
governem, nem têm leis uns com os outros. E se não têm esta letra \textsc{r} na sua
pronunciação, é porque não têm rei que os reja, e a quem obedeçam, nem obedecem a ninguém,
nem o pai ao filho, nem o filho ao pai, e cada um vive ao som da sua vontade; para dizerem
Francisco dizem Pancisco,\footnote{ Em Varnhagen (1851 e 1879), ``Pancico''.} para dizerem
Lourenço dizem Rorenço, para dizerem Rodrigo dizem Rorigo;\footnote{ Em Varnhagen (1851 e
1879), ``Rodigo''.} e por este modo pronunciam todos os vocábulos em que entram estas
três letras.

\paragraph{[151] Que trata do sítio e arrumação das aldeias, e as quantidades dos principais
delas}\quad
Em cada aldeia dos Tupinambas há um principal, a que seguem somente na guerra onde lhe dão
alguma obediência, pela confiança que têm em seu esforço e experiência, que no tempo de
paz cada um faz o que obriga seu apetite. Este principal há de ser valente homem para o
conhecerem por tal, e aparentado e benquisto, para ter quem ajude a fazer suas roças, mas
quando as faz com ajuda de seus parentes e chegados, ele lança primeiro mão do serviço que
todos. Quando este principal assenta a sua aldeia, busca sempre um sítio alto e desabafado
aos ventos, para que lhe lave as casas, e que tenha a água muito perto, e que a terra
tenha disposição para derredor da aldeia fazerem suas roças e granjearias; e como escolhe
o sítio a contentamento dos mais antigos, faz o principal sua casa muito comprida, coberta
de palma, a que os índios chamam pindoba, e as outras casas da aldeia se fazem também
muito compridas e arrumadas, de maneira que lhes fica no meio um terreiro quadrado, onde
fazem bailes e os seus ajuntamentos; e em cada aldeia há um cabeça, que há de ser índio
antigo e aparentado, para lhe os outros que vivem nestas casas terem respeito; e não vivem
mais nesta aldeia, que enquanto lhes não apodrece a palma das casas, que lhes dura três,
quatro anos. E como lhes chove muito nelas, passam a aldeia para outra parte. E nestas
casas não há nenhuns repartimentos, mais que os tirantes; que entre um e outro é um rancho
onde se agasalha cada parentela, e o principal toma o seu rancho primeiro, onde se ele
arruma com sua mulher e filhos, mancebas, criados solteiros, e algumas velhas que o
servem, e pela mesma ordem vai arrumando a gente da sua casa, cada parentela em seu lanço;
de onde se não poderão mudar, salvo se for algum mancebo solteiro, e casar, porque em tal
caso se irá para o lanço onde está sua mulher; e por cima destes tirantes das casas lançam
umas varas arrumadas bem juntas, a que chamam juraus,\footnote{ Em Varnhagen (1851 e 1879),
``juráos''. Jirau é uma armação ou estrado de madeira.} em que guardam suas alfaias e
seus legumes, que se aqui curam ao fumo, para não apodrecerem; e da mesma maneira se
arrumam e ordenam nas outras casas; e em umas e outras a gente que se agasalha em cada
lanço destes. Quando comem é no chão, em cócaras,\footnote{ No manuscrito da
\textsc{bgjm}, ``em covas''.} e todos juntos, e os principais deitados nas redes. Nestas
casas tem este gentio ajuntamento, sem se pegarem uns dos outros, mas sempre o macho com
fêmea. Se estas aldeias estão em frontaria de seus contrários, e em lugares de guerra, faz
este gentio de roda da aldeia uma cerca de pau a pique muito forte, com suas portas e
seteiras, e afastado da cerca vinte e trinta palmos fazem derredor dela uma rede de
madeira, com suas entradas de fora para entre ela e a cerca, para que, se lhe os
contrários entrarem dentro, lhe saírem, e ao recolher se embaracem de maneira que os
possam flechar e desbaratar, como acontece muitas vezes.\footnote{ No manuscrito da
\textsc{bgjm} o período termina em ``os possam flechar''.}

\paragraph{[152] Que trata da maneira dos casamentos dos Tupinambas e seus amores}\quad
A mulher verdadeira dos Tupinambas é a primeira que o homem teve e conversou, e não têm em
seus casamentos outra cerimônia mais que dar o pai a filha a seu genro, e como têm
ajuntamento natural, ficam casados; e os índios principais têm mais de uma mulher, e o que
mais mulheres tem, se tem por mais honrado e estimado; mas elas dão todas a obediência à
mais antiga, e todas a servem, a qual tem armado sua rede junto da do marido, e entre uma
e outra tem sempre fogo aceso; e as outras mulheres têm as suas redes, em que dormem, mais
afastadas, e fogo entre cada duas redes; e quando o marido se quer ajuntar com qualquer
delas, vai"-se lançar com ela na rede, onde se detém só aquele espaço deste contentamento,
e torna"-se para o seu lugar; e sempre há entre estas mulheres ciúmes, mormente a mulher
primeira; porque pela maior parte são mais velhas que as outras, e de menos gentileza, o
qual ajuntamento é público diante de todos. E quando o principal não é o maior da aldeia
dos índios das outras casas, o que tem mais filhos é mais rico e mais estimado, e mais
honrado de todos, porque são as filhas mui requestadas dos mancebos que as namoram; os
quais servem os pais das damas dois e três anos primeiro que lhas deem por mulheres; e não
as dão senão aos que melhor os servem, a quem os namoradores fazem a roça, e vão pescar e
caçar para os sogros que desejam de ter, e lhe trazem a lenha do mato; e como os sogros
lhes entregam as damas, eles se vão agasalhar no lanço dos sogros com as mulheres, e
apartam"-se dos pais, mães e irmãos, e mais parentela com que antes estavam; e por nenhum
caso se entrega a dama a seu marido enquanto lhe não vem seu costume; e como lhe vem é
obrigada a moça a trazer atado pela cinta um fio de algodão, e em cada bucho dos braços
outro, para que venha à notícia de todos. E como o marido lhe leva a frol, é obrigada a
noiva a quebrar estes fios, para que seja notório que é feita dona; e ainda que uma moça
destas seja deflorada por quem não seja seu marido, ainda que seja em segredo, há de
romper os fios da sua virgindade, que de outra maneira cuidará que a leva logo o diabo, os
quais desastres lhes acontecem muitas vezes; mas o pai não se enoja por isso, porque não
falta quem lha peça por mulher com essa falta; e se algum principal da aldeia pede a outro
índio a filha por mulher, o pai lha dá sendo menina; e aqui se não estende o preceito
acima, porque ele a leva para o seu lanço, e a vai criando até que lhe venha seu costume,
e antes disso por nenhum caso lhe toca.

\paragraph{[153] Que trata dos afeites deste gentio}\quad
Costumam os mancebos Tupinambas se depenarem os cabelos de todo o corpo, e não deixar
mais que os da cabeça, que trazem tosquiados de muitas feições, o que faziam antes que
tivessem tesouras, com umas canas, que por natureza cortam muito; e alguns o trazem
cortado por cima das orelhas, e muito bem aparado; os quais cobrem os membros genitais com
alguma coisa por galantaria, e não pelo cobrir; e pintam"-se de lavores pretos, que fazem
com tinta de jenipapo, e se têm damas, elas têm cuidado de os pintar; também trazem na
cabeça umas penas amarelas, pegadas pelos pés com cera, e arrecadas de ossos nas orelhas,
e grandes contas brancas, que fazem de búzios, lançadas ao pescoço; aos quais as mesmas
damas rapam a testa com umas caninhas, e lhes arrancam os cabelos da barba, pestanas,
sobrancelhas, e os mais cabelos de todo corpo, como já fica dito. E quando se estes
mancebos querem fazer bizarros, arrepiam o cabelo para cima com almécega, onde lhe pegam
umas peninhas amarelas pegadas nele, e sobraçam contas brancas. E põem nas pernas e nos
braços umas manilhas de penas amarelas, e seu diadema das mesmas penas na cabeça. As moças
também se pintam de tinta de jenipapo, com muitos lavores, a seu modo, mui louçãs; e põem
grandes ramais de contas de toda a sorte ao pescoço e nos braços; e põem nas pernas, por
baixo do joelho, umas tapicuras,\footnote{ Em Varnhagen (1851 e 1879), ``tapacurás''.} que
são do fio do algodão, tinto de vermelho, tecido de maneira que lhas não podem tirar, o
que tem três dedos de largo; o que lhes põem as mães enquanto são cachopas, para que lhes
engrossem as pernas pelas barrigas, enquanto crescem, as quais as trazem nas pernas
enquanto são namoradas, mas de maneira que as possam tirar, ainda que com trabalho; e
enquanto são solteiras pintam"-nas as mães, e depois de casadas os maridos, se lhes querem
bem; as quais moças são barbeadas de todos os cabelos que os mancebos tiram, por outras
mulheres. Estas índias também curam os cabelos para que sejam compridos, grossos e pretos,
os quais para terem isto os untam muitas vezes com óleo de cocos bravos.

\paragraph{[154] Que trata da criação que os Tupinambas dão aos filhos e o que fazem quando
lhes nascem}\quad
Quando estas índias entram em dores de parir, não buscam parteiras, não se guardam do ar,
nem fazem outras cerimônias, parem pelos campos e em qualquer outra parte como uma
alimária; e em acabando de parir, se vão ao rio ou fonte, onde se lavam, e as crianças que
pariram; e vêm"-se para casa, onde o marido se deita logo na rede, onde está muito coberto,
até que seca o umbigo da criança; em o qual visitam seus parentes e amigos, e lhes trazem
presentes de comer e beber, e a mulher lhe faz muitos mimos, enquanto o marido está assim
parido, o qual está muito empanado para que lhe não dê o ar; e dizem que se lhe der o ar
que fará muito nojo à criança, e que se se erguerem e forem ao trabalho que lhes morrerão
os filhos, e eles que serão doentes da barriga; e não há quem lhes tire da cabeça que da
parte da mãe não há perigo, senão da sua; porque o filho lhe saiu dos lombos, e que elas
não põem da sua parte mais que terem guardada a semente no ventre onde se cria a criança.

Como nascem os filhos aos Tupinambas, logo lhes põem o nome que lhe parece; os quais nomes
que usam entre si são de alimárias, peixes, aves, árvores, mantimentos, peças de armas e
doutras coisas diversas; aos quais furam logo o beiço debaixo, onde lhes põem, depois que
são maiores, pedras por gentileza.

Não dão os Tupinambas a seus filhos nenhum castigo, nem os doutrinam, nem os repreendem
por coisa que façam; aos machos ensinam"-nos a atirar com arcos e flechas ao alvo, e depois
aos pássaros; e trazem"-nos sempre às costas até a idade de sete e oito anos, e o mesmo às
fêmeas; e uns e outros mamam na mãe até que torna a parir outra vez; pelo que mamam muitas
vezes seis e sete anos; às fêmeas ensinam as mães a enfeitar"-se, como fazem as
portuguesas, e a fiar algodão, e a fazer o mais serviço de suas casas conforme o seu costume.


\paragraph{[155] Em que se declara o com que se os Tupinambas fazem bizarros}\quad
Para se os Tupinambas fazerem bizarros usam de muitas bestialidades mui estranhas, como é
fazerem depois de homens três e quatro buracos nos beiços de baixo, onde metem pedras com
grandes pontas para fora; e outros furam os beiços de cima, também como os de baixo, onde
também metem pedras como nos de baixo; também alguns furam as ventas, em que metem outras
pedras com pontas para fora; e outros furam as faces, onde metem umas pedras redondas,
verdes e pardas, que ficam inseridas nas faces, como espelhos de borracha;\footnote{ Em
Varnhagen (1851 e 1879), ``e outros furam os beiços de cima, também como os de baixo, onde
também metem pedras redondas, verdes e pardas, que ficam''.} em as quais há alguns que têm
nas faces dois e três buracos, em que metem pedras com pontas para fora; e há alguns que
têm todos estes buracos, que, com as pedras neles, parecem os demônios; os quais sofrem
estas dores por parecerem temerosos a seus contrários.

Usam também entre si umas carapuças de penas amarelas e vermelhas, que põem na cabeça, que
lha cobre até as orelhas; os quais fazem colares para o pescoço de dentes dos contrários,
onde trazem logo juntos dois, três mil dentes, e nos pés uns cascavéis de certas ervas da
feição da castanha, cujo tinido se ouve muito longe. Ornam"-se mais estes índios, para suas
bizarrices, de uma roda de penas de ema, que atam sobre as ancas, que lhes faz tamanho
vulto que lhes cobre as costas todas de alto a baixo; e para se fazerem mais feios se
tingem todos de jenipapo, que parecem negros da Guiné, e tingem os pés de uma tinta
vermelha muito fina, e as faces; e põem sobraçadas muitas contas de búzios, e outras
pequenas de penas nos braços; e quando se ataviam com todas estas peças, levam uma espada
de pau marchetada com cascas de ovos de pássaros de cores diversas, e na empunhadura umas
penas grandes de pássaros, e certas campainhas de penas amarelas, a qual espada lançam,
atada ao pescoço, por detrás; e levam na mão esquerda seu arco e flechas, com dentes de
tubarão; e na direita um maracá, que é um cabaço cheio de pedrinhas, com seu cabo, com que
vai tangendo e cantando; e fazem estas bizarrices para quando na sua aldeia há grandes
vinhos, ou em outra, onde vão folgar; pelas quais andam cantando e tangendo sós, e depois
misturados com outros; com os quais atavios se fazem temidos e estimados.

\paragraph{[156] Que trata da luxúria destes bárbaros}\quad
São os Tupinambas tão luxuriosos que não há pecado de luxúria que não cometam; os quais
sendo de muito pouca idade têm conta com mulheres, e bem mulheres; porque as velhas, já
desestimadas dos que são homens, granjeiam estes meninos, fazendo"-lhes mimos e regalos, e
ensinam"-lhes a fazer o que eles não sabem, e não os deixam de dia, nem de noite. É este
gentio tão luxurioso que poucas vezes têm respeito às irmãs e tias, e, porque este pecado
é contra seus costumes, dormem com elas pelos matos, e alguns com suas próprias filhas; e
não se contentam com uma mulher, mas têm muitas, como já fica dito, pelo que morrem muitos
de esfalfados. E em conversação não sabem falar senão nestas sujidades, que cometem cada
hora; os quais são tão amigos da carne que se não contentam, para seguirem seus apetites,
com o membro genital como a natureza formou; mas há muitos que lhe costumam pôr o pelo de
um bicho tão peçonhento, que lho faz logo inchar, com o que têm grandes dores, mais de
seis meses, que se lhe vão gastando espaço de tempo; com o que se lhes faz o seu cano tão
disforme de grosso, que os não podem as mulheres esperar, nem sofrer; e não contentes
estes selvagens de andarem tão encarniçados neste pecado, naturalmente cometido, são muito
afeiçoados ao pecado nefando, entre os quais se não têm por afronta; e o que se serve de
macho, se tem por valente, e contam esta bestialidade por proeza; e nas suas aldeias pelo
sertão há alguns que têm tenda pública a quantos os querem como mulheres públicas.

Como os pais e as mães veem os filhos com meneios para conhecer mulher, eles lhas buscam,
e os ensinam como a saberão servir; as fêmeas muito meninas esperam o macho, mormente as
que vivem entre os portugueses. Os machos destes Tupinambas não são ciosos; e ainda que
achem outrem com as mulheres, não matam a ninguém por isso, e quando muito espancam as
mulheres pelo caso. E as que querem bem aos maridos, pelos contentarem, buscam"-lhes moças
com que eles se desenfadem, as quais lhes levam à rede onde dormem, onde lhes pedem muito
que se queiram deitar com os maridos, e as peitam para isso; coisa que não faz nenhuma
nação de gente, senão estes bárbaros.

\paragraph{[157] Que trata das cerimônias que usam os Tupinambas nos seus parentescos}\quad
Costumam os Tupinambas que quando algum morre que é casado, é obrigado o irmão mais velho
a casar com sua mulher, e quando não tem irmão, o parente mais chegado pela parte
masculina; e o irmão da viúva é obrigado a casar com sua filha se a tem; e quando a mãe da
moça não tem irmão, pertence"-lhe por marido o parente mais chegado da parte de sua mãe; e
se não quer casar com esta sua sobrinha, não tolherá a ninguém dormir com ela, e depois
lhe dá o marido que lhe vem à vontade.

O tio, irmão do pai da moça, não casa com a sobrinha, nem lhe toca quando fazem o que
devem, mas tem"-na em lugar de filha, e ela como a pai lhe obedece, depois da morte do pai,
e pai lhe chama; e quando estas moças não têm tio, irmão de seu pai, tomam em seu lugar o
parente mais chegado; e a todos os parentes da parte do pai em todo o grau chamam pai, e
eles a ela filha; mas ela obedece ao mais chegado parente, sempre; e da mesma maneira
chamam os netos ao irmão e primo de seu avô e eles a eles netos, e aos filhos dos netos, e
netas de seus irmãos e primos; e da parte da mãe também os irmãos e primos dela chamam aos
sobrinhos filhos, e eles aos tios pais; mas não lhes têm tamanho acatamento como aos tios
da parte do pai; e preza"-se muito este gentio de seus parentes e o que mais parentes e
parentas tem, é mais honrado e temido, e trabalha muito pelos chegar para si, e fazer
corpo com eles em qualquer parte em que vivem; e quando qualquer índio aparentado tem
agasalhado seus parentes em sua casa e lanço, quando há de comer, deita"-se na sua rede
onde lhe põem o que há de comer em uma vasilha, e assentam"-se em cócaras, suas mulheres e
filhos, e todos seus parentes, grandes e pequenos; e todos comem juntos do que tem na
vasilha, que está no meio de todos.

\paragraph{[158] Que trata do modo de comer e do beber dos Tupinambas}\quad
Já fica dito como os principais dos Tupinambas quando comem, estão deitados na rede, e
comem com eles os parentes,\footnote{ Em Varnhagen (1851 e 1879), ``e como comem com
eles''.} e os agasalha consigo; entre os quais comem também os seus criados e escravos,
sem lhe terem nenhum respeito; antes quando o peixe ou carne não é que sobeje, o principal
o reparte por quinhões iguais, e muitas vezes fica ele sem nada, os quais estão todos em
cócaras, com a vasilha em que comem todos no chão no meio deles, e enquanto comem não
bebem vinho, nem água, o que fazem depois de comer.

Quando os Tupinambas comem à noite, é no chão como está dito, e virados com as costas para
o fogo, e ficam todos às escuras; e não praticam em coisa alguma quando comem, senão
depois de comer; e quando têm quê, toda a noite não fazem outra coisa, até que os vence o
sono; e por outra parte mantém"-se este gentio com nada, e anda logo dois e três dias sem
comer, pelo que os que são escravos dão pouco trabalho a seus senhores pelo mantimento,
antes eles mantêm os senhores fazendo"-lhes suas roças, e caçando, e pescando"-lhes
ordinariamente.

Este gentio não come carne de porco, dos que se criam em casa, senão os escravos criados
entre os brancos; mas comem a carne dos porcos"-do"-mato e da água; os quais também não
comem azeite, senão os ladinos; toda a caça que este gentio come, não a esfola, e
chamuscam"-na toda ou pelam"-na na água quente, a qual comem assada ou cozida e as tripas
mal lavadas; ao peixe não escamam nem lhe tiram as tripas, e assim como vem do mar ou dos
rios, assim o cozem ou assam: o sal de que usam, com que temperam o seu comer, e em que
molham o peixe e carne fazem"-no da água salgada que cozem tanto em uma vasilha sobre o
fogo, até que se coalha e endurece, com o que se remedeiam; mas é sobre o preto, e
requeima.

Este gentio é muito amigo de vinho, assim machos como fêmeas, o qual fazem de todos os
seus legumes, até da farinha que comem; mas o seu vinho principal é de uma raiz a que
chamam aipim, que se coze, e depois pisam"-na e tornam"-na a cozer, e como é bem cozida,
buscam as mais formosas moças da aldeia para espremer estes aipins com as mãos, e algum
mastigado com a boca, e depois espremido na vasilha, que é o que dizem que lhe põem a
virtude, segundo a sua gentilidade; a esta água e sumo destas raízes lançam em grandes
potes, que para isso têm, onde se este vinho coze, e está até que se faz azedo; e como o
está bem, o bebem com grandes cantares, e cantam e bailam toda uma noite às vésperas do
vinho, e ao outro dia pela manhã começam a beber, bailar e cantar; e as moças solteiras da
casa andam dando o vinho em uns meios cabaços, a que chamam cuias, aos que andam cantando,
os quais não comem nada enquanto bebem, o que fazem de maneira que vêm a cair de bêbados
por esse chão; e o que faz mais desatinos nestas bebedices é o mais estimado dos outros,
em os quais se fazem sempre brigas; porque aqui se lembram de seus ciúmes, e castigam por
isso as mulheres, ao que acodem os amigos, e jogam às tiçoadas uns com os outros.

São costumados a almoçar primeiro que se vão às suas roças a trabalhar, onde não comem
enquanto andam no trabalho, senão depois que se vão para casa.\footnote{ Em Varnhagen
(1851 e 1879), ``que se vem para casa''.}

\paragraph{[159] Em que se declara o modo da granjearia dos Tupinambas e suas habilidades}\quad
Quando os Tupinambas vão às suas roças, não trabalham senão das sete horas da manhã até ao
meio"-dia, e os muito diligentes até horas de véspera; e não comem neste tempo senão depois
destas horas, que se vêm para suas casas; os machos costumam a roçar os matos, e os
queimam e alimpam a terra deles; e as fêmeas plantam o mantimento e o alimpam; os machos
vão buscar a lenha com que se aquentam e se servem, porque não dormem sem fogo, ao longo
das redes, que é a sua cama; as fêmeas vão buscar a água à fonte e fazem de comer; e os
machos costumam ir lavar as redes aos rios, quando estão sujas.

Não fazem os Tupinambas entre si outras obras"-primas que balaios de folhas da palma, e
outras vasilhas da mesma folha ao seu modo e do seu uso; fazem arcos e flechas, e alguns
empalhados e lavrados de branco e preto, feitio de muito artifício; fazem cestos de umas
varas que chamam cipós,\footnote{ Em Varnhagen (1851 e 1879), ``fazem cestos de varas, a
que chamam samburá''.} e outras vasilhas em lavores, como as de rota"-da"-Índia; fazem
carapuças e capas de penas de pássaros, e outras obras de pena de seu uso, e sabem dar
tinta de vermelho e amarelo às penas brancas; e também contrafazem as penas dos papagaios
com sangue de rãs, arrancando"-lhes as verdes, e fazem"-lhes nascer outras, amarelas; fazem
mais estes índios, os que são principais, redes lavradas de lavores de esteiras, e de
outros laços, e umas cordas tecidas, a que chamam muçuranas, de algodão, que têm o feitio
dos cabos de cabresto que vêm de Fez.

Quando este gentio quer tomar muito peixe nos rios de água doce e nos esteiros de água
salgada, os atravessam com uma tapagem de varas, e batem o peixe de cima para baixo; onde
lhe lançam muita soma de umas certas ervas pisadas, a que chamam timbó, com o que se
embebeda o peixe de maneira que se vem acima da água como morto; onde tomam às mãos muita
soma dele.

As mulheres deste gentio não cozem, nem lavam; somente fiam algodão, de que não fazem
teias, como puderam, porque não sabem tecer; fazem deste fiado as redes em que dormem, mas
não são lavradas, e umas fitas com passamanes e algumas mais largas, com que enastram os
cabelos. As mulheres já de idade têm cuidado de fazerem a farinha de que se mantêm, e de
trazerem a mandioca das roças às costas para a casa; e as que são muito velhas têm cuidado
de fazerem vasilhas de barro a mão como são os potes em que fazem os vinhos, e fazem
alguns tamanhos que levam tanto como uma pipa, em os quais e em outros, menores, fervem os
vinhos que bebem; fazem mais estas velhas, panelas, púcaros e alguidares a seu uso, em que
cozem a farinha, e outros em que a deitam e em que comem, lavrados de tintas de cores; a
qual louça cozem em uma cova que fazem no chão; e põem"-lhe a lenha por cima; e têm e creem
estas índias que se cozer esta louça outra pessoa, que não seja a que a fez, que há de
arrebentar no fogo; as quais velhas ajudam também a fazer a farinha que se faz no seu
lanço. As fêmeas destes gentios são muito afeiçoadas a criar cachorros para os maridos
levarem à caça, e quando elas vão fora levam"-nos às costas; as quais também folgam de
criar galinhas e outros pássaros em suas casas. As quais, quando andam com seu costume,
alimpam"-se com um bordão que têm sempre junto de si, que levam na mão quando vão fora de
casa; e não se pejam de se alimparem diante da gente, nem de as verem comer piolho, o que
fazem quando se catam nas cabeças umas às outras; e como os encontra a que os busca, os dá
à que os trazia na cabeça, que logo os trinca entre os dentes, o que não fazem pelos
comer, mas em vingança de as morderem.

\paragraph{[160] Que trata de algumas habilidades e costumes dos Tupinambas}\quad
São os Tupinambas grandes flecheiros, assim para as aves como para a caça dos porcos,
veados e outras alimárias, e há muitos que matam no mar e nos rios de água doce o peixe à
flecha; e desta maneira matam mais peixe que outros à linha; os quais não arreceiam
arremeter grandes cobras, que matam, e a lagartos que andam na água, tamanhos como eles,
que tomam vivos a braços.

Costumam mais estes índios, quando vêm de caçar ou pescar, partirem sempre do que trazem
com o principal da casa em que vivem, e o mais entregam a suas mulheres, ou a quem tem o
cuidado de os agasalhar no seu lanço.

Têm estes índios mais que são homens enxutos, mui ligeiros para saltar e trepar, grandes
corredores e extremados marinheiros, como os metem nos barcos e navios, onde com todo o
tempo ninguém toma as velas como eles; e são grandes remadores, assim nas suas canoas, que
fazem de um só pau, que remam em pé vinte e trinta índios, com o que as fazem voar.

São também muito engenhosos para tomarem quanto lhes ensinam os brancos, como não for
coisa de conta, nem de sentido, porque são para isso muito bárbaros; mas para carpinteiros
de machado, serradores, oleiros, carreiros e para todos os ofícios de engenhos de açúcar,
têm grande destino, para saberem logo estes ofícios; e para criarem vacas têm grande mão e
cuidado.

Têm estes Tupinambas uma condição muito boa para frades franciscanos, porque o seu fato, e
quanto têm, é comum a todos os da sua casa que querem usar dele; assim das ferramentas,
que é o que mais estimam, como das suas roupas, se as têm, e do seu mantimento; os quais,
quando estão comendo, pode comer com eles quem quiser, ainda que seja contrário, sem lho
impedirem nem fazerem por isso carranca.

Também as moças deste gentio que se criam e doutrinam com as mulheres portuguesas, tomam
muito bem o cozer e lavrar, e fazem todas as obras de agulha que lhes ensinam, para o que
têm muita habilidade, e para fazerem coisas doces, e fazem"-se extremadas cozinheiras; mas
são muito namoradas e amigas de terem amores com os homens brancos.

São os Tupinambas grandes nadadores e mergulhadores, e quando lhes releva, nadam três e
quatro léguas; e são tais que, se de noite não têm com que pescar, se deitam na água e
como sentem o peixe consigo, o tomam às mãos de mergulho; e da mesma maneira tiram polvos
e lagostins das concavidades do fundo do mar, ao longo da costa.

\paragraph{[161] Que trata dos feiticeiros e dos que comem terra para se matarem}\quad
Entre este gentio Tupinamba há grandes feiticeiros, que têm este nome entre eles, por lhes
meterem em cabeça mil mentiras; os quais feiticeiros vivem em casa apartada cada um por
si, a qual é muito escura e tem a porta muito pequena, pela qual não ousa ninguém entrar
em sua casa, nem de lhe tocar em coisa dela; os quais, pela maior parte, não sabem nada, e
para se fazerem estimar e temer tomam este ofício, por entenderem com quanta facilidade se
mete em cabeça a esta gente qualquer coisa; mas há alguns que falam com os diabos, que os
espancam muitas vezes, os quais os fazem muitas vezes ficar em falta com o que dizem; pelo
que não são tão cridos dos índios, como temidos. A estes feiticeiros chamam os Tupinambas
pajés; os quais se escandalizam de algum índio por lhe não dar sua filha ou outra coisa
que lhe pedem, e lhe dizem: ``Vai, que hás de morrer'', ao que chamam ``lançar a morte'',
e são tão bárbaros que se vão deitar nas redes pasmados, sem quererem comer; e de pasmo se
deixam morrer, sem haver quem lhes possa tirar da cabeça que podem escapar do mandado dos
feiticeiros, aos quais dão alguns índios suas filhas por mulheres, com medo deles, por se
assegurarem suas vidas. Muitas vezes acontece aparecer o diabo a este gentio, em lugares
escuros, e os espanca de que correm de pasmo; mas a outros não faz mal, e lhes dá novas de
coisas sabidas.

Tem este gentio outra barbaria muito grande, que se tomam qualquer desgosto, se anojam de
maneira que determinam de morrer; e põem"-se a comer terra, cada dia uma pouca, até que vêm
a definhar e inchar do rosto e olhos, e morrer disso, sem lhe ninguém poder valer, nem
desviar de se quererem matar; o que afirmam que lhes ensinou o diabo, e que lhes aparece,
como se determinam a comer a terra.

\paragraph{[162] Que trata das saudades dos Tupinambas, e como choram e cantam}\quad
Costumam os Tupinambas que vindo qualquer deles de fora, em entrando pela porta, se vai
logo deitar na sua rede, ao qual se vai logo uma velha ou velhas, e põem"-se em cócaras
diante dele a chorá"-lo em altas vozes; em o qual pranto lhe dizem as saudades que dele
tinham, com sua ausência, e os trabalhos que uns e outros passaram; a que os machos lhes
respondem chorando em altas vozes, e sem pronunciarem nada, até que se enfadam, e mandam
às velhas que se calem, ao que estas obedecem; e se o chorado vem de longe, o vêm chorar
desta maneira todas as fêmeas mulheres daquela casa, e as parentas que vivem nas outras, e
como acabam de chorar, lhe dão as boas"-vindas e trazem"-lhe de comer, em um alguidar,
peixe, carne e farinha, tudo junto posto no chão, o que ele assim deitado come; e como
acaba de comer lhe vêm dar as boas"-vindas todos os da aldeia um e um, e lhe perguntam como
lhe foi pelas partes por onde andou; e quando algum principal vem de fora, ainda que seja
da sua roça, o vêm chorar todas as mulheres de sua casa, uma e uma, ou de duas em duas, e
lhe trazem presentes para comer, fazendo"-lhe as cerimônias acima ditas.

Quando morre algum índio, a mulher, mãe e parentas o choram com um tom mui lastimoso, o
que fazem muitos dias; em o qual choro dizem muitas lástimas, e magoam a quem as entende
bem; mas os machos não choram, nem se costuma entre eles chorar por ninguém que lhes
morra.

Os Tupinambas se prezam de grandes músicos, e, ao seu modo, cantam com sofrível tom, os
quais têm boas vozes; mas todos cantam por um tom, e os músicos fazem motes de improviso,
e suas voltas, que acabam no consoante do mote; um só diz a cantiga, e os outros respondem
com o fim do mote, os quais cantam e bailam juntamente em uma roda, em a qual um tange um
tamboril, em que não dobra as pancadas; outros trazem um maracá na mão, que é um cabaço,
com umas pedrinhas dentro, com seu cabo por onde pegam; e nos seus bailes não fazem mais
mudanças, nem continências mais que bater no chão com um só pé ao som do tamboril; e assim
andam todos juntos à roda, e entram pelas casas uns dos outros; onde têm prestes vinho,
com que os convidar; e às vezes anda um par de moças cantando entre eles, entre as quais
há também mui grandes músicas, e por isso mui estimadas.

Entre este gentio são os músicos mui estimados, e por onde quer que vão, são bem
agasalhados, e muitos atravessaram já o sertão por entre seus contrários, sem lhes fazerem
mal.

\paragraph{[163] Que trata como os Tupinambas agasalham os hóspedes}\quad
Quando entra algum hóspede em casa dos Tupinambas, logo o dono do lanço da casa, onde ele
chega, lhe dá a sua rede e a mulher lhe põe de comer diante, sem lhe perguntarem quem é,
nem de onde vem, nem o que quer; e como o hóspede come, lhe perguntam pela sua língua:
``Vieste já?'', e ele responde ``Sim'', as quais boas"-vindas lhe vêm dar todos os que o
querem fazer, e depois disso praticam muito devagar. E quando algum hóspede estrangeiro
entra em alguma destas aldeias, vem pregando, e assim anda correndo toda a aldeia até
quando dá com a casa do principal, e sem falar a ninguém, deita"-se em uma qualquer que
acha mais à mão, onde lhe põem logo de comer, e como acaba de comer, lhe manda o principal
armar uma rede junto da porta do seu lanço de uma banda,\footnote{ No manuscrito da
\textsc{bgjm}, ``ninguém, deita"-se em uma rede junto da porta do seu lanço de uma
banda''.} e ele arma a sua da outra banda, ficando a porta no meio para caminho de quem
quiser entrar, e assim os da aldeia lhe vêm dar as boas"-vindas, como acima está declarado;
e neste lugar se põe a praticar o principal com o hóspede muito devagar, em redor dos
quais se vêm assentar os índios da aldeia, que querem ouvir novas, onde ninguém não
responde, nem pergunta coisa alguma, até que o principal acabe de falar, e como dá fim às
práticas, lhe diz que descanse de seu vagar; e depois, se o principal despede do hóspede,
vêm outros falar com ele, para saberem novas daquelas partes de onde o hóspede vem; e ao
outro dia se ajunta este principal em outra casa, onde se ajuntam os anciãos da aldeia, e
praticam sobre a vinda do índio estrangeiro, e sobre as coisas que contou de onde vinha; e
lançam suas contas se vem de bom título ou não; e se é seu contrário, de maravilha escapa
que não o matem, e lhe façam seu ofício com muita festa e regozijo; ao qual hóspede choram
as velhas também antes que coma, como atrás fica declarado.

\paragraph{[164] Que trata do uso que os tupinambas têm em seus conselhos e das cerimônias
que neles usam}\quad
Quando o principal da aldeia quer praticar algum negócio de importância, manda recado aos
índios de mais conta, os quais se ajuntam no meio do terreiro da aldeia, onde em estacas
que têm para isso metidas no chão, armam suas redes derredor da do principal, onde também
se chegam os que querem ouvir estas práticas, porque entre eles não há segredo; os quais
se assentam todos em cócaras, e como tudo está quieto, propõe o principal sua prática, a
que todos estão muito atentos; e como acaba sua oração, respondem os mais antigos cada um
por si; e quando um fala, calam"-se todos os outros, até que vêm a concluir no que hão de
fazer; sobre o que têm suas alterações, muitas vezes. E alguns dos principais que estão
neste conselho, levam algumas cangoeiras de fumo, de que bebem; o que começa de fazer o
principal primeiro; e para isso leva um moço, que lhe dá a cangoeira acesa, e como lhe
toma a salva, manda a cangoeira a outro que não a tem, e assim se revezam todos os que não
a têm, com ela; o que estes índios fazem por autoridade, como os da Índia comem o bétele,
em semelhantes ajuntamentos; o que também fazem muitos homens brancos, e todos os
mamelucos; porque tomam este fumo por mantença, e não podem andar sem ele na boca, aos
quais dana o bafo e os dentes, e lhes faz mui ruins cores. Esta cangoeira de fumo é um
canudo que se faz de uma folha de palma seca, e tem dentro três e quatro folhas secas de
erva"-santa, a que os índios chamam petume, a qual cangoeira atam pela banda mais apertada
com o fio, onde estão as folhas de petume, e acendem esta cangoeira pela parte das folhas
de petume, e como tem brasa, a metem na boca, e sorvem para dentro o fumo, que logo lhe
entra pelas cachagens, mui grosso, e pelas goelas, e sai"-lhe pelas ventas fora com muita
fúria, como não podem sofrer este fumo, tiram a cangoeira fora da boca.

\paragraph{[165] Que trata de como se este gentio cura em suas \mbox{enfermidades}}\quad
São os Tupinambas mui sujeitos à doença das boubas, que se pegam uns aos outros, mormente
enquanto são meninos, porque se não guardam de nada; e têm para si que as hão de ter tarde
ou cedo, e que o bom é terem"-nas enquanto são meninos, os quais não fazem outro remédio
senão fazê"-las secar,\footnote{ No manuscrito da \textsc{bgjm}, ``uns aos outros, mormente
enquanto são meninos, aos quais não fazem outro remédio senão fazê"-las secar''.} quando
lhe saem para fora, o que fazem com as tingirem com jenipapo; e quando isto não basta,
curam"-lhes estas bostelas das boubas com a folha de caraoba, de cuja virtude temos já
feito menção, e como se estas bostelas secam, têm para si que estão sãos deste mau humor,
e na verdade não têm dores nas juntas como se elas secam.

Em alguns tempos e lugares, mais que outros, são estes índios doentes de terçãs e quartãs,
que lhes nascem de andar pela calma, sem nada na cabeça, e de quando estão mais suados se
banharem com água fria, metendo"-se nos rios e nas fontes, muitas vezes ao dia pelo tempo
da calma; ou quando trabalham, que estão cansados e suados; às quais febres não fazem
nenhuma cura senão comendo uns mingaus, que são uns caldos de farinha de carimã, como já
fica dito, que são muito leves e sadios; e untam"-se com água do jenipapo, com o que ficam
todos tintos de preto, ao que têm grande devoção.

Curam estes índios algumas postemas e bexigas com sumo de ervas de virtude, que há entre
eles, com que fazem muitas curas muito notáveis, como já fica dito atrás; e quando se
sentem carregados da cabeça, sarjam nas fontes, e aos meninos sarjam"-nos nas pernas, quando
têm febre, mas em seco; o que fazem as velhas com um dente de cotia muito agudo, que têm
para isso.

Curam as grandes feridas e flechadas com umas ervas, que chamam embayba,\footnote{ Em
Varnhagen (1851 e 1879), ``cabureíba''. Embaíba ou embaúba.} que é milagrosa, e com
outras ervas, de cujas virtudes fica dito atrás no seu título; com as quais curam o cano,
que se lhes enche muitas vezes de câncer; e as flechadas penetrantes e outras feridas, de
que se veem em perigo, curam por um estranho modo, fazendo em cima do fogo um leito de
varas largas umas das outras, sobre as quais deitam os feridos, com as feridas boca abaixo
em cima deste fogo, pelas quais com a quentura se lhes sai todo o sangue que têm dentro e
a umidade; e ficam as feridas sem nenhuma umidade; as quais depois curam com óleo e o
bálsamo, ou ervas, de que já fizemos menção, com o que têm saúde poucos dias; e não há
entre este gentio médicos assinalados, mas são"-no muito bons os recuchilhados.


Destes índios andarem sempre nus, e das fregueirices que fazem dormindo no chão, são
muitas vezes doentes de corrimentos, a que eles chamam caruaras,\footnote{ Em Varnhagen
(1851 e 1879), ``caivarás''.} de que lhes doem as juntas; das quais são os feiticeiros
grandes médicos, chupando"-lhes com a boca o lugar onde lhes dói, onde às vezes lhes metem
os dentes, e tiram da boca algum pedaço de ferro, ou outra coisa, que lhes metem na cabeça
tirar daquele lugar onde chupavam, e que quando lhes doía lhes saíra fora, onde lhes
tingem com jenipapo, com que dizem que se acha logo bom.

\paragraph{[166] Que trata do grande conhecimento que os Tupinambas têm da terra}\quad
Têm os Tupinambas grande conhecimento da terra por onde andam, pondo o rosto no sol, por
onde se governam; com o que atinam grandes caminhos pelo deserto, por onde nunca andaram;
como se verá pelo que aconteceu já na Bahia, de onde mandaram dois índios destes
Tupinambas degredados pela justiça por seus delitos, para o Rio de Janeiro, onde foram
levados por mar; os quais se vieram de lá, cada um por sua vez, fugidos, afastando"-se
sempre do povoado, por não ser sentidos por seus contrários; e vinham sempre caminhando
pelos matos; e desta maneira atinaram com a Bahia, e chegaram à sua aldeia, de onde eram
naturais a salvamento, sendo caminho mais de trezentas léguas.

Costuma este gentio, quando anda pelo mato sem saber novas do lugar povoado, deitar"-se no
chão, e cheirar o ar, para ver se lhe cheira a fogo, o qual conhecem pelo faro a mais de
meia légua, segundo a informação de quem com eles trata mui familiarmente; e como lhe
cheira a fogo, se sobem às mais altas árvores que acham, em busca de fumo, o que alcançam
com a vista de mui longe, o qual vão seguindo, se lhes vem bem ir onde ele está; e se lhes
convém desviar"-se dele, o fazem antes que sejam sentidos; e por os Tupinambas terem este
conhecimento da terra e do fogo, se faz muita conta deles, quando se oferece irem os
portugueses à guerra a qualquer parte, onde os Tupinambas vão sempre adiante, correndo a
terra por serem de recado, mostrando à mais gente o caminho por onde hão de caminhar, e o
lugar onde se hão de aposentar cada noite.

\paragraph{[167] Que trata de como os Tupinambas se apercebem para irem à guerra}\quad
Como os Tupinambas são muito belicosos, todos os seus fundamentos são como farão guerra
aos seus contrários; para o que se ajuntam no terreiro da sua aldeia as pessoas mais
principais, e fazem seus conselhos, como fica declarado; onde assentam a que parte hão de
ir dar a dita guerra, e em que tempo; para o que se notifica a todos que se façam prestes
de arcos e flechas, e alguns paveses,\footnote{ Pavês: escudo longo e largo que protege o
corpo.} que fazem de um pau mole e muito leve, e as mulheres entendem em lhes fazerem a
farinha que hão de levar, a que chamam de guerra; porque dura muito, para se fazer a dita
guerra, de onde tomou o nome; e como todos estão prestes de suas armas e mantimentos, à
noite antes da partida anda o principal pregando ao redor das casas, e nesta pregação lhes
diz onde vão, e a obrigação que têm de ir tomar vingança de seus contrários, pondo"-lhes
diante a obrigação que têm para o fazerem e para pelejarem valorosamente;\footnote{ No
manuscrito da \textsc{bgjm}, ``lhes diz onde vão, e a obrigação que tem para o fazerem e
para pelejarem valorosamente''.} prometendo"-lhes vitória contra seus inimigos, sem nenhum
perigo da sua parte, do que ficará deles memória para os que após eles vierem cantar em
seus louvores; e que pela manhã comecem de caminhar. E em amanhecendo, depois de
almoçarem, toma cada um seu quinhão de farinha às costas, e a rede em que há de dormir,
seu pavês e arco e flechas na mão, e outros levam além disto uma espada de pau a tiracolo.
Os roncadores levam tamboril,\footnote{ No manuscrito da \textsc{bgjm}, ``flechas na mão,
e os roncadores levam além disto uma espada de pau a tiracolo, outros levam tamboril''.}
outros levam buzinas, que vão tangendo pelo caminho, com que fazem grande estrondo, como
chegam à vista dos contrários. E os principais deste gentio levam consigo as mulheres
carregadas de mantimentos, e eles não levam mais que a sua rede e armas às costas, e arco
e flechas na mão. E antes que se abalem, faz o principal capitão da dianteira, que eles
têm por grande honra, o qual vai mostrando o caminho e o lugar onde hão de dormir cada
noite. E a ordenança com que se põe a caminho, é um diante do outro, porque não sabem
andar de outra maneira; e como saem fora dos seus limites, e entram pela terra dos
contrários, levam ordinariamente suas espias adiante que são sempre mancebos muito
ligeiros, que sabem muito bem este ofício; e com muito cuidado, os quais não caminham cada
dia mais de légua e meia até duas léguas, que é o que se pode andar até as nove horas do
dia, que é o tempo em que aposentam seu arraial, o que fazem perto d'água, fazendo suas
choupanas, a que chamam tejupares, as quais fazem arruadas, deixando um caminho pelo meio
delas; e desta maneira vão fazendo suas jornadas, fazendo fogos nos tajupares.

\paragraph{[168] Que trata de como os Tupinambas dão em seus contrários}\quad
Tanto que os Tupinambas chegam duas jornadas da aldeia de seus contrários não fazem fogo
de dia, por não serem sentidos deles pelos fumos que se vê de longe;\footnote{ Em
Varnhagen (1851 e 1879), ``que se vêm de longe''.} e ordenam"-se de maneira que possam dar
nos contrários de madrugada, e em conjunção de lua cheia para andarem a derradeira jornada
de noite pelo luar, e tomarem seus contrários desapercebidos e descuidados; e em chegando
à aldeia dão todos juntos tamanho urro, gritando, que fazem com isso e com suas buzinas e
tamboris grande espanto; e desta maneira dão o seu salto nos contrários; e no primeiro
encontro não perdoam a grande nem a pequeno, para o que vão apercebidos de uns paus à
feição de arrochos, com uma quina por uma ponta, com o que da primeira pancada que dão na
cabeça ao contrário, lha fazem em pedaços. E há alguns destes bárbaros tão carniceiros que
cortam aos vencidos, depois de mortos, suas naturas, assim aos machos como às fêmeas, as
quais levam para darem a suas mulheres que as guardam depois de mirradas no fogo, para nas
suas festas as darem de comer aos maridos por relíquias, o que lhes dura muito tempo; e
levam os contrários que não mataram na briga, cativos, para depois os matarem em terreiro
com as festas costumadas.

No despojo desta guerra não tem o principal coisa certa, e cada um leva o que pode
apanhar, e quando os vencedores se recolhem, põem fogo às casas da aldeia em que deram,
que são cobertas de palmas até o chão. E recolhem"-se logo andando todo o que lhes resta do
dia, e toda a noite pelo luar com o passo mais apressado, trazendo suas espias detrás, por
se arrecearem de se ajuntarem muitos do contrário, e virem tomar vingança do acontecido a
seus vizinhos, como cada dia lhes acontece. E sendo caso que os Tupinambas achem seus
contrários apercebidos com a sua cerca feita, e eles se atrevem a os cercar, fazem"-lhes
por derredor outra contracerca de rama e espinhos muito liada com madeira que metem no
chão, a que chamam caissá,\footnote{ Em Varnhagen (1851 e 1879), ``caiçá''.} pela qual
enquanto verde não há coisa que os rompa, e ficam com ela seguros das flechas dos
contrários, a qual caissá fazem bem chegada à cerca dos contrários, e de noite falam mil
roncarias, e jogam as pulhas de parte a parte, até que os Tupinambas abalroam a cerca ou
levantam cerco, se se não atrevem com ele, ou por lhes faltar o mantimento.

\paragraph{[169] Que trata de como os contrários dos Tupinambas dão sobre eles quando se
recolhem}\quad
Acontece muitas vezes aos Tupinambas, quando se vêm recolhendo para suas casas, dos
assaltos que deram em seus contrários, ajuntar"-se grande soma deles, e virem"-lhes no
alcance até lhes não poderem fugir; e ser"-lhes necessário esperá"-los, o que fazem ao longo
da água, onde se fortificam fazendo sua cerca de caissá; o que fazem com muita pressa para
dormirem ali seguros de seus contrários, mas com boa vigia; onde muitas vezes são cercados
e apertados dos contrários; mas os cercados veem por detrás desta cerca a quem está de
fora, para empregarem todas as suas flechas à vontade, e os de fora não veem quem lhes
atira; e se não vêm apercebidos para os abalroarem, ou de mantimentos, para continuarem
com o cerco, se tornam a recolher, por não poderem abalroar aos Tupinambas como queriam.

E estes assaltos, que os Tupinambas vão dar aos Tupinaes e outros contrários seus, lhes
acontece também a eles por muitas vezes, do que ficam muito maltratados, se não são
avisados primeiro, e apercebidos; mas as mais das vezes eles são os que ofendem a seus
inimigos, e são mais prevenidos quando veem nestas afrontas de mandar pedir socorro a seus
vizinhos, e lho veem logo dar com muita presteza.

Quando os Tupinambas estão cercados de seus contrários, as pessoas de mais autoridade
dentre eles lhes andam pregando de noite para que se esforcem e pelejem como bons
cavaleiros, e que não temam seus contrários, porque muito depressa se verão vingados deles
porque lhes não tardará o socorro muito; e as mesmas pregações costumam fazer quando eles
têm cercado seus contrários, e os querem abalroar; e antes que deem o assalto, estando
juntos todos à noite atrás, passeia o principal derredor dos seus, e lhes diz em altas
vozes o que hão de fazer, e os avisa para que se apercebam, e estejam alerta; e as mesmas
pregações lhes faz, quando andam fazendo as cercas de caissá, para que se animem, e façam
aquela obra com muita pressa.

E quando os Tupinambas pelejam no campo, andam saltando de uma banda para outra, sem
estarem nunca quedos, assobiando, dando com a mão no peito, guardando"-se das flechas que
lhes lançam seus contrários, e lançando"-lhes as suas com muita fúria.

\paragraph{[170] Em que se declara como o Tupinamba que matou o contrário toma logo nome, e
as cerimônias que nisto fazem}\quad
Costuma"-se, entre os Tupinambas, que todo aquele que mata contrário, toma logo nome entre
si, mas não o diz senão a seu tempo, que manda fazer grandes vinhos; e como estão para se
poderem beber, tingem"-se à véspera à tarde de jenipapo, e começam à tarde a cantar, e toda
a noite, e depois que têm cantado um grande pedaço, anda toda a gente da aldeia rogando ao
matador, que diga o nome que tomou, ao que se faz de rogar, e, tanto que o diz, se ordenam
novas cantigas, fundadas sobre a morte daquele que morreu, e em louvores do que matou, o
qual, como se acabam aquelas festas e vinhos, se recolhe para a sua rede, como anojado,
por certos dias, e não come neles certas coisas, que têm por agouro se as comer dentro
daquele tempo.

Todo Tupinamba que matou na guerra ou em outra qualquer parte algum contrário, tanto que
vem para casa, e é notório aos moradores dela da tal morte do contrário, costumam, em o
matador entrando em casa, arremessarem"-se todos ao seu lanço e tomarem"-lhe as armas e
todas as suas alfaias de seu uso, ao que ele não há de resistir por nenhum caso, e há de
deixar levar tudo sem falar palavra.

E como o matador faz estas festas deixa crescer o cabelo por dó alguns dias, e como é
grande, ordena outros vinhos para tirar o dó; ao que faz nas vésperas cantadas, e ao dia
que se hão de beber os vinhos, se tosquia o matador e tira o dó; tornando"-se a encher e
tingir de jenipapo, o qual também se risca em algumas partes do corpo com o dente de
cotia, em lavores; e dão por estas sarjaduras uma tinta com que ficam vivas, e enquanto o
riscado vive, o têm por grande bizarria; e há alguns índios que tomaram tantos nomes, e se
riscaram tantas vezes que não têm parte onde não esteja o corpo riscado.

Costumam também as irmãs dos matadores fazerem as mesmas cerimônias que fizeram seus
irmãos, tosquiando"-se e tingindo"-se do jenipapo, e darem alguns riscos em si; e fazem o
mesmo pelos primos a que também chamam irmãos, e fazem também suas festas com seus vinhos,
como eles; e para se não sentir a dor do riscar, se lavam primeiro muito espaço com água
muito quente, com que lhes entesa a carne, e não sentem as sarjaduras; mas muitos ficam
dela tão maltratados que se põem em perigo de morte.

\paragraph{[171] Que trata do tratamento que os Tupinambas fazem aos que cativam, e da
mulher que lhes dão}\quad
Os contrários que os Tupinambas cativam na guerra, ou de outra qualquer maneira, metem"-nos
em prisões, as quais são cordas de algodão grossas, que para isso têm mui louçãs, a que
chamam musuranas,\footnote{ Em Varnhagen (1851 e 1879), ``muçuranas''.} as quais são
tecidas como os cabos dos cabrestos de África; e com elas os atam pela cinta e pelo
pescoço, onde lhes dão muito bem de comer, e lhes fazem bom tratamento, até engordarem, e
estão estes cativos para se poderem comer, que é o fim para que os engordam; e como os
Tupinambas têm estes contrários quietos e bem seguros nas prisões, dão a cada um por
mulher a mais formosa moça que há na sua casa, com quem se ele agasalha, todas as vezes
que quer, a qual moça tem cuidado de o servir, e de lhe dar o necessário para comer e
beber, com o que cevam cada hora, e lhe fazem muitos regalos. E se esta moça emprenha do
que está preso, como acontece muitas vezes, como pare, cria a criança até idade que se
pode comer, que a oferece para isso ao parente mais chegado, que lho agradece muito, o
qual lhe quebra a cabeça em terreiro com as cerimônias que se adiante seguem, onde toma
nome; e como a criança é morta, a comem assada com grande festa, e a mãe é a primeira que
come desta carne, o que tem por grande honra, pelo que de maravilha escapa nenhuma criança
que nasce destes ajuntamentos, que não matem; e a mãe que não come seu próprio filho, a
que estes índios chamam cunhambira, que quer dizer filho do contrário, têm"-na em ruim
conta, e em pior, se o não entregam seus irmãos ou parentes com muito contentamento. Mas
também há algumas que tomaram tamanho amor aos cativos que as tomaram por mulheres, que
lhes deram muito jeito para se acolherem e fugirem das prisões, que eles cortam com alguma
ferramenta, que elas às escondidas deram e lhes foram pôr antes de fugirem no mato
mantimentos para o caminho;\footnote{ Em Varnhagen (1851 e 1879), `` às escondidas lhes
deram, e lhes foram pôr no mato, antes de fugirem, mantimentos''.} e estas tais criaram
seus filhos com muito amor, e não os entregaram aos parentes para os matarem,\footnote{ Em
Varnhagen (1851 e 1879), ``entregaram a seus parentes''.} antes os guardaram e defenderam
deles até serem moços grandes, que como chegam a essa idade logo escapam da fúria dos seus
contrários.

Muitas vezes deixam os Tupinambas de matar alguns contrários que cativaram por serem
moços, e se quererem servir deles, aos quais criam e fazem tão bom tratamento que andam de
maneira que podem fugir, o que eles não fazem por estarem à sua vontade; mas depois que
este gentio teve comércio com os portugueses, folgam de terem escravos para lhes venderem;
e, às vezes, depois de os criarem, os matam por fazerem uma festa destas.

\paragraph{[172] Que trata da festa e aparato que os Tupinambas fazem para matarem em
terreiro seus contrários}\quad
Como os Tupinambas veem que os contrários que têm cativos estão já bons para matar,
ordenam de fazer grandes festas a cada um, para as quais há grandes ajuntamentos de
parentes e amigos, que para isso são chamados de trinta e quarenta léguas, para a vinda
dos quais fazem grandes vinhos, que bebem com grandes festas; mas fazem"-na muito maiores
para o dia do sacrifício do que há de padecer, com grandes cantares, e à véspera em todo
dia cantam e bailam, e ao dia se bebem muitos vinhos pela manhã, com motes que dizem sobre
a cabeça do que há de padecer,\footnote{ No manuscrito da \textsc{bgjm}, ``pela manhã, que
dizem sobre a cabeça que há de padecer''.} que também bebe com eles. E os que cantam suas
cantigas vituperando o que há de padecer e exalçando o matador, dizendo suas proezas e
louvores; e antes que bebam os vinhos untam o cativo todo com mel de abelhas, e por cima
deste mel o empenam todo com penas de cores, e pintam"-no a lugares de jenipapo, e os pés
com uma tinta vermelha, e metem"-lhe uma espada de pau nas mãos para que se defenda de quem
o quer matar com ela, como puder; e como estes cativos veem chegada a hora em que hão de
padecer, começam a pregar e dizer grandes louvores de sua pessoa, dizendo que já estão
vingados de quem os há de matar, contando grandes façanhas suas e mortes que deram aos
parentes do matador, ao qual ameaçam e a toda a gente da aldeia, dizendo que seus parentes
os vingarão. E começam a levar este preso a um terreiro fora da aldeia, que para esta
execução está preparado, e metem"-no entre dois mourões, que estão metidos no chão,
afastados um do outro por vinte palmos, pouco mais ou menos, os quais estão furados, e por
cada furo metem as pontas das cordas com que o contrário vem preso, onde fica preso como
touro de cordas, onde lhe as velhas dizem que se farte de ver o sol, pois tem o fim tão
chegado; ao que o cativo responde com grande coragem, que pois ele tem vingança da sua
morte tão certa, que aceita o morrer com muito esforço. E antes de lhe chegar a execução,
contemos como se prepara o matador.

\paragraph{[173] Que trata de como se enfeita e aparata o matador}\quad
Costumam os Tupinambas, primeiro que o matador saia do terreiro, enfeitarem"-no muito bem,
pintados com lavores de jenipapo todo o corpo, e põem"-lhe na cabeça uma carapuça de penas
amarelas e um diadema, manilhas nos braços e pernas, das mesmas penas, grandes ramais de
contas brancas sobraçadas, e seu rabo de penas de ema nas ancas e uma espada de pau de
ambas as mãos muito pesada, marchetada com continhas brancas de búzios, e pintada com
cascas de ovos de cores, assentado tudo, em lavores ao seu modo, sobre cera, o que fica
mui igualado e bem feito; e no cabo desta espada têm grandes penachos de penas de pássaros
feitas em molho e dependuradas da empunhadura, a que eles chamam embaguadura.\footnote{ Em
Varnhagen (1851 e 1879), ``embagadura''.}

E como o matador está prestes para receber esta honra, que entre o gentio é a maior que
pode ser, ajuntam"-se seus parentes e amigos e vão"-no buscar à sua casa; de onde o vêm
acompanhando com grandes cantares e tangeres dos seus búzios, gaitas e tambores,
chamando"-lhe bem"-aventurado, pois chegou a ganhar tamanha honra, como é vingar a morte de
seus antepassados, e de seus irmãos e parentes; e com este estrondo entra no terreiro da
execução, onde está o que há de padecer, que o está esperando com grande coragem com uma
espada de pau na mão, diante de quem chega o matador, e lhe diz que se defenda, porque vem
para o matar, a quem responde o preso com mil roncarias; mas o solto remete a ele com a
sua espada de ambas as mãos, da qual se quer desviar o preso para alguma banda, mas os que
têm cuidado das cordas puxam por elas de feição que o fazem esperar a pancada; e acontece
muitas vezes que o preso, primeiro que morra, chega com a sua espada ao matador que o
trata muito mal, sem embargo de lhe não deixarem as cordas chegar a ele; por mais que o
pobre trabalhe, não lhe aproveita, porque tudo é dilatar a vida mais dois credos, onde a
rende nas mãos do seu inimigo, que lhe faz a cabeça em pedaços com sua espada; e como se
acaba esta execução, tiram"-no das cordas e levam"-no para onde se costuma repartir esta
carne.

E acabado o matador de executar sua ira no cativo, toma logo entre si nome
algum,\footnote{ Em Varnhagen (1851 e 1879), ``entre si algum nome''.} o qual declara
depois com as cerimônias que ficam ditas atrás; e vai"-se do terreiro recolher para o seu
lanço, onde tira as armas e petrechos com que se enfeitou; e a mesma honra ficam recebendo
aqueles que primeiro pegaram dos cativos na guerra, do que tomam também nome
novo,\footnote{ Em Varnhagen (1851 e 1879), ``também novo nome''.} com as mesmas festas e
cerimônias que já ficam ditas; o que se não faz com menos alvoroço que aos próprios
matadores.

\paragraph{[174] Em que se declara o que os Tupinambas fazem do contrário que mataram}\quad
Acabado de morrer este preso, o espedaçam logo os velhos da aldeia, e tiram"-lhe as tripas
e fressura, que mal lavadas, cozem e assam para comer; e reparte"-se a carne por todas as
casas da aldeia e pelos hóspedes que vieram de fora a ver estas festas e matanças, a qual
carne se coze logo para se comer nos mesmos dias de festa, e outra assam muito afastada do
fogo de maneira que fica muito mirrada, a que este gentio chama moquém, a qual se não come
por mantimento, senão por vingança; e os homens mancebos e mulheres moças provam"-na
somente, e os velhos e velhas são os que se metem nesta carniça muito, e guardam alguma da
assada do moquém por relíquias, para com ela de novo tornarem a fazer festas, se se não
oferecer tão cedo matarem outro contrário. E os hóspedes que vieram de fora a ver esta
festa levam o seu quinhão de carne que lhe deram do morto, assada do moquém para as suas
aldeias, onde como chegam fazem grandes vinhos para, com grandes festas, segundo sua
gentilidade, os beberem sobre esta carne humana que levam, a qual repartem por todos da
aldeia, para a provarem, e se alegrarem em vingança de seu contrário que padeceu, como
fica dito.

Acontece muitas vezes cativar um Tupinamba a um contrário na guerra, onde o não quis matar
para o trazer cativo para a sua aldeia, onde o faz engordar com as cerimônias já
declaradas para o deixar matar a seu filho quando é moço e não tem idade para ir à guerra,
o qual mata em terreiro, como fica dito, com as mesmas cerimônias; mas atam as mãos ao que
há de padecer, para com isto o filho tomar nome novo e ficar armado cavaleiro, e mui
estimado de todos. E se este moço matador, ou outro algum, se não quer riscar quando toma
novo nome, contentam"-se com se tingir de jenipapo, e deixar crescer o cabelo e tosquiá"-lo,
com as cerimônias atrás declaradas; e os que se riscam quando tomam nome novo, a cada nome
que tomam fazem sua feição de lavor, que para eles é grande bizarria, para que se veja
quantos nomes tem.

\paragraph{[175] Que trata das cerimônias que os Tupinambas fazem quando morre algum, e como
os enterram}\quad
É costume entre os Tupinambas que, quando morre qualquer deles, o levam a enterrar
embrulhado na sua rede em que dormia, e o parente mais chegado lhe há de fazer a cova; e
quando o levam a enterrar vão"-no acompanhando mulher, filhos e parentes,\footnote{ Em
Varnhagen (1851 e 1879), ``mulher, filhas e parentes''.} se os tem, os quais o vão
pranteando até a cova, com os cabelos soltos sobre o rosto, e estão"-no pranteando até que
fica bem coberto de terra; de onde se tornam para sua casa, onde a viúva chora o marido
por muitos dias; e se morrem as mulheres destes Tupinambas, é costume que os maridos lhes
façam a cova, e ajudem a levar às costas a defunta, e se não tem já marido o irmão ou
parente mais chegado lhe faz a cova.

E quando morre algum principal da aldeia em que vive, e depois de morto alguns dias, ao
qual antes de o enterrarem fazem as cerimônias seguintes.\footnote{ Em Varnhagen (1851 e
1879), ``alguns dias, antes de o enterrarem''.} Primeiramente o untam com mel todo, e por
cima do mel o empenam com penas de pássaros de cores, e põem"-lhe uma carapuça de penas na
cabeça com todos os mais enfeites que eles costumam trazer nas suas festas; e têm"-lhe
feito na mesma casa e lanço onde ele vivia, uma cova muito funda e grande, com sua
estacada por derredor, para que tenha a terra que não caia sobre o defunto, e armam"-lhe a
sua rede embaixo, de maneira que não toque o morto no chão; em a qual rede o metem assim
enfeitado, e põem"-lhe junto da rede seu arco e flechas, e a sua espada, e o
maraqua\footnote{ Em Varnhagen (1851 e 1879), ``maracá''.} com que costumava tanger, e
fazem"-lhe fogo ao longo da rede para se aquentar, e põem"-lhe de comer em um alguidar, e a
água em um cabaço;\footnote{ Em Varnhagen (1851 e 1879), ``e a água em um cabaço, como
galinha''.} e como esta matalotagem está feita, e lhe põem também sua cangoeira de fumo na
mão, lançam"-lhe muita soma de madeira igual no andar da rede de maneira que não toque no
corpo, e sobre esta madeira muita soma de terra, com ramas debaixo, primeiro, para que não
caia terra sobre o defunto; sobre a qual sepultura vive a mulher, como dantes. E quando
morre algum moço, filho de algum principal, que não tem muita idade, metem"-no em cócaras,
atados os joelhos com a barriga, em um pote em que ele caiba, e enterram o pote na mesma
casa debaixo do chão, onde o filho e o pai, se é morto, são chorados muitos dias.

\paragraph{[176] Que trata do sucessor ao principal que morreu, e das cerimônias que faz sua
mulher, e por morte dela também\protect\footnote{ Em Varnhagen (1851 e 1879), ``sua
mulher, e as que se fazem por morte dela também''.}}\quad
Costumam os Tupinambas, quando morre o principal da aldeia, elegerem entre si quem suceda
em seu lugar, e se o defunto tem filho que lhe possa suceder, a ele aceitam por seu
cabeça; e quando não é para isso, ou o não tem, aceitam um seu irmão em seu lugar; e não
os tendo que tenham partes para isso, elegem um parente seu, se é capaz de tal cargo, e
tem as partes atrás declaradas.

É costume entre as mulheres dos principais Tupinambas, ou de outro qualquer índio, a
mulher cortar os cabelos por dó, e tingirem"-se toda de jenipapo; as quais choram seus
maridos muitos dias, e são visitadas de suas parentas e amigas; e todas as vezes que o
fazem, tornam com a viúva a prantear de novo o defunto, as quais deixam crescer o cabelo
até que lhes dá pelos olhos, e se não casa com outro, logo faz sua festa com vinhos, e
torna"-se a tosquiar para tirar o dó, e tinge"-se de novo do jenipapo.

Costumam os índios, quando lhes morrem as mulheres, deixarem crescer o cabelo, no que não
têm tempo certo, e tingem"-se do jenipapo por dó; e quando se querem tosquiar, se tornam a
tingir de preto à véspera da festa dos vinhos, que fazem a seu modo, cantando toda a
noite, para a qual se ajunta muita gente para estes cantares, e o viúvo tosquia"-se à
véspera, à tarde, e ao outro dia há grandes revoltas de cantar e bailar, e beber muito; e
o que neste dia mais bebeu fez maior valentia, ainda que vomite e perca o juízo. Nestas
festas se cantam as proezas do defunto ou defunta, e do que tira o dó, e o mesmo dó tomam
os irmãos, filhos, pai e mãe do defunto, e cada um por si faz sua festa, quando tira o dó
apartado, ainda que o tragam por uma mesma pessoa. Mas este sentimento houveram de ter os
vivos dos mortos, quando estavam doentes; mas são tão desamoráveis os Tupinambas que,
quando algum está doente, e a doença é comprida, logo aborrece a todos os seus, e curam
dele muito pouco; e como o doente chega a estar mal, é logo julgado por morto; e não
trabalham os seus mais chegados por lhe dar a vida, antes o desamparam, dizendo que pois
há de morrer, e não tem remédio, que para que é dar"-lhe de comer, nem curar dele; e tanto
é isto assim que morrem muitos ao desamparo; e levam a enterrar outros ainda vivos, porque
como chega a perder a fala dão"-no logo por morto; e entre os portugueses aconteceu muitas
vezes fazerem trazer de junto da cova escravos seus para casa, por as mulheres os julgarem
por mortos, muitos dos quais tiveram saúde e viveram depois muitos anos.

\paragraph{[177] Que trata de como entre os Tupinambas há muitos mamelucos que descendem dos
franceses, e de um índio que se achou muito alvo}\quad
Ainda que pareça fora de propósito o que se contém neste capítulo, pareceu decente
escrever aqui o que nele se contém, para se melhor entender a natureza e condição dos
Tupinambas, com os quais os franceses, alguns anos antes que se povoasse a Bahia, tinham
comércio; e quando se iam para França com suas naus carregadas de pau de tinta, algodão e
pimenta, deixavam entre os gentios alguns mancebos para aprenderem a língua e poderem
servir na terra, quando tornassem da França para lhes fazer seu resgate; os quais se
amancebaram na terra, onde morreram, sem se quererem tornar para a França, e viveram como
gentios com muitas mulheres, dos quais e dos que vinham todos os anos à Bahia e ao rio de
Serigipe, em naus da França, se inçou a terra de mamelucos, que nasceram, viveram e
morreram como gentios; dos quais há hoje muitos seus descendentes, que são louros, alvos e
sardos, e havidos por índios Tupinambas, e são mais bárbaros que eles. E não é de espantar
serem estes descendentes dos franceses alvos e louros, pois que saem a seus avós; mas é de
maravilhar trazerem do sertão, entre outros Tupinambas, um menino de idade de dez anos
para doze, no ano de 1586, que era tão alvo, que de o ser muito não podia olhar para a
claridade; e tinha os cabelos da cabeça, pestanas e sobrancelhas tão alvas como algodão,
com o qual vinha seu pai, com quem era tão natural, que toda pessoa que o via, o julgava
por esse sem o conhecer; e não era muito preto, e a mãe, que vinha na companhia, era muito
preta; e pelas informações que se então tomaram dos outros Tupinambas da companhia,
achou"-se que o pai deste índio branco não descendia dos franceses, nem eles foram àquelas
partes, de onde esta gente vinha, nunca; e ainda que este menino era assim branco, era
muito feio.

Nesta povoação onde este índio branco veio ter, que é de Gabriel Soares, aconteceu um caso
estranho a uma índia Tupinamba, que havia pouco viera do sertão, a qual ia para uma roça a
buscar mandioca, levando um filho de um ano às costas, que ia chorando, do qual se enfadou
a mãe de maneira que lhe fez uma cova com um pau no chão, e o enterrou vivo; e foi"-se a
índia com as outras à roça, que seria dali distância de um bom tiro de bombarda; e
arrancou a mandioca, que ia buscar; e tornou"-se com ela para casa, que seria de onde a
criança ficava enterrada, outro tiro de bombarda; sobre o que as outras índias, que viram
fazer esta crueldade de mãe,\footnote{ Em Varnhagen (1851 e 1879), ``que viram esta
crueldade''.} estando fazendo farinha, se puseram a praticar, maravilhando"-se do caso
acontecido, o que ouviram outras índias da mesma casa ladinas, e foram"-no contar a sua
senhora, e logo se informou do caso como acontecera, e sabendo a verdade dele mandou a
toda pressa desenterrar a criança, que ainda acharam viva, e por ser pagã a fez batizar
logo, a qual viveu depois seis meses.

\subsection{Daqui por diante se vai continuando com a vida e costumes dos Tupinaes, e
outras castas de gentio da Bahia que vive pela terra dentro do seu sertão, dos quais
diremos o que pudemos alcançar deles; e começando logo nos Tupinaes}

\paragraph{[178] Que trata de quem são os Tupinaes}\quad
Tupinaes é uma gente do Brasil semelhante no parecer, vida e costumes dos Tupinambas e na
linguagem não têm mais diferença uns dos outros, do que têm os moradores de Lisboa dos
de entre Douro e Minho; mas a dos Tupinambas é a mais polida; e pelo nome tão semelhante
destas duas castas de gentio se parece bem claro que antigamente foi esta gente toda uma,
como dizem os índios antigos desta nação; mas têm"-se por tão contrários uns dos outros que
se comem aos bocados, e não cansam de se matarem em guerras, que continuamente têm, e não
tão somente são inimigos os Tupinaes dos Tupinambas, mas são"-no de todas as outras nações
do gentio do Brasil, e entre todas elas lhes chamam ``tabayaras'',\footnote{ Em Varnhagen
(1851 e 1879), ``tapuras''.} que quer dizer contrários. Os quais Tupinaes nos tempos
antigos viveram ao longo do mar, como fica dito no título dos Tupinambas, que os lançaram
dele para o sertão, onde agora vivem, e terão ocupado uma corda de terra de mais de
duzentas léguas; mas ficam entressachados com eles, em algumas partes, alguns Tapuyas, com
quem têm também contínua guerra.

São os Tupinaes mais atraiçoados que os Tupinambas, e mais amigos de comer carne humana,
em tanto que se lhes não acha nunca escravo dos contrários que cativam, porque todos matam
e comem, sem perdoarem a ninguém. E quando as fêmeas emprenham dos contrários, em parindo
lhes comem logo a criança, a que também chamam cunhamebira;\footnote{ Em Varnhagen (1851 e
1879), ``cunhãembira''.} e a mesma mãe ajuda logo a comer o mesmo filho que
pariu.\footnote{ Em Varnhagen (1851 e 1879), ``comer o filho que pariu''.}

\paragraph{[179] Que trata de alguns costumes e trajes dos Tupinaes}\quad
\mbox{Costumam} entre os Tupinaes trazerem os homens os cabelos da cabeça compridos até lhes
cobrirem as orelhas, muito aparados sobre elas, e desafogado por diante; e outros o trazem
copado sobre as orelhas, como crenchas;\footnote{ Crencha: trança de cabelo.} e alguns
tosquiam a dianteira até as orelhas sobre pentem, por detrás o cabelo comprido; e a seu
modo, de uma maneira e outra, fica muito afeiçoado.

São os Tupinaes mais fracos de ânimo que os Tupinambas, de menos trabalho, fé e
verdade;\footnote{ Em Varnhagen (1851 e 1879), ``de menos trabalho, de menos fé e
verdade''.} são músicos de natureza, e grandes cantores de chacotas, quase pelo modo dos
Tupinambas; bailam, caçam e pescam como eles, e pelejam em saltos, como eles; mas não são
pescadores no mar, como se acham nele, pelo não haverem em costume, por ser gente do
sertão, e esmorecerem; e não pescam senão nos rios de água doce.\footnote{ No manuscrito
da \textsc{bgjm}, ``por ser gente do sertão; e não pescam''.}

Estes Tupinaes andaram antigamente correndo toda a costa do Brasil, de onde foram lançados
sempre do outro gentio,\footnote{ Em Varnhagen (1851 e 1879), ``foram sempre lançados''.}
com quem ficavam vizinhando, por suas ruins condições; do que ficaram mui odiados de todas
as outras nações do gentio.

Traz este gentio os beiços furados, e pedras neles e no rosto, como os Tupinambas; e,
ainda, se fazem mais furos nele, e se fazem mais bizarros; e quando se enfeitam o fazem na
forma dos Tupinambas, e trazem ao pescoço colares de dentes dos contrários como eles. E na
guerra usam dos mesmos tambores, trombetas, buzinas que costumam trazer os Tupinambas; os
quais são muito mais sujeitos ao pecado nefando do que são os Tupinambas, e os que servem
de machos se prezam muito disso, e o tratam, quando se dizem seus louvores.

Quando este gentio anda algum caminho, ou se acha em parte onde lhe falta fogo, esfregando
um pau rijo que para isso trazem com flechas fendidas, fazem acender esfregando muito com
as mãos até que se levanta labareda; o qual logo pega nas flechas, e desta maneira se
remedeiam; do que também se aproveitam os Tupinambas, quando têm necessidade de fogo.

Estes Tupinaes são os fronteiros dos Tupinambas, com os quais foram sempre apertando até
que os fizeram ir vizinhar com os Tapuyas, com quem têm sempre guerra sem entenderem em
outra coisa, da qual saem como lhes ordena a fortuna. Deste gentio Tupinae há já muito
pouco, em comparação do muito que houve, o qual se consumiu com fomes e guerras que
tiveram com seus vizinhos, de uma parte e da outra. Costumam estes índios nos seus
cantares tangerem com um canudo de uma cana de seis a sete palmos de comprido, e tão
grosso que cabe um braço, por grosso que seja, por dentro dele; o qual canudo é aberto
pela banda de cima, e quando o tangem vão tocando com o fundo do canudo no chão, e toa
tanto como os seus tambores, da maneira que os eles tangem.

\paragraph{[180] Em que se declara quem são os Amoypiras e onde vivem}\quad
Convém arrumarmos aqui os Amoypiras,\footnote{ Em Varnhagen (1851 e 1879), ``Amoipiras''.}
porque descendem dos Tupinambas e por estarem na fronteira dos Tupinaes, além do rio de
São Francisco; e passamos pelos Tapuyas, que ficam em meio para uma das bandas, por
estarem espalhados por toda a terra, de quem temos muito que dizer ao diante, no cabo
desta história da vida e costumes do gentio.

Quando os Tupinaes viviam ao longo do mar, residiam os Tupinambas no sertão, onde certas
aldeias deles foram fazendo guerra aos Tapuyas, que tinham por vizinhos, a quem foram
perseguindo por espaço de anos tão rijamente que entraram tanto pela terra adentro que
foram vizinhar com o rio de São Francisco. E neste tempo, outros Tupinambas fizeram
despejar aos Tupinaes de junto do mar da Bahia, como já fica dito, os quais os meteram
tanto pela terra adentro, afastando"-se dos Tupinambas, que tomaram os caminhos àqueles que
iam seguindo os Tapuyas, pelo que não puderam tornar para o mar por terem diante os
Tupinaes, que como se sentiram desapoderados dos Tupinambas,\footnote{ Em Varnhagen (1851
e 1879), ``se sentiram desapressados dos''. No manuscrito da \textsc{bgjm}, ``se sentiram
desapercebidos, digo desapoderados dos''.} que os lançaram fora da ribeira do mar, e
souberam destoutros Tupinambas que seguiram os Tapuyas, deram"-lhes nas costas e apertaram
com eles rijamente, o que também fizeram da sua parte os Tapuyas, fazendo"-lhes crua
guerra, ao que os Tupinambas não podiam resistir; e vendo"-se tão apertados de seus
contrários, assentaram de se passarem à outra banda do rio de São Francisco, onde se
contentaram da terra, e assentaram ali sua vivenda, chamando"-se Amoypiras, por o seu
principal se chamar Amoypira; onde esta gente multiplicou de maneira que tem senhoreado ao
longo deste rio de São Francisco, a que o gentio chama o Pará, mais de cem léguas, onde
agora vivem; e ficam"-lhe em frontaria, destoutra parte do rio, de um lado os Tapuyas, e de
outro os Tupinaes, que se fazem cruel guerra uns aos outros, passando com embarcações ao
seu modo à outra banda, dando grandes assaltos,\footnote{ No manuscrito da \textsc{bgjm},
``grandes faltos''.} nos contrários, os Amoypiras aos Tapuyas, que atravessam o rio em
almadias, que fazem da casca de árvores grandes, cujo feitio fica atrás declarado.

\paragraph{[181] Que trata da vida e costumes dos Amoypiras}\quad
Têm os Amoypiras a mesma linguagem dos Tupinambas; e a diferença que têm é em alguns nomes
próprios, que no mais entendem"-se muito bem; e têm os mesmos costumes e gentilidades; mas
são atraiçoados e de nenhuma fé, nem verdade.

Na terra onde este gentio vive estão mui faltos de ferramentas, por não terem comércio com
os portugueses; e apertados da necessidade cortam as árvores com umas ferramentas de
pedra, que para isso fazem; com o que, ainda que com muito trabalho, roçam o mato para
fazerem suas roças; do que também se aproveitavam antigamente todo o outro gentio, antes
que comunicasse com gente branca.

E para plantarem na terra a sua mandioca e legumes, cavam nela com uns paus tostados
agudos, que lhes servem de enxadas; os quais Amoypiras trazem o cabelo da cabeça copado e
aparado ao longo das orelhas, e as mulheres trazem os cabelos compridos como os
Tupinambas. Pesca este gentio com uns espinhos tortos que lhe servem de anzóis, com que
matam muito peixe, e a flecha, para o que são mui certeiros, e para matarem muita caça.

Trazem os Amoypiras os beiços furados e pedras neles como os Tupinambas; pintam"-se de
jenipapo e enfeitam"-se como eles; e usam na guerra tambores que fazem de um só pau, que
cavam por dentro com fogo, tanto até que ficam mui delgados, os quais toam muito bem; na
mesma guerra usam de trombetas que fazem de uns búzios grandes furados, ou da cana da
perna das alimárias que matam, a qual lavram e engastam em um pau. Em tudo o mais seguem
os costumes dos Tupinambas, assim na guerra como na paz, dos quais fica dito largamente no
seu título.

Estes Amoypiras têm por vizinhos no sertão detrás de si outro gentio, a que chamam
ubirajaras, com quem têm guerra ordinariamente, e se matam e comem uns aos outros com
muita crueldade, sem perdoarem as vidas, quando se cativam.

\paragraph{[182] Que trata brevemente da vivenda dos ubirajaras e seus costumes}\quad
Pelo sertão da Bahia, além do rio de São Francisco, partindo com os Amoypiras da outra
banda do sertão, vive uma certa nação de gente bárbara, a que chamam ubirajaras, que quer
dizer ``senhores dos paus'', os quais se não entendem na linguagem com outra nenhuma nação
do gentio; têm contínua guerra com os Amoypiras, e cativam"-se, matam"-se e comem"-se uns aos
outros, sem nenhuma piedade.

Estes ubirajaras não viram nunca gente branca, nem têm notícia dela, e é gente muito
bárbara, da estatura e cor do outro gentio, e trazem os cabelos muito compridos assim os
machos como as fêmeas, e não consentem em seu corpo nenhuns cabelos que, em lhes nascendo,
não arranquem.

Fazem estes ubirajaras suas lavouras, como fica dito dos Amoypiras, e pescam nos rios com
os mesmos espinhos e com outras armadilhas, que fazem com ervas; e matam muita caça com
certas armadilhas que fazem, em que lhes cai facilmente.\footnote{ Em Varnhagen (1851 e
1879), ``lhes facilmente cai''.}

A peleja dos ubirajaras é a mais notável do mundo, porque a fazem com uns paus tostados
muito agudos,\footnote{ Em Varnhagen (1851 e 1879), ``notável do mundo, como fica dito,
porque a''.} de comprimento de três palmos, pouco mais ou menos cada um, e são agudos de
ambas as pontas, com os quais atiram a seus contrários como com punhais; e são tão certos
com eles que não erram tiro, com o que têm grande chegada; e desta maneira matam também a
caça, que, se lhes espera o tiro, não lhes escapa, os quais com estas armas se defendem de
seus contrários tão valorosamente como seus vizinhos com arcos e flechas; e quando vão à
guerra, leva cada um seu feixe destes paus com que peleja, e com estas armas são muito
temidos dos Amoipiras, com os quais têm sempre guerra por uma banda, e pela outra com umas
mulheres, que dizem ter uma só teta, que pelejam com arco e flechas, e se governam e regem
sem maridos, como se diz das amazonas, das quais não podemos alcançar mais informações,
nem da vida e costumes destas mulheres.

\subsection{Começa a vida e costumes dos Tapuyas}

Como a atenção com que nos ocupamos nestas lembranças foi para mostrar bem o muito que há
que dizer das grandezas da Bahia de Todos os Santos,\footnote{ Em Varnhagen (1851 e 1879),
``há que dizer da Bahia de Todos os Santos''.} cabeça do Estado do Brasil, é necessário
que não fique por declarar a vida e costumes dos Tapuyas, primeiros possuidores desta
província da Bahia, de quem começamos a dizer o que se pode alcançar deles, começando no
capítulo que se segue.

\paragraph{[183] Que trata da terra que os Tapuyas possuíram e possuem hoje em dia}\quad
Até agora tratamos de todas as castas de gentio que vivia ao largo do mar da costa do
Brasil, e de algumas nações que vivem pelo sertão, de que tivemos notícia, e deixamos de
falar dos Tapuyas, que é o mais antigo gentio que vive nesta costa, do qual ela foi toda
senhoreada, desde a boca do Rio da Prata até a do Rio das Amazonas, como se vê do que está
hoje povoado e senhoreado deles; porque da banda do Rio da Prata senhoreiam ao longo da
costa mais de cento e cinquenta léguas, e da parte do rio das Amazonas senhoreiam para
contra o sul mais de duzentas léguas e pelo sertão vêm povoando por uma corda de terra por
cima de todas as nações do gentio nomeadas, desde o Rio da Prata até o das Amazonas, e
toda a mais costa senhorearam nos tempos atrás, de onde por espaço de tempo foram lançados
de seus contrários; por se eles dividirem e inimizarem uns com os outros, por onde se não
favoreceram, e os contrários tiveram forças para pouco a pouco os irem lançando na ribeira
do mar de que eles eram possuidores.

Atrás fica dito como foram lançados os Tapuyas da Bahia e seu limite pelos Tupinaes, os
quais se foram recolhendo para o sertão por espaço de tempo, onde até agora vivem
divididos em bandos, não se acomodando uns com os outros, antes têm cada dia diferenças e
brigas, e se matam muitas vezes em campo; por onde se diminuem em poder, para não poderem
resistir a seus contrários, com as forças necessárias; por se fiarem muito em seu esforço
e ânimo, não entendendo o que está tão entendido, que o esforço dos poucos não pode
resistir ao poder dos muitos.

\paragraph{[184] Que trata de quem são os Tapuyas, quem são os Maraquas}\quad
Como os Tapuyas são tantos e estão tão divididos em bandos, costumes e linguagem, para se
poder dizer deles muito, era necessário de propósito e devagar tomar grandes informações
de suas divisões, vida e costumes; mas, pois ao presente não é possível, trataremos de
dizer dos que vizinham com a Bahia, sobre quem se fundaram todas estas informações que
neste caderno estão relatadas; começando logo que os mais chegados Tapuyas aos povoadores
da Bahia são uns que se chamam de alcunha os Maraquas,\footnote{ Em Varnhagen (1851 e
1879), ``Maracás''.} os quais são homens robustos e bem acondicionados, trazem o cabelo
crescido até as orelhas e copado, e as mulheres os cabelos compridos atados detrás, o qual
gentio fala sempre de papo tremendo com a fala, e não se entende com outro nenhum gentio
que não seja Tapuya.

Quando estes Tapuyas cantam, não pronunciam nada, por ser tudo garganteado, mas a seu
modo; são entoados e prezam"-se de grandes músicos, a quem o outro gentio folga muito de
ouvir cantar. São estes Tapuyas grandes flecheiros, assim para a caça como para seus
contrários, e são muito ligeiros e grandes corredores, e grandes homens de pelejarem em
campo descoberto, mas pouco amigos de abalroar cercas; e quando dão em seus contrários, se
se eles recolhem em alguma cerca, não se detêm muito em os cercar, antes se recolhem logo
para suas casas, as quais têm em aldeias ordenadas, como costumam os Tupinambas.

Estes Tapuyas não comem carne humana, e se tomam na guerra alguns contrários,\footnote{ No
manuscrito da \textsc{bgjm}, ``e se tomam alguns contrários''.} não os matam; mas
servem"-se deles como de seus escravos, e por tais os vendem agora aos portugueses que com
eles tratam e comunicam.

São estes Tapuyas muito folgazões e não trabalham nas roças, como os Tupinambas, nem
plantam mandioca, nem comem senão legumes, que lhes as mulheres plantam, e granjeiam em
terras sem mato grande, a que põem o fogo para fazerem suas sementeiras; os homens
ocupam"-se em caçar, a que são muito afeiçoados.

Costuma este gentio não matar a ninguém dentro em suas casas, e se seus contrários,
fugindo"-lhes da briga, se acolhem a elas, não os hão de matar dentro, nem fazer"-lhes
nenhum agravo, por mais irados que estejam; e esperam que saiam para fora, ou, se lhes
passa a ira e aceitam"-nos por escravos, ao que são mais afeiçoados que a matá"-los, como
lhes fazem a eles.

São os Tapuyas contrários de todas as outras nações do gentio, por terem guerra com eles
ao tempo que viviam junto do mar, de onde por força de armas foram lançados; os quais são
homens de grandes forças, andam nus, como o mais gentio, e não consentem em si mais
cabelos que os da cabeça, e trazem os beiços furados e pedras neles, como os Tupinambas.

Estes Tapuyas são conquistados, pela banda do rio de Seregipe, dos Tupinambas que vivem
por aquelas partes; e por outra parte os vêm saltear os Tupinaes, que vivem da banda do
poente; e vigiam"-se ordinariamente de uns e dos outros; e está povoado deste gentio por
esta banda cinquenta ou sessenta léguas de terra; entre os quais há uma serra,\footnote{
No manuscrito da \textsc{bgjm}, ``há umas serras''.} onde há muito salitre e pedras
verdes, de que eles fazem as que trazem metidas nos beiços por bizarria.

\paragraph{[185] Em que se declara o sítio em que vivem outros Tapuyas, e de parte de seus
costumes}\quad
Pelo sertão da mesma Bahia, para a banda do poente oitenta léguas do mar, pouco mais ou
menos, estão umas serras que se estendem por uma banda e por a outra, e para o sertão mais
de duzentas léguas, tudo povoado de Tapuyas contrários destes de que até agora tratamos,
que se dizem os Maraquas, mas todos falam, cantam e bailam de uma mesma feição, e têm os
mesmos costumes no proceder da sua vida e gentilidades, com muito pouca diferença.

Estes Tapuyas têm guerra por uma banda com os Tupinaes, que lhes ficam a um lado muito
vizinhos, e por outra parte a têm com Amoipiras, que lhes ficam em fronteira da outra
banda do rio de São Francisco, e matam"-se uns aos outros cruelmente, dos quais se vigiam
de contínuo, contra quem pelejam com arcos e flechas, o que sabem tão bem manejar como
todo o gentio do Brasil. São estes Tapuyas grandes homens de fazer guerra a seus
contrários, e são mais esforçados que conquistadores, e mais fiéis que os Tupinaes.

Vivem estes Tapuyas em suas aldeias em casas bem arrumadas e tapadas pelas paredes de
pau"-a-pique,\footnote{ Em Varnhagen (1851 e 1879), ``bem tapadas pelas paredes, e armados
de pau"-a-pique''.} a seu modo, muito fortes, por amor dos contrários as não entrarem e
tomarem de súbito, em as quais dormem em redes, como os Tupinambas, com fogo à ilharga,
como faz todo o gentio desta comarca.

Não costuma este gentio plantar mandioca, nem fazer lavouras senão de milhos e outros
legumes, porque não têm ferramentas com que roçar o mato e cavar a terra, e por falta
delas quebram o mato pequeno às mãos, e às árvores grandes põem fogo ao pé de onde está
lavrado até que as derruba, e cavam a terra com paus agudos para plantarem suas
sementeiras; e o mais tempo se mantêm com frutas silvestres e com caça, a que são muito
afeiçoados.

Costuma este gentio Tapuya trazerem os machos os cabelos da cabeça tão compridos que lhe
dão pela cinta,\footnote{ Em Varnhagen (1851 e 1879), ``Costume deste gentio Tapuia é
trazerem aos machos os cabelos''.} e às vezes os trazem entrançados ou enastrados com
fitas de fio de algodão, que são como passamanes, mas muito largas; e as fêmeas andam
tosquiadas e trazem tingidas derredor de si umas franjas de fio de algodão, que têm os
cadilhos tão compridos que bastam para lhes cobrirem suas vergonhas, o que não trazem
nenhumas mulheres do gentio destas partes.

\paragraph{[186] Em que se declaram alguns costumes dos Tapuyas destas partes}\quad
Estes Tapuyas que vivem nesta comarca são muito músicos, e cantam pela maneira dos
primeiros; trazem os beiços debaixo furados, e neles umas pedras verdes roliças e
compridas, que lavram devagar, roçando"-as com outras pedras tanto até que as aperfeiçoam à
sua vontade.

Não pescam estes índios nos rios à linha, porque não têm anzóis; mas, para matarem peixe,
colhem uns ramos de umas ervas como vides, mas mui compridos e brandos, e tecem"-nos como
rede, os quais deitam no rio e tapam"-no de uma parte à outra; e uns têm mão nesta rede e
outros batem a água em cima, de onde o peixe foge e vem"-se descendo até dar nela, onde se
ajunta; e tomam às mãos o peixe pequeno,\footnote{ Em Varnhagen (1851 e 1879), ``pequeno
peixe''.} e o grande matam às flechadas, sem errarem um.

Costumam estes Tapuyas, para fazerem sal, queimarem uma serra de salitre, que está entre
eles, de onde tomam aquela cinza; e a terra queimada, lançam"-na na água do rio em
vasilhas, a qual fica logo salgada, e põem"-na ao fogo, onde a cozem e ferve tanto até que
se coalha, e fica feito o sal em um pão; e com este sal temperam seus manjares; mas o
salitre torna logo a crescer na serra para cima, mas não é tão alvo como o que não foi
queimado.

Entre estes Tapuyas há outros mais chegados ao rio de São Francisco, que estão com eles
desavindos, que são mais agrestes e não vivem em casas, e fazem sua vivenda em furnas onde
se recolhem; e têm uma destas serras mui áspera onde fazem sua habitação; os quais têm os
mesmos costumes que os de cima.

Corre esta corda dos Tapuyas toda esta terra do Brasil pelas cabeceiras do outro gentio, e
há entre eles diferentes castas, com mui diferentes costumes, e são contrários uns dos
outros; entre os quais há grandes discórdias, por onde se fazem guerra muitas vezes e se
matam sem nenhuma piedade.

\subsection{Daqui por diante se declara o grande cômodo que a Bahia tem para se
fortificar, e os metais que se nela dão}

Não parece despropósito arrumar à sombra do que está dito da Bahia de Todos os Santos, os
grandes aparelhos e cômodos que tem para se fortificar, como convém ao serviço de el"-rei
nosso senhor e ao bem da terra, para se poder resistir a quem a quiser ofender; o que
começamos a declarar pelo capítulo que se segue.

\paragraph{[187] Em que se declara a pedra que tem a Bahia para se poder fortificar}\quad
A primeira coisa que convém para se fortificar a Bahia é que tem pedra de alvenaria e
cantaria, de que há em todo o seu circuito muita comodidade, e grande quantidade para se
poder fazer grandes muros, fortalezas e outros edifícios; porque derredor da cidade há
muita pedra preta, assim ao longo do mar como pela terra, a qual é de pedreiras boas de
quebrar, com a qual se fazem paredes mui bem liadas; e pelos limites desta cidade há muita
pedra molar, como a de alvenaria de Lisboa, com que se faz boa obra; e ao longo do mar,
meia légua da cidade, e em muitos lugares mais afastados, há muitas lajes de pedra mole
como tufo,\footnote{ Em Varnhagen (1851 e 1879), ``há muitas lagoas de pedra''.} de que se
fazem cunhais em obra de alvenaria, com as quais se liam os edifícios que se na terra
fazem, e se afeiçoam os cunhais destas lajes com pouco trabalho, por estarem cortados pela
natureza conforme o para que são necessários.

Quando se edificou a cidade do Salvador, se aproveitaram os edificadores e povoadores dela
de uma pedra cinzenta boa de lavrar, que iam buscar por mar ao porto de Itapitanga, que
está sete léguas da cidade na mesma Bahia, da qual fizeram as colunas da Sé, portais e
cunhais e outras obras de meio relevo, e muitas campas e outras obras proveitosas; mas
depois se descobriu outra pedreira melhor,\footnote{ No manuscrito da \textsc{bgjm},
``depois de se descobrir''.} que se arranca dos arrecifes que se cobrem com a preamar das
marés de águas"-vivas ao longo do mar, a qual pedra é alva e dura, que o tempo nunca gasta,
mas trabalhosa de lavrar que gasta as ferramentas muito; de que se fazem obras mui primas
e formosas, e campas de sepulturas mui grandes; e parece a quem nisto tem atentado que
esta pedra se faz da areia congelada; porque ao longo dos mesmos arrecifes, bem chegados a
eles, é tudo rochedo de pedra preta, e estoutra é muito branca, depois de lavrada; mas não
é muito macia, a qual quando a lavram faz sempre uma grã areenta, e acham"-se muitas vezes
no âmago destas pedras cascas de ostras e de outro marisco, e uns seixinhos de areia, pelo
que se tem que esta pedra se formou de areia e que se congelou com a frialdade da água do
mar, o que é fácil de crer, porque se acham por estas praias limos enfarinhados de areia,
que está congelada e dura como pedra, e alguns paus de ramos de árvores também cobertos
desta massa tão dura como se foram de pedra.

\paragraph{[188] Em que se declara o cômodo que tem a Bahia para se poder fazer muita cal,
como se faz}\quad
A maior parte da cal que se faz na Bahia é das cascas das ostras, de que há tanta
quantidade que se faz dela muita cal, e que é alvíssima, e tão boa como a de
Alcântara;\footnote{ Em Varnhagen (1851 e 1879), ``alvíssima, e lisa também, como''.} e
fazem"-se dela guarnições de estuque mui alvas e primas; e a cal que se faz das ostras é
mais fácil de fazer que a de pedras; porque gasta pouca lenha e com lhe fazerem fogo que
dure dez, doze horas, fica muito bem cozida, e é tão forte que se quer caldeada, e ao
caldear ferve em pulos como a cal de pedra de Lisboa. Quanto mais que, quando não houvera
este remédio tão fácil, na ilha de Taparica, que está defronte da cidade, estão três
fornos de cal, onde se faz muita, que se vende a cruzado o moio; a qual cal é muito
estranha, porque se faz de umas pedras que se criam no mar neste sítio desta ilha e em
outras partes, as quais são muito crespas e artificiosas para outras curiosidades, e não
nascem em pedreiras, mas acham"-se soltas em muita quantidade. Estas pedras são sobre o
leve, por serem por dentro organizadas com alfebas. Esta pedra se enforna em fornos de
arcos, como os em que coze a louça, com sua abóbada fechada por cima da mesma pedra, mas
sobre os arcos está o forno todo cheio de pedra, e o fogo mete"-se"-lhe por baixo dos arcos
com lenha grossa, e coze em uma noite e um dia, e coze muito bem; cuja cal é muito alva, e
lia a obra que se dela faz como a de Portugal, e caldeiam"-na da mesma maneira; mas não
leva tanta areia como a cal que se faz das ostras e de outro qualquer marisco, de que
também se faz muito alva e boa, e para todas as obras. Quanto mais que, quando não houvera
remédio tão fácil para se fazer infinidade de cal, como o que está dito, com pouco
trabalho se podia fazer muita cal, porque na Bahia, no rio de Jaguaripe, e em outras
partes, há muita pedra lioz,\footnote{ Lioz: pedra calcária branca e compacta que é
utilizada em estatuária e em construções arquitetônicas.} como a de Alcântara, com umas
veias vermelhas, a qual pedra é muito dura, de que se fará toda obra"-prima, quanto mais
cal, para o que se tem já experimentado e coze muito bem; e se se não valem dela para
fazerem cal é porque acham estoutro remédio muito perto, e muito fácil; e para as mesmas
obras e edifícios que forem necessários, tem a Bahia muito barro de que se faz muita e boa
telha, e muito tijolo de toda a sorte; do que há em cada engenho um forno de tijolo e
telha, em os quais se coze também muito boa louça e formas que se faz do mesmo barro.

\paragraph{[189] Em que se declara os grandes aparelhos que a Bahia tem para se nela fazerem
grandes armadas}\quad
Pois sobejam aparelhos à Bahia para se poder fortificar, entenda"-se que lhe não faltam
para se poder fazer grandes armadas com que se possa defender e ofender a quem contra o
sabor de Sua Majestade se quiser apoderar dela, para o que tem tantas e tão maravilhosas e
formosas madeiras, para se fazerem muitas naus, galeões e galés, para quem não faltarão
remos, com que se eles possam remar, muito extremados, como já fica dito atrás; pois para
se fazer muito tabuado para estas embarcações sobeja cômodo para isso, porque há muitas
castas de madeiras, que se serram muito bem, como em seu lugar fica dito; para as quais o
que falta são serradores, de que há tantos na Bahia escravos de diversas pessoas, que,
convindo ao serviço de Sua Majestade trabalharem todos e fazer tabuado, ajuntar"-se"-ão pelo
menos quatrocentos serradores escravos mui destros, e duzentos escravos carpinteiros de
machado; e ajuntar"-se"-ão mais quarenta carpinteiros da ribeira, portugueses e mestiços,
para ajudarem a fazer as embarcações, os quais se ocupam em fazer navios que se na terra
fazem, caravelões, barcas de engenho e barcos de toda a sorte. O que resta agora de
madeira para fazerem estas naus e galés são mastros e vergas; disto há mais aparelho na
Bahia que nas províncias de Flandres; porque há muitos mastros inteiros para se
emastrearem naus de toda a sorte, e muitas vergas, o que tudo é mais forte do que os de
pinho e de mais dura, mas são mais pesados, o que tudo se achará à borda da água.

Bem sei que me estão já perguntando pela pregadura para essas armadas, ao que respondo que
na terra há muito ferro de veias para se poder lavrar, mas que enquanto se não lavra será
necessário vir de outra parte; mas se a necessidade for muita, há tantas ferramentas na
terra de trabalho, tantas ferragens dos engenhos que se poderão juntar mais de cem mil
quintais\footnote{ Quintal: antiga medida de massa que equivalia a quatro arrobas.} de ferro; e
porque tarde já em lhe dar ferreiro, digo que em cada engenho há um ferreiro com sua
tenda, e com os mais que têm tenda na cidade e em outras partes se pode juntar cinquenta
tendas de ferreiros, com seus mestres obreiros.

\paragraph{[190] Em que se apontam os quais aparelhos que há para se fazerem estas armadas}\quad
Parecerá impossível achar"-se na Bahia aparelho de estopa para se calafetarem as naus,
galeões e galés que se podem fazer nela, para o que tem facílimo remédio; porque há nos
matos desta província infinidade de árvores que dão embira, como temos dito quando falamos
da propriedade delas, a qual embira lhe sai da casca que é tão grossa como um dedo; como
está pisada é muito branda, e desta embira se calafetam as naus que se fazem no Brasil, e
todas as embarcações; de que há tanta quantidade como já dissemos atrás; a qual para
debaixo da água é muito melhor que estopa, porque não apodrece tanto, e incha muito na
água; e as costuras que se calafetam com a embira\footnote{ Em Varnhagen (1851 e 1879),
``envira''.} ficam muito mais fixas do que as que se calafetam com estopa, do que há muita
quantidade na terra. E se cuidar\footnote{ Cuidar: pensar.} quem ler estes apontamentos
que não haverá oficiais que calafetem estas embarcações, afirmo"-lhe que há
estantes na Bahia mais de duas dúzias, e achar"-se"-ão nos navios, que sempre estão no
porto, dez ou doze, que são calafates das mesmas naus, e há muitos escravos, também, na
terra, que são calafates por si sós, e à sombra de quem o sabem bem fazer.

Breu para se brearem estas embarcações não temos na terra, mas é por falta de se não dar
remédio a isto; porque ao longo do mar, em terras baixas de areia, é tudo povoado de umas
árvores que se chamam camasari,\footnote{ Em Varnhagen (1851 e 1879), ``camaçari''.} que
entre a casca e o âmago lançam infinidade de resina branca, grossa como terebentina de
Beta, a qual é tão pegajosa que se não tira das mãos senão com azeite quente, a qual, se
houver quem lhe saiba fazer algum cozimento, será muito boa para se brearem com ela os
navios, e far"-se"-á tanta quantidade que poderão carregar naus desta resina; e porque se
não podem brear as naus sem se misturar com a resina graxa, na Bahia se faz muita de
tubarões, lixas e outros peixes, com que se alumiam os engenhos e se breiam os barcos que
há na terra, e que é bastante para se adubar o breu para muitas naus. Quanto mais que se à
Bahia forem biscainhos ou outros homens que saibam armar às baleias, em nenhuma parte
entram tantas como nela, onde residem seis meses do ano e mais, de que se fará tanta graxa
que não haja embarcações que a possam trazer à Espanha.

\paragraph{[191] Em que se apontam os mais aparelhos que faltam para as embarcações}\quad
Pois que temos aparelho para lançar as embarcações que se podem fazer na Bahia ao mar,
convém que lhe demos os aparelhos com que estas embarcações possam navegar; e demos"-lhe
primeiro as bombas, que se fazem na terra muito boas, de duas peças porque têm extremadas
madeiras para elas; e para navios pequenos há umas árvores que a natureza furou por
dentro, que servem de bombas nos navios da costa, as quais são muito boas.

Pois os poleames se fazem de uma árvore que chamam jenipapo, que é muito bom de lavrar, e
nunca fende como está seco, de que se farão de toda a sorte. Enxárcia\footnote{ Enxárcia:
cordoalha que, em embarcações, sustenta os mastros e permite acesso às vergas.} para as
embarcações tem a Bahia em muita abastança, porque se faz da mesma embira com que
calafetam, antes de se amassar, aberta em febras\footnote{ Febra: pedaço ou fibra de
madeira.} a mão, a qual se fia tão bem como o linho, e é mais durável e mais rija que a de
esparto,\footnote{ Esparto: arbusto nativo do Mediterrâneo usado para obtenção de fibras.}
e tão boa como a do Cairo; e desta mesma embira se fazem amarras muito fortes e grossas e
de muita dura; e há na terra embira em abastança para se poder fazer muita quantidade de
enxárcia e amarras; e para amarras tem a terra outro remédio das barbas de umas palmeiras
bravas que lhes nascem ao pé, de comprimento de quinze e vinte palmos, de que se fazem
amarras muito fortes e que nunca apodrecem, de que há muita quantidade pelos matos para se
fazerem muitas quando cumprir. Pelo que não falta mais agora para estas armadas que as
velas, para o que há facílimo remédio, quando as não houver de lonas e pano de breu, pois
em todos os anos se fazem grandes carregações de algodão, de que se dá muito na terra; do
qual podem fazer grandes teais de pano grosso, que é muito bom para velas, de muita dura e
muito leves, de que andam velejados os navios e barcos da costa; e dentro da Bahia trazem
muitos barcos as velas de pano de algodão que se fia na terra, para o que há muitas
tecedeiras, que se ocupam em tecer teais de algodão, que se gastam em vestidos dos índios,
escravos de Guiné e outra muita gente branca de trabalho.

\paragraph{[192] Em que se aponta o aparelho que na Bahia tem para se fazer pólvora, e muita
picaria e armas de algodão}\quad
Pois temos dito o aparelho que a Bahia tem para se fortificar e defender dos corsários, se
a forem cometer; saibamos se tem alguns aparelhos naturais da terra com que se possam
ofender seus inimigos, não falando nos arcos e flechas do gentio, com o que os escravos da
Guiné, mamelucos e outros muitos homens bravos naturais de terra sabem pelejar, do que há
tanta quantidade nesta província; mas digamos das maravilhosas armas de algodão que se
fazem na Bahia, geralmente por todas as casas dos moradores, as quais não passa besta, nem
flecha nenhuma; do que se os portugueses querem antes armar que de cossoletes, nem
couraças; porque a flechada que dá nestas armas resvala por elas e faz dano aos
companheiros; e deste estofado de algodão armam os portugueses os corpos e fazem do mesmo
estofado celados para a cabeça, e muito boas adargas.

Fazem também na Bahia paveses e rodelas de copaiba, de que fizemos menção quando falamos
da natureza desta árvore, as quais rodelas são tão boas como as do adargueiro, e
d'avantagem por serem mais leves e estopentas, do que se farão infinidade delas muito
grandes e boas.

Dão"-se na Bahia muitas hastes de lanças do comprimento que quiserem, as quais são mais
pesadas que as de faia, mas são muito mais fortes e formosas; e das árvores de que estas
hastes tiram, há muitas de que se pode fazer muita picaria, e infinidade de dardos de
arremesso, que os Tupinambas sabem muito bem fazer.

E chegando ao principal, que é a pólvora, em todo o mundo se não sabe que haja tão bom
aparelho para ela como na Bahia, porque tem muitas serras que não têm outra coisa senão
salitre, o qual está em pedra alvíssima sobre a terra, tão fino que assim pega o fogo dele
como de pólvora mui refinada; pelo que se pode fazer na Bahia tanta quantidade dela que se
possa dela trazer tanta para a Espanha, com que se forneçam todos os Estados de que Sua
Majestade é rei e senhor, sem esperar que lhe venha da Alemanha, nem de outras partes, de
onde trazem este salitre com tanta despesa e trabalho, do que se deve de fazer muita
conta.

\paragraph{[193] Em que se declara o ferro, aço e cobre que tem a Bahia}\quad
Bem por culpa de quem a tem não há na Bahia muitos engenhos de ferro, pois o ela está
mostrando com o dedo em tantas partes, para o que Luís de Brito levou aparelhos para fazer
um engenho de ferro por conta de Sua Alteza e oficiais deste mister; e o porquê se não
fez, não serve de nada dizer"-se; mas não se deixou de fazer por falta de ribeiras de água,
pois a terra tem tantas e tão capazes para tudo; nem por falta de lenha e carvão, pois em
qualquer parte onde os engenhos de ferro assentarem há disto muita abundância. Também na
Bahia, trinta léguas pela terra adentro, há algumas minas descobertas sobre a terra de
mais fino aço que o de Milão; o qual está em pedra sem outra nenhuma mistura de terra nem
pedra; e não tem que fazer mais que lavrar"-se em vergas para se poder fazer obra com ele,
do que há muita quantidade que está perdido, sem haver quem ordene de o aproveitar; e
desta pedra de aço se servem os índios para amolarem as suas ferramentas com ela a mão.

E cinquenta ou sessenta léguas pela terra adentro tem a Bahia uma serra muito grande
escalvada, que não tem outra coisa senão cobre, que está descoberto sobre a terra em
pedaços, feitos em concavidades, crespo, que não parece senão que foi já fundido, ou, ao
menos, que andou fogo por esta serra, com que se fez este lavor no cobre, do que há tanta
quantidade que se não acabará nunca. E nesta serra estiveram por vezes alguns índios
Tupinambas e muitos mamelucos, e outros homens que vinham do resgate, os quais trouxeram
mostras deste cobre em pedaços, que se não foram tantas as pessoas que viram esta serra se
não podia crer senão que o derreteram no caminho de algum pedaço de caldeira que levavam;
mas todos afirmaram estar este cobre daquela maneira descoberto na serra.

\paragraph{[194] Em que se trata das pedras verdes e azuis que se acham no sertão da Bahia}\quad
Deve"-se também notar que se acham também no sertão da Bahia umas pedras azuis"-escuras
muito duras e de grande fineza, de que os índios fazem pedras que metem nos beiços, e
fazem"-nas muito roliças e de grande lustro, roçando"-as com outras pedras, das quais se
podem fazer peças de muita estima e grande valor, as quais se acham muito grandes; e entre
elas há algumas que têm umas veias aleonadas que lhes dão muita graça.

No mesmo sertão há muitas pedreiras de pedras verdes coalhadas, muito rijas, de que também
o gentio faz pedras para trazer nos beiços,\footnote{ Em Varnhagen (1851 e 1879), ``de que
o gentio também faz''.} roliças e compridas, as quais lavram como as de cima, com o que
ficam muito lustrosas; do que se podem lavrar peças muito ricas e para se estimarem entre
príncipes e grandes senhores, por terem a cor muito formosa; e podem se tirar da pedreira
pedaços de sete e oito palmos, e estas pedras têm grande virtude contra a dor de cólica.

Em muitas outras partes da Bahia, nos cavoucos que fazem as invernadas na terra, se acham
pedaços de finíssimo cristal, e de mistura algumas pontas oitavadas como diamante,
lavradas pela natureza que têm muita formosura e resplandor. E não há dúvida senão que
entrando bem pelo sertão desta terra há serras de cristal finíssimo, que se enxerga o
resplandor delas de muito longe, e afirmaram alguns portugueses que as viram que parecem
de longe as serras da Espanha quando estão cobertas de neve, os quais e muitos mamelucos e
índios que viram essas serras dizem que está tão bem criado e formoso este cristal em
grandeza, que se podem tirar pedaços inteiros de dez, doze palmos de comprido, e de grande
largura e fornimento, do qual cristal pode vir à Espanha muita quantidade para poderem
fazer dele obras mui notáveis.

\paragraph{[195] Em que se declara o nascimento das esmeraldas e safiras}\quad
Em algumas partes do sertão da Bahia se acham esmeraldas mui limpas e de honesto tamanho,
as quais nascem dentro em cristal, e como elas crescem muito, arrebenta o cristal; e os
índios quando as acham dentro nele, põem"-lhe o fogo para o fazerem arrebentar, de maneira
que lhe possam tirar as esmeraldas de dentro, com o que elas perdem a cor e muita parte do
seu lustro, das quais esmeraldas se servem os índios nos beiços, mas não as podem lavrar
como as pedras ordinárias que trazem nos beiços, de que já falamos. E entende"-se que assim
como estas esmeraldas se acham sobre a terra são finas, que o serão muito mais as que se
buscarem debaixo dela, e de muito preço, porque a que a terra despede de si deve ser a
escória das boas que ficam debaixo, as quais se não buscaram até agora por quem lhe
fizesse todas as diligências, nem chegaram a elas mais que mamelucos e índios, que se
contentavam de trazerem as que acharam sobre a terra, e em uma das partes onde se acham
estas esmeraldas, que é ao pé de uma serra, onde é de notar muito o seu nascimento; porque
ao pé desta serra, da banda do nascente, se acham muitas esmeraldas dentro no cristal
solto onde elas nascem; de onde trouxeram uns índios amostras, coisa muito para ver;
porque, como o cristal é mui transparente, trespassam as esmeraldas com seu resplandor da
outra banda, às quais lhes ficam as pontas da banda de fora, que parece que as meteram a
mão pelo cristal. E ao pé da mesma serra, da banda do poente, se acham outras pedras muito
escuras que também nascem no cristal, as quais mostram um roxo cor de púrpura muito fino,
e tem"-se grande presunção de estas pedras poderem ser muito finas e de muita estima. E
perto desta serra está outra de quem o gentio conta que cria umas pedras muito vermelhas,
pequenas e de grande resplandor.

Afirmam os índios Tupinambas, os Tupinaes, Tamoios e Tapuyas e os índios que com eles
tratam neste sertão da Bahia e no da capitania de São Vicente, que debaixo da terra se
cria uma pedra do tamanho e redondeza de uma bola, a qual arrebenta debaixo da terra; e
que dá tamanho estouro como uma espingarda, ao que acodem os índios e cavam a terra, onde
toou este estouro, onde acham aquela bola arrebentada, em quartos como romã, e que lhe
saem de dentro muitas pontas cristalinas do tamanho de cerejas, as quais são de uma banda
oitavadas e lavradas mui sutilmente em ponta como diamante, e da outra banda onde pegavam
da bola, tinham uma cabeça tosca, das quais trouxeram do sertão amostras delas ao
governador Luís de Brito, que quando as viu teve pensamento que seriam diamantes; mas um
diamante de um anel entrava por elas, e a casca da bola era de pedra não muito alva, e
ruivaça, por fora.

\paragraph{[196] Em que se declara a muita quantidade de ouro e prata que há na comarca da
Bahia}\quad
Dos metais de que o mundo faz mais conta, que é o ouro e prata, fazemos aqui tão pouca,
que os guardamos para o remate e fim desta história, havendo"-se de dizer deles primeiro,
pois esta terra da Bahia tem dele tanta parte quanto se pode imaginar; do que podem vir à
Espanha cada ano maiores carregações do que nunca vieram das Índias Ocidentais, se Sua
Majestade for disto servido, o que se pode fazer sem se meter nesta empresa muito cabedal
de sua fazenda, de que não tratamos miudamente por não haver para quê, nem fazer ao caso
da tenção destas lembranças, cujo fundamento é mostrar as grandes qualidades do Estado do
Brasil, para se haver de fazer muita conta dele, fortificando"-lhe os portos principais,
pois têm tanto cômodo para isso como no que toca à Bahia está declarado; o que se devia
pôr em efeito com muita instância, pondo os olhos no perigo em que está de chegar à
notícia dos luteranos parte do conteúdo neste \textit{Tratado}, para fazerem suas armadas,
e se irem povoar esta província, onde com pouca força que levem de gente bem armada se
podem senhorear dos portos principais, porque não hão de achar nenhuma resistência neles,
pois não têm nenhum modo de fortificação, de onde os moradores se possam defender nem
ofender a quem os quiser entrar. E se Deus permitir por nossos pecados, que seja isto,
acharão todos os cômodos que temos declarado e muitos mais para se fortificarem, porque
hão de fazer trabalhar os moradores nas suas fortificações com as suas pessoas, com seus
escravos, barcos, bois, carros e tudo mais necessário, e com todos os mantimentos que
tiverem por suas fazendas, o que lhes há de ser forçado fazer para com isso resgatarem as
vidas; e com a força da gente da terra se poderão apoderar e fortificar de maneira que não
haja poder humano com que se possam tirar do Brasil estes inimigos, donde podem fazer
grandes danos a seu salvo em todas as terras marítimas da Coroa de Portugal e Castela, o
que Deus não permitirá; de cuja bondade confiemos que deixará estar estes inimigos de
nossa santa fé católica com a cegueira que até agora tiveram de não chegar à sua notícia o
conteúdo neste \textit{Tratado}, para que lhe não façam tantas ofensas estes infiéis, como
lhe ficarão fazendo se se senhorearem desta terra, que Deus deixe crescer em Seu santo
serviço; com o que o Seu santo nome seja exalçado, para que Sua Majestade a possa possuir
por muitos e felizes anos com grandes contentamentos. Amém

\end{linenumbers}
